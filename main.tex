\documentclass[12pt]{report}
\usepackage[utf8]{inputenc}
\usepackage{graphicx}
\graphicspath{ {C:/Users/utente/Desktop/misc/} }
\usepackage[english]{babel}

\usepackage{titling}
\setlength{\droptitle}{-10em}
\usepackage{soul}
\usepackage[usenames,dvipsnames]{xcolor}
\usepackage{amsthm}
\usepackage{bm,amsmath}
\usepackage {tikz, tikz-cd }
\usepackage{quiver}
\usetikzlibrary {positioning, shapes.geometric, arrows, decorations.text}
\usetikzlibrary { shapes }
\usetikzlibrary { patterns }
\usepackage{tabularx}
\usepackage{caption}
\usepackage{subcaption}
\usepackage{bbold}
\usepackage{stmaryrd}
\usepackage{amssymb}
\usepackage{yfonts}
\usepackage{mathtools}
\usepackage{enumitem}
\usepackage{trimclip,adjustbox}


\makeatletter
\newcommand*\bigcdot{\mathpalette\bigcdot@{1.5}}
\newcommand*\bigcdot@[2]{\mathbin{\vcenter{\hbox{\scalebox{#2}{$\m@th#1\bullet$}}}}}
\makeatother


\theoremstyle{plain}
\newtheorem{thm}{Theorem}
\newtheorem{cor}[thm]{Corollary}
\newtheorem{lem}[thm]{Lemma}
\newtheorem{prop}[thm]{Proposition}
\newtheorem*{thm*}{Theorem}
\newtheorem*{lem*}{Lemma}
\newtheorem*{prop*}{Proposition}
\newtheorem*{cor*}{Corollary}

\theoremstyle{definition}
\newtheorem{definition}{Def.}[section]
\newtheorem{ex}[thm]{Example}


\theoremstyle{remark}
\newtheorem{remark}{Remark}


\newcommand{\VDashA}{%
	\mathrel{\text{\clipbox{0pt 0pt {.8\width} 0pt}{$\Vdash$}}\mkern.9mu}\vDash
}

\newcommand{\VDashB}{%
	\mathrel{
		\text{\clipbox{0pt 0pt {.8\width} 0pt}{$\Vdash$}}
		\mkern.9mu
		\text{\adjustbox{width=.87\width,height=\height}{$\vDash$}}
	}
}



\title{
	{\includegraphics{unimi.png}} \\
	\vspace{0.3cm}
	{\large Dipartimento di Matematica Federigo Enriques}\\
	\vspace{1cm}
	{\large \textbf{Tesi di Laurea Magistrale in Matematica }}\\
	\vspace{1cm}
		{\emph{Topos semantics for three-valued Gödel-Dummett Logic}}\\
}

\author{\\ \\ Author: Alberto Panerai \\ 
	Student ID: 980237 \\ Supervisor: Prof. Stefano Aguzzoli}

\date{26/09/2023}



 \begin{document}
 	\maketitle
 	
 	${}$ \newpage
 	\chapter*{Abstract}
 	This work builds upon Professor S.Aguzzoli \& P.Codara's paper \cite{towards}. \newline
 	Its primary aim is to explore the topos semantics of a particular fragment of super-intuitionistic and fuzzy logic namely \emph{three-valued Gödel-Dummett Logic} $\mathcal{G}_3$. 
 	The dual-algebraic semantics of $\mathcal{G}_3$ is given by the sub-category of \emph{finite forests} of height at most two $\mathbb{FF}_2$, a.k.a. \emph{bushes}. Building upon the topos-semantics of \emph{bushes} at the propositional layer, we arrive through a new path at first-order topos semantics for $\mathcal{G}_3$.\newline
 In chapter 1 we present this logic after a more general introduction of intuitionistic logic and its principal semantics given by Heyting algebras and Kripke frames, which will come up later on. \newline
 In chapter 2 we give a detailed account of \emph{finite forests} and in particular the sub-category of \emph{bushes}, which forms a topos and is dually equivalent to finite three-valued Gödel algebras.
Continuing where \cite{towards} left off, we provide a novel counterexample to show why $\mathbb{FF}_k$, i.e., \emph{finite forests} of height at most $k$ with $k$ greater than two, fails to be a topos. An original Python code is also used to aid in these constructions. \newline
 In chapter 3, following the development outlined in \cite{goldblatt} and giving whenever necessary some generalities of topos semantics, we examine the propositional layer of the topos semantics given by \emph{bushes} and propose some results regarding its internal and external logics. \newline
 In chapter 4 we follow up from chapter 3 and move on to first order logic. We start by giving an account by \cite{goldblatt} of first order topos models and make some considerations on predicates and modeling three-valued Gödel-Dummett logic before taking an alternative type-theoretic approach to quantifiers due to \cite{lambekscott} and applying it to our topos of bushes. Conclusive remarks are made on this application and the recovery of first order Gödel-Dummett logic. \newline   
 In chapter 5,  we compare our findings with alternative topos semantics for $\mathcal{G}_3$ given by \emph{variable sets} and \emph{sheaves} and make some new remarks on the matter. \newline
 The conclusion is a short summary of what has been achieved throughout this work, its justification and a few comments on further areas of research. \newline\newline
 %CHECK
 \hl{(Highlighted sections or sub-sections contain the original results made by the author.)
}
 	\chapter*{Dedication}
 	To Mum and Dad. \newline
 	Without their love and support none of this would be possible.
 	\newpage
 		${}$ \newpage
 	
 	\chapter*{Declaration}
 	I hereby declare that this thesis represents my own work which has
 	been done after registration for the degree of \emph{Laurea Magistrale in Matematica} at \emph{Università degli Studi di Milano}, and has not been
 	previously included in a thesis or dissertation submitted to this or any
 	other institution for a degree, diploma or other qualifications.
 	\newpage
 		${}$ \newpage
 		
 	\chapter*{Acknowledgements}
 	I want to thank again my family and anyone who helped me in my mathematical journey. \newline
 	Special thanks are due to my thesis advisor Professor S. Aguzzoli who showed me a truly fascinating area of research and has been very supportive throughout this endeavor. \newline
 	I also thank Professors V. Marra, S. Ghilardi and N.Gambino for their wonderful lectures and discussions which in no small part helped me in this endeavor.  
 		\newpage
 	${}$ \newpage
 	\tableofcontents
 	
 \chapter{Introduction}
 
	\section{Foreword}
		
	This thesis touches on the subjects of category theory and \emph{non-classical} mathematical logic.
 	It aims to study the \emph{Topoi} semantics of a particular \emph{non classical} logic named Gödel-Dummett Logic. \newline
	To introduce what Gödel-Dummett Logic is we first give an account (guided by \cite{goldblatt} \& \cite{awodey}) of what \emph{Intuitionistic Propositional Logic} and more broadly \emph{Intuitionism} is. 
	\newline
	\newline
%	CHECK
	(An introduction to the basic notions of category theory and mathematical logic is not given in this work.
	For this we refer to \cite{awodey},\cite{maclane},\cite{goldblatt} among others)
	

	\newpage
	
	\section{A Non-Classical World}
	
	\emph{Intuitionistic} or \emph{constructive} logic is commonly described as \emph{classical} logic without Aristotle's \emph{law of excluded middle}, a.k.a. \emph{tertium non datur}:
	\begin{definition}
		(LEM) : $A \lor \neg A$.
	\end{definition} 
	Or equivalently without the law of double negation elimination \footnote{used for \emph{reductio ad absurdum} proofs.}:
	\begin{definition}
		(DNE) : $\neg \neg A \Rightarrow A$.
	\end{definition}
	L.E.J. Brouwer in the first half of the twentieth century observed that (LEM) was abstracted from finite to infinite situations without much justification.\newline Take the following statement about the existence of infinitely many twin primes a.k.a. the \emph{twin prime conjecture} $\mathcal{P}$ in number theory:
	\begin{gather*}
		\forall x \in \mathbb{N} \; (A(x) \lor \neg A(x)) \\
		A(x) := \exists y \in \mathbb{N} ( (y > x) \land (prime(y) \land prime(y+2)) )
	\end{gather*}
	The problem with this assertion is that the \emph{twin prime conjecture} $\mathcal{P}$ like many other unsolved problems in Mathematics has not yet been proven or dis-proven so there is no proof either of $\mathcal{P}$ or $\neg \mathcal{P}$ and hence at this present \emph{state of knowledge} there can be no \emph{constructive} or \emph{effective} proof of $\mathcal{P} \lor \neg\mathcal{P}$. \newline For Brouwer and constructivists alike the claim of $\mathcal{P} \lor \neg\mathcal{P}$ is simply not acceptable. The crucial difference with the classical case is that formulae are only considered \emph{true} when we have proof or direct \emph{evidence of truth}. \newline
	This \emph{reductive} notion of truth has not been without controversy. Doing away with (LEM) or (DNE) is problematic as they are so commonly used in mathematical practice. \newline
	 For example consider a \emph{non-constructive} proof of the statement: "there exist irrational numbers $a,b$ such that $a^b$ is rational".
	 \begin{prop}
	 	$\exists\; a,b \in \mathbb{R} \setminus \mathbb{Q} : a^b \in \mathbb{Q}$.
	 \end{prop}
 \begin{proof}
 	$\sqrt{2}^{\sqrt{2}}$ is either rational or irrational. \newline
 	 In the first case we take $a = b = \sqrt{2}$. \newline
 	In the second case we take $a = \sqrt{2}^{\sqrt{2}}, b= \sqrt{2}$ and we are done.
 \end{proof}
 What's wrong? \newline
	Note that in the above proof the status of $\sqrt{2}^{\sqrt{2}}$ is \emph{undefined} and hence no \emph{constructive} proof is given for its rationality or lack thereof. \newline
	Definitive evidence of the irrationality of $\sqrt{2}^{\sqrt{2}}$ has been given by the Gelfond-Schneider Theorem which establishes \emph{constructively} that for any complex algebraic numbers $a,b \neq 1$ with $b$ irrational, $a^b$ is in fact a transcendental and hence irrational number. So the original statement has a \emph{constructive} proof.\newline 

	The \emph{BHK-interpretation} named after Brouwer,Heyting and Kolmogorov provides a framework for \emph{intuitionistic logic}.\newline In classical logic the meaning of statements is given by stating for example the following \emph{truth conditions}: 
	\begin{itemize}
		\item $\phi \land \chi$ is true iff $\phi$ is true and $\chi$ is true.
		\item $\phi \lor \chi$ is true iff $\phi$ is true or $\chi$ is true.
		\item $\neg \phi$ is true iff $\phi$ is not true.
	\end{itemize}
	In the BHK-interpretation the notion of \emph{proof}\footnote{not to be construed as syntactic or formal proof but as intuitive/informal proof or convincing mathematical argument.} replaces that of classical \emph{truth}.
	The usual logical formulae are now interpreted as:
	\begin{itemize}
		\item No proof of \emph{false} $\bot$ exists. 
		\item a proof of $\phi \;\land\; \chi$ consists of a pair of proofs, i.e., \newline   "$< \text{proof of } \phi$ ; $\text{proof of } \chi >$".
		\item a proof of $\phi \lor \chi$ is either a proof of one or the other, i.e., \newline  "proof of $\phi$ $|$ a proof of $ \chi $".
		\item a proof of $\phi \Rightarrow \chi $ consists of a \emph{method of converting} a proof of $\phi$ into a proof of $\chi$
		\item a proof of $\neg \phi$ consists of a \emph{method of converting} a proof of $\phi$ into a proof of \emph{false} (in other words "$\phi$ has no proof").
	\end{itemize}
	Moving on to first order logic and quantifiers if we assume the variables to range over a domain \emph{D}:
	\begin{itemize}
		\item a proof of $\forall x A(x)$ is a \emph{construction} that transforms any $d \in D$ into a proof of $A(d)$.
		\item a proof of $\exists x A(x)$ is a pair $<d$ ; $\text{proof of } A(d) >$.
	\end{itemize}
	
	
	How do we now formalize Intuitionistic Logic starting from the propositional layer?

	\newpage

	\section{From \textbf{IPL} to \textbf{CPL}}
	
	We begin with a presentation of the basic \emph{syntax} of \emph{Intuitionistic Propositional Logic} or \textbf{IPL}. \newline
	We do this by specifying a \emph{Formal Language} presented as an \emph{alphabet} composed of a denumerable set of propositional variables $\textbf{Prop}:=\{..,p,q,r,..\}$ and the connectives symbols $\neg, \land, \lor, \Rightarrow$. \footnote{we could also add symbols for brackets (, ) } 
	\newline
	We define \emph{inductively} Formulae $\mathbf{Form} = \{\alpha,\beta,..\phi,\chi,\psi,..\}$ in the following way:
	\begin{definition}[propositional formula]
		$\mathbf{Form}$ := $p$ $|$ $\bot$ $|$ $ F \land F $ $|$ $ F \lor F $ $|$ $ F \Rightarrow F $.
		\begin{enumerate}[label=\roman*]
			\item $\bot \in \mathbf{Form}$.
			\item  If $p\in \mathbf{Prop}$ then $p \in \mathbf{Form}$.
			\item  If $\alpha, \beta \in \mathbf{Form}$ then $(\alpha \land \beta) \in \mathbf{Form}$.
			\item  If $\alpha, \beta \in \mathbf{Form}$ then $(\alpha \lor \beta) \in \mathbf{Form}$.
			\item  If $\alpha, \beta \in \mathbf{Form}$ then $(\alpha \Rightarrow \beta) \in \mathbf{Form}$.
		\end{enumerate}
	\end{definition} 
	The constant $\bot$ represents falsity, $\land$ conjunction, $\lor$ disjunction and $\Rightarrow$ implication.
	We also define the following operators for truth, negation and double implication respectively:
	\begin{definition}
		$\top$ := $\bot \Rightarrow \bot  \;\;$,	$\; \neg \alpha$ := $\alpha \Rightarrow \bot \;\;$,	$\; \alpha \Leftrightarrow \beta$ := $(\alpha \Rightarrow \beta) \land (\beta \Rightarrow \alpha)$.
	\end{definition}

	We characterize \emph{\textbf{IPL}} by its axioms. \newline
	 The axioms for \textbf{IPL} are instances of the following schemata: \newline (Here \{\ A,B,C,..\} are meta-variables/placeholders for any propositional formula).
	
		\begin{enumerate}
			\item $ A \Rightarrow (A \land A) $.
			\item $ (A \land B) \Rightarrow (B \land A) $.
			\item $ (A \Rightarrow B) \Rightarrow ((A \land C) \Rightarrow (B \land C)) $.
			\item $ ((A \Rightarrow B) \land (B \Rightarrow C)) \Rightarrow (A \Rightarrow C) $.
			\item $ B \Rightarrow (A \Rightarrow B) $.
			\item $ (A \land (A \Rightarrow B)) \Rightarrow B $.
			\item $ A \Rightarrow (A \lor B) $.
			\item $ (A \lor B) \Rightarrow (B \lor A) $.
			\item $ ((A \Rightarrow C) \land (B \Rightarrow C)) \Rightarrow ((A \lor B) \Rightarrow C) $.
			\item $ \neg A \Rightarrow (A \Rightarrow B) $.
			\item $ ((A \Rightarrow B) \land (A \Rightarrow \neg B)) \Rightarrow \neg A $.
		\end{enumerate}
				
	Having specified the axioms, now we introduce a \emph{Propositional Calculus}\footnote{due to \emph{A. Heyting}.} for \textbf{IPL}. 
	We do this by defining a relation of \emph{entailment} or  \emph{(syntactic) derivation} between formulae
	 $\vdash$ .  \newline
	All we need to specify this \emph{derivation} is a single rule of inference that is \emph{the rule of detachment} a.k.a. \emph{Modus Ponens} [MP] .
	
	\begin{definition}[MP] For every $\alpha,\beta \in \mathbf{Form}$: \newline
		If $\vdash \alpha$ and $\vdash (\alpha \Rightarrow \beta)$, then $\vdash \beta$.
		 \newline
		which means: 
		\emph{"From $\alpha$ and $\alpha \Rightarrow \beta$ one can derive $\beta$".}
	\end{definition}
	
	$\textbf{TH}_{\textbf{IPL}}$, i.e., the \emph{Theorems} for \textbf{IPL} or \emph{provable} formulae in \textbf{IPL} denoted simply by $\vdash \phi$ \footnote{one could also use the notation $\vdash_{\textbf{IPL}}$. However we will refrain from doing so if the context is unambiguous.} are defined as all the formulae that one can derive or \emph{deduce} from $\top$ "truth" or from any instance of the axiom schemata of \textbf{IPL} in a \emph{finite} number of steps using [MP].
	\newline The relation $\alpha \vdash \beta$ in this context is characterized as $\vdash (\alpha \Rightarrow \beta)$.
	\newline
		
		
		The \emph{algebraic Semantics} for \textbf{IPL} can be obtained through the construction of a \emph{Lindenbaum-Tarski algebra} $\mathcal{A}$. \newline
		$\mathcal{A}$ is formed by taking the equivalence classes of formulae $\phi$ by the \emph{double-entailment} relation ($\phi \dashv\vdash \chi$ means "$\phi\vdash\chi$ and $\chi\vdash\phi$"):

\[ \phi \equiv \chi \text{ iff } \phi \dashv\vdash \chi\] 


		Furthermore $\mathcal{A}$ admits a well defined ordering given by entailment:
		\begin{definition}
			$[\phi] \leq [\chi] \;$	iff		$\; \phi \vdash \chi $.
		\end{definition}
		We then define the following operations in $\mathcal{A}$ induced by the logical connectives:
		\begin{itemize}
			\item $1 := [\top]$.
			\item $0 := [\bot]$.
			\item $[\phi] \land [\chi] := [\phi \land \chi]$.
			\item $[\phi] \lor [\chi] := [\phi \lor \chi]$.
			\item $[\phi] \Rightarrow [\chi] := [\phi \Rightarrow \chi]$.
	\end{itemize}

	\begin{definition}
		$\mathcal{A}$ := $(\textbf{Form}/\equiv\;, \leq)$.
	\end{definition}
	Since for every $\phi \in \textbf{Form}$ $(\bot \vdash \phi)$ and $(\phi \vdash \top)$ it follows that
	$\mathcal{A}$ is a bounded \emph{lattice} with minimum  $0 = [\bot]$ and maximum $1 = [\top]$.

\begin{prop}
	A formula $\phi$ is \emph{provable} $\top \vdash \phi$ iff $[\phi] = 1$.
\end{prop}
 		$\mathcal{A}$ with these operations is a \emph{Heyting algebra}. \newline
 		The class of Heyting algebras will be denoted by $\mathbf{HA}$. \newline
 		 What exactly is a \emph{Heyting algebra}?
 	\newline
 		A \emph{Heyting algebra} $\mathcal{H}$ is first of all a \emph{partially ordered set}, a.k.a. \emph{poset} \footnote{recall that this means that the ordering is \emph{reflexive,antisymmetric and transitive}.. not necessarily \emph{total}..}
 		$(\mathcal{H}, \leq)$ endowed with constants 0,1, binary operations \emph{join} $\lor$ and \emph{meet} $\land$
 		that form a \emph{bounded lattice}. For this the following must hold for all $a,b,c \in \mathcal{H}$:
 		 		\begin{enumerate}
 			 			\item $0 \leq a$.
 			 			\item $a \leq 1$.
 			  			\item $a \lor b \leq c \;$ iff $\;a \leq c$ and $b \leq c$.
 						\item $c \leq a \land b \;$ iff $\;c \leq a$ and $c \leq b$.
 				\end{enumerate} 
 	
 		\begin{definition}[Heyting algebra]\label{heytingalg}
 		The bounded lattice $(\mathcal{H}, \lor, \land, 0, 1)$ is said to be a \emph{Heyting algebra} if it admits a binary operator $\Rightarrow$ for which $(a \Rightarrow b) \in \mathcal{H}$, a.k.a. an \emph{exponential element}\footnote{or \emph{pseudo-complement of a with respect to b}.} such that:\newline
 			For every $a,b,c \in \mathcal{H}$ \newline there exists $(a \Rightarrow b) \in \mathcal{H}$  such that: 
 			\begin{equation*}
 				c \leq (a \Rightarrow b) \;\text{ iff }\; (c \land a) \leq b.
 			\end{equation*}
 		\end{definition}
 %check 
 \begin{remark}
 	The previous condition on exponential elements corresponds to the categorical notion of an \emph{adjoint pair} between the functors $(- \land a)$ and $(a \Rightarrow -)$ which gives rise to \emph{cartesian closed categories}.\footnote{see \cite{awodey} or \cite{maclane} for details.}
 \end{remark}
 		
 			This exponential element can also be taken as the \emph{greatest} element with this property, i.e.:
 		\begin{prop}
 			$a \Rightarrow b \equiv sup\{ c \in \mathcal{H} : c \land a \leq b \}$.	
 		\end{prop}
 	

For an analogue of classical negation, 
 		we can define for every $b \in \mathcal{H}$ a \emph{pseudo-complement} of b:
 		\begin{definition}[pseudo-complement]
 			$\neg b := (b \Rightarrow 0)$.
 		\end{definition}

\begin{remark}
	An equivalent categorical definition for a Heyting algebra, given in \cite{awodey}, is that of a \emph{cartesian closed poset}. 
\end{remark}
		We now specify the environment for the algebraic semantics of \textbf{IPL}:
	
		An $\mathcal{H}$-\emph{Interpretation} $\mathcal{I}: \mathbf{Prop} \rightarrow \mathcal{H}$ of \textbf{IPL} in a Heyting algebra $\mathcal{H}$, a.k.a. $\mathcal{H}$-\emph{valuation} is an \emph{assignment} of the propositional variables $p,q,r..$ of $\mathbf{Prop}$ to elements of $\mathcal{H}$ denoted by $\llbracket p \rrbracket_{\mathcal{H}}, \llbracket q \rrbracket_{\mathcal{H}}, \llbracket r \rrbracket_{\mathcal{H}}$...\newline
		From now on, if the context is clear $\llbracket - \rrbracket_{\mathcal{H}}$ will simply be denoted by $\llbracket - \rrbracket$. \newline 
		The interpretation  $\mathcal{I}$ is extended recursively to $\mathbf{Form}$:
		\begin{definition}[H.A. interpretation]
			Given an assignment $\mathcal{I}: \mathbf{Prop} \rightarrow \mathcal{H}$ with $p \mapsto \llbracket p \rrbracket$,\newline $\mathcal{I}: \mathbf{Form} \rightarrow \mathcal{H}$ is defined as follows:
			\begin{gather*}
			\llbracket \phi \land \chi \rrbracket := \llbracket \phi \rrbracket \land \llbracket \chi \rrbracket .\\
			\llbracket \phi \lor \chi \rrbracket := \llbracket \phi \rrbracket \lor \llbracket \chi \rrbracket .
			\\
			\llbracket \phi \Rightarrow \chi \rrbracket := \llbracket \phi \rrbracket \Rightarrow \llbracket \chi \rrbracket .\\
			\llbracket \neg \phi \rrbracket := \neg\llbracket\phi\rrbracket. \\
			 \llbracket \bot \rrbracket := 0. 
			 \\ \llbracket \top \rrbracket := 1.
			\end{gather*}
		\end{definition}

		We can talk about what it means for a formula $\phi$ to be \emph{valid} in $ \mathcal{H} $:
		
		\begin{definition}[validity in HA]
		$\phi$ is valid in $\mathcal{H}$ denoted by $\mathcal{H} \models \phi$ when $\llbracket \phi \rrbracket = 1$ for any chosen assignment for the $\mathcal{H}$-interpretation $\mathcal{I}$.
		\end{definition}
	
				There is a \emph{canonical} interpretation of \textbf{IPL} in $\mathcal{A}$ given simply by $\llbracket p \rrbracket := [p]$ and inductively by $\llbracket \phi \rrbracket := [\phi]$ that \emph{validates} only the provable formulae.
	
			If a formula $\phi$ is \emph{always true} it is called a \emph{tautology}:
		
		\begin{definition}[tautology in \textbf{IPL}]
			A formula $\phi$ is a \emph{tautology}\footnote{tautology in \textbf{IPL}.} or \emph{Intuitionistically valid} denoted by $\vDash \phi$ if it is valid in every Heyting algebra $\mathcal{H}$.
		\end{definition}	
		
		\begin{remark}
			If a formula $\phi$ is a \emph{tautology}, one has in particular in $\mathcal{A}$ that $1 = \llbracket \phi \rrbracket = [\phi]$ and so $\vdash \phi$. 
		\end{remark}	
		
		We can show that all axioms are tautologies and that the inference rule (MP) \emph{preserves validity}, i.e., "if $\vDash \phi$ and $\vDash (\phi \Rightarrow \psi)$ then $\vDash \psi$". \newline
		This gives us the following result:
		
		\begin{prop}
			$\vdash \phi$ iff  $\vDash \phi$, i.e., "$\phi$ is provable\footnote{in \textbf{IPL}.} iff $\phi$ is \emph{intuitionistically} valid".
		\end{prop}
		
		which means that:
		
		\begin{prop}
			\begin{equation*}
				\vdash_{IPL} \alpha \text{ iff } H \vDash \alpha \text{ for all } H \in \mathbf{HA}.
			\end{equation*}
			,i.e. , \emph{Heyting algebras} provide \emph{sound} and \emph{complete} semantics for \textbf{IPL}
		\end{prop}

		What about \emph{Classical Propositional Logic} or \textbf{CPL}? 
	\newline
	We keep the same syntax and calculus used in the intuitionistic case but add an axiom "$12. \; (A \lor \neg A)$" which corresponds to the Law of Excluded Middle (LEM).\newline
	We use the following notation to show this fact:
	
		\[ \textbf{CPL} = \textbf{IPL} + (A \lor \neg A) \]
 
	This time a provable formulae in \textbf{CPL}, i.e., $\vdash_{CPL} \phi$ \footnote{again, from now on we omit to specify $\vdash_{\textbf{CPL}}$.} can also be derived from any instance of (LEM) and thus:
	\begin{prop}
		$\textbf{TH}_\textbf{IPL} \subset \textbf{TH}_\textbf{CPL}$.
	\end{prop}
	We will see that this inclusion is strict. \newline
	
		Now, we can construct the Lindenbaum-Tarski algebra $\mathcal{A'}$ as before by taking the equivalence classes of formulae by double entailment and repeating the same definitions for its ordering and operations. 
		The following still holds:
		\begin{prop}
			A formula $\phi$ is \emph{provable} $\vdash \phi$ iff $[\phi] = 1$.
		\end{prop}
		$\mathcal{A'}$ is a special case of a \emph{Heyting algebra} called a \emph{Boolean algebra}. \newline
		What is a Boolean algebra? 
		\begin{definition}
			A Boolean algebra $(\mathcal{B}, \lor, \land, \Rightarrow, 0, 1)$ is a Heyting algebra where for every $a \in \mathcal{B}$ the pseudo-complement of $a$, i.e., $\neg a$ is a \emph{complement} of $a$ which means:
			\begin{enumerate}
				\item $a \land (\neg a) = 0$.
				\item $a \lor (\neg a) = 1$. 
			\end{enumerate}
		\end{definition}
				
		We denote the class of Boolean algebras by $\mathbf{BA}$. \newline
	\newline
		 			Note that with these definitions
		 					every Boolean algebra is a Heyting algebra though in general a Heyting algebra need not be Boolean. In fact we will see examples in which this is the case.
		 			
		The algebraic semantics for \textbf{CPL} are given in a very similar way to the classical case.
		An \emph{Interpretation}  or $\mathcal{B}$-valuation $\mathcal{V}$ of \textbf{CPL} in a Boolean algebra $\mathcal{B}$ is again an assignment of propositional variables to elements of $\mathcal{B}$ denoted by $\llbracket p \rrbracket, \llbracket q \rrbracket, \llbracket r \rrbracket$.. The interpretation extends recursively to all formulae as before with the addendum $\llbracket \neg \phi \rrbracket = \neg \circ \llbracket \phi \rrbracket$.
		
		More commonly, we deal with a \emph{valuation} $\mathcal{V}$ in the two-element Boolean algebra of \emph{truth values} $\mathbb{2}=\{0 < 1\}$ which specifies a \emph{value assignment}, i.e., any function $V : \textbf{Prop} \rightarrow \mathbb{2}$. Also by our recursive definition of $\mathcal{V}$:
		
		\begin{remark}
			$V : \textbf{Prop} \rightarrow \mathbb{2}$ is \emph{lifted} in a unique way to $V : \textbf{Form} \rightarrow \mathbb{2}$ a so-called \emph{truth}-function on any formula.
		\end{remark}

		We can talk about what it means for a formula $\phi$ to be \emph{valid} in $ \mathcal{B} $:
		
		\begin{definition}[validity in BA]
			$\phi$ is valid in $\mathcal{B}$ denoted by $\mathcal{B} \models \phi$ when $\llbracket \phi \rrbracket = 1$ for any chosen assignment for the $\mathcal{B}$-valuation $\mathcal{V}$.
		\end{definition}
	There is a \emph{canonical} interpretation of \textbf{CPL} in $\mathcal{A}$ given simply by $\llbracket p \rrbracket := [p]$ and inductively by $\llbracket \phi \rrbracket := [\phi]$ that \emph{validates} only the provable formulae.
		
		If a formula $\phi$ is \emph{always true} it is called a \emph{tautology}:
		
		\begin{definition}[tautology in \textbf{CPL}]
			A formula $\phi$ is a \emph{tautology}\footnote{tautology in \textbf{CPL}.} or \emph{classically valid} denoted by $\vDash \phi$ if it is valid in every Boolean algebra $\mathcal{B}$.
		\end{definition}	
				
In a similar fashion as before, we conclude that:
		
		\begin{prop}
			\begin{equation*}
				\vdash_{CPL} \alpha \text{ iff } B \vDash \alpha \text{ for all } B \in \mathbf{BA}.
			\end{equation*}		,i.e. ,
			\emph{Boolean algebras} provide \emph{sound} and \emph{complete} semantics for \emph{Classical Propositional Logic}
		\end{prop}
		
		
		To sum up:
		\begin{itemize}
			\item Heyting algebras provide a (sound and complete) algebraic semantics for Intuitionistic Propositional Logic (\textbf{IPL}).
			\item 	$\textbf{CPL} = \textbf{IPL} + (A \lor \neg A)$ and 	$\textbf{TH}_\textbf{IPL} \subset \textbf{TH}_\textbf{CPL}$ . 
			\item $\mathbf{BA} \subset \mathbf{HA}$ Boolean algebras are a particular class of Heyting algebras  which provide a (sound and complete) algebraic semantics for Classical Propositional Logic (\textbf{CPL}).
		\end{itemize} 
	
	
			The canonical examples of Boolean algebras are:
			
			\begin{ex}[Power-Set]
					$( \mathcal{P}(X),  \subseteq )$, i.e., the set of sub-sets, a.k.a. \emph{Power-Set} of a set $X$ with the set-inclusion partial ordering is a \emph{Boolean algebra} with $0 := \emptyset$, $1 := X$, $\lor := \cup$, $\land := \cap$ and $\neg := (\;)^\complement $.
				\end{ex}
			
			\begin{ex}[$\mathbb{2}$]
				We encountered a special case of the \emph{power-set} Boolean algebra, which corresponds to the power-set of the singleton set $\mathcal{P}(\{*\})$, i.e., the \emph{two-element} Boolean algebra of \emph{truth-values} $\mathbb{2} := \{0 < 1\}$, where $0$ and $1$ represent $false$ and $true$ respectively. \newline
				Note that here the Law of Excluded Middle (LEM) clearly holds:\newline $\mathbb{2} \models (\alpha \lor \neg \alpha)$.
			\end{ex}
			
			\begin{ex}[$free_1(B)$]
				The Boolean algebra \emph{freely generated}\footnote{this can be seen as the Lindenbaum-Tarski algebra with just one propositional variable.} by the element $p$. 
			\end{ex}
			
			\begin{figure}[h]
				\centering
				\begin{tikzpicture}[thick,scale=0.6, every node/.style={scale=0.8}]
					\node (A) at (0,0) {0};
					\node (B) at (0,3) {1};
					\draw[line width=.01in] (A) -- (B);
				
					\node (A) at (8,0) {0};
					\node (B) at (6,2) {$p$};
					\node (C) at (10,2) {$\neg p$};
					\node (D) at (8,4) {1};
					\draw[line width=.01in] (A) -- (B);
					\draw[line width=.01in] (A) -- (C);
					\draw[line width=.01in] (C) -- (D);
					\draw[line width=.01in] (B) -- (D);
				\end{tikzpicture}
				\caption{ $\mathbb{2}$ (left) and $free_1(B)$ (right). }
			\end{figure}
			
For Heyting algebras the canonical example is:

\begin{ex}[Open Sets]
	The lattice of \emph{Open Sets} $(\mathcal{O},\subseteq)$ (ordered by inclusion) of a topological space $(X,\tau)$  is a Heyting algebra with $0 := \emptyset$, $1 := X$, $\lor := \cup$, $\land := \cap$ and: \newline $U \Rightarrow V := (U^\complement \cup V)^\circ$, i.e., the \emph{largest open sub-set} of $U^\complement \cup V$. \newline
	This means that whenever $W$ is open, $W \subseteq (U^\complement \cup V)^\circ$ iff $(U \cap W) \subseteq V$.
\end{ex}

The following example shows that not every Heyting algebra need be Boolean   $\mathbf{BA} \subsetneq \mathbf{HA}$ and that 	$\textbf{TH}_\textbf{IPL} \subsetneq \textbf{TH}_\textbf{CPL}$ :

\begin{ex}[The case of $\neg\neg a \neq a$]
	Note that in the Heyting algebra of Open Sets the \emph{pseudo-complement} of any open set corresponds to the interior of its complement, i.e., $\neg U := (U \Rightarrow \emptyset) = (U^\complement)^\circ$. \newline
	Note that in a Boolean algebra "$\neg\neg a = a$" must hold for every element. \newline
	$a \leq \neg \neg a$ holds in $(\mathcal{O},\subseteq)$ since $\neg \neg a = ((a \Rightarrow 0) \Rightarrow 0)$ and $a \land (a \Rightarrow 0) \leq 0$ by \emph{evaluation}. \newline\newline
	The converse $\neg\neg a \leq a$ however in general does not hold.\newline
	Take for example the open sets in $[0,1]$ with the induced standard topology. \newline
	By unfolding the definition of $\neg$ we obtain : $\neg\neg (0,1) = [0,1] \nsubseteq (0,1)$.\newline\newline
	Finally, if we consider (in the induced topology of $[0,1]$)  the open set $[0,1/2)$ and  $[0,1/2)^\complement = (1/2,1]$ we observe that $[0,1/2) \cup [0,1/2)^\complement \subsetneq [0,1]$, i.e., \newline LEM $A \lor \neg A$ is not \emph{intuitionistically} valid. In other words:
	\begin{equation*}
		\nvdash_{IPL} A \lor \neg A.
	\end{equation*}
	 
\end{ex}
	
\begin{ex}[$free_1(H)$]
		The Heyting algebra \emph{freely generated}\footnote{this again can be seen as the Lindenbaum-Tarski algebra with just one propositional variable.} by the element $p$. \newline
		This forms an infinite lattice called the \emph{Rieger-Nishimura ladder}. 
	\end{ex} 
	
	
	\begin{figure}[h]
		\centering
		\begin{tikzpicture}[thick,scale=0.6, every node/.style={scale=0.6}]
			\node (A) at (0,0) {$0$};
			\node (B) at (-2,2) {$P_1 := \neg p$};
			\node (C) at (2,2) {$P_2 := p$};
			\node (D) at (0,4) {$P_4 := p \lor \neg p$};
			\node (E) at (4,4) {$P_3 := \neg \neg p$};
			\node (F) at (2,6) {$P_6 := \neg p \lor \neg \neg p$};
			\node (G) at (-2,6) {$P_5 := \neg \neg p \Rightarrow p$};
			\node (H) at (0,8) {$P_8 := P_5 \lor P_3$};
			\node (I) at (4,8) {$P_7 := P_5 \Rightarrow P_4$};
			\node (J) at (2,10) {$P_{10} := P_5 \lor P_7$};
			\node (J') at (3,11) {};
			\node (K) at (-2,10) {$P_{9} := P_7 \Rightarrow P_6$};
			\node (L) at (0,12) {$P_{12} := P_7 \lor P_9$};
			\node (L') at (1,13) {};
			\node (L'') at (-1,13) {};
			\node (M) at (0,14) {.......................................};
			\node (N) at (0,15) {.......................................};
			\node (0) at (0,16) {.......................................};
			\node (P) at (0,17) {$1$};
			\draw[line width=.01in] (A) -- (B);
			\draw[line width=.01in] (A) -- (C);
			\draw[line width=.01in] (B) -- (D);
			\draw[line width=.01in] (C) -- (D);
			\draw[line width=.01in] (C) -- (E);
			\draw[line width=.01in] (E) -- (F);
			\draw[line width=.01in] (D) -- (F);
			\draw[line width=.01in] (D) -- (G);
			\draw[line width=.01in] (G) -- (H);
			\draw[line width=.01in] (F) -- (H);
			\draw[line width=.01in] (F) -- (I);
			\draw[line width=.01in] (H) -- (J);
			\draw[line width=.01in] (I) -- (J);
			\draw[line width=.01in] (H) -- (K);
			\draw[line width=.01in] (K) -- (L);
			\draw[line width=.01in] (J) -- (L);
			\draw[dotted, line width=.01in] (J) -- (J');
			\draw[dotted, line width=.01in] (L) -- (L');
			\draw[dotted, line width=.01in] (L) -- (L'');
		\end{tikzpicture}
		\caption{$free_1(H)$ displayed as the \emph{Rieger-Nishimura ladder} in which we used auxiliary meta-variables $P_n$ to label the nodes. }
	\end{figure}
	
	
	
	
	
\newpage
	
	\section{States of Knowledge}
	
	We introduce, following \cite{goldblatt}, a new semantics for IPL given by \emph{S.Kripke} which will give us a better insight into our area of study. \newline
	The structure we shall be concerned with is a poset \textbf{P} called a \emph{Kripke Frame}  which represents a finite set of \emph{possible worlds}, a.k.a. \emph{states of knowledge} with a so-called \emph{temporal ordering}.\newline
	(Propositional) Formulae are now interpreted as sub-sets of this poset which represents the states at which the sentence is \emph{true} . \newline
	A formula, we shall see, is not true or false per se but rather \emph{true at a certain state of knowledge} and once true remains true in all \emph{future states}, i.e., we have \emph{the persistence of truth in time}. \newline
	This accords well with the intuitionistic point of view in which a formula is true when its truth has been \emph{constructively determined} at some state or \emph{stage} and \emph{constructive knowledge} once established lasts forever. \newline
	Note that this temporal ordering of states of knowledge need not be \emph{linear}. This embodies the idea that from the \emph{present} state, there might be more than one possible \emph{future} state. Consider for example a world in which \emph{Fermat's Last Theorem} is determined to be true and one in which it is shown to be false. 
	
	\begin{definition}[Kripke Frame]
	A Kripke Frame $\textbf{P} := (P, \sqsubseteq)$ is a finite set of \emph{possible worlds}
	ordered by a partial (so-called \emph{temporal}) ordering $\sqsubseteq$.
	\end{definition} 
	
	Having fixed a Kripke Frame,
	we introduce the notion of an \emph{hereditary} sub-set of \textbf{P}:
	
	\begin{definition}[hereditary sub-set]
		$A \subseteq P$ is \emph{hereditary} if it is \emph{upwards-closed} under $\sqsubseteq$, i.e., if $p \in A$ and $p \sqsubseteq q$ we have that $q \in A$. \newline
		The collection of all these hereditary up-sets will be denoted by $\textbf{P}^+$.
	\end{definition}  
	
	Semantics is now given by assigning to each propositional variable a hereditary sub-set by a \textbf{P}-valuation $\mathcal{V}$, i.e., a function $\mathcal{V} : \textbf{Prop} \rightarrow \textbf{P}^+$. 
	
	\begin{remark}
		$\mathcal{V}(\textbf{p})$ formalizes the idea of \emph{"the set of states at which $\textbf{p}$ is determined to be true"} and being an hereditary sub-set this means that the \emph{knowledge of this truth is persistent in time}.
	\end{remark}
	
	We define a \emph{Kripke Model} $\mathcal{M}$ and give a formal definition of what it means for a formula to be \emph{true} at a particular state:
	
	\begin{definition}[Kripke Model]
		A \emph{Kripke Model} based on \textbf{P} is $\mathcal{M} := (\textbf{P}, \mathcal{V})$ where $\mathcal{V}$ is a \textbf{P}-valuation. \newline
		We define inductively a \emph{forcing relation} $\mathcal{M} \VDashA_w \alpha$ \newline, i.e., \emph{"in $\mathcal{M}$ world $w$ forces formula $\alpha$"} or also \emph{"the formula $\alpha$ is true in $\mathcal{M}$ at  $w \in P$"}:
		\begin{enumerate}
			\item $\mathcal{M} \VDashA_w \textbf{p}$ iff $w \in \mathcal{V}(\textbf{p})$.
			\item $\mathcal{M} \VDashA_w (\alpha \land \beta)$ iff $\mathcal{M} \VDashA_w \alpha$ and $\mathcal{M} \VDashA_w \beta$.
			\item $\mathcal{M} \VDashA_w (\alpha \lor \beta)$ iff either $\mathcal{M} \VDashA_w \alpha$ or $\mathcal{M} \VDashA_w \beta$.
			\item $\mathcal{M} \VDashA_w (\neg\alpha)$ iff for all $s$ such that $w \sqsubseteq s$ it is not the case that $\mathcal{M} \vDash_s \alpha$.
			\item $\mathcal{M} \VDashA_w (\alpha \Rightarrow \beta)$ iff for all $s$ such that $w \sqsubseteq s$, if $\mathcal{M} \vDash_s \alpha$ then $\mathcal{M} \vDash_s \beta$.
		\end{enumerate}
		With the expression $\mathcal{M} \vDash \alpha$, i.e., \emph{$\alpha$ is true in $\mathcal{M}$}, we indicate that $\mathcal{M} \VDashA_w \alpha$ holds for all $w \in \textbf{P}$.
	\end{definition}
	
	\begin{remark}
		The truth of $\neg \alpha$ at state $w$ means that $\alpha$ is never verified at any later stage.
	\end{remark}
	
	The notion of \emph{validity} is readily given:
	
	\begin{definition}[Kripke Validity]
		A formula $\alpha$ is valid on the frame \textbf{P}, denoted by $\textbf{P} \VDashA \alpha$, if for every model $\mathcal{M}$ based on \textbf{P} we have $\mathcal{M} \VDashA \alpha$. 
	\end{definition}
	
	Let us now consider the hereditary set of states at which $\alpha$ is \emph{true} in $\mathcal{M}$:
	
	\begin{definition}
		$\mathcal{M}(\alpha) := \{ w : \mathcal{M} \VDashA_w \alpha\}$.
	\end{definition} 
	
	Also we introduce the following operations:  
	\begin{definition}
		For any hereditary sets $S,  T$:
	\begin{gather*}
		\neg S := \{w: \text{ for all } z, \; w \sqsubseteq z, \; z \notin S\}. \\
		S \Rightarrow T := \{ w: \text{ for all } z, \; w \sqsubseteq z, \;\text{ if } z \in S \text{ then } z \in T \}.
	\end{gather*}	
	\end{definition}
	
	
	With these definitions: 
	\begin{prop}
		For any hereditary $S,T,U$: \newline
		$ U \subseteq (S \Rightarrow T) $ iff $ (S \cap U) \subseteq T $ and
		$\neg S = S \Rightarrow \emptyset$.
	\end{prop}
	
	The definition of a Kripke Model we gave earlier can be given in an equivalent fashion by requiring that:
	
	\begin{enumerate}[label=(\roman*)]
		\item $\mathcal{M}(\textbf{p}) = \mathcal{V}(\textbf{p})$.
		\item $\mathcal{M}(\alpha \land \beta) = \mathcal{M}(\alpha) \cap \mathcal{M}(\beta)$.
		\item $\mathcal{M}(\alpha \lor \beta) = \mathcal{M}(\alpha) \cup \mathcal{M}(\beta)$.
		\item $\mathcal{M}(\neg\alpha) = \neg\mathcal{M}(\alpha)$.
		\item $\mathcal{M}(\alpha \Rightarrow \beta) = \mathcal{M}(\alpha) \Rightarrow \mathcal{M}(\beta)$.
	\end{enumerate}
	
	Note that since the intersection $\cap$ and union $\cup$ of hereditary sets is hereditary:
	
	\begin{thm}
		$(\textbf{P}^+, \subseteq )$ with the operations $\cap$,$\cup$ and $\Rightarrow$ is a Heyting algebra.
	\end{thm}
	
	Any \textbf{P}-valuation $\mathcal{V}: \textbf{Prop} \rightarrow \textbf{P}^+$ can now be seen as a $\textbf{P}^+$-valuation for the Heyting algebra  $\textbf{P}^+$. \newline
	We have $\mathcal{V}(\textbf{p}) = \mathcal{M}(\textbf{p})$ and by extending  $\mathcal{V}$ inductively to arbitrary formulae\footnote{$\mathcal{V}(\alpha \land \beta) := \mathcal{V}(\alpha) \cap \mathcal{V}(\beta)$, $\mathcal{V}(\alpha \lor \beta) := \mathcal{V}(\alpha) \cup \mathcal{V}(\beta)$, $\mathcal{V}(\alpha \Rightarrow \beta) := \mathcal{V}(\alpha) \Rightarrow \mathcal{V}(\beta)$ etc..}:
	\begin{lem}
		$\mathcal{M}(\alpha) = \mathcal{V}(\alpha)$ for any formula $\alpha$.
	\end{lem}
	Putting all this together:
	
	\begin{prop}
		$\mathcal{M} \VDashA \alpha$ iff $\mathcal{M}(\alpha)=P$ iff $\mathcal{V}(\alpha)=P$.
	\end{prop}
	
	Also, since $P$ is the top element of the lattice $\textbf{P}^+$, we obtain a link between \emph{Kripke validity} on the frame \textbf{P} and \emph{Heyting algebra validity} on $\textbf{P}^+$:
	
	\begin{thm}
		$\textbf{P} \VDashA \alpha$ iff $\textbf{P}^+ \models \alpha$.
	\end{thm}
	
Returning to the Semantics of IPL, we have the following result:
	
	\begin{thm}
		 \begin{equation*}
			\vdash_{IPL} \alpha\text{ iff }\textbf{P} \VDashA \alpha \text{ for any frame }\textbf{P}.
		\end{equation*} i.e.,
		Kripke Semantics is sound and complete for IPL.
\end{thm}

	Soundness comes from the fact that if $\vdash_{IPL} \alpha$ then for any Heyting algebra $H$ we have $H \models \alpha$. 
	For any frame \textbf{P} we thus have $\textbf{P}^+ \models \alpha$ and so $\textbf{P} \VDashA \alpha$. \newline
	For Completeness, for $p\in \textbf{P}$ we introduce the set of sentences \emph{known to be true at p}, a.k.a. the \emph{extension} of $p$ :
	
	\begin{definition}[extension of $p$]
		 $\Gamma_p := \{ \alpha : \mathcal{M} \VDashA_p \alpha \}$ denotes the \emph{extension} of $p$.
		This set of sentences satisfies the following properties:
		\begin{enumerate}
			\item (soundness) If $\vdash_{IPL} \alpha$ then $\alpha \in \Gamma_p$.
			\item (closure under MP) If $\vdash_{IPL} \alpha \Rightarrow \beta$ and $\alpha \in \Gamma_p$ then $\beta \in \Gamma_p$.
			\item (consistency) There exists an $\alpha$ such that $\alpha \notin \Gamma_p$.
			\item (primality) If $\alpha \lor \beta \in \Gamma_p$ then $\alpha \in \Gamma_p$ or $\beta \in \Gamma_p$.
		\end{enumerate}
	\end{definition}
	
		$\Gamma_p$ is also called a \emph{state-description} of the state $p\in \textbf{P}$ by discerning the sentences known to be true at $p$.
	
	\begin{definition}[full set]
		A set $\Gamma \subseteq \textbf{Form}$ that satisfies the previous conditions 1.-4. is called \emph{full}.
	\end{definition}
	
	We can now introduce the so-called \emph{canonical model} for IPL:
	
	\begin{definition}[canonical frame]
		The canonical frame for IPL is the collection of \emph{all} full sets ordered by inclusion $\textbf{P}_{IPL} := (P_{IPL}, \subseteq)$.
	\end{definition}
	
	\begin{definition}[canonical model]
		The canonical model for IPL is given by $\mathcal{M}_{IPL}:=(\textbf{P}_{IPL}, V_{IPL})$ where:
		\begin{equation*}
			V_{IPL} : \textbf{p}_i \mapsto \{\Gamma: \textbf{p}_i \in \Gamma\}
		\end{equation*}
		i.e., the canonical valuation assigns to every proposition the set of full sets that contain it.
	\end{definition}
	
	If we add the following results:
	
	\begin{lem}
		$\mathcal{M} \VDashA_\Gamma \alpha$ iff $\alpha \in \Gamma$.
	\end{lem}
	
	\begin{lem}[Lindenbaum]
		$\vdash_{IPL} \alpha$ iff $\alpha$ is a member of every full set.
	\end{lem}
	
	We obtain the desired completeness result:
	
	\begin{thm}
		$\vdash_{IPL} \alpha$ iff $\mathcal{M}_{IPL} \VDashA \alpha$ iff $\textbf{P}_{IPL} \VDashA \alpha$.
	\end{thm}
	
	Kripke Semantics can also provide a \emph{topological} interpretation of \emph{intuitionism} as for any frame \textbf{P} the hereditary sets $\textbf{P}^+$ form a \emph{topology}:
	
	\begin{prop}
		$\textbf{P}^+$ is the Heyting algebra of Open Sets for the topology just described.\newline
		In particular one has $\neg S = (S^\complement)^\circ$ the largest hereditary subset of $S^\complement$ and 
	\end{prop}
	

	Another nice feature of Kripke semantics is that one can determine the validity (or lack thereof) of formulae by looking at the \emph{structure} of the frames. \newline 
	
To illustrate this point,
	a few simple examples of  Kripke Models are illustrated:
		
	\begin{ex}
		$\mathcal{T} := \{ \textbf{2}, \mathcal{V} \}$. \newline
		Where $\textbf{2} :=\{ 0 < 1 \}$ and  $\mathcal{V}: \textbf{p} \mapsto {1}$.\newline
		We can ask ourselves if the instance of LEM "$\textbf{p} \lor \neg \textbf{p}$" is valid on this frame.\newline
		The answer is \emph{no}. To see this:\newline
		Note that $\mathcal{T} \nVDash_0 \textbf{p} $ but $\mathcal{T} \VDashA_1 \textbf{p} $ with $0\leq 1$.\newline
		So, by definition we also have  $\mathcal{T} \nVDash_0 \neg\textbf{p} $ and so  $\mathcal{T} \nVDash_0 (\textbf{p} \lor \neg \textbf{p}) $ i.e., LEM is not valid on this frame.
	\end{ex}
	
	
	\begin{figure}[h]
		\centering
		\begin{tikzpicture}[thick,scale=0.6, every node/.style={scale=0.8}]
			\node (A) at (0,0) {$\bigcdot_0$};
			\node (B) at (0,3) {$\bigcdot_1$};
			\node (b) at (-1,3) {\textbf{p}};
			\draw[line width=.01in] (A) -- (B);
			\end{tikzpicture}
		\caption{ Kripke Model $\mathcal{T}$. The propositions \emph{true} at $0,1$ are listed beside the node labeled by the state $0,1$.}
	\end{figure}
	
	
	
	\begin{ex}
		$\mathcal{K} := \{ \textbf{W}, \mathcal{V} \}$. \newline
		Where $\textbf{W} :=\{r < w_1 < w_3, r < w_2 < w_3\}$ and  $\mathcal{V}: \textbf{p} \mapsto {w_1,w_3}, \textbf{q} \mapsto {w_2,w_3 }$. \newline
		Similarly as before the LEM is not valid on this frame though its weaker version\footnote{wLEM := $(\neg \alpha \lor \neg \neg \alpha)$.} $ \neg\textbf{p} \lor \neg\neg \textbf{p} $ holds. \newline
		
		What can we say of the classically valid \emph{pre-linearity axiom}, i.e., \newline  "$(\alpha \Rightarrow \beta) \lor (\beta \Rightarrow \alpha)$"? \newline
		Since $\mathcal{T} \VDashA_{w_1} \textbf{p} $, $\mathcal{T} \nVDash_{w_1} \textbf{q} $ and $\mathcal{T} \VDashA_{w_1} \textbf{q} $, $\mathcal{T} \nVDash_{w_1} \textbf{p} $ we have that: \newline$\nvDash_r (\textbf{p} \Rightarrow \textbf{q}) \lor (\textbf{q} \Rightarrow \textbf{p})$.\newline
		This tells us that the pre-linearity axiom is not valid on this frame and thus is not intuitionistically valid, i.e., 
		\begin{equation*}
			\nvdash_{IPL} (\alpha \Rightarrow \beta) \lor (\beta \Rightarrow \alpha).
		\end{equation*}
		However this formula is valid for example on the previous frame $\mathcal{T}$.   
	\end{ex}
	
	
	\begin{figure}[h]
		\centering
		\begin{tikzpicture}[thick,scale=0.6, every node/.style={scale=0.8}]
			\node (A) at (0,0) {$\bigcdot_r$};
			\node (B) at (-3,3) {$\bigcdot_{w_1}$};
			\node (b) at (-4,3) {\textbf{p}};
			\node (C) at (3,3) {$\bigcdot_{w_2}$};
			\node (c) at (4,3) {\textbf{q}};
			\node (D) at (0,6) {$\bigcdot_{w_3}$};
			\node (d) at (-1.5,6) {\textbf{p},\textbf{q}};
			\draw[line width=.01in] (A) -- (B);
			\draw[line width=.01in] (A) -- (C);
			\draw[line width=.01in] (B) -- (D);
			\draw[line width=.01in] (C) -- (D);
		\end{tikzpicture}
		\caption{ Kripke Model $\mathcal{K}$. The propositions \emph{true} at $w_i$ are listed beside the node labeled by the state $w_i$.}
	\end{figure}
	
	
We can also recover \emph{classical} validity: 
	
	\begin{ex}
		Recall the classical validity of the form $\mathbb{2} \models \alpha$, i.e., validity on the two-element Boolean algebra $\mathbb{2}=\{0<1\}$, means that $\llbracket \alpha \rrbracket = 1$ for any truth assignment of the propositional variables $V: \textbf{Prop} \rightarrow \mathbb{2}$. If we now specify: \newline  
		
		$\mathcal{C} := \{ \{ w \}, \tilde{V} \}$. \newline
		Where $\mathcal{V}: \textbf{p} \mapsto {w}$ iff $V: \textbf{p} \mapsto 1$.\newline
		So classical validity is the same as Kripke-validity on this discrete frame, i.e.,
		$\{ w \} \VDashA \alpha$ iff $\mathbb{2} \models \alpha$.
	\end{ex}
	
	
	\begin{figure}[h]
		\centering
		\begin{tikzpicture}[thick,scale=0.6, every node/.style={scale=0.8}]
			\node (A) at (0,0) {$\bigcdot_w$};
			\node (a) at (-1,0) {$\textbf{p}_i$};
		\end{tikzpicture}
		\caption{ Kripke Model $\mathcal{C}$ in which $\textbf{p}_i$ stands for all the propositions valued at 1.}
	\end{figure}
	
	
	These examples suggest the following definitions:
	
	\begin{definition}[discrete]
		The frame \textbf{P} is \emph{discrete}, i.e., has $z \sqsubseteq w$ iff $z=w$,  iff  \newline $\textbf{P} \VDashA (\alpha \lor \neg \alpha)$.
	\end{definition}
	
	
	\begin{definition}[directed]
		The frame \textbf{P} is \emph{directed}, i.e., if $z \sqsubseteq w$ and $z \sqsubseteq v$ then there exists an $s$ with $w \sqsubseteq s$ and $v \sqsubseteq s$, iff  \newline $\textbf{P} \VDashA (\neg \alpha \lor \neg \neg \alpha)$.
	\end{definition}
	
	
	\begin{definition}[weakly linear]
		The frame \textbf{P} is \emph{weakly linear}, i.e., if $z \sqsubseteq w$ and $z \sqsubseteq v$ then either $w \sqsubseteq v$ or $v \sqsubseteq w$, iff  \newline $\textbf{P} \VDashA (\alpha \Rightarrow \beta) \lor (\beta \Rightarrow \alpha)$.
	\end{definition}
	
	
	\newpage
	
		\section{An Intermediate and Fuzzy Logic}
		\label{intermediate}
The following introduction, which draws heavily from \cite{fuzzy} \& \cite{firstorder} among others, is a recollection of known results.
		As the name suggests:
		
		\begin{definition}[intermediate propositional logic]
			 An \emph{intermediate}, a.k.a. \emph{super-intuitionistic} propositional logic $\mathcal{L}$ is such that: $\textbf{Th}_{\textbf{IPL}} \subset \textbf{Th}_\mathcal{L} \subset \textbf{Th}_{\textbf{CPL}}$, i.e., its theorems include all the \textbf{IPL}-theorems and are included in the \textbf{CPL}-theorems.
			 
		\end{definition}
		
		An Intermediate Propositional Logic is Gödel-Dummett Propositional Logic $\mathcal{G}$, obtained by extending the standard \textbf{IPL} with the \emph{pre-linearity} axiom we encountered earlier:
		
		\begin{definition}
			 $\mathcal{G} := \textbf{IPL} \; + \; ((A \Rightarrow B) \lor (B \Rightarrow A))$.
		\end{definition}
	
	We will also use the following notation to express this fact:
	\begin{equation*}
		{\textbf{IPL}} \subset \mathcal{G} \subset {\textbf{CPL}}
	\end{equation*}
	
		The algebraic semantics of $\mathcal{G}$ is given by the variety, i.e., an equationally definable class, $\mathbb{G}$ of \emph{Gödel algebras}:
		
		\begin{remark} For every formula $\alpha$ one has : 
			$\vdash_{\mathcal{G}} \alpha$ iff $G \vDash \alpha$ for all $G \in \mathbb{G}$. 
		\end{remark}
		
		A Gödel algebra \emph{G} is a Heyting algebra that satisfies \emph{pre-linearity}:
	
		\begin{definition}
			 $\emph{G} := (G, \lor, \land, \Rightarrow, 0, 1) \;$ such that $ \forall x,y,z \in G :$ 
			 	\begin{enumerate}
			 	\item $0 \leq x$.
			 	\item $x \leq 1$.
			 	\item $x \lor y \leq z \;$ iff $\;x \leq z$ and $y \leq z$.
			 	\item $z \leq x \land y \;$ iff $\;z \leq x$ and $z \leq y$.
			 	\item $z \leq (x \Rightarrow y) \;$ iff $\; (z \land x) \leq y$.
			 	\item $(x \Rightarrow y) \lor (y \Rightarrow x) = 1$.
			 	\end{enumerate}  
		\end{definition}
	
		 Note that the pre-linearity axiom restricts the space of possible algebraic semantics to Heyting algebras built on top of a total order.\newline
		 
		 In fact, one can show:
		 \begin{prop}
		 	Any infinite \emph{chain} $C$ with minimum and maximum elements where: \footnote{memento : $\neg x := x \Rightarrow 0 $.} 
		 		\begin{gather*}
		 		x \Rightarrow y :=\begin{cases}
		 			max (C) & \text{if }x \leq y\\
		 			y &	\text{if }x > y \\
		 		\end{cases}     \\
		 		x \lor y := max(x,y) \\ x \land y := min(x,y) \\
		 		\neg x:=\begin{cases}
		 			max (C) &	\text{if }x = min (C) \\
		 			min (C) & \text{if }x \neq min (C)\\
		 		\end{cases}     
		 	\end{gather*}
		 	provides \emph{sound \& complete semantics} for all formulae $\alpha$. 
		 	 \[ \mathcal{G} \vdash \alpha \; \text{ iff } \; C \models \alpha \]
		 \end{prop} 
		A special case of the above is:
		\begin{definition}[standard model]\label{standard}
				The standard \emph{model} is the real unit interval $([0,1],max,min,\Rightarrow,0,1)$:
			\[ \mathcal{G} \vdash \alpha \; \text{ iff } \; [0,1] \models \alpha. \]
		\end{definition}
		
		Furthermore, if we take a look at its Kripke semantics we have another characterization:
		\begin{prop}
			\begin{equation*}
				\mathcal{G} \vdash \alpha \text{ iff }\textbf{W} \VDashA \alpha\text{ for any \emph{weakly linear} frame \textbf{W}}.
			\end{equation*}
		\end{prop}
		\newpage
		 With regards to \emph{first-order logic}:
		 \newline\newline
		 We fix a first-order language $\mathcal{L}$ with the usual symbols for connectives and quantifiers $\forall,\exists$ and countable sets of predicate symbols $\textfrak{P}$, function symbols $\textfrak{F}$ for every arity $k>0$ and variables $\textfrak{V}$.\newline
		
		 As we just saw, the standard model takes the set $[0,1] \subset \mathbb{R}$. We define the following:
		 \begin{definition}[Gödel set]
		 	A \emph{Gödel set} is a closed sub-set $\mathcal{V} \subseteq [0,1]$ containing 0 and 1.
		 \end{definition}
		 
		 If $\textfrak{U}$ is the \emph{universe} or \emph{domain} of the \emph{interpretation} $\mathcal{I}$ we extend the language $\mathcal{L}$ to $\mathcal{L}^\textfrak{U}$ with constant symbols $\bar{u}$ \footnote{a.k.a. \emph{the name of u}.} for each element $u \in \textfrak{U}$. 
		 
		 \begin{definition}[first order interpretation]
		 	Having fixed a \emph{Gödel set} $\mathcal{V}$. \newline
		 	An \emph{interpretation} $\mathcal{I}$ into $\mathcal{V}$ is given by:
		 	\begin{enumerate}
		 		\item a non-empty set $\textfrak{U} = \textfrak{U}^\mathcal{I}$, i.e., the \emph{universe} of $\mathcal{I}$.
		 		\item a function $\textfrak{p}^\mathcal{I} : \textfrak{U}^k \rightarrow \mathcal{V}$ for each k-ary predicate symbol $\textfrak{p} \in \textfrak{P}$.
		 		\item a function $\textfrak{f}^\mathcal{I} : \textfrak{U}^k \rightarrow \textfrak{U}$ for each k-ary function symbol $\textfrak{f} \in \textfrak{F}$. Each new constant symbol $\bar{u}$ is interpreted as the element $u$.
		 	\end{enumerate}
		 \end{definition}
		 
		 The first-order semantics are thus:
		 \newpage
		 \begin{definition}[first order semantics]\label{fosemantics}
		 	Having fixed an interpretation $\mathcal{I}$: \newline
		 	Any term $t = f(\bar{u}_1,..,\bar{u}_k)$ \footnote{thanks to the extended language we can restrict ourselves to \emph{ground}-terms, i.e., terms without variables.} is realized inductively as $\mathcal{I}(t) = f^{\mathcal{I}} (u_1,..,u_k)$. \newline
		 	As for atomic formulae $A \equiv P(t_1,..,t_n)$ we define $\mathcal{I}(A) \equiv P^{\mathcal{I}} (t_1^{\mathcal{I}},..,t_n^\mathcal{I})$ and as for composite formulae:
		 	\begin{itemize}
		 		\item $\mathcal{I}(\bot) := 0$. 
		 		\item $\mathcal{I}(A \land B) := min(\mathcal{I}(A),\mathcal{I}(B))$.
		 		\item $\mathcal{I}(A \lor B) := max(\mathcal{I}(A),\mathcal{I}(B))$.
		 		\item $\mathcal{I}(A \Rightarrow B) :=\begin{cases}
		 			1 & \text{if }\mathcal{I}(A) \leq \mathcal{I}(B)\\
		 			\mathcal{I}(B) &	else 
		 		\end{cases}     $.
		 		\item $\mathcal{I}(\forall x.A(x)) := inf \{\mathcal{I}(A(\bar{u})) : u \in \textfrak{U}\}$.
		 		\footnote{$A(u)$ is the formula obtained from $A(x)$ by substituting each \emph{free} occurrence of the variable $x$ in $A$ with $u$. }
		 		\item $\mathcal{I}(\exists x.A(x)) := sup \{\mathcal{I}(A(\bar{u})) : u \in \textfrak{U}\}$.
		 	\end{itemize} 
		 	If $\mathcal{I}(A)=1$, we say that $\mathcal{I}$ \emph{satisfies A} and denote this by $\models_{\mathcal{I}} A$. \newline
		 A formula $A$ is said to be \emph{valid in the first order Gödel logic $\mathcal{G}_\mathcal{V}$} whenever $\models_{\mathcal{I}} A$ for all possible interpretations $\mathcal{I}$.
		 \end{definition} 
 	 In Classical Logic the truth values are binary $\{0,1\}$ or $\{true,false\}$. \newline On the other hand, Intuitionistic and Intermediate Logics provide a framework for arbitrarily many $n>2$ truth values.
		 
		 \begin{remark}
		 	Consider the proposition \emph{"Antarctica is large"}. Binary truth values are clearly unsatisfactory to express its validity. We would be better served with \emph{degrees of truth} from 0 (\emph{false}) to 1 (\emph{true}).
		 \end{remark}
		 
		 In fact the standard model for $\mathcal{G}$ in \ref{standard} can provide a continuum of truth values between $0$ false and $1$ true. \newline
		 %CHECK
		 Gödel-Dummett Logic is also a so-called \emph{fuzzy logic}. We briefly digress and give a better understanding of this notion following \cite{metamath}: 
		 \newline \emph{"Antarctica is large"} or \emph{"The patient is young"} are examples of \emph{fuzzy} propositions which are \emph{true to some degree}.\newline
		 The standard set used to \emph{encode} the truth degrees is the real unit interval $[0,1]$ with its standard order.\newline Similarly to the classical case, most fuzzy logics are \emph{truth functional}, i.e., the truth degrees of compound formulae is a function of the truth degrees of the compounds like we saw in (\ref{standard}).\newline
		 This is expressed as: \emph{Each connective $c$ of arity n has a truth function $f_c: [0,1]^n \rightarrow [0,1]$ determining for any formulae $\phi_1,\phi_2,..\phi_n$ the \emph{truth degree} of $c(\phi_1,\phi_2,..,\phi_n)$ from the \emph{truth degrees} of $\phi_1,\phi_2,..,\phi_n$}\newline
		 Moreover, this over-arching principle must be observed: 
		 \emph{Each many valued logic must be a generalization of classical two-value logic.} 
\newline This means for example that if we introduce a new connective symbol \& for \emph{strong} conjunction and denote by $*$ its binary truth function, the following must hold: \[1 * 1 = 1, \;\; 1 * 0 = 0 = 0 * 1, \;\; 0*0 = 0 \;\; \]
In analogy to the classical case, we would also like for $*$ to be non-decreasing in both arguments, 1 to be its unit element and 0 its zero element. We can thus define a more general $*$ as:
\begin{definition}[t-norm]
	A \emph{t-norm} is a binary operation $*$ on $[0,1]$ satisfying:
	\begin{enumerate}[label=(\roman*)]
		\item $*$ is commutative and associative, i.e., for all $x,y,z\in [0,1]$,
		\begin{gather*}
			x * y = y * x \\
			(x * y) * z = x * (y * z)
		\end{gather*}
		\item $*$ is non-decreasing in both arguments, i.e., \newline
		for all $x,x_1,x_2,y,y_1,y_2\in [0,1]$,
		\begin{gather*}
			x_1 \leq x_2 \text{ implies } x_1 * y \leq x_2 * y \\
			y_1 \leq y_2 \text{ implies } x * y_1 \leq x * y_2
		\end{gather*}
		\item for all $x \in [0,1]$ $1*x=x$ and $0*x=0$.		
	\end{enumerate}
	The t-norm $*$ is \emph{continuous} if it is so as a continuous mapping $[0,1]^2 \rightarrow [0,1]$. 
\end{definition}

	Recall that for the standard model semantics (\ref{standard}) of Gödel-Dummett Logic the conjunction $p \land q$ was realized as $min(\llbracket p \rrbracket, \llbracket q \rrbracket)$, indeed distinguished examples of continuous t-norms are:
	\begin{ex}\label{tnorms} ${ }$
		\begin{enumerate}[label=(\roman*)]
			\item \emph{\L{}ukasiewicz} t-norm: $x*y := max(0,x+y-1)$
			\item \emph{Gödel} t-norm: $x*y := min(x,y)$
			\item \emph{Goguen/Product} t-norm: $x*y := x \cdot y$
		\end{enumerate}	
	\end{ex}
	With regards to implication, In classical logic $\phi \Rightarrow \psi$ is true iff the truth value of $\phi$ is less than or equal to the truth value of $\psi$. This leads us to desire that a truth function $x \Rightarrow y$ should be \emph{non-increasing in x and non-decreasing in y}. \newline Also we would like a so-called \emph{fuzzy modus ponens}
	whereby from \emph{lower bounds} of truth degree $x$ of $\phi$ and $x\Rightarrow y$ truth degree of $\phi \Rightarrow \psi$ one should be able to deduce a truth degree $y$ of $\psi$. We take the t-norm $*$ to be the operation which \emph{computes the lower bound for y}, i.e., 
	\[(\textit{Fuzzy MP}) \;\; \text{If } a \leq x \text{ and } b \leq x \Rightarrow y, \text{ then }a * b \leq y \]
	In particular, if $a=x$ and $b=z$ we obtain:
		 \[\text{If } z \leq x \Rightarrow y, \text{ then }x * z \leq y \]
	Also, if we want to define $x \Rightarrow y$ to be \emph{as large as possible} we may require the converse, i.e.,
	\[\text{If } x * z \leq y, \text{ then }z \leq x \Rightarrow y \]
	which gives the condition that $*$ and $\Rightarrow$ must form an \emph{adjoint pair}:\footnote{it is no accident that this condition is very similar to the one for pseudo-complements in (\ref{heytingalg}).}
			 \[x * z \leq y \text{ iff }z \leq x \Rightarrow y \]
	making $x \Rightarrow y$ the \emph{maximal z} satisfying $x*z \leq y$, i.e.,
		\[ x \Rightarrow y = sup \{z \;|\; x*z \leq y\} \]	
In fact, each continuous t-norm $*$ uniquely determines a so-called \emph{residuum} $x \Rightarrow y$:
\begin{lem}
	Let $*$ be a continuous t-norm. There is a unique binary operation $\Rightarrow$ satisfying for all $x,y,z \in [0,1]$
	the condition $x * z \leq y, \text{ iff }z \leq x \Rightarrow y $ determined by $x \Rightarrow y := max\{z \;|\; x *z \leq y\}$.
\end{lem}
We now have: 
\begin{lem}
	For each continuous t-norm $*$ and its residuum $\Rightarrow$, the following hold for all $x,y \in [0,1]$: \begin{gather*}
		x \leq y \text{ iff } (x\Rightarrow y)=1 \\
		(1 \Rightarrow x) = x \\
		\text{If } x \leq y \text{ then } x=y*(y\Rightarrow x)
	\end{gather*}
\end{lem}
 \newpage
Returning to the t-norms in (\ref{tnorms}):
\begin{thm}
	The residua of our t-norms are:
	 \begin{enumerate}[label=(\roman*)]
	 	\item \emph{\L{}ukasiewicz} implication: $x \Rightarrow y = 
	 	\begin{cases}
	 		1 - x + y & \text{ if } x > y \\
	 		1 & \text{ if } x \leq y
	 	\end{cases}$
	 	\item \emph{Gödel} implication : $x \Rightarrow y = 
	 		\begin{cases}
	 			y & \text{ if } x > y \\
	 			1 & \text{ if } x \leq y
	 		\end{cases}$
	 	\item \emph{Goguen/Product} implication: $x \Rightarrow y = 
	 	\begin{cases}
	 		y/x & \text{ if } x > y \\
	 		1 & \text{ if } x \leq y
	 	\end{cases}$
	 \end{enumerate}	
\end{thm} 
 
 The residuum $\Rightarrow$ defines a so-called \emph{pre-complement} $(-)x := x \Rightarrow 0$ which generalizes classical negation:
 
 \begin{lem}
 	The pre-complements of our t-norms are:
 	 \begin{enumerate}[label=(\roman*)]
 		\item \emph{\L{}ukasiewicz} negation: $(-)x = 1 - x$
 		\item \emph{Gödel} negation: $(-)x =  \begin{cases}
 				0 &\text{ if }x > 0 \\ 1 &\text{ if }x = 0;
 			\end{cases} $
 		\item \emph{Goguen/Product} negation: $(-)x =  \begin{cases}
 			0 &\text{ if }x > 0 \\ 1 &\text{ if }x = 0;
 		\end{cases} $
 	\end{enumerate}	
 \end{lem}
 We now define the so-called \emph{Basic Many-Valued Logic} $\mathcal{BL}$: \newline
 Having fixed a continuous t-norm $*$ we introduce a propositional calculus $PC(*)$ with variables $p_1,p_2,..$, connectives \&,$\rightarrow$ and a constant $\bar{0}$.\newline
 Formulae are defined in the usual inductive way: \emph{each propositional variable is a formula, $\bar{0}$ is a formula, if $\phi$ and $\psi$ are formulae so are $\phi$ $\&$ $\psi$ and $\phi \rightarrow \psi$.} \newline
 Further operations are defined as:
 \begin{gather*}
 	\phi \land \psi := \phi \; \& \; (\phi \rightarrow \psi) \\
 	\phi \lor \psi := ((\phi \rightarrow \psi)\rightarrow \psi) \land ((\psi \rightarrow \phi)\rightarrow \phi) \\
 	\neg \phi := \phi \rightarrow \bar{0} \\
 	\phi \equiv \psi := (\phi \rightarrow \psi) \; \& \;(\psi \rightarrow \phi) \\
 	\bar{1} := \bar{0} \rightarrow \bar{0}
 \end{gather*}

 \begin{definition}[axioms of $\mathcal{BL}$]  The axioms of our \emph{Basic Logic} are:
 	\begin{enumerate}[label=(A\arabic*)]
 		\item $(\phi \rightarrow \psi)\rightarrow((\psi \rightarrow \chi)\rightarrow(\phi \rightarrow \chi))$
 		\item $(\phi \; \& \; \psi)\rightarrow \phi $
 		\item $(\phi \; \& \; \psi)\rightarrow (\psi \; \& \; \phi) $
 		\item $(\phi \; \& \; (\phi \rightarrow \psi))\rightarrow (\psi \; \& \; (\psi \rightarrow \phi)) $
 		\item $ (\phi \rightarrow (\psi \rightarrow \chi)) \rightarrow ((\phi \; \& \; \psi) \rightarrow \chi) $
 		\item $ ((\phi \; \& \; \psi) \rightarrow \chi) \rightarrow (\phi \rightarrow (\phi \rightarrow \chi)) $
 		\item $((\phi \rightarrow \psi)\rightarrow \chi)\rightarrow(((\psi \rightarrow \psi)\chi)\rightarrow \chi) $
 		\item $\bar{0} \rightarrow \phi $ 
 	\end{enumerate}
 \end{definition}
 The only deduction rule is \emph{Modus Ponens}.\newline
 
 An \emph{evaluation of propositional variables} is an assignment $e$ of each propositional variable $p$ to a truth value $e(p) \in [0,1]$. 
 \newline 
 $*$ and its residuum $\Rightarrow$ become the truth functions of the \emph{strong} conjunction \& and implication:
   \begin{gather*}
   		e(\bar{0}):= 0 \\
   		e(\phi \rightarrow \psi) := e(\phi) \Rightarrow e(\psi) \\
   		e(\phi \;\&\; \psi) := e(\phi)*e(\psi)
   \end{gather*}
 Note that, from the previous definitions, one can show that:
 \begin{lem} For any formulae $\phi, \psi$:
 	\begin{gather*}
 		e(\phi \land \psi) = min( e(\phi),e(\psi) ) \\
 		e(\phi \lor \psi) = max( e(\phi),e(\psi) )
 	\end{gather*}
 \end{lem}
 A formula $\phi$ is called a \emph{1-tautology} if $e(\phi)=1$ for each evaluation $e$.
 \newline
 Note that all axioms of $\mathcal{BL}$ can be shown to be \emph{1-tautologies} in each $PC(*)$ and since \emph{Modus Ponens} preserves 1-tautologies \footnote{i.e., If $\phi$ and $\phi \rightarrow \psi$ are 1-tautologies then so is $\psi$.}, the following result holds:
 \begin{lem}
 	All formulae provable in $\mathcal{BL}$ are \emph{1-tautologies} in each $PC(*)$.
 \end{lem}
 The \emph{algebraization} of $\mathcal{BL}$ can now be achieved: \newline
 We introduce the following structure:
 \begin{definition}[BL-algebra]
 	A \emph{BL-algebra} is a \emph{residuated} \emph{prelinear} lattice:\newline
 	A \emph{residuated lattice} is given by an algebra $(L,\cap,\cup,*,\Rightarrow,0,1)$ with binary operations $\cap,\cup,*,\Rightarrow$ and constants $0,1$ such that:
 	\begin{enumerate}[label=(\roman*)]
 		\item $(L,\cap,\cup,*,\Rightarrow,0,1)$ is a lattice endowed with a partial ordering $\leq$ with largest element 1 and least element 1.
 		\item $(L,*,1)$ is a commutative \emph{semigroup}\footnote{i.e., * is commutative,associative and $1*x=x$ for any element $x$.} with unit element 1.
 		\item $*$ and $\Rightarrow$ form an \emph{adjoint pair}\footnote{i.e., $z \leq (x\Rightarrow y)$ iff $x*z \leq y$ for all $x,y,z$.}. 
 	\end{enumerate}
 	A residuated lattice becomes a \emph{BL-algebra} iff the following identies hold:
 	\begin{enumerate}[label=(\roman*)]
 		\setcounter{enumi}{3}
 		\item $x \cap y = x * (x \Rightarrow y)$
 		\item $(x \Rightarrow y) \cup (y \Rightarrow x) =1$
 	\end{enumerate}
 \end{definition}
 
 \begin{definition}
 	Let \textbf{L} be a BL-algebra. We can define an \textbf{L}-evaluation by taking an evaluation $e$ of propositional variables and extending it to all formulae in the usual way:
 	\begin{gather*}
 		e(\bar{0}) := 0 \\
 		e(\phi \rightarrow \psi) := e(\phi) \Rightarrow e(\psi) \\
 		e(\phi \; \& \; \psi) := e(\phi) * e(\psi)
 	\end{gather*}
 	hence:
 	\begin{gather*}
 		e(\phi \land \psi) = e(\phi) \cap e(\psi) \\
 		e(\phi \lor \psi) = e(\phi) \cup e(\psi) \\
 		e(\neg \phi) = e(\phi) \Rightarrow 0
 	\end{gather*}
 	$\phi$ is called an \textbf{L}-\emph{tautology} if $e(\phi)=1 \text{ for each \textbf{L}-evaluation $e$}$.
 \end{definition}
 
 
 %check
 For the variety of \emph{BL-algebras} it can be shown that the following hold:
 \begin{prop}
 	\begin{enumerate}
 		\item For each t-norm $*$, the real unit interval endowed with the truth functions for the connectives $([0,1],min,max,*,\Rightarrow,0,1)$ is a linearly-ordered BL-algebra.
 		\item The Lindenbaum-Tarski algebra for $\mathcal{BL}$ is a (not linearly-ordered) BL-algebra.
 	\end{enumerate}
 \end{prop}

 \begin{thm}
 	 The algebraic semantics of $\mathcal{BL}$ is sound and complete for \emph{BL-algebras} and in particular for \emph{linearly ordered BL-algebras}:
 	\begin{gather*}
 		\mathcal{BL} \vdash \phi \;\text{ iff }\;  \phi \text{ is an \textbf{L}-tautology for each BL-algebra \textbf{L}} \\ \text{ iff }\;  \phi \text{ is an \textbf{L'}-tautology for each linearly ordered BL-algebra \textbf{L'}}
 	\end{gather*}
 \end{thm}
 
 If we extend $\mathcal{BL}$ with the double-negation axiom $\neg\neg \phi \rightarrow \phi$ we obtain \emph{\L{}ukasiewicz propositional logic}.\newline
 If we add a new binary operation symbol $\odot$ and a couple of new axioms:\newline $\neg\neg \chi \rightarrow ((\phi \odot \chi \rightarrow \psi \odot \chi)\rightarrow (\phi \rightarrow \psi))$, $\;\phi \land \neg \phi \rightarrow \bar{0}$ we obtain \emph{Product/Goguen logic}.\newline\newline
 If we extend $\mathcal{BL}$ with the \emph{idempotency} axiom:
 \[ \phi \rightarrow (\phi \;\&\; \phi)\] 
 we obtain our Gödel-Dummett logic $\mathcal{G}$, i.e.,
 \[ \mathcal{G} = \mathcal{BL} + (\phi \rightarrow (\phi \;\&\; \phi))\]
  The first consequence of this is:
 \begin{lem}
 	$\mathcal{G} \vdash (\phi \;\&\; \psi) \equiv (\phi \land \psi)$ 
 \end{lem}
 In other words, strong conjunction \& is equivalent to conjunction $\land$. Thus we choose to get rid of \& one of the two symbols and keep the other.\newline
 In fact, recalling that the Gödel t-norm is defined as $x*y := min(x,y)$:
 \begin{remark}
 	we can define a Gödel algebra \textbf{G} as a BL-algebra $(G,\cup,\cap,*,\Rightarrow,0,1)$ with idempotent multiplication, i.e., satisfying the identity $x*x=x$.\newline
 	Since now $* \;=\; \land$, we display \textbf{G} simply as $(G,\cup,\cap,\Rightarrow,0,1)$.
 \end{remark}
  A similar approach starting this time from $\mathcal{MTL}$ \emph{monoidal t-norm based logic}, i.e., the logic of \emph{left-continuous t-norms and their residua}, introduced in \cite{mtl}, and adding the \emph{idempotency} axiom produces the same result:
    \[ \mathcal{G} = \mathcal{MTL} + (\phi \rightarrow (\phi \;\&\; \phi))\]
%check
We now return to the main part of our introduction: \newline
In this work we are primarily interested in \emph{finite-valued} Gödel-Dummett Logic $\mathcal{G}_n$ for $n \in \mathbb{N}$.\newline
		 In 1932's \emph{Zum intuitionistischen aussagenkalkül} \cite{godel} \emph{K.Gödel} showed that:
		 
		 \begin{thm}
		 	IPL cannot be viewed as a system of \emph{many-valued} logic.
		 	 \end{thm}
		 	i.e., we cannot find a \emph{finite} set $M$ of \emph{truth values}, a subset $D \subset M$ of designated values \footnote{in the classical case we are used to $M= \{0,1\} $, $D= \{1\}$ and the usual interpretation of the connectives in the Boolean algebra $\mathbb{2}$. } and an interpretation of the connectives $\land,\lor,\Rightarrow,\neg$ such that:
		 	$\vdash_{IPL} \alpha$ iff, for all valuations $\mathcal{V}$ into $M$, $\mathcal{V}(\alpha) \in D$. 
		\newline
		 \newline
			 We begin by defining $\mathcal{G}_n$ by adding to the axioms of $\mathcal{G}$ a statement to "limit" the number of distinct elements or \emph{truth values} in a model to at most $n > 0$:
			 \begin{definition}
			 	\[ \mathcal{G}_n := \mathcal{G} \; + \; {F_{n+1}} \]
			 	Where\footnote{$\bigvee$ indicates the iterated $\lor$ connective.}  \[ F_n := \bigvee_{0 \leq i < j < n} (\textbf{p}_i \Leftrightarrow \textbf{p}_j) \]
			 \end{definition}
			 
			 Consider a possible assignment of the propositional variables which maps $\textbf{p}_i$ and $\textbf{p}_j$ to the same element $e$ (for some $\textbf{p}_i \Leftrightarrow \textbf{p}_j$ in $F_n$).\newline
			 Observe that the formula $(a \Leftrightarrow a) \lor b$ is provable in IPL.\footnote{recall also that $\vdash_{IPL} (A \lor B)$ iff either $\vdash_{IPL} A$ or $\vdash_{IPL} B$.}
		\newline
			 It follows that:
			 
			 \begin{lem}
			 	 $F_n$ is satisfied in any \emph{realization} with fewer than $n$ elements which is a model for IPL, i.e., in which every theorem of IPL is satisfied.  
			 \end{lem}
			 
			We thus construct a \emph{n-chain} Heyting algebra called $\emph{C}_n$. \newline Its $n>1$ elements are equidistant points on $[0,1]$ from $0$ to $1$, i.e.:
			\begin{equation*}
				\{1-\frac{1}{k}\}_{k=1}^{n-1} \cup \{1\}=\{0,\frac{1}{n-1},\frac{2}{n-1}..,1-\frac{1}{n-1},1\}.
			\end{equation*}
			The \emph{designated} element is $1$:
			 \begin{definition}[$\emph{C}_n$]
			 	\begin{gather*}
			 		\emph{C}_n := \{0 < \frac{1}{n-1} < \frac{2}{n-1} <..< 1-\frac{1}{n-1} < 1\}.  \\
			 		a \Rightarrow b :=\begin{cases}
			 			1 & \text{if }a \leq b\\
			 			b &	\text{if }a > b \\
			 		\end{cases}     \\
			 		a \lor b := max(a,b) \\ a \land b := min(a,b) \\
			 		\neg a:=\begin{cases}
			 			1 &	\text{if }a = 0 \\
			 			0 & \text{if }a \neq 0\\
			 		\end{cases}     
			 	\end{gather*}
		 		\end{definition}
	
	Recalling that Heyting algebras provide a sound and complete semantics for IPL:
\newline
		For $\emph{C}_n$ all theorems of IPL and all formulae $F_k$ with $k > n$ are satisfied since there is no way to choose $n+1$ values without repeating some of them, i.e.,
		\begin{lem}
			$\emph{C}_n \models \mathcal{G}_h$ for any $h \geq n$. 
		\end{lem}
		 On the other hand, $F_r$ with $r \leq n$ are \emph{not} satisfied because it is always possible to choose distinct values for the $r$ variables, i.e.,
		 \begin{lem}
		 	 $\emph{C}_n \nvDash \mathcal{G}_s$ with $s < n$.
		 \end{lem}
				It follows that no $F_n$ with $n \geq 1$ is provable in IPL and also:
\newline
			Since  $F_n \vdash F_{n+1}$ ($F_{n+1}$ is a "weaker" condition than $F_n$) and $\emph{C}_{n+1} \nvDash \mathcal{G}_n$ whereas $\emph{C}_{n+1} \models \mathcal{G}_{n+1}$, we have:\footnote{in general it is the case that if $\mathcal{L} \subseteq \mathcal{L}'$ then the Models of $\mathcal{L}'$ are  Models of $\mathcal{L}$, i.e., $Mod(\mathcal{L}') \subseteq Mod(\mathcal{L})$.}
			\begin{lem}
				For every $n \in \mathbb{N} : \mathcal{G}_{n+1} \subset \mathcal{G}_n$ and  
				$\mathcal{G}_{n} \nsubseteq \mathcal{G}_{n+1}$.
			\end{lem}
		Note also that we return to classical logic if $n=2$: 
		\begin{remark}
			Since any Heyting algebra $H$ with $H \models \mathcal{G}_2$ is such that $H \models F_3$, i.e., $H$ must have only two\footnote{we exclude the case of the \emph{degenerate} Heyting algebra in which $\bot\equiv\top$.} elements $\bot,\top$, i.e., $H \cong \mathbb{2}$ and	$\mathcal{G}_2 = \textbf{CPL}$.
		\end{remark}
		
		In fact, the collection $\{ \mathcal{G}_n \}_{n\in \mathbb{N}}$ forms a \emph{descending proper chain} of \emph{intermediate logics}:
		
		\begin{prop}
			$\textbf{IPL} \subsetneq \mathcal{G} \subsetneq... \subsetneq \mathcal{G}_{n+1} \subsetneq \mathcal{G}_n \subsetneq ....\subsetneq \mathcal{G}_3 \subsetneq \mathcal{G}_2 = \textbf{CPL}$.
		From this we obtain:
		\begin{equation*}
			\mathcal{G} = \bigcap_{k\geq2}  \mathcal{G}_k. 
		\end{equation*}
		\end{prop}

				We can also show that $\mathcal{G}_n$ are \emph{semantically characterized} by these n-chains, i.e.:
				\begin{prop}
					Let \emph{H} be a Heyting algebra that models $\mathcal{G}_n$: 
					\begin{gather*}
						H \models \mathcal{G}_n \; \text{ iff } \; H \cong C_{k}\text{ for some }k \leq n. \\
						\mathcal{G}_n \vdash \phi \;\text{ iff }\; C_n \models \phi.
					\end{gather*}
				\end{prop}
			\newpage	
			Now let us take a closer look at $\mathcal{G}_3$ or \emph{Gödel-Dummett (Propositional) three-valued logic} also known as the logic of \emph{Here and There} which will come up later in this work. 
				\newline
				The three \emph{truth values} \textfrak{T} are the elements of $C_3$: The designated value $1$ which we rename \textcolor{OliveGreen}{t}, a.k.a. \textcolor{OliveGreen}{true},  $0$ we rename \textcolor{red}{f}, a.k.a. \textcolor{red}{false} and the non-classical value $\frac{1}{2}$ we rename \textcolor{cyan}{$*$}, a.k.a. \textcolor{cyan}{not-false}.\newline
				
				The following remark is made by \emph{A.Heyting} in \cite{heyting}:
		%CHECK		
				\begin{remark}
					The truth values of $\mathcal{G}_3$ can be \emph{thought of} as follows:
					\begin{itemize}
						\item \textcolor{OliveGreen}{t} denotes a \emph{correct} assertion.
						\item \textcolor{cyan}{$*$} denotes an assertion that \emph{cannot be false but whose correctness has yet to be proven}. \footnote{by proof we mean an intuitionistic one.}
						\item \textcolor{red}{f} denotes a \emph{false} assertion.
					\end{itemize}
				\end{remark}
				
				\begin{definition}[truth values for $\mathcal{G}_3$]
 $\textfrak{T}:= \{0, \frac{1}{2}, 1\} \cong \{\textcolor{red}{f},\textcolor{cyan}{*},\textcolor{OliveGreen}{t}\}$.
					\begin{figure}[h]
						\centering
						\begin{tikzpicture}[thick,scale=0.6, every node/.style={scale=0.8}]
							\node (A) at (0,0) {\textcolor{red}{f}};
							\node (B) at (0,2) {\textcolor{cyan}{*}};
							\node (C) at (0,4) {\textcolor{OliveGreen}{t}};
							\draw[line width=.01in] (A) -- (B);
							\draw[line width=.01in] (B) -- (C);
						\end{tikzpicture}
						\caption{ 	$C_3 := \{0, \frac{1}{2}, 1\} \cong\{ \text{\textcolor{red}{f}} < \textcolor{cyan}{*} < \text{\textcolor{OliveGreen}{t}} \}$.}
					\end{figure}
				\end{definition}

				Given a valuation of the propositional variables into these truth values \emph{V}: $\textbf{Prop}\rightarrow \textfrak{T}$ with $V: \top \mapsto \text{\textcolor{OliveGreen}{t}} $ and $\bot \mapsto \text{\textcolor{red}{f}}$. \newline
				In accord with the semantics of $C_3$, this valuation is extended to propositional formulae denoted by $A,B,..$ as:
				\begin{align*}
					V(A \land B) := min\{ V(A), V(B) \} \\
					V(A \lor B) := max\{ V(A), V(B) \} \\
					V(A \Rightarrow B) :=
					\begin{cases}
						\textcolor{OliveGreen}{t} & \text{	if	}  V(A) \leq V(B) \\
						V(B) & \text{	if	} V(A)>V(B)\\
					\end{cases}  \\
					V(\neg A) :=\begin{cases}
						\text{\textcolor{OliveGreen}{t}} &	\text{if }V(A) = \text{\textcolor{red}{f}} \\
						\text{\textcolor{red}{f}} & \text{if }V(A) \neq \text{\textcolor{red}{f}}\\
					\end{cases}        
				\end{align*}

				The \emph{truth tables} for the connectives $\neg, \land, \lor, \Rightarrow$  are thus given by:
				
				\begin{figure}[h]
					\centering
					\begin{subfigure}[h]{0.2\textwidth}

								\begin{tabular}{||c || c ||}  
									\hline
									 & $\neg $ \\  
									\hline\hline
									\textcolor{OliveGreen}{t} & \textcolor{red}{f}  \\ 
									\hline
									\textcolor{cyan}{$*$} & \textcolor{red}{f} \\
									\hline
									\textcolor{red}{f} & \textcolor{OliveGreen}{t}  \\
									\hline
								\end{tabular}
					\caption{}
					\end{subfigure}
				\hfil
				\centering
				\begin{subfigure}[h]{0.2\textwidth}
					\begin{tabular}{||c || c | c | c ||}  
						\hline
						$ \land $ & \textcolor{OliveGreen}{t} & \textcolor{cyan}{$*$} & \textcolor{red}{f} \\  
						\hline\hline
						\textcolor{OliveGreen}{t} & \textcolor{OliveGreen}{t} & \textcolor{cyan}{$*$} & \textcolor{red}{f}  \\ 
						\hline
						\textcolor{cyan}{$*$} & \textcolor{cyan}{$*$} & \textcolor{cyan}{$*$} & \textcolor{red}{f} \\
						\hline
						\textcolor{red}{f} & \textcolor{red}{f} & \textcolor{red}{f} & \textcolor{red}{f}  \\
						\hline
					\end{tabular}
					\caption{}
				\end{subfigure}
			
				\hfil
				\centering
			\begin{subfigure}[h]{0.3\textwidth}
				\begin{tabular}{||c || c | c | c ||}  
					\hline
					$ \lor $ & \textcolor{OliveGreen}{t} & \textcolor{cyan}{$*$} & \textcolor{red}{f} \\  
					\hline\hline
					\textcolor{OliveGreen}{t} & \textcolor{OliveGreen}{t} & \textcolor{OliveGreen}{t} & \textcolor{OliveGreen}{t}  \\ 
					\hline
					\textcolor{cyan}{$*$} & \textcolor{OliveGreen}{t} & \textcolor{cyan}{$*$} & \textcolor{cyan}{$*$} \\
					\hline
					\textcolor{red}{f} & \textcolor{OliveGreen}{t} & \textcolor{cyan}{$*$} & \textcolor{red}{f}  \\
					\hline
				\end{tabular}
				\caption{}
			\end{subfigure}
			\hfil
			\centering
		\begin{subfigure}[h]{0.3\textwidth}
			\begin{tabular}{||c || c | c | c ||}  
				\hline
				$ \Rightarrow $ & \textcolor{OliveGreen}{t} & \textcolor{cyan}{$*$} & \textcolor{red}{f} \\  
				\hline\hline
				\textcolor{OliveGreen}{t} & \textcolor{OliveGreen}{t} & \textcolor{cyan}{$*$} & \textcolor{red}{f}  \\ 
				\hline
				\textcolor{cyan}{$*$} & \textcolor{OliveGreen}{t} & \textcolor{OliveGreen}{t} & \textcolor{red}{f} \\
				\hline
				\textcolor{red}{f} & \textcolor{OliveGreen}{t} & \textcolor{OliveGreen}{t} & \textcolor{OliveGreen}{t}  \\
				\hline
			\end{tabular}
			\caption{}
		\end{subfigure}
				
				\end{figure}
				
			
			\begin{remark}
				Notice that, since $\mathcal{G}_3 \subset \mathcal{G}_2$, the truth tables for \textcolor{OliveGreen}{t} and \textcolor{red}{f} are the same as in classical logic.	\newline
				Also, the classical $p \lor \neg p$ (\emph{Tertium non datur}) here is no longer a \emph{tautology} given that an assignment of \textcolor{cyan}{$*$} to $p$ yields a truth value of \textcolor{cyan}{$*$} for "$ \textcolor{cyan}{*} \lor (\neg \textcolor{cyan}{*}) $" . \newline
			\end{remark}	
			
			The extension of $\mathcal{G}_3$ to First Order Logic is obtained by applying \ref{fosemantics}:
			
			\begin{prop}[first order $\mathcal{G}_3$]\label{fog3}
				Taking as Gödel set $\textfrak{T} = \{0, \frac{1}{2}, 1\}$, the interpretation $\mathcal{I}$ with an associated universe \textfrak{U} is extended to quantified formulae as:
				\begin{itemize}
					\item $\mathcal{I}(\forall x.A(x)) := min \{\mathcal{I}(A(u)) : u \in \textfrak{U}\}$.
					\item $\mathcal{I}(\exists x.A(x)) := max \{\mathcal{I}(A(u)) : u \in \textfrak{U}\}$.
				\end{itemize}
			\end{prop}
			
				\newpage
			${}$ \newpage
		
		
		
		
		
		
		
	
	
 
 \chapter{Algebras and Forests}
 
In \ref{intermediate} we brought up the variety $\mathbb{G}$ of \emph{Gödel algebras} as the algebraic semantics of Gödel-Dummett (Propositional) Logic $\mathcal{G}$. \newline
We saw that \emph{finite-valued} Gödel-Dummett Logic $\mathcal{G}_n$ has a semantic characterization given by the n-chains $\emph{C}_n$. \newline
In fact, if we take the \emph{subdirect} product of k-chains with $k \leq n$, the resulting \emph{sub-variety} is called $\mathbb{G}_n$ or \emph{n-valued Gödel algebras} which model the logic $\mathcal{G}_n$. \newline
Using a Stone-like \emph{Duality} between \emph{finite} Gödel algebras and a soon to be introduced category of \emph{Finite Forests} we aim to explore the Topos semantics of a sub-category of \emph{Bushes} and \emph{re-discover} our target logic $\mathcal{G}_3$ at the propositional and first-order levels. 
\newline
We start by giving the following introduction to \emph{Finite Forests} and their duality with \emph{Finite Gödel algebras} building on the works of \cite{towards}, \cite{manyval}, \cite{recursive} and \cite{computing} among others.

\newpage
\section{Finite Forests}
\label{chapter02}

\subsection{First Steps}

We need the following preliminary definitions:

\begin{definition}[\emph{down-set} of $S\subseteq F$] 
	$\downarrow S := \{ x \in F \;|\; x \leq y \text{ for some } y\in S \}$.
\end{definition}

\begin{definition}[\emph{up-set} of $S\subseteq F$] 
	$\uparrow S := \{ x \in F \;|\; x \geq y \text{ for some } y\in S \}$.
\end{definition}

Now, what do we mean by \emph{finite forests}? 

\begin{definition}[\emph{finite forest F}]
	\emph{F} is a finite poset in which the \emph{down-set} of each element $x \in F$, i.e., $ \downarrow x :=  \downarrow \{x\} $ is a \emph{chain}\footnote{totally ordered sub-poset.} with the order inherited from \emph{F}.
\end{definition}

\begin{definition}[\emph{finite tree T}]
	$T$ is a finite forest with a minimum element called \emph{root} or \emph{bottom}.
\end{definition}

\begin{definition}[\emph{sub-forest G}] 
	A sub-forest $G$ of a finite forest $F$ is a \emph{downward-closed sub-poset G of F}, i.e., $G=\downarrow G$.\newline $F$ in this case is also called a \emph{super-forest} of the finite forest $G$.
\end{definition}

\begin{figure}[h]
	\centering
	\begin{tikzpicture}[thick,scale=0.5, every node/.style={scale=0.9}]
		\node (A)  at (0,0) {a0};
		\node (B)  at (3,0) {\textcolor{orange}{$b_0$}};
		\node (C)  at (3,3) {\textcolor{orange}{$b_1$}};
		\node (D)  at (6,0) {\textcolor{orange}{$c_0$}};
		\node (E)  at (6,3) {c1};
		\node (F)  at (12,0) {\textcolor{orange}{$d_0$}};
		\node (G)  at (12,3) {\textcolor{orange}{$d_1$}};
		\node (H)  at (9,3) {\textcolor{orange}{$d_2$}};
		\node (I)  at (15,3) {d3};
		\node (J)  at (9,6) {d4};
		\node (K)  at (15,6) {d5};
		\draw[line width=.03in, orange] (B) -- (C);
		\draw[line width=.03in] (D) -- (E);
		\draw[line width=.03in, orange] (F) -- (G);
		\draw[line width=.03in, orange] (F) -- (H);
		\draw[line width=.03in] (F) -- (I);
		\draw[line width=.03in] (H) -- (J);
		\draw[line width=.03in] (I) -- (K);
	\end{tikzpicture}
	\caption{A finite forest \emph{F} and its sub-forest \textcolor{orange}{$G$} in which the nodes are ordered bottom to top and displayed accordingly. \newline The figure is meant to represent the finite poset\newline $F = \{ a_0,\; b_0 < b_1,\; c_0 < c_1,\; d_0 < d_1 < d_4,\; d_0 < d_2,\; d_0 < d_3 < d_5 \}$ and its sub-forest \newline $\textcolor{orange}{G}=\{ \textcolor{orange}{b_0} < \textcolor{orange}{b_1},\; \textcolor{orange}{c_0},\; \textcolor{orange}{d_0} < \textcolor{orange}{d_1},\;\textcolor{orange}{d_0} < \textcolor{orange}{d_2}\}$.  }
\end{figure}

\newpage
\subsection{The Categories $ \mathbb{FF}_*$}

From now on a slight abuse of notation will be used: "$C\in \mathbb{C}$" instead of "object $C$ in $\mathbb{C}$".
\newline
We introduce appropriate \emph{morphisms} or \emph{arrows} between finite forests \emph{F} and \emph{G} in the form of \emph{order-preserving open maps}:
\begin{definition}[\emph{open map}]
	A map $ f: F \rightarrow G $ is \emph{open} if it carries any down-set of $ S \subseteq F $ to a down-set of $ T \subseteq G $ :$f(\downarrow S) = (\downarrow T)$.
\end{definition}

\begin{remark}
	$f(\downarrow S) = (\downarrow T)$ is equivalent to $\forall x \in F : f(\downarrow x) = \downarrow f(x)$. 
\end{remark}

\begin{figure}[h] 
	\centering
	\begin{tikzpicture}[thick,scale=0.5, every node/.style={scale=0.9}]
		\node (A) at (0,0) {\textcolor{red}{$a_0$}};
		\node (B) at (3,0) {\textcolor{OliveGreen}{$b_0$}};
		\node (C) at (3,3) {\textcolor{SkyBlue}{$b_1$}};
		\node (D) at (6,0) {\textcolor{red}{$c_0$}};
		\node (E) at (6,3) {\textcolor{red}{$c_1$}};
		\node (F) at (12,0) {\textcolor{OliveGreen}{$d_0$}};
		\node (G) at (12,3) {\textcolor{SkyBlue}{$d_2$}};
		\node (H) at (9,3) {\textcolor{OliveGreen}{$d_1$}};
		\node (I) at (15,3) {\textcolor{SkyBlue}{$d_3$}};
		\node (J) at (9,6) {\textcolor{SkyBlue}{$d_4$}};
		\node (K) at (15,6) {\textcolor{SkyBlue}{$d_5$}};
		
		\node (u) at (16,1) {};
		\node (v) at (22,1) {};
		
		\node (L) at (23,0) {\textcolor{red}{$e_0$}};
		\node (M) at (26,0) {\textcolor{OliveGreen}{$f_0$}};
		\node (N) at (26,3) {\textcolor{SkyBlue}{$f_1$}};
		
	\draw[->, dotted, line width=.01in] (u) -- node[anchor=north] {$h$} (v);
	
		\draw[line width=.03in, SkyBlue] (B) -- (C);
		\draw[line width=.03in, red] (D) -- (E);
		\draw[line width=.03in, SkyBlue] (F) -- (G);
		\draw[line width=.03in,OliveGreen] (F) -- (H);
		\draw[line width=.03in, SkyBlue] (F) -- (I);
		\draw[line width=.03in, SkyBlue] (H) -- (J);
		\draw[line width=.03in, SkyBlue] (I) -- (K);
		
		\draw[line width=.03in, SkyBlue] (M) -- (N);
	\end{tikzpicture}
	\caption{An arrow $h$ between forests \emph{F} (left) and \emph{G} (right). \newline
		The nodes of \emph{F} are mapped to the nodes of corresponding color in \emph{G}. \newline For example $a_0, b_0 \mapsto f_0$ and $b_1 \mapsto f_1$.}
\label{fig:arrow}
\end{figure}
 

	We define the category in question:
	\begin{definition}[category $\mathbb{FF}$]
		$\mathbb{FF}$ is the category formed by taking finite forests as objects and order-preserving open maps as arrows.
	\end{definition}
	Of particular interest to us are finite forests of \emph{height} at most $n \geq 0$.
	
	\begin{definition}[\emph{height}]
	The \emph{height} of a finite forest \emph{F} is the maximum cardinality of a downset $\downarrow x$ for $ x \in F$.
	\end{definition} 
	
	In the example \emph{F} in figure~\ref{fig:arrow} has height 3 whilst \emph{G} has height 2. 
	
	\begin{definition}[category $\mathbb{FF}_n$]
		The (full) subcategory $\mathbb{FF}_n$ of $\mathbb{FF}$ has finite forests of height at most $n$ as objects and open maps between them as arrows.\newline
		The objects of $\mathbb{FF}_2$ are called \emph{bushes}.
	\end{definition}
	
	Also the \emph{finite trees} form the subcategory $\mathbb{T}$. \newline\newline
	Let $\mathbb{FF}_*$ stand for either $\mathbb{FF}$ or $\mathbb{FF}_n$ for $n>0$.
	\newline
	The following properties are valid  in $\mathbb{FF}_*$: \footnote{here we omit to verify that these constructions satisfy the categorical properties of Terminal and Initial objects, Products, Co-products and so on. For reference see \cite{towards} \& \cite{recursive}.}
	
	\begin{thm}[Terminal]
		The \emph{terminal} object is given by the singleton forest denoted as $\textbf{1} := \{ \bullet \}$.
	\end{thm}
	 
	 \begin{thm}[Initial]
	 	The \emph{initial} object is given by the empty forest denoted as $\textbf{0} := \{\}$.
	 \end{thm} 

The \emph{Co-product} or \emph{Sum} between two forests $F$ and $G$ is readily given. 

\begin{figure}[h]
	\centering
	\begin{tikzpicture}[thick,scale=0.8, every node/.style={scale=0.9}]
		\node (A)  at (0,0) {$\bigcdot$};
		\node (B)  at (3,0) {$\bot$};
		\node (C)  at (3,3) {$\bigcdot$};
		
		\draw[line width=.01in] (B) -- (C);
		
	\end{tikzpicture}
	\caption{$\Omega$.}
	\label{fig:Omega}
\end{figure}


\begin{thm}[Co-product]
	The \emph{Co-product} $F+G$ between $F$ and $G$ is obtained by taking the \emph{disjoint union} of the two posets and the inclusion maps $\iota_F : F \hookrightarrow F+G$ and $\iota_G : G \hookrightarrow F+G $. 
\end{thm}

For example $\Omega := \textbf{1} + \textbf{1}_{\bot}$ is displayed in figure~\ref{fig:Omega}.


\begin{cor}
	Each forest $F$ in $\mathbb{FF}_*$ uniquely determines a finite family of trees $\{F_i\}_{i=1}^N$ such that $F = \sum_{i=1}^{N} F_i$. \footnote{by $\sum$ we mean the coproduct of the summands.}
\end{cor}

	
\newpage

Now for each finite forest $F$ we write $F_{\bot}$ as the tree obtained by appending a new bottom/root element as the new minimum.\newline 
In fact:
\begin{remark}
	every tree $T$ is of the form $T = T'_{\bot}$ for some finite forest $T'$.
\end{remark}

(Abuse of notation: oftentimes "$=$" in these cases is used instead of "$\cong$"). \newline
In $\mathbb{FF}$ the \emph{product} $F \times G$ between $F$ and $G$ is defined in a recursive manner \newline
We require the following properties to hold for our product:
\begin{lem} ${}$
	\begin{itemize}
		\item ($\times$ by \textbf{0} is \textbf{0}) $\forall F$: $F \times \textbf{0} = \textbf{0} = \textbf{0} \times F$  
		\item (\textbf{1} as neutral element of $\times$) $\forall F$: $F \times \textbf{1} = F = \textbf{1} \times F$
		\item (distributive law for $\times$) $\forall F,G,H $: $ F \times (G+H) = (F \times G) + (F \times H) $
	\end{itemize}
\end{lem}
The recursive formula for $F_{\bot} \times G_{\bot} $ is given by:
\begin{definition}[Product]
	\begin{equation}
		F_{\bot} \times G_{\bot} := ( (F \times G_{\bot}) + ( F \times G ) + (F_{\bot} \times G) )_{\bot}
	\end{equation}
\end{definition}
 The projection maps are also given recursively following the construction of the product object :
 	\begin{definition}[Projections]
 		The maps $ \pi_{F_\bot} : 	F_{\bot} \times G_{\bot} \twoheadrightarrow F_{\bot}$ and $\pi_{G_\bot} : F_{\bot} \times G_{\bot} \twoheadrightarrow G_{\bot} $ are defined as follows:
 		\newline
 		Let $t_0$ be the root of $F_{\bot} \times G_{\bot}$ and $r_0,s_0$ the roots respectively of $F_{\bot}$ and $G_{\bot}$.
 		\begin{gather*}
 			\pi_{F_\bot}: t_0 \mapsto r_0 \\
 			\pi_{G_\bot} : t_0 \mapsto s_0 
 		\end{gather*}
 		Let $F_0$ and $G_0$ stand for $F_\bot$ and $G_\bot$. \newline 
 		Recalling the representation of each finite forest as a finite sum of trees, $F_\bot = (\sum_{i=1}^{N} F_i)_\bot$ and $G_\bot = (\sum_{j=1}^{M} G_j)_\bot$.\newline
 		Each element $x \in  (F \times G_0) + ( F \times G ) + (F_0 \times G) $ belongs to a unique tree $F_i \times G_j$ with $(i+j) >0$.\newline
 		Also let $\iota_{F_i} : F_i \hookrightarrow F_\bot$ the set-inclusion of the support of $F_i$ into $F_\bot$  and $\iota_{G_j}$ the analogous map for $G_j$. Then:
 		\begin{gather*}
 			\pi_{F_\bot}: x \mapsto \iota_{F_i} ( \pi_{F_i}(x) ) \\
 			\pi_{G_\bot}: x \mapsto \iota_{G_j} ( \pi_{G_j}(x) ) 
 		\end{gather*}
 	\end{definition}
 	
 	\begin{lem}
 			The object $F_{\bot} \times G_{\bot}$ together with $\pi_{F_\bot}$ and $\pi_{G_\bot}$ is proven in \cite{recursive} to be the desired product object in $\mathbb{FF}$ together with the associated projections.
 	\end{lem}
 
 \begin{figure}[h]
 	\centering
 	\begin{tikzpicture}[thick,scale=0.5, every node/.style={scale=0.9}]
 		\node (A)  at (0,0) {$a_1b_0$};
 		\node (B)  at (0,3) {$a_1b_1$};
 		\node (C)  at (3,0) {$a_1b_1$};
 		\node (D)  at (6,0) {$a_0b_1$};
 		\node (E)  at (6,3) {$a_1b_1$};
 		\node (F)  at (3,-3) {$a_0b_0$};
 		
 		\draw[line width=.03in] (A) -- (B);
 		\draw[line width=.03in] (D) -- (E);
 		\draw [dashed] [ line width=.03in] (F) -- (A);
 		\draw [dashed] [ line width=.03in] (F) -- (C);
 		\draw [dashed] [ line width=.03in] (F) -- (D);
 		
 	\end{tikzpicture}
 	\caption{The product of $1_\bot = \{a_0 < a_1\} = A$ with $1_\bot = \{b_0 < b_1\} = B$ computed following the recursive formula:\newline
 		$ 1_\bot \times 1_\bot = ( (1 \times 1_\bot) + (1 \times 1) + ( 1_\bot \times 1 )  )_\bot = ( 1_\bot + 1 + 1_\bot )_\bot $.\newline The labeling of the nodes specifies the projections: $a_i b_j$ indicates that this node is taken to $a_i$ by the 'left' projection $\pi_{A}$ and to $b_j$ by the 'right' projection $\pi_B$.}
 \end{figure}
 
 Recalling the previous remarks, we have $ F = \sum_{i=1}^{N} F_i $ and $ G = \sum_{j=1}^{M} G_j $ with each $ F_i,  G_j $ being a tree equal to $(F_i')_\bot,  (G_j')_\bot$ for some finite forests $F_i',  G_j'$.
 
 \begin{thm}[Product in $\mathbb{FF}$] 
 	$\forall F,G \in \mathbb{FF}$ such that $ F = \sum_{i=1}^{N} F_i $ and $ G = \sum_{j=1}^{M} G_j $ the product is given recursively as:   \[ F \times_\mathbb{FF} G = \sum_{i,j=1}^{N,M} (F_i')_\bot \times_\mathbb{FF} (G_j')_\bot \] together with the projection maps $\pi_F$ and $\pi_G$ defined as before for each summand.
 \end{thm}
 
 
 The product in $\mathbb{FF}_n$ is obtained by \emph{trimming} the product in $\mathbb{FF}$.
 
 \begin{thm}[Product in $\mathbb{FF}_n$]\label{thm:prodffn}
 	$\forall F, G \in  \mathbb{FF}_n$, \newline $ F \times_{\mathbb{FF}_n} G $ is the sub-forest of all nodes of height $ \leq n $ of $ F \times_\mathbb{FF} G $ together with the projection maps $\pi_F$ and $\pi_G$ \emph{restricted} to $F \times_{\mathbb{FF}_n} G $.
 \end{thm}
 
 (From now on $\times$ will be used usually instead of $\times_\mathbb{FF}$). \newline
 For the purpose of this work we only display selected instances of  products and projections in the following figures.



\begin{figure}[h]
	\centering
	\begin{tikzpicture}[thick,scale=0.5, every node/.style={scale=0.9}]
		
		\node (a) at (0,-2) { };
		\node (b) at (-4,-2) { };
		\draw[->>, dotted, line width=.01in] (a) -- node[anchor=north] {$\pi_{A}$} (b);
		
\node (c) at (6,-2) { };
\node (d) at (10,-2) { };
\draw[->>, dotted, line width=.01in] (c) -- node[anchor=north] {$\pi_{B}$} (d);

\node (a') at (-1,-7) { };
\node (b') at (-4,-5) { };
\draw[->>, dotted, line width=.01in] (a') -- node[anchor=north] {$\pi_{A}'$} (b');

\node (c') at (7,-7) { };
\node (d') at (10,-5) { };
\draw[->>, dotted, line width=.01in] (c') -- node[anchor=north] {$\pi_{B}'$} (d');

	
		\node (A0) at (-6,-5) {\textcolor{purple}{$a_0$}};
		\node (A1) at (-6,-2) {\textcolor{pink}{$a_1$}};
		
		\node (A) at (0,0) {\textcolor{pink}{$\bullet$} \textcolor{blue}{$ \bullet $} };
		\node (B) at (0,3) {\textcolor{pink}{$\bullet$} \textcolor{SkyBlue}{$ \bullet $} };
		\node (C) at (3,0) {\textcolor{pink}{$\bullet$} \textcolor{SkyBlue}{$ \bullet $} };
		\node (D) at (6,0) {\textcolor{purple}{$\bullet$} \textcolor{SkyBlue}{$ \bullet $} };
		\node (E) at (6,3) {\textcolor{pink}{$\bullet$} \textcolor{SkyBlue}{$ \bullet $} };
		\node (F) at (3,-3) {\textcolor{purple}{$\bullet$} \textcolor{blue}{$ \bullet $} };
		
		\node (B0) at (12,-5)  {\textcolor{blue}{$b_0$}};
		\node (B1) at (12,-2) {\textcolor{SkyBlue}{$b_1$}};
		
		\draw[line width=2in, thick] (A0) -- (A1);
		
		\draw[line width=2in, thick] (0,0) -- (0,3);
		\draw[line width=2in, thick] (6,0) -- (6,3);
		\draw [line width=2in, thick] (3,-3) -- (0,0);
		\draw [ line width=2in, thick] (3,-3) -- (3,0);
		\draw [ line width=2in, thick] (3,-3) -- (6,0);
		
		\draw[line width=2in, thick] (B0) -- (B1);
		
		\node (A) at (0,-6) {\textcolor{pink}{$\bullet$} \textcolor{blue}{$ \bullet $} };
		\node (C) at (3,-6) {\textcolor{pink}{$\bullet$} \textcolor{SkyBlue}{$ \bullet $} };
		\node (D) at (6,-6) {\textcolor{purple}{$\bullet$} \textcolor{SkyBlue}{$ \bullet $} };
		\node (F) at (3,-9) {\textcolor{purple}{$\bullet$} \textcolor{blue}{$ \bullet $} };
		
		\draw [line width=2in, thick] (3,-9) -- (0,-6);
		\draw [ line width=2in, thick] (3,-9) -- (3,-6);
		\draw [ line width=2in, thick] (3,-9) -- (6,-6);
	\end{tikzpicture}
	\caption{At each node of the product forest (top center) $ 1_\bot \times_{\mathbb{FF}} 1_\bot$ the color of the left and right dot specifies respectively the projection $\pi_A$ from $ 1_\bot \times_{\mathbb{FF}} 1_\bot$ to  $1_\bot = \{a_0,a_1\}$ (left) and the projection $\pi_B$ from $ 1_\bot \times 1_\bot$ to $1_\bot = \{b_0,b_1\}$ (right).\newline
		Below we have 
		the product forest for $\mathbb{FF}_2 $ (bottom center) $ 1_\bot \times_{\mathbb{FF}_2} 1_\bot$ and the associated projections $\pi_{A}'$ and $\pi_{B}'$.}
		\label{fig:product}
\end{figure}


The product $ F \times G $ of two finite forests $F$ and $G$ is \emph{not} the usual cartesian product of the underlying posets. \newline
In fact the cardinality of the underlying set of the product $ | F \times G | = 6$ in the case of $F$ and $G$ of the form $1_\bot$ is not the product of the cardinalities of $ | F | = 2 $ and $ | G | = 2 $.  \newline
(If the context is specified $\times$ will replace $\times_{\mathbb{FF}_n}$). \newline
In the category of \emph{bushes} $\mathbb{FF}_2$ however the underlying set $ | A \times B | $ of the product of two bushes $A$ and $B$ \emph{is} the cartesian product $| A | \times | B | $ of the underlying sets of $A$ and $B$.\newline In figure figure~\ref{fig:product} we have $4 = | 1_\bot \times 1_\bot | = | 1_\bot | \times | 1_\bot | = 2 \times 2 $.

\newpage
\section{The Duality between Algebras and Forests}

We now consider the category $\mathbb{G}$ of Gödel algebras and their homomorphisms and the full subcategory $\mathbb{G}_{fin}$ of \emph{finite}\footnote{finite cardinality.} Gödel algebras. In turn, for each $ n>0 $ the full subcategory $(\mathbb{G_n})_{fin}$ of $\mathbb{G_n}$ contains the \emph{finite n-valued} algebras. \newline
The remarkable result, essentially due to \emph{A.Horn}\footnote{see also chapter IX \cite{fuzzy}.}, that we shall use henceforth is the \emph{dual equivalence} between the category $\mathbb{FF}$ of Finite Forests and the category $\mathbb{G}_{fin}$ of finite Gödel algebras realized by the contra-variant functors \emph{Spec} and \emph{Sub}.

\begin{thm}[Finite spectral duality for Gödel algebras]
	\[ \mathbb{G}_{fin} \simeq \mathbb{FF}^{\textit{op}} \]
\end{thm}

To see how these functors operate we need to introduce the following notions:
\begin{definition}[proper filter]
	A proper filter $\textfrak{f}$ for a Gödel algebra $\mathbf{A}$: $\emptyset \neq \textfrak{f} \subsetneq \mathbf{A}$ is an up-set of $\mathbf{A}$ closed under \emph{meets} i.e.,  $\forall x,y \in \textfrak{f}: x \land y \in \textfrak{f}$.
\end{definition}
\begin{definition}[prime filter]
	A prime filter $\textfrak{p}$ is a proper filter with $0 \notin \textfrak{p}$ and whenever $ (y \lor z) \in \textfrak{p} $ either $y \in \textfrak{p}$ or $z \in \textfrak{p}$. 
\end{definition}
\begin{definition}[principal filter]
	\textfrak{f} is \emph{principal} if \textfrak{f} = $\uparrow x_\textfrak{f}$ \footnote{$\uparrow x := \{y \in \mathbf{A} \;|\; y \geq x\}$.} for some $x_\textfrak{f} \in \textbf{A}$.
\end{definition}
\begin{definition}[prime spectrum]
	 The set of prime filters of $\mathbf{A}$ is also called the \emph{prime spectrum} of $\mathbf{A}$, a.k.a. $Spec(A)$ and is partially ordered by reverse-inclusion $\supseteq$.
\end{definition}
\begin{definition}[join-irreducible]
	$x \in \mathbf{A} $ is join-irreducible if $x \neq 0$ and whenever $ x= y \lor z $ then either $x=z$ or $x=y$.
\end{definition}

The functor \emph{Spec} assigns to an algebra \textbf{A} the prime spectrum of \textbf{A} with its reverse ordering.\newline The functor \emph{Sub} takes the set of sub-forests of \textbf{F} and forms a finite Gödel algebra with intersection, a new "implication" operator and the empty forest.\newline Both the functors act on morphisms by taking pre-images.

\begin{definition}[Spec]
	$\emph{Spec} : \mathbb{G}_{fin} \rightarrow \mathbb{FF}$ \newline
	\newline
			$	\textbf{A} \longmapsto (\emph{Spec}(\textbf{A})  = \{ \textfrak{p}\subseteq \textbf{A} \;|\; \textfrak{p} \text{ prime filter} \}, \supseteq )$ \newline
				$ \textbf{A} \xrightarrow{f} \textbf{B} \longmapsto Spec(\textbf{B}) \xrightarrow{f^{-1}\{\}} Spec(\textbf{A}) $
\end{definition}

\begin{definition}[Sub]
	$ \emph{Sub} :  \mathbb{FF} \rightarrow \mathbb{G}_{fin}$\newline \newline
	 $\textbf{F} \longmapsto \emph{Sub}(\textbf{F}) = (\{\downarrow G \subseteq F\},\cup,\cap,\Rightarrow,\emptyset ) $ with $ H \Rightarrow G := F \setminus \uparrow(H \setminus G) $. \newline
	 $ \textbf{H} \xrightarrow{h} \textbf{G} \longmapsto Sub(G) \xrightarrow{h^{-1}\{\}} Sub(H) $
\end{definition}

Also, if we restrict the functors to the category of finite n-valued algebras and forests of height n-1:

\begin{thm}
	\[ (\mathbb{G_n})_{fin} \simeq \mathbb{FF}^{\textit{op}}_{n-1} \] 
	For instance: $ (\mathbb{G_3})_{fin} \simeq \mathbb{FF}^{\textit{op}}_{2} $, i.e., the category of bushes.
\end{thm}

We use the following result observed in \cite{recursive}:
\begin{lem}
	In each finite Gödel algebra \textbf{A}, all filters \textfrak{f} are principal. \newline Every prime filter $\mathfrak{p}$ is equal to $\uparrow x_\mathfrak{p}$ for some join-irreducible element $x_\mathfrak{p}$.
\end{lem}

To understand how this duality works we start by taking the dual of the \emph{free} Gödel algebra $\mathcal{F}_1$ on one generator $x$.

\begin{figure}[h]
	\centering
	\begin{tikzpicture}[thick,scale=1.2, every node/.style={scale=0.8}]
		\node (A) at (0,0) {0};
		\node (B) at (1,1) {$\neg x$};
		\node (C) at (-1,1) {$x$};
		\node (D) at (0,2) {$x \lor \neg x$};
		\node (E) at (-2,2) {$\neg\neg x$};
		\node (F) at (-1,3) {1};
		
		\draw[line width=.01in] (A) -- (B);
		\draw[line width=.01in] (A) -- (C);
		\draw[line width=.01in] (B) -- (D);
		\draw[line width=.01in] (C) -- (D);
		\draw[line width=.01in] (C) -- (E);
		\draw[line width=.01in] (D) -- (F);
		\draw[line width=.01in] (E) -- (F);
		
		\node (f) at (3.5,0) {$\uparrow \neg x$};
		\node (t) at (5,0) {$\uparrow x$};
		\node (n) at (5,2) {$\uparrow \neg\neg x$};
		
		\draw[line width=.01in] (t) -- (n);
		
		\node (A') at (10,0) {$\emptyset$};
		\node (B') at (11,1) {$\{ \uparrow\neg x \}$};
		\node (C') at (9,1) {$\{ \uparrow x \}$};
		\node (D') at (10,2) {$\{ \uparrow x, \uparrow\neg x \}$};
		\node (E') at (8,2) {$\{ \uparrow\neg\neg x \}$};
		\node (F') at (9,3) {$ Spec(\mathcal{F}_1) $};

		\draw[line width=.01in] (A') -- (B');
		\draw[line width=.01in] (A') -- (C');
		\draw[line width=.01in] (B') -- (D');
		\draw[line width=.01in] (C') -- (D');
		\draw[line width=.01in] (C') -- (E');
		\draw[line width=.01in] (D') -- (F');
		\draw[line width=.01in] (E') -- (F');
	\end{tikzpicture}
	\caption{ $\mathcal{F}_1$ (left),  Spec($\mathcal{F}_1$)  (center) and Sub(Spec($\mathcal{F}_1$)) (right) }
\end{figure}


Note that $\Omega := \textbf{1} + \textbf{1}_\bot = Spec(\mathcal{F}_1)$. This finite forest will be ubiquitous in the upcoming chapters.
\newline
What this duality tells us is that:
\begin{remark}
	Each finite Gödel algebra \textbf{A} is isomorphic to the Gödel algebra of all sub-forests of  $(Spec(\textbf{A}),\supseteq)$ taken as a finite forest.
\end{remark}

The knowledge of a dual category to the category of Gödel algebras in which products and co-products are readily computed provides us with a valuable tool for the study of the structure of these algebras. For example:
\begin{remark}
	The free algebra on $k>0$ generators  $\mathcal{F}_k = k \cdot [\mathcal{F}_1]$ is the \emph{k-th co-power} \footnote{i.e., the co-product iterated k times over the same object.} of the free algebra on one generator, as such its dual is $ [Spec(\mathcal{F}_1)]^k $ the \emph{k-th power} \footnote{i.e., the product iterated k times over the same object.} of $Spec(\mathcal{F}_1)$.
\end{remark}
 
 	In particular we have: 
 
\begin{remark}
	As a consequence of $ (\mathbb{G_3})_{fin} \simeq \mathbb{FF}^{\textit{op}}_{2}$:
 \[\mathcal{F}_1 \cong Sub(Spec(\mathcal{F}_1)) = Sub(\textbf{1}+\textbf{1}_\bot) = Sub(\Omega).\] 
\end{remark}

%check
In fact, this duality can be seen as a \emph{generalization} of the finite case of \emph{Stone's Representation Theorem} (\cite{stone}) for Boolean algebras whereby \emph{finite Boolean algebras are dually equivalent to finite sets}.
\newline
The latter says that classical propositional logic \emph{CPL} can be seen as the logic of \emph{finite sets}:

\begin{remark}
	A formula $\phi$ of \emph{CPL} seen as a \emph{characteristic function} from a finite set/domain/universe $X$ determines a \emph{subset} $\{x \in X \;|\; \phi(x) = 1\}$ of elements for which \emph{$\phi$ is true}.\newline
	Similarly, a formula $\phi$ of $\mathcal{G}$ determines a \emph{sub-forest} of a finite forest for which \emph{$\phi$ is true}.    
\end{remark}
To be more precise:
\begin{remark}
	For every formula $\phi$ which contains propositional variables $p_1,..,p_n$: \newline
	Finite Forests provide (sound and complete) semantics of Gödel-Dummett Propositional Logic $\mathcal{G}$,  i.e., \footnote{here $\models_{H.A.}$ refers to Heyting algebra validity.}
	\begin{equation*}
		\mathcal{G} \vdash \phi \;\text{ iff }\; Sub(Spec(\mathcal{F}_n)) \models_{H.A.} \phi \;\text{ iff }\; \llbracket \phi \rrbracket_{Sub(Spec(\mathcal{F}_n))}=Spec(\mathcal{F}_n).
	\end{equation*}
\end{remark}


\newpage
Using the same \emph{Stone-style} duality as before we obtain:

\begin{figure}[h]
	\centering
	\begin{tikzpicture}[thick,scale=1.2, every node/.style={scale=0.8}]
		\node (A) at (0,0) {0};
		\node (B) at (1,1) {$\neg x$};
		\node (C) at (-1,1) {$x$};
		\node (D) at (0,2) {1};
		
		\draw[line width=.01in] (A) -- (B);
		\draw[line width=.01in] (A) -- (C);
		\draw[line width=.01in] (B) -- (D);
		\draw[line width=.01in] (C) -- (D);
		
		
		\node (f) at (3.5,0) {$\uparrow \neg x$};
		\node (t) at (5,0) {$\uparrow x$};
		
		
		\node (A') at (8,0) {$\emptyset$};
		\node (B') at (9,1) {$\{ \uparrow\neg x \}$};
		\node (C') at (7,1) {$\{ \uparrow x \}$};
		\node (D') at (8,2) {$2$};
		
		
		\draw[line width=.01in] (A') -- (B');
		\draw[line width=.01in] (A') -- (C');
		\draw[line width=.01in] (B') -- (D');
		\draw[line width=.01in] (C') -- (D');
		
	\end{tikzpicture}
	\caption{Free boolean algebra on one generator $\mathcal{B}_1$ (left), the finite forest $ \textbf{2} = \textbf{1} + \textbf{1} $ (center) and the Hasse diagram of the sub-forests of $\textbf{2}$ (right). }
\end{figure}

In fact:
\begin{remark}
	We re-discover under a different guise the familiar correspondence between the Free Boolean algebra on $x$ $\mathcal{B}_1$ and the two prime filters $\uparrow x$ and $\uparrow \neg x$ of the \emph{Stone Space} $\mathbf{2}$  by observing that: 
	\[ \mathbb{BA}_{fin} \cong (\mathbb{G_2})_{fin} \simeq \mathbb{FF}^{\textit{op}}_{1} = \mathbb{Set}_{fin}^{op} \]  
\end{remark}
 

\newpage

\section{Forests, Bushes and Topoi} 

Why \emph{Topoi}? \newline
For classical logic all objects and sub-objects would be represented by sets and sub-sets. This, incidentally, is the case for finite forests of height 1, i.e., $\mathbb{FF}_1$, which is equivalent to the category of finite sets $\mathbb{Set}_{fin}$.  However for finite forests of height greater than 1 something different is required. \newline
Topoi as such are a categorical generalization of sets and provide an ideal framework for non-classical logic.
In a Topos we wish to abstract notions of sub-sets, elements and set constructions like products and exponentiation. \newline

Recall the definition of an \emph{(elementary) Topos}.
\begin{definition}[Topos]
	A topos $\mathcal{E}$ is a category $\mathcal{E}$ such that:
	\begin{enumerate}
		\item $\mathcal{E}$ is finitely complete and co-complete.
		\item $\mathcal{E}$ has a sub-object classifier.
		\item $\mathcal{E}$ has exponential objects.
	\end{enumerate}
	
\end{definition}
 
We have seen that the categories $\mathbb{FF}, \mathbb{FF}_{k}$ for $k\geq1$ have initial and terminal objects, finite products and co-products. \newline
 \emph{Finite completeness} follows from the construction of \emph{equalizers} for trees in $\mathbb{T}$ by taking the inclusion-maximal sub-tree contained in the equalizing sub-poset and generalizing to $\mathbb{FF}$. \newline Finite completeness and co-completeness also follows from the duality with Gödel algebras exploiting the universal algebra fact that locally finite varieties are finitely complete and co-complete.
 
 \begin{lem}[finite forests are finitely complete and co-complete]\label{lem:ffcomplete}
 	$\mathbb{FF}, \mathbb{FF}_{k}$ for $k\geq1$ are finitely complete and co-complete.
 \end{lem}
 

What about the \emph{sub-object classifier}?
\newpage
\subsection{Sub-object Classifiers}

Recall what it means for a category $\mathbb{C}$ to have a \emph{sub-object classifier}:

\begin{definition}[Sub-object classifier]
	Let $\mathbb{C}$ be a category with a terminal object $\mathbf{1}$. 
	A \emph{sub-object classifier} for $\mathbb{C}$ is an object $\Omega$ together with an arrow $true: \mathbf{1} \rightarrow \Omega$ that satisfies the following $\Omega$-axiom 
\end{definition}

\begin{definition}[$\Omega$-axiom]
	For each sub-object $s : A \rightarrowtail B $ there is a unique \emph{characteristic} arrow $\chi_s : B \rightarrow \Omega$ making the following commutative diagram (a.k.a. the characteristic diagram of $s$) a \emph{pullback} of $\chi_s$ and $true$ (the unique arrow from $A$ to the terminal object $1$ is named $!_A$):
	\begin{figure}[h]
		\centering 
		\begin{tikzcd}
			A && B \\
			& {{}} \\
			1 && \Omega
			\arrow["s"', tail, from=1-1, to=1-3]
			\arrow["{\chi_s}"', from=1-3, to=3-3]
			\arrow["{!_A}", dashed, from=1-1, to=3-1]
			\arrow["true", from=3-1, to=3-3]
			\arrow[draw=none, from=1-1, to=2-2]
			\arrow[ from=1-1, to=2-2, phantom, "\scalebox{1.5}{$\lrcorner$}",  very near start, color=black]
		\end{tikzcd}\
		\caption{the characteristic diagram of $s$.}
	\end{figure}   
	In other words, there must be a 1:1 correspondence between sub-objects and characteristic arrows:
	\begin{equation*}
		Sub(B) \cong \mathbb{C}(B,\Omega).
	\end{equation*}
	
\end{definition}


For $\mathbb{Set}_{fin}$ the sub-object classifier is $\textbf{2} = \{0,1\}$ the two element set together with the map \emph{true} : $ \textbf{1} = \{0\} \rightarrow \textbf{2} \;,  \; 0 \mapsto 1$. \newline
 Equivalently in $\mathbb{FF}_1$ the classifier is $\textbf{2} = \textbf{1} + \textbf{1}$ (displayed as the nodes \textcolor{red}{\textbf{f}} and \textcolor{OliveGreen}{\textbf{t}}) and the arrow \emph{true} : $ \textbf{1} = \{\bullet\} \rightarrow \textbf{2} \;,  \; \bullet \mapsto \textcolor{OliveGreen}{\textbf{t}}$. \newline
 The $\mathbb{FF}_1$-equivalent of each finite set is a \emph{finite anti-chain} represented by a sum of \textbf{1}s. \newline
  If we take a sub-set of a finite set say $A=\{a,b,d\} \subset B=\{a,b,c,d,e\}$ where $\subset$ is given by the inclusion arrow $\iota_A$, the \emph{characteristic} arrow $\chi_A$ sends the nodes of $A$ to \textcolor{OliveGreen}{\textbf{t}} and those of $B \setminus A$ to \textcolor{red}{\textbf{f}}. \newline The characteristic diagram in this case is given by: (coloring the domain of \emph{true} and $\chi_A$ is meant to show that \textcolor{OliveGreen}{$\bullet$} and \textcolor{red}{$\bullet$} map to $\textcolor{OliveGreen}{\textbf{t}}$ and $\textcolor{red}{\textbf{f}}$ respectively in \textbf{2}).


\begin{figure}[h]
	\centering
\begin{tikzcd}[scale=0.6]
	{\bullet \bullet \;\bullet} && {\textcolor{OliveGreen}{\bullet} \textcolor{OliveGreen}{\bullet} \textcolor{red}{\bullet} \textcolor{OliveGreen}{\bullet} \textcolor{red}{\bullet}} \\
	& {{}} \\
	{\textcolor{OliveGreen}{\bullet}} && {\textcolor{red}{\textbf{f}}\;\;\;\; \textcolor{OliveGreen}{\textbf{t}} }
	\arrow["\iota_A"', tail, from=1-1, to=1-3]
	\arrow["{\chi_A}"', from=1-3, to=3-3]
	\arrow["{!_A}", dashed, from=1-1, to=3-1]
	\arrow["true", from=3-1, to=3-3]
	\arrow[draw=none, from=1-1, to=2-2]
	\arrow[ from=1-1, to=2-2, phantom, "\scalebox{1.5}{$\lrcorner$}",  very near start, color=black]
\end{tikzcd}
	\caption{Characteristic diagram for  $\{a,b,d\} \subset \{a,b,c,d,e\}$.}
\end{figure}

%\begin{figure}[h]
%	\centering
%	\begin{tikzcd}[scale=0.6]
%		A = \{a, b, d\} && B = \{\textcolor{OliveGreen}{a}, \textcolor{OliveGreen}{b}, \textcolor{red}{c}, \textcolor{OliveGreen}{d}, \textcolor{red}{e}\} \\
%		& {{}} \\
%		1 = \{\textcolor{OliveGreen}{*}\} && 2 = \{ \textcolor{red}{\textbf{f}},\; \textcolor{OliveGreen}{\textbf{t}} \}
%		\arrow["\iota_A"', tail, from=1-1, to=1-3]
%		\arrow["{\chi_A}"', from=1-3, to=3-3]
%		\arrow["{!_A}", dashed, from=1-1, to=3-1]
%		\arrow["true", from=3-1, to=3-3]
%		\arrow[draw=none, from=1-1, to=2-2]
%		\arrow[ from=1-1, to=2-2, phantom, "\scalebox{1.5}{$\lrcorner$}",  very near start, color=black]
%	\end{tikzcd}
%	\caption{Characteristic diagram for  $\{a,b,d\} \subset \{a,b,c,d,e\}$.}
%\end{figure}



	\newpage
	Generalizing to Finite Forests $\mathbb{FF}$: We know, thanks to \cite{towards}, that the object $\Omega = \textbf{1} + \textbf{1}_\bot = Spec(\mathcal{F}_1)$ \emph{assumes the role} of $\textbf{2}$ together with an appropriate arrow \emph{true} and is in fact a sub-object classifier for $ \mathbb{FF}$.
	
\begin{thm}[Sub-object classifier for $\mathbb{FF}$]
	$\Omega$ together with $true: \mathbf{1} \rightarrow \Omega$ is the sub-object classifier for $\mathbb{FF}$. \newline
	$true$ is defined as the unique map that carries $\bullet$ to the root of $\textbf{1}_\bot \subset \Omega$.
\end{thm}

Observe that $\Omega$ is an object of $\mathbb{FF}_2$ and $1$ is the terminal object of every $\mathbb{FF}_k$ $k>0$ so:

\begin{cor}\label{cor:subobjcl}
	$\Omega$ and $true$ (as defined for $\mathbb{FF}$) form the sub-object classifier of $\mathbb{FF}_k$ for any $k>1$.
\end{cor}

\begin{figure}[h]
	\centering
\begin{tikzpicture}[scale=0.4]
			\node (A) at (-14,0) {$a_0$};
			\node (B) at (-10,0) {$b_0$};
			\node (C) at (-10,4) {$b_1$};
			\draw[line width=.03in] (B) -- (C);
	
	
	\node (A') at (0,0) {$\uparrow \neg x$};
	\node (B') at (4,0) {$\uparrow x$};
	\node (C') at (4,4) {$\uparrow \neg\neg x$};
	\draw[line width=.03in] (B') -- (C');
	
		\node (A'') at (14,0) {\textcolor{red}{\textbf{f}}};
		\node (B'') at (18,0) {\textcolor{OliveGreen}{\textbf{t}}};
		\node (C'') at (18,4) {\textcolor{SkyBlue}{$*$}};
		\draw[SkyBlue, line width=.03in] (B'') -- (C'');
	\end{tikzpicture}
\caption{We recollect the different representations of $\Omega$ as $1+1_\bot$ (left), $Spec(\mathcal{F}_1)$ (center) and also give a new one (right) using the labels "\textcolor{red}{\textbf{f}}","\textcolor{OliveGreen}{\textbf{t}}" and "\textcolor{SkyBlue}{$*$}" .}
\end{figure}
\newpage
To see why this holds:\newline
Given a sub-object, say $f: F \hookrightarrow G$, which determines a sub-forest $f[F]$ of $G$, there is a unique $\chi_f: G \rightarrow \Omega $ making the characteristic diagram a pullback. 
\newline
An instance of this is given in which $f : \alpha_j \mapsto a_j, \beta_j \mapsto b_j,  \delta_0 \mapsto d_0$ for $ j=0,1 $ (the usual coloring notation on the domain applies).

\begin{figure}[h]
	\centering
	\begin{tikzcd}
		\begin{tikzpicture}[scale=0.4]
			\node (A0) at (0,0) {\textcolor{black}{$\alpha_0$}};
			\node (A1) at (0,3) {\textcolor{black}{$\alpha_1$}};
			\node (B0) at (2,0) {\textcolor{black}{$\beta_0$}};
			\node (B1) at (2,3) {\textcolor{black}{$\beta_1$}};
			\node (C0) at (4,0) {\textcolor{black}{$\delta_0$}};
			\draw[line width=.03in] (A0) -- (A1);
			\draw[line width=.03in] (B0) -- (B1);
		\end{tikzpicture} && 
		\begin{tikzpicture}[scale=0.4]
		\node (A0) at (0,0) {\textcolor{OliveGreen}{$a_0$}};
		\node (A1) at (0,3) {\textcolor{OliveGreen}{$a_1$}};
		\node (B0) at (3,0) {\textcolor{OliveGreen}{$b_0$}};
		\node (B1) at (3,3) {\textcolor{SkyBlue}{$b_2$}};
		\node (B2) at (5,3) {\textcolor{OliveGreen}{$b_1$}};
		\node (B3) at (3,6) {\textcolor{SkyBlue}{$b_3$}};
		\node (C0) at (7,0) {\textcolor{red}{$c_0$}};
		\node (C1) at (7,3) {\textcolor{red}{$c_1$}};
		\node (D0) at (9,0) {\textcolor{OliveGreen}{$d_0$}};
		\node (E0) at (11,0) {\textcolor{red}{$e_0$}};
		\node (E1) at (11,3) {\textcolor{red}{$e_1$}};
		\node (E2) at (10,6) {\textcolor{red}{$e_2$}};
		\node (E3) at (12,6) {\textcolor{red}{$e_3$}};
		
		\draw[OliveGreen, line width=.03in] (A0) -- (A1);
		\draw[SkyBlue,line width=.03in] (B0) -- (B1);
		\draw[OliveGreen,line width=.03in] (B0) -- (B2);
		\draw[SkyBlue, line width=.03in] (B1) -- (B3);
		\draw[red, line width=.03in] (C0) -- (C1);
		\draw[red, line width=.03in] (E0) -- (E1);
		\draw[red, line width=.03in] (E1) -- (E2);
		\draw[red, line width=.03in] (E1) -- (E3);
	\end{tikzpicture} \\
		& {{}} \\
		\textcolor{OliveGreen}{\bullet} && \begin{tikzpicture}[scale=0.4]
			\node (A) at (0,0) {\textcolor{red}{\textbf{f}}};
			\node (B) at (3,0) {\textcolor{OliveGreen}{\textbf{t}}};
			\node (C) at (3,3) {\textcolor{SkyBlue}{$*$}};
			\draw[SkyBlue, line width=.03in] (B) -- (C);
		\end{tikzpicture}
		\arrow["{\chi_f}"', from=1-3, to=3-3]
		\arrow["{!_F}", dashed, from=1-1, to=3-1]
		\arrow["true", from=3-1, to=3-3]
		\arrow[draw=none, from=1-1, to=2-2]
		\arrow["f"', tail, from=1-1, to=1-3]
		\arrow[draw=none, from=1-1, to=2-2]
		\arrow[ from=1-1, to=2-2, phantom, "\scalebox{1.5}{$\lrcorner$}",  very near start, color=black]
	\end{tikzcd}
\caption{Characteristic diagram of $f: F \hookrightarrow G$. }	
\end{figure}

Notice that $\chi_f$ assigns the value \textcolor{OliveGreen}{\textbf{t}} to the nodes of the sub-forest $f[F]$,  the value \textcolor{SkyBlue}{\textbf{$*$}} to any node that is not in the sub-forest but is in the up-set of $f(F)$ so  and the value \textcolor{red}{\textbf{f}} to any node that is not in the sub-forest nor in its up-set. 

\newpage
\subsection{Exponentials}

We now come to \emph{exponentiation}.

Recall what it means for a category $\mathbb{C}$ to have an \emph{exponential object}:
\begin{definition}[exponential objects]
	$\forall A, B \in \mathbb{C}$ there exists an exponential object $B^A \in \mathbb{C}$ and a map $eval : B^A \times A \rightarrow B$ such that the  \emph{universal mapping property} or UMP holds: For any $g :C \times A \rightarrow B$ there is a unique $\hat{g} :C \rightarrow B^A$ making the following diagram commute:
	
	\begin{figure}[h]
		\centering
		\begin{tikzcd}
			{B^A \times A} && B \\
			\\
			{C \times A}
			\arrow["g"', from=3-1, to=1-3]
			\arrow["{\hat{g}\times id_A}"', from=3-1, to=1-1]
			\arrow["eval"', from=1-1, to=1-3]
		\end{tikzcd}
		\caption{UMP}
	\end{figure}
	
\end{definition}

 \begin{remark}
 	The UMP can also be read as: $\forall g :C \times A \rightarrow B\;$ $\exists! \;\hat{g} :C \rightarrow B^A$ called an \emph{encoding of g} such that $eval \circ (\hat{g}\times id_A) = g$.
 \end{remark}


\begin{remark}
	This can also be seen as an \emph{adjunction} $ (-) \times A \dashv (-)^{A} $ between the \emph{product} and \emph{exponent} functors by a fixed object $A$ which gives rise to a 1:1 correspondence for any object $B$:
	\begin{gather*}
		Hom(C \times A, B) \cong Hom(C,B^A) \\
		g \xleftrightarrow{1:1} \hat{g}
	\end{gather*} 
\end{remark}
 
 Remember also that for finite sets $\mathbb{Set}_{fin}$  the exponential is defined as  $B^A := \{f : A \rightarrow B\}$ i.e., the set of functions from $A$ to $B$. \newline
 $\hat{g}$ is the result of the operation of \emph{currying} i.e., $\hat{g}(c) := g(c,-) : A \rightarrow B$ and \emph{eval} is \emph{evaluation} of a function $f$ in $a$ is $eval(f,a) := f(a)$ so that $eval \circ (\hat{g}\times id_A)(c,a) = eval(\hat{g}(c),a)=(\hat{g}(c))(a)=g(c,a)$.\newline
 \newline
 We now proceed outlining the steps used in \cite{towards}:\newline
 \newline
 Do exponential objects exist in $\mathbb{FF}_*$? \newline
 
 \emph{If} they exist then, since $\mathbb{FF}_*$ is a \emph{distributive} category:
 \begin{lem}
 	For all objects $ F,G,H $ in $\mathbb{FF}_* :$
 	\[ F^{(G+H)} \cong F^G \times F^H \]
 \end{lem}

Recall that any $F \in \mathbb{FF}_*$ can be written as $F= \sum_{i=1}^{N} T_i $ for some trees $\{T_i\}_{i=1}^N$, this allows us to reduce the study of the existence of exponentiation to the case $F^T$ for some tree $T$.\newline

Since $ C \cong (C \times 1) \rightarrow F$ is adjoint to $ C \rightarrow F^1 $ for any $C \in \mathbb{FF}_*$, we have $F^1 \cong F$ for every $F \in \mathbb{FF}_*$.
We now have:
\begin{remark}
	$\forall F,G \in \mathbb{FF}_* : (F+G)^1 \cong F^1 + G^1$.
\end{remark}
This can be generalized.
It can be shown that $F^T + G^T$ \emph{behaves as} $(F+G)^T$. \footnote{i.e., for each $f: H \times T \rightarrow F + G$ there is a unique $\hat{f}: H \rightarrow F^T + G^T$ such that $(e_F + e_G)\circ(\hat{f} \times id_T)=f$ with $e_F,e_G$ the evaluation maps for $F^T$ and $G^T$.}
\newline
What this entails is:
\begin{lem}$\forall F,G \in \mathbb{FF}_*, T \in \mathbb{T} :$
	\[ (F+G)^T \cong F^T + G^T\]
\end{lem}

In fact reducing further the study to the existence of the exponential $T^S$ for both $T,S \in \mathbb{T}$,
we now have for $F,G \in \mathbb{FF}_*$:
\begin{thm}
	For $ F = \sum_{i=1}^{n} T_i $ and $ G = \sum_{j=1}^{m} S_j :$
	\[ F^G \cong \prod_{j=1}^{m} \sum_{i=1}^{n} (T_i^{S_j}) \] 
\end{thm}

\newpage

Exponentiation for $\mathbb{FF}_1$ has been settled as $\mathbb{FF}_1 \simeq Set_{fin}$. \newline
The next "level up" is $\mathbb{FF}_2$ or the category of \emph{Bushes}. \newline
Each \emph{bush} in $\mathbb{FF}_2 \cap \mathbb{T}$ is of the form $B_\bot$ where $B$ is a finite anti-chain i.e., a finite set. \newline

By the previous considerations, we arrive at the main result of \cite{towards}: \newline
(Note that $ |A_\bot|$ denotes the cardinality of the underlying poset and $n\textbf{F}$ is shorthand for the n-th co-power of $\textbf{F}$).
\begin{thm}[\emph{bushes} have exponential objects]\label{thm:bushesexp}
 The following formula holds for all $A_\bot$ and $B_\bot$ $\in \mathbb{FF}_2 \cap \mathbb{T}$ :
\begin{equation}\label{exponent}
	B_\bot ^{A_\bot} \cong | B_\bot |^{|A|} (\;(\; (|B_\bot|^{|A_\bot|} -1)\mathbf{1} )_\bot\;)
\end{equation}

Using distributivity we can generalize this formula to arbitrary Bushes $F$ and $G$, written as sums of trees in $\mathbb{FF}_2$, $F=\sum_{i=1}^{m}(F_i)_\bot$ and $G=\sum_{j=1}^{n}(G_j)_\bot$:
\begin{equation}
	F ^ G \cong \prod_{j=1}^{n} \sum_{i=1}^{m} ( | (F_i)_\bot |^{|G_j|} (\;(\; (|(F_i)_\bot|^{|(G_j)_\bot|} -1)\mathbf{1} )_\bot\;) )
\end{equation}
\end{thm} 

 The intuition behind this fact is that trees in $\mathbb{FF}_2$ behave with respect to arrows nearly as finite sets. This is because arrows $f$ from $B_\bot$ to a target $C_\bot$ are  basically (Set-)functions with the only constraint that the root of $B_\bot$  be sent to the root of $C_\bot$.\newline
 
 Putting together the results in Theorem~\ref{thm:bushesexp},Corollary~\ref{cor:subobjcl} \& Lemma~\ref{lem:ffcomplete} we obtain:
 \begin{cor}[\emph{bushes} form a topos]
 	$\mathbb{FF}_2$/\emph{bushes} is an elementary topos.
 \end{cor}
 
 	We wrap up by giving a few examples of exponentials for Bushes:
 
 \begin{ex}
 	\begin{gather*}
 		\Omega^{\textbf{1}_\bot} \cong \textbf{1} + 2(3 \cdot \textbf{1})_\bot. \\
 		(\textbf{1}_\bot) ^{\Omega} \cong 2(7 \cdot \textbf{1})_\bot. \\
 		 \Omega^{\Omega} \cong \textbf{1} + \textbf{1}_\bot + 2(3 \cdot \textbf{1})_\bot + 2(7 \cdot \textbf{1})_\bot.  
 	\end{gather*}
 \end{ex}
 
 
\newpage
\subsection{\hl{An Exponential Example}}

 Let us examine how these exponential objects work with the example of arrows from $\textbf{1}_\bot \times \textbf{1}_\bot$ to $\textbf{1}_\bot$. \newline
 The formula \ref{exponent} says that the exponential object should be:
 \[ (\textbf{1}_\bot) ^ {\textbf{1}_\bot} \cong 2^1 ( ( (2^2-1)\textbf{1} )_\bot ) = 2( 3\cdot\textbf{1})_\bot \]

The universal mapping property (UMP) applied to this case states that the following diagram should commute.

\begin{figure}[h]
	\centering
	\begin{tikzcd}[scale=0.5]
		{\begin{tikzpicture}[scale=0.4]
				\node (A) at (0,0) {$e_0$};
				\node (B) at (-3,3) {$e_3$};
				\node (C) at (0,3) {$e_2$};
				\node (D) at (3,3) {$e_1$};
				\draw[line width=.03in] (A) -- (B);
				\draw[line width=.03in] (A) -- (C);
				\draw[line width=.03in] (A) -- (D);
				
				\node (A') at (9,0) {$e_0'$};
				\node (B') at (6,3) {$e_3'$};
				\node (C') at (9,3) {$e_2'$};
				\node (D') at (12,3) {$e_1'$};
				\draw[line width=.03in] (A') -- (B');
				\draw[line width=.03in] (A') -- (C');
				\draw[line width=.03in] (A') -- (D');
			\end{tikzpicture} \times \begin{tikzpicture}[scale=0.4]
			\node (A) at (0,0) {$a_0$};
			\node (B) at (0,3) {$a_1$};
			
			\draw[line width=.03in] (A) -- (B);
			
		\end{tikzpicture}} && 
	\begin{tikzpicture}[scale=0.4]
	\node (A) at (0,0) {$b_0$};
	\node (B) at (0,3) {\textcolor{gray!60}{$b_1$}};

	\draw[gray!60, line width=.03in] (A) -- (B);

	\end{tikzpicture} \\
		\\
		{\begin{tikzpicture}[scale=0.4]
				\node (A) at (0,0) {$c_0$};
				\node (B) at (0,3) {$c_1$};
				
				\draw[line width=.03in] (A) -- (B);
				
			\end{tikzpicture} \times \begin{tikzpicture}[scale=0.4]
			\node (A) at (0,0) {$a_0$};
			\node (B) at (0,3) {$a_1$};
			
			\draw[line width=.03in] (A) -- (B);
			
		\end{tikzpicture}}
		\arrow["g"', from=3-1, to=1-3]
		\arrow["{\hat{g}\times id_A}"', from=3-1, to=1-1]
		\arrow["eval"', from=1-1, to=1-3]
	\end{tikzcd}
	\caption{UMP diagram for $A=B=C=\textbf{1}_\bot$. 	(Bushes)}
\end{figure}

To understand how to construct the \emph{adjoint} arrow $\hat{g} : \textbf{1}_\bot \rightarrow (\textbf{1}_\bot)^{\textbf{1}_\bot} $ of $g : \textbf{1}_\bot \times \textbf{1}_\bot \rightarrow \textbf{1}_\bot$ let us take a look at all the possible maps (black dots are mapped to the root $b_0$ of $B=\textbf{1}_\bot$ whilst gray dots to $b_1$):
\newline
For example, one of these maps $g_1$ is:
\begin{figure}[h]
	\begin{tikzpicture}[scale=0.25]
		
		\node (A) at (0,0) {$\textcolor{black}{\bullet}$};
		\node (B) at (-4,5) {$\textcolor{gray!60}{\bullet}$};
		\node (C) at (0,5) {$\textcolor{gray!60}{\bullet}$};
		\node (D) at (4,5) {$\textcolor{black}{\bullet}$};
		\node (d) at (5,2) {};
		
		\draw[gray!60, line width=.03in] (A) -- (B);
		\draw[gray!60, line width=.03in] (A) -- (C);
		\draw[line width=.03in] (A) -- (D);
		
		\node (e) at (17,2) {};
		\node (E) at (18,0) {$\textcolor{black}{\bullet}$};
		\node (F) at (18,5) {$\textcolor{gray!60}{\bullet}$};
		\draw[gray!60, line width=.03in] (E) -- (F);
		
		\draw[->, dotted, line width=.01in] (d) -- node[anchor=north] {$g_1$} (e);
	\end{tikzpicture}
	\caption{$C \times A = \textbf{1}_\bot \times \textbf{1}_\bot$ is displayed on the left and $B = \textbf{1}_\bot$ on the right. The map $g_1$ sends the nodes on the left object to nodes with matching color on the right object.}	
\end{figure}

%\begin{figure}[h]
%	\begin{tikzpicture}[scale=0.25]
%		
%		\node (A) at (0,0) {$\bot$};
%		\node (B) at (0,5) {$\textcolor{orange}{\bullet}$};
%		\node (d) at (1,2) {};
%		
%		\draw[orange, line width=.03in] (A) -- (B);
%		
%		\node (e) at (10,2) {};
%		
%		\node (E) at (12,0) {$e_0$};
%		\node (F) at (9,5) {$e_3$};
%		\node (G) at (12,5) {$e_2$};
%		\node (H) at (15,5) {$\textcolor{orange}{e_1}$};
%		\draw[line width=.03in] (E) -- (F);
%		\draw[line width=.03in] (E) -- (G);
%		\draw[orange, line width=.03in] (E) -- (H);
%		
%		\node (E') at (22,0) {$e_7$};
%		\node (F') at (19,5) {$e_6$};
%		\node (G') at (22,5) {$e_5$};
%		\node (H') at (25,5) {$e_4$};
%		\draw[line width=.03in] (E') -- (F');
%		\draw[line width=.03in] (E') -- (G');
%		\draw[line width=.03in] (E') -- (H');
%		
%		\draw[->, dotted, line width=.01in] (d) -- node[anchor=north] {$\hat{g_1}$} (e);
%	\end{tikzpicture}
%	\caption{}
%	\end{figure}




\begin{figure}[h]
	\centering
	\begin{tikzpicture}[scale=0.25]
		\node (A) at (0,0) {$\textcolor{black}{\bullet} $};
		\node (N) at (2,0) {$g_0$};
		\node (B) at (-3,5) {$\textcolor{black}{\bullet}$};
		\node (C) at (0,5) {$\textcolor{black}{\bullet}$};
		\node (D) at (3,5) {$\textcolor{black}{\bullet}$};
		\draw[line width=.03in] (A) -- (B);
		\draw[line width=.03in] (A) -- (C);
		\draw[line width=.03in] (A) -- (D);
		
		\node (A') at (9,0) {$\textcolor{black}{\bullet}$};
		\node (N') at (11,0) {$g_1$};
		\node (B') at (6,5) {$\textcolor{gray!60}{\bullet}$};
		\node (C') at (9,5) {$\textcolor{gray!60}{\bullet}$};
		\node (D') at (12,5) {$\textcolor{black}{\bullet}$};
		\draw[gray!60, line width=.03in] (A') -- (B');
		\draw[gray!60, line width=.03in] (A') -- (C');
		\draw[line width=.03in] (A') -- (D');
		
		\node (A'') at (18,0) {$\textcolor{black}{\bullet}$};
		\node (N'') at (20,0) {$g_2$};
		\node (B'') at (15,5) {$\textcolor{black}{\bullet}$};
		\node (C'') at (18,5) {$\textcolor{gray!60}{\bullet}$};
		\node (D'') at (21,5) {$\textcolor{black}{\bullet}$};
		\draw[line width=.03in] (A'') -- (B'');
		\draw[gray!60,line width=.03in] (A'') -- (C'');
		\draw[line width=.03in] (A'') -- (D'');
	
		\node (A''') at (27,0) {$\textcolor{black}{\bullet}$};
		\node (N''') at (29,0) {$g_3$};
		\node (B''') at (24,5) {$\textcolor{gray!60}{\bullet}$};
		\node (C''') at (27,5) {$\textcolor{black}{\bullet}$};
		\node (D''') at (30,5) {$\textcolor{black}{\bullet}$};
		\draw[gray!60,line width=.03in] (A''') -- (B''');
		\draw[line width=.03in] (A''') -- (C''');
		\draw[line width=.03in] (A''') -- (D''');
		
		\node (A'''') at (0,-7) {$\textcolor{black}{\bullet}$};
		\node (N'''') at (2,-7) {$g_4$};
		\node (B'''') at (-3,-2) {$\textcolor{gray!60}{\bullet}$};
		\node (C'''') at (0,-2) {$\textcolor{gray!60}{\bullet}$};
		\node (D'''') at (3,-2) {$\textcolor{gray!60}{\bullet}$};
		\draw[gray!60, line width=.03in] (A'''') -- (B'''');
		\draw[gray!60, line width=.03in] (A'''') -- (C'''');
		\draw[gray!60, line width=.03in] (A'''') -- (D'''');
		
		\node (A''''') at (9,-7) {$\textcolor{black}{\bullet}$};
		\node (N''''') at (11,-7) {$g_5$};
		\node (B''''') at (6,-2) {$\textcolor{gray!60}{\bullet}$};
		\node (C''''') at (9,-2) {$\textcolor{black}{\bullet}$};
		\node (D''''') at (12,-2) {$\textcolor{gray!60}{\bullet}$};
		\draw[gray!60, line width=.03in] (A''''') -- (B''''');
		\draw[line width=.03in] (A''''') -- (C''''');
		\draw[gray!60, line width=.03in] (A''''') -- (D''''');
		
		\node (A'''''') at (18,-7) {$\textcolor{black}{\bullet}$};
		\node (N'''''') at (20,-7) {$g_6$};
		\node (B'''''') at (15,-2) {$\textcolor{black}{\bullet}$};
		\node (C'''''') at (18,-2) {$\textcolor{black}{\bullet}$};
		\node (D'''''') at (21,-2) {$\textcolor{gray!60}{\bullet}$};
		\draw[line width=.03in] (A'''''') -- (B'''''');
		\draw[ line width=.03in] (A'''''') -- (C'''''');
		\draw[gray!60, line width=.03in] (A'''''') -- (D'''''');
		
		\node (A''''''') at (27,-7) {$\textcolor{black}{\bullet}$};
		\node (N''''''') at (29,-7) {$g_7$};
		\node (B''''''') at (24,-2) {$\textcolor{black}{\bullet}$};
		\node (C''''''') at (27,-2) {$\textcolor{gray!60}{\bullet}$};
		\node (D''''''') at (30,-2) {$\textcolor{gray!60}{\bullet}$};
		\draw[line width=.03in] (A''''''') -- (B''''''');
		\draw[gray!60, line width=.03in] (A''''''') -- (C''''''');
		\draw[gray!60, line width=.03in] (A''''''') -- (D''''''');
\end{tikzpicture}
	\caption{All the possible maps $g_j : \textbf{1}_\bot \times \textbf{1}_\bot \rightarrow \textbf{1}_\bot$ for j=0,..,7. 	(Bushes)}
\end{figure}

Notice that:
\begin{lem}
	$|Hom(\textbf{1}_\bot \times \textbf{1}_\bot, \textbf{1}_\bot)|=8$.
\end{lem}
  This result could be computed by simply counting the number of distinct set-functions from $3\cdot \textbf{1}$ to $\textbf{1}_\bot$ which is $2^3$ since as we saw $\textbf{1}_\bot \times \textbf{1}_\bot = (3\cdot \textbf{1})_\bot$ and there are no constraints on where the non-root elements are assigned. \newline

How to obtain the adjoint maps $\hat{g}_j$?\newline

The \emph{product map} $\hat{g}_j \times id_A$ by definition must satisfy:
\begin{enumerate}
	\item $\pi_{B^A}\circ(\hat{g}_j \times id_A) = \hat{g}_j$.
	\item $\pi_{A}\circ(\hat{g}_j \times id_A) = id_A$.
\end{enumerate}
From the first condition, the fibers for $i=0,1$ $\pi_C^{-1}\{c_i\}$ of $C \times A$ must be mapped to the corresponding fibers of $\pi_{B^A}^{-1}\{\hat{g}_j(c_i)\}$ of $B^A \times A$. \newline
From the second condition, the fibers for $i=0,1$ $\pi_A^{-1}\{a_i\}$ of $C \times A$ must be mapped to the corresponding fibers of $\pi_A^{-1}\{a_i\}$ of $B^A \times A$. \newline

The \emph{eval} map can now be constructed on the product $B^A \times A$.


\begin{figure}[h]
	\begin{tikzcd}
		\begin{tikzpicture}[scale=0.3]
			\node (H) at (-22,3) {\textcolor{gray!60}{$e_3 a_0$}};
			\node (G) at (-19,3) {$e_2 a_0$};
			\node (F) at (-16,3) {\textcolor{gray!60}{$e_1 a_0$}};
			\node (A) at (-10,-4) {$e_0 a_0$};
			\node (B) at (-11,3) {$e_3 a_1$};
			\node (C) at (-8,3) {\textcolor{gray!60}{$e_2 a_1$}};
			\node (D) at (-5,3) {\textcolor{gray!60}{$e_1 a_1$}};
			\node (E) at (0,3) {$e_0 a_1$};
			
			\draw[line width=.03in] (A) -- (B);
			\draw[gray!60, line width=.03in] (A) -- (C);
			\draw[gray!60, line width=.03in] (A) -- (D);
			\draw[line width=.03in] (A) -- (E);
			\draw[gray!60, line width=.03in] (A) -- (F);
			\draw[line width=.03in] (A) -- (G);
			\draw[gray!60, line width=.03in] (A) -- (H);
			
			\node (H') at (4,3) {$e_3' a_0$};
			\node (G') at (7,3) {\textcolor{gray!60}{$e_2' a_0$}};
			\node (F') at (10,3) {\textcolor{gray!60}{$e_1' a_0$}};
			\node (A') at (16,-4) {$e_0'a_0$};
			\node (B') at (15,3) {$e_3'a_1$};
			\node (C') at (18,3) {$e_2'a_1$};
			\node (D') at (21,3) {\textcolor{gray!60}{$e_1' a_1$}};
			\node (E') at (26,3) {\textcolor{gray!60}{$e_0' a_1$}};
			
			\draw[line width=.03in] (A') -- (H');
			\draw[gray!60, line width=.03in] (A') -- (G');
			\draw[gray!60, line width=.03in] (A') -- (F');
			\draw[line width=.03in] (A') -- (B');
			\draw[line width=.03in] (A') -- (C');
			\draw[gray!60, line width=.03in] (A') -- (D');
			\draw[gray!60, line width=.03in] (A') -- (E');
			
			
		\end{tikzpicture} \\
		\\
		\begin{tikzpicture}[scale=0.3]
			\node (A) at (0,0) {$b_0$};
			\node (B) at (0,4) {\textcolor{gray!60}{$b_1$}};
			
			\draw[gray!60, line width=.03in] (A) -- (B);
			
		\end{tikzpicture}
		\arrow["eval"', from=1-1, to=3-1]
	\end{tikzcd}
	\caption{The map $eval : (\textbf{1}_\bot)^{\textbf{1}_\bot}\times \textbf{1}_\bot \rightarrow \textbf{1}_\bot$. 
		%(The pre-image of the root $b_0$ is displayed in black and that of $b_1$ in gray)
			(Bushes) \newline (The usual labeling notation on nodes is used for product objects whereby the left and right letters specify the left and right projections respectively). 
	}
\end{figure}

Starting off from the map $g_0$ which sends every node to the root $b_0$ we have that $\hat{g}_0 : C=\textbf{1}_\bot \rightarrow \textbf{1}_\bot^{\textbf{1}_\bot}$ must send the nodes $c_0$ and $c_1$ of $C$ to a root (either $e_0$ or $e_0'$) of one of the trees $T_1$($\cong (3\cdot\textbf{1})_\bot$) or $T_2$ ($\cong (3\cdot\textbf{1})_\bot$). \newline

Without loss of generality we can say that $\hat{g}_0$ maps $c_0,c_1$ to the root $e_0$ of $T_1$. \newline
By a similar argument we find that $\hat{g}_1 :c_0 \mapsto e_0$ \footnote{recall that every root must be mapped to a root.} and $c_1 \mapsto e_1$. \newline
Also $\hat{g}_2 : c_0 \mapsto e_0, c_1 \mapsto e_2$ and $\hat{g}_3 : c_0 \mapsto e_0, c_1 \mapsto e_3$.   
The rest follow in a similar fashion with the maps $\hat{g}_k$ $k=4,..,7$ sending $c_0$ to $e_0'$ instead.\newline
By requiring that the UMP diagram commute, the fibers for $j=0,..,7$ of $g_j^{-1}\{b_i\}$ for $i=0,1$ must be preserved by the mapping $\hat{g}_j \times id_A$ .
\newpage
Note that $\hat{g}_j$ is entirely determined by $\hat{g}_j(c_1)$ since we know that $c_0$ maps to the root of the tree that contains $\hat{g}_j(c_1)$.
This suggests the following:
\begin{lem}
	The exponential object $(\textbf{1}_\bot) ^ {\textbf{1}_\bot}$ \emph{encodes} the maps $\{g_j\}_{j=0}^7$.
\end{lem}
%check
It is worth pausing here and make a more general categorical consideration which will come in useful later on:
 \begin{prop}\label{representing}
 	In $\mathbb{FF_2}$/\emph{bushes} we have that $\textbf{1}_\bot$ is the representing object i.e., 
 	\[ \mathbb{FF_2}(\textbf{1}_\bot, F) \cong |F| \]
 	whereby $|F|$ denotes the underlying set of the finite forest $F$.
 \end{prop}
 \begin{proof}
 	Let $\textbf{1}_\bot = \{ \bot < \bullet\}$ and $|F| = \{a_j\}_{j=1}^n$ where each $a_j$ is either
 	a root $r_j$ or on top of (a unique) one i.e., $r_j < a_j$. \newline
 	The bijection is realized as we saw earlier by $f_j \leftrightarrow a_j$ where:
 	\[f_j := \bullet \mapsto a_j \;\;\text{ and } \bot \mapsto r_j\]
 \end{proof}
 
 
We resume our example: \newline
  This \emph{encoding} can be visualized by labeling each node that corresponds to $\hat{g}_j(c_1)$ with the map $g_j$ so that the following figure is obtained:

\begin{figure}[h]
	\centering
	\begin{tikzpicture}[scale=0.35, every node/.style={scale=0.8}]
		\node (A) at (0,0) {$g_0$};
			\node (C0) at (0,-1) {\textcolor{black}{$\bullet$}};
			\node (B0) at (-2,-1) {\textcolor{black}{$\bullet$}};
			\node (D0) at (2,-1) {\textcolor{black}{$\bullet$}};
			\node (A0) at (0,-3) {\textcolor{black}{$\bullet$}};
			\draw[line width=.02in] (A0) -- (B0);
			\draw[line width=.02in] (A0) -- (C0);
			\draw[line width=.02in] (A0) -- (D0);
		\node (B) at (-5,5) {$g_3$};
			\node (C3) at (-5,8) {\textcolor{black}{$\bullet$}};
			\node (B3) at (-7,8) {\textcolor{gray!60}{$\bullet$}};
			\node (D3) at (-3,8) {\textcolor{black}{$\bullet$}};
			\node (A3) at (-5,6) {\textcolor{black}{$\bullet$}};
			\draw[gray!60, line width=.02in] (A3) -- (B3);
			\draw[line width=.02in] (A3) -- (C3);
			\draw[line width=.02in] (A3) -- (D3);
		\node (C) at (0,5) {$g_2$};
			\node (C2) at (0,8) {\textcolor{gray!60}{$\bullet$}};
			\node (B2) at (-2,8) {\textcolor{black}{$\bullet$}};
			\node (D2) at (2,8) {\textcolor{black}{$\bullet$}};
			\node (A2) at (0,6) {\textcolor{black}{$\bullet$}};
			\draw[line width=.02in] (A2) -- (B2);
			\draw[gray!60, line width=.02in] (A2) -- (C2);
			\draw[line width=.02in] (A2) -- (D2);
		\node (D) at (5,5) {$g_1$};
			\node (C1) at (5,8) {\textcolor{gray!60}{$\bullet$}};
			\node (B1) at (3,8) {\textcolor{gray!60}{$\bullet$}};
			\node (D1) at (7,8) {\textcolor{black}{$\bullet$}};
			\node (A1) at (5,6) {\textcolor{black}{$\bullet$}};
			\draw[gray!60, line width=.02in] (A1) -- (B1);
			\draw[gray!60, line width=.02in] (A1) -- (C1);
			\draw[line width=.02in] (A1) -- (D1);
			
		\draw[line width=.01in] (A) -- (B);
		\draw[line width=.01in] (A) -- (C);
		\draw[line width=.01in] (A) -- (D);
		
		\node (A') at (15,0) {$g_7$};
			\node (C0') at (15,-1) {\textcolor{gray!60}{$\bullet$}};
			\node (B0') at (13,-1) {\textcolor{black}{$\bullet$}};
			\node (D0') at (17,-1) {\textcolor{gray!60}{$\bullet$}};
			\node (A0') at (15,-3) {\textcolor{black}{$\bullet$}};
			\draw[line width=.02in] (A0') -- (B0');
			\draw[gray!60, line width=.02in] (A0') -- (C0');
			\draw[gray!60, line width=.02in] (A0') -- (D0');
		\node (B') at (10,5) {$g_6$};
			\node (C1') at (10,8) {\textcolor{black}{$\bullet$}};
			\node (B1') at (8,8) {\textcolor{black}{$\bullet$}};
			\node (D1') at (12,8) {\textcolor{gray!60}{$\bullet$}};
			\node (A1') at (10,6) {\textcolor{black}{$\bullet$}};
			\draw[ line width=.02in] (A1') -- (B1');
			\draw[ line width=.02in] (A1') -- (C1');
			\draw[gray!60, line width=.02in] (A1') -- (D1');		
		\node (C') at (15,5) {$g_5$};
			\node (C2') at (15,8) {\textcolor{black}{$\bullet$}};
			\node (B2') at (13,8) {\textcolor{gray!60}{$\bullet$}};
			\node (D2') at (17,8) {\textcolor{gray!60}{$\bullet$}};
			\node (A2') at (15,6) {\textcolor{black}{$\bullet$}};
			\draw[gray!60, line width=.02in] (A2') -- (B2');
			\draw[line width=.02in] (A2') -- (C2');
			\draw[gray!60, line width=.02in] (A2') -- (D2');		
		\node (D') at (20,5) {$g_4$};
			\node (C3') at (20,8) {\textcolor{gray!60}{$\bullet$}};
			\node (B3') at (18,8) {\textcolor{gray!60}{$\bullet$}};
			\node (D3') at (22,8) {\textcolor{gray!60}{$\bullet$}};
			\node (A3') at (20,6) {\textcolor{black}{$\bullet$}};
			\draw[gray!60, line width=.02in] (A3') -- (B3');
			\draw[gray!60, line width=.02in] (A3') -- (C3');
			\draw[gray!60, line width=.02in] (A3') -- (D3');
		\draw[line width=.01in] (A') -- (B');
		\draw[line width=.01in] (A') -- (C');
		\draw[line width=.01in] (A') -- (D');
	\end{tikzpicture}
	\caption{The encoded maps in $(\textbf{1}_\bot) ^ {\textbf{1}_\bot}$.	(Bushes)}
\end{figure}

The category of \emph{bushes} $\mathbb{FF}_2$ is a \emph{topos}.
\newline
What about \emph{higher} finite forests $\mathbb{FF}_k$ with $k\geq 3$ ?
\newline\newline
(Henceforth we denote $\mathbb{FF}_k$ with $k\geq 3$ by $\mathbb{FF}_{k\geq3}$.)



\newpage
\section{When Forests fail to be a Topos} 
\label{counterex}

As the title suggests in this section we are going to prove that finite forests $\mathbb{FF}_k$ higher than bushes with $k>2$ \emph{fail} to be a topos by detailing step by step a novel constructive counter-example. \newline\newline
The failure occurs when we require the existence of exponential objects.
Recall that exponentiation for Bushes was obtained thanks to the fact that every tree in $\mathbb{FF}_2$ was of the form $F_\bot$ with $F$ a finite anti-chain or set. Thus every map between trees $B_\bot \rightarrow C_\bot$ was essentially a set-function with the only requirement that the root node be mapped to the root of the target. This property is lost when we move to $\mathbb{FF}_k$ with $k>2$. \newline
In fact, as we shall prove: $\mathbb{FF}_k$ with $k>2$ generally fails to have exponential objects.
\newpage
\subsection{\hl{Counter-example for $\mathbb{FF}_3$}}

Let's start by re-examining the previous example of $g: \textbf{1}_\bot \times \textbf{1}_\bot \rightarrow \textbf{1}_\bot$. Though this time $\times$ is taken not to be $\times_{\mathbb{FF}_2}$ but rather $\times_{\mathbb{FF}}$. \newline
Let us take a look at some of the possible maps.

For example, one of these maps $g_1$ is:
\begin{figure}[h]
	\centering
	\begin{tikzpicture}[scale=0.25]
		
		\node (A) at (0,0) {$\textcolor{black}{\bullet}$};
		\node (B) at (-4,5) {$\textcolor{gray!60}{\bullet}$};
		\node (B') at (-4,10) {$\textcolor{gray!60}{\bullet}$};
		\node (C) at (0,5) {$\textcolor{gray!60}{\bullet}$};
		\node (D) at (4,5) {$\textcolor{black}{\bullet}$};
		\node (D') at (4,10) {$\textcolor{black}{\bullet}$};
		\node (d) at (5,2) {};
		
		\draw[gray!60, line width=.03in] (A) -- (B);
		\draw[gray!60, line width=.03in] (B) -- (B');
		\draw[gray!60, line width=.03in] (A) -- (C);
		\draw[line width=.03in] (A) -- (D);
		\draw[line width=.03in] (D) -- (D');
		
		\node (e) at (17,2) {};
		\node (E) at (18,0) {$\textcolor{black}{\bullet}$};
		\node (F) at (18,5) {$\textcolor{gray!60}{\bullet}$};
		\draw[gray!60, line width=.03in] (E) -- (F);
		
		\draw[->, dotted, line width=.01in] (d) -- node[anchor=north] {$g_1$} (e);
	\end{tikzpicture}
\caption{
	As before black dots are mapped to the root $b_0$ of $B=\textbf{1}_\bot$ whilst gray dots to $b_1$.}
\end{figure}


\begin{figure}[h]
	\centering
	\begin{tikzpicture}[scale=0.2]
		\node (A) at (0,0) {$\textcolor{black}{\bullet} $};
		\node (N) at (2,0) {$g_0$};
			\node (B) at (-3,5) {$\textcolor{black}{\bullet}$};
				\node (E) at (-3,10) {$\textcolor{black}{\bullet}$};
			\node (C) at (0,5) {$\textcolor{black}{\bullet}$};
			\node (D) at (3,5) {$\textcolor{black}{\bullet}$};
				\node (F) at (3,10) {$\textcolor{black}{\bullet}$};
		\draw[line width=.03in] (A) -- (B);
			\draw[line width=.03in] (B) -- (E);
		\draw[line width=.03in] (A) -- (C);
		\draw[line width=.03in] (A) -- (D);
			\draw[line width=.03in] (D) -- (F);
		
		\node (A') at (9,0) {$\textcolor{black}{\bullet}$};
		\node (N') at (11,0) {$g_1$};
			\node (B') at (6,5) {$\textcolor{gray!60}{\bullet}$};
				\node (E') at (6,10) {$\textcolor{gray!60}{\bullet}$};
			\node (C') at (9,5) {$\textcolor{gray!60}{\bullet}$};
			\node (D') at (12,5) {$\textcolor{black}{\bullet}$};
				\node (F') at (12,10) {$\textcolor{black}{\bullet}$};
		\draw[gray!60, line width=.03in] (A') -- (B');
			\draw[gray!60, line width=.03in] (B') -- (E');
		\draw[gray!60, line width=.03in] (A') -- (C');
		\draw[line width=.03in] (A') -- (D');
			\draw[line width=.03in] (D') -- (F');
		
		\node (A'') at (18,0) {$\textcolor{black}{\bullet}$};
		\node (N'') at (20,0) {$g_2$};
			\node (B'') at (15,5) {$\textcolor{black}{\bullet}$};
				\node (E'') at (15,10) {$\textcolor{black}{\bullet}$};
			\node (C'') at (18,5) {$\textcolor{gray!60}{\bullet}$};
			\node (D'') at (21,5) {$\textcolor{black}{\bullet}$};
				\node (F'') at (21,10) {$\textcolor{black}{\bullet}$};
		\draw[line width=.03in] (A'') -- (B'');
			\draw[line width=.03in] (B'') -- (E'');
		\draw[gray!60,line width=.03in] (A'') -- (C'');
		\draw[line width=.03in] (A'') -- (D'');
			\draw[line width=.03in] (D'') -- (F'');
		
		\node (A''') at (27,0) {$\textcolor{black}{\bullet}$};
		\node (N''') at (29,0) {$g_3$};
			\node (B''') at (24,5) {$\textcolor{gray!60}{\bullet}$};
				\node (E''') at (24,10) {$\textcolor{gray!60}{\bullet}$};
			\node (C''') at (27,5) {$\textcolor{black}{\bullet}$};
			\node (D''') at (30,5) {$\textcolor{black}{\bullet}$};
				\node (F''') at (30,10) {$\textcolor{black}{\bullet}$};
			\draw[gray!60,line width=.03in] (A''') -- (B''');
				\draw[gray!60,line width=.03in] (B''') -- (E''');
			\draw[line width=.03in] (A''') -- (C''');
			\draw[line width=.03in] (A''') -- (D''');
				\draw[line width=.03in] (D''') -- (F''');
		
		\node (A'''') at (0,-12) {$\textcolor{black}{\bullet}$};
		\node (N'''') at (2,-12) {$g_4$};
			\node (B'''') at (-3,-7) {$\textcolor{black}{\bullet}$};
				\node (E'''') at (-3,-2) {$\textcolor{gray!60}{\bullet}$};
			\node (C'''') at (0,-7) {$\textcolor{black}{\bullet}$};
			\node (D'''') at (3,-7) {$\textcolor{black}{\bullet}$};
				\node (F'''') at (3,-2) {$\textcolor{gray!60}{\bullet}$};
			\draw[line width=.03in] (A'''') -- (B'''');
				\draw[gray!60, line width=.03in] (B'''') -- (E'''');
			\draw[line width=.03in] (A'''') -- (C'''');
			\draw[line width=.03in] (A'''') -- (D'''');
				\draw[gray!60, line width=.03in] (D'''') -- (F'''');
		
		\node (A''''') at (9,-12) {$\textcolor{black}{\bullet}$};
		\node (N''''') at (11,-12) {$g_5$};
			\node (B''''') at (6,-7) {$\textcolor{gray!60}{\bullet}$};
				\node (E''''') at (6,-2) {$\textcolor{gray!60}{\bullet}$};
			\node (C''''') at (9,-7) {$\textcolor{gray!60}{\bullet}$};
			\node (D''''') at (12,-7) {$\textcolor{black}{\bullet}$};
				\node (F''''') at (12,-2) {$\textcolor{gray!60}{\bullet}$};
			\draw[gray!60, line width=.03in] (A''''') -- (B''''');
				\draw[gray!60, line width=.03in] (B''''') -- (E''''');
			\draw[gray!60, line width=.03in] (A''''') -- (C''''');
			\draw[line width=.03in] (A''''') -- (D''''');
				\draw[gray!60, line width=.03in] (D''''') -- (F''''');
		
		\node (A'''''') at (18,-12) {$\textcolor{black}{\bullet}$};
		\node (N'''''') at (20,-12) {$g_6$};
			\node (B'''''') at (15,-7) {$\textcolor{black}{\bullet}$};
				\node (E'''''') at (15,-2) {$\textcolor{gray!60}{\bullet}$};
			\node (C'''''') at (18,-7) {$\textcolor{black}{\bullet}$};
			\node (D'''''') at (21,-7) {$\textcolor{black}{\bullet}$};
				\node (F'''''') at (21,-2) {$\textcolor{black}{\bullet}$};
		\draw[line width=.03in] (A'''''') -- (B'''''');
			\draw[gray!60, line width=.03in] (B'''''') -- (E'''''');
		\draw[ line width=.03in] (A'''''') -- (C'''''');
		\draw[line width=.03in] (A'''''') -- (D'''''');
			\draw[line width=.03in] (D'''''') -- (F'''''');
		
		\node (A''''''') at (27,-12) {$\textcolor{black}{\bullet}$};
		\node (N''''''') at (29,-12) {$g_7$};
			\node (B''''''') at (24,-7) {$\textcolor{black}{\bullet}$};
				\node (E''''''') at (24,-2) {$\textcolor{black}{\bullet}$};
			\node (C''''''') at (27,-7) {$\textcolor{black}{\bullet}$};
			\node (D''''''') at (30,-7) {$\textcolor{black}{\bullet}$};
				\node (F''''''') at (30,-2) {$\textcolor{gray!60}{\bullet}$};
			\draw[line width=.03in] (A''''''') -- (B''''''');
				\draw[line width=.03in] (B''''''') -- (E''''''');
			\draw[line width=.03in] (A''''''') -- (C''''''');
			\draw[line width=.03in] (A''''''') -- (D''''''');
				\draw[gray!60, line width=.03in] (D''''''') -- (F''''''');
				
		\node (n) at (0,-20) {....};
		
			\node (A0) at (9,-25) {$\textcolor{black}{\bullet}$};
			\node (N0) at (11,-25) {$g_{12}$};
				\node (B0) at (6,-20) {$\textcolor{black}{\bullet}$};
					\node (E0) at (6,-15) {$\textcolor{black}{\bullet}$};
				\node (C0) at (9,-20) {$\textcolor{gray!60}{\bullet}$};
				\node (D0) at (12,-20) {$\textcolor{gray!60}{\bullet}$};
					\node (F0) at (12,-15) {$\textcolor{gray!60}{\bullet}$};
				\draw[line width=.03in] (A0) -- (B0);
					\draw[line width=.03in] (B0) -- (E0);
				\draw[gray!60, line width=.03in] (A0) -- (C0);
				\draw[gray!60, line width=.03in] (A0) -- (D0);
					\draw[gray!60, line width=.03in] (D0) -- (F0);
					
	\node (n') at (18,-20) {....};
	\end{tikzpicture}
	\caption{The maps $g_j : \textbf{1}_\bot \times \textbf{1}_\bot \rightarrow \textbf{1}_\bot$ for $ j=0,..,17 $. }
\end{figure}

Notice that in this case:
\begin{lem}
	$|Hom(\textbf{1}_\bot \times \textbf{1}_\bot, \textbf{1}_\bot)|=18=|Hom(\textbf{1}_\bot, (\textbf{1}_\bot)^{\textbf{1}_\bot})|$.
\end{lem}
The previous result can be obtained in a number of ways. One of them is to count all the possible \emph{extensions} in $\mathbb{FF}$ of the old maps $g_j$ used in $\mathbb{FF}_2$. Another is to count all the possible sub-forests or down-sets of $\textbf{1}_\bot \times \textbf{1}_\bot$ since $B$ has only two elements and the pre-image of the root $b_0$ uniquely determines the map. \newline
The exponential object $(\textbf{1}_\bot)^{\textbf{1}_\bot}$ for $\mathbb{FF}_k$ $k>2$ is not readily given by some formula as in the case of Bushes. 
However, we have reason to state that in this case:
\begin{lem}\label{lem:superforest}
The \emph{candidate} for $(\textbf{1}_\bot)^{\textbf{1}_\bot}$ must be a \emph{super-forest} of the same-name exponential object found for the category of Bushes i.e., of $ 2(3 \cdot \textbf{1})_\bot $.
\end{lem}
This is due to the fact that $\textbf{1}_\bot \times_{\mathbb{FF}_2} \textbf{1}_\bot$ is by definition a sub-forest (of nodes of height at most 2) of $\textbf{1}_\bot \times_{\mathbb{FF}} \textbf{1}_\bot$ and the maps \emph{encoded} by $\textbf{1}_\bot \times_{\mathbb{FF}} \textbf{1}_\bot$ are all \emph{extensions} of maps already encoded by $\textbf{1}_\bot \times_{\mathbb{FF}_2} \textbf{1}_\bot$. 

\begin{remark}
	Another approach to Lemma~\ref{lem:superforest} is obtained by looking at arrows \emph{into} the  \emph{candidate} exponential object "$(\textbf{1}_\bot)^{\textbf{1}_\bot}$".\newline We can start by the simple case of an arrow from \textbf{1} and proceed from there by examining the arrows from $\textbf{1}_\bot$, $(\textbf{1}_\bot)_\bot$ and so forth.
	\newline
	Note that effectively this is an application of the \emph{Yoneda Lemma}\footnote{see \cite{awodey} for reference.} as we are determining the structure of an unknown object by looking at its \emph{relations}/arrows with other objects $C$:
	\begin{equation*}
	X \cong (\textbf{1}_\bot)^{\textbf{1}_\bot} \; \text{ iff } \;\forall C \in \mathbb{FF}_* : Hom(C,X) \cong Hom (C,(\textbf{1}_\bot)^{\textbf{1}_\bot}). 
	\end{equation*}
\end{remark}
Now,\newline
by adjunction we have: 
\[ |Hom(\textbf{1}, (\textbf{1}_\bot)^{\textbf{1}_\bot})|=|Hom(\textbf{1} \times \textbf{1}_\bot, \textbf{1}_\bot)| = |Hom(\textbf{1}_\bot, \textbf{1}_\bot)| =2\]

Since a root can be mapped only to another root this tells us that our candidate for $(\textbf{1}_\bot)^{\textbf{1}_\bot}$ is the sum of two trees $T_0 + T_1$.
\newline
 Also by the previous remark since there should be precisely 18 maps from $\textbf{1}_\bot$ to $(\textbf{1}_\bot)^{\textbf{1}_\bot}$ we know that between them these trees have $18 - 2 = 16$ (the 2 constant maps on roots have been subtracted) \emph{first-level branches} i.e., branches from the root node. \newline
The same arguments about \emph{fibers} we used in the case of \emph{bushes} apply here. 
For example $\hat{g}_0$ must send the nodes of $C=\textbf{1}_\bot$ to a root say $e_0$ of the tree $T_0$ and $\hat{g}_{12}$ to the root $e_0'$ of $T_1$.. etc.
 \newpage


The universal mapping property (UMP) in this case states that the following diagram should commute:
\newline
(Recall that $A = \{ \textcolor{blue}{a_0} < \textcolor{SkyBlue}{a_1} \} = \textbf{1}_\bot, B = \{ \textcolor{black}{b_0} < \textcolor{gray!60}{b_1}\} = \textbf{1}_\bot \text{ and } C = \{ \textcolor{Bittersweet}{c_0} < \textcolor{YellowOrange}{c_1}\} = \textbf{1}_\bot,$). 
(Note that our candidate for $(\textbf{1}_\bot)^{\textbf{1}_\bot}$ is purposefully left incomplete as only the first two levels are shown).\newline
The dotted lines in $\textbf{1}_\bot \times \textbf{1}_\bot$ show that $\textbf{1}_\bot \times_{\mathbb{FF}} \textbf{1}_\bot$ is a super-forest of $\textbf{1}_\bot \times_{\mathbb{FF}_2} \textbf{1}_\bot$. \newline
The nodes of $\textbf{1}_\bot \times \textbf{1}_\bot$ will be labeled with Greek letters and the usual coloring notation will apply to display projections.  
 \begin{figure}[h]
 	\raggedleft
 	\begin{tikzcd}[scale=0.4]
 		{\begin{tikzpicture}[scale=0.4]
 				\node (I) at (-18,4) {$e_{11}$};
 				\node (J) at (-16,4) {$e_{10}$};
 				\node (K) at (-14,4) {$e_9$};
 				\node (L) at (-12,4) {$e_8$};
 				\node (A) at (-8,0) {$e_0$};
 				\node (B) at (-10,4) {$e_3$};
 				\node (C) at (-8,4) {$e_2$};
 					\node (C1) at (-8,6) {.....};
 				\node (D) at (-6,4) {$e_1$};
 				\node (E) at (-4,4) {$e_7$};
 				\node (F) at (-2,4) {$e_6$};
 				\node (G) at (0,4) {$e_5$};
 				\node (H) at (2,4) {$e_4$};
 				\draw[line width=.03in] (A) -- (B);
 				\draw[line width=.03in] (A) -- (C);
 				\draw[line width=.03in] (A) -- (D);
 				\draw[line width=.03in] (A) -- (E);
 				\draw[line width=.03in] (A) -- (F);
 				\draw[line width=.03in] (A) -- (G);
 				\draw[line width=.03in] (A) -- (H);
 				\draw[line width=.03in] (A) -- (I);
 				\draw[line width=.03in] (A) -- (J);
 				\draw[line width=.03in] (A) -- (K);
 				\draw[line width=.03in] (A) -- (L);
 
 				\node (A') at (4,-3) {$e_0'$};	
 					\node (B') at (0,1) {$e_5'$};
 					\node (C') at (2,1) {$e_3'$};
 					\node (D') at (4,1) {$e_2'$};
 						\node (D1') at (4,3) {...};
 					\node (E') at (6,1) {$e_1'$};
 					\node (F') at (8,1) {$e_4'$};
 				\draw[line width=.03in] (A') -- (B');
 				\draw[line width=.03in] (A') -- (C');
 				\draw[line width=.03in] (A') -- (D');
 				\draw[line width=.03in] (A') -- (E');
 				\draw[line width=.03in] (A') -- (F');	
 		\end{tikzpicture} \times 
 	\begin{tikzpicture}[scale=0.4]
 				\node (A) at (0,0) {\textcolor{blue}{$a_0$}};
 				\node (B) at (0,3) {\textcolor{SkyBlue}{$a_1$}};
 				
 				\draw[SkyBlue, line width=.03in] (A) -- (B);
 				
 		\end{tikzpicture}} && 
 		\begin{tikzpicture}[scale=0.4]
 			\node (A) at (0,0) {$b_0$};
 			\node (B) at (0,3) {\textcolor{gray!60}{$b_1$}};
 			
 			\draw[gray!60, line width=.03in] (A) -- (B);
 			
 		\end{tikzpicture} \\
 		\\
 		{\begin{tikzpicture}[thick,scale=0.45, every node/.style={scale=0.9}]
	 				\node (a)  at (-0.5,-0.5) {\textcolor{YellowOrange}{\bullet}};
	 				\node (A)  at (0,0) {\beta};
	 				\node (a')  at (0.5,-0.5) {\textcolor{blue}{\bullet}};	
	 					\node (b)  at (-0.5,2.5) {\textcolor{YellowOrange}{\bullet}};
	 					\node (B)  at (0,3) {\epsilon};
	 					\node (b')  at (0.5,2.5) {\textcolor{SkyBlue}{\bullet}};
	 				\node (c)  at (2.5,-0.5) {\textcolor{YellowOrange}{\bullet}};
	 				\node (C)  at (3,0) {\gamma};
	 				\node (c')  at (3.5,-0.5) {\textcolor{SkyBlue}{\bullet}};
	 				\node (d)  at (5.7,-0.5) {\textcolor{Bittersweet}{\bullet}};
	 				\node (D)  at (6,0) {\delta};
	 				\node (d')  at (6.5,-0.5) {\textcolor{SkyBlue}{\bullet}};
	 					\node (e)  at (5.5,2.5) {\textcolor{YellowOrange}{\bullet}};	
	 					\node (E)  at (6,3) {\zeta};
	 					\node (e')  at (6.5,2.5) {\textcolor{SkyBlue}{\bullet}};
 				\node (f)  at (2.5,-3.5) {\textcolor{Bittersweet}{\bullet}};
 				\node (F)  at (3,-3) {\alpha};
 				\node (f')  at (3.5,-3.5) {\textcolor{blue}{\bullet}};
 				
 				\draw [dotted] [line width=.03in] (A) -- (B);
 				\draw [dotted] [line width=.03in] (D) -- (E);
 				\draw [ line width=.03in] (F) -- (A);
 				\draw [ line width=.03in] (F) -- (C);
 				\draw [ line width=.03in] (F) -- (D);
 				
 		\end{tikzpicture}}
 		\arrow["g"', from=3-1, to=1-3]
 		\arrow["{\hat{g}\times id_A}"', from=3-1, to=1-1]
 		\arrow["eval"', from=1-1, to=1-3]
 	\end{tikzcd}
 	\caption{UMP diagram for $A=B=C=\textbf{1}_\bot$.  }
 \end{figure}
 \newline
(From now on the candidate for the exponential object in $\mathbb{FF}_3$ will be denoted by $"B^A"$ or simply by $B^A$). 
\newpage
The map \emph{eval} can be fully described as we saw in the case of Bushes. However we will concentrate our attention on a specific part of its domain in $"(\textbf{1}_\bot)^{\textbf{1}_\bot}" \times \textbf{1}_\bot$ namely the first levels of $T_0 \times \textbf{1}_\bot$ and display only this part in detail leaving the rest purposefully incomplete.
\begin{figure}[h]
	\begin{tikzcd}
		\begin{tikzpicture}[scale=0.3]
				\node (H') at (-35,5) {...};
			\node (H) at (-34,3) {\textcolor{gray!60}{$e_3 a_0$}};
				\node (H1) at (-34,8) {\textcolor{gray!60}{$e_3 a_1$}};
			\node (G) at (-31,3) {$e_2 a_0$};
				\node (G1) at (-31,8) {$e_2 a_1$};
			\node (F) at (-28,3) {\textcolor{gray!60}{$e_1 a_0$}};
				\node (F1) at (-28,8) {\textcolor{gray!60}{$e_1 a_1$}};
			\node (A3) at (-25,3) {$e_7 a_0$};	
				\node (A3') at (-25,8) {$e_7 a_1$};
				\node (A3'') at (-24,5) {...};
			\node (A4) at (-22,3) {$e_4 a_0$};	
				\node (A4') at (-22,8) {\textcolor{gray!60}{$e_4 a_1$}};	
			\node (A) at (-18,-5) {$e_0 a_0$};
				\node (B') at (-19,2) {...};
			\node (B) at (-17,3) {$e_3 a_1$};
			\node (C) at (-14,3) {\textcolor{gray!60}{$e_2 a_1$}};
			\node (D) at (-11,3) {\textcolor{gray!60}{$e_1 a_1$}};
				\node (D') at (-11,5) {...};
			\node (A1) at (-4,3) {$e_4 a_1$};
				\node (A1') at (-8,2) {...};
			\node (A2) at (-8,3) {$e_7 a_1$};
			\node (E) at (3,2) {$e_0 a_1$};
				\node (E1) at (9,7) {\textcolor{gray!60}{$e_4 a_1$}};
					\node (E1') at (5,5) {...};
				\node (E2) at (5,7) {\textcolor{gray!60}{$e_7 a_1$}};
				\node (E3) at (2,7) {$e_1 a_1$};
				\node (E4) at (-1,7) {$e_2 a_1$};	
				\node (E5) at (-4,7) {$e_3 a_1$};
					\node (E5') at (-4,5) {...};		
			\draw[line width=.03in] (A) -- (B);
			\draw[gray!60, line width=.03in] (A) -- (C);
			\draw[gray!60, line width=.03in] (A) -- (D);
			\draw[line width=.03in] (A) -- (A1);
			\draw[line width=.03in] (A) -- (A2);
			\draw[line width=.03in] (A) -- (E);
				\draw[gray!60, line width=.03in] (E) -- (E1);
				\draw[gray!60, line width=.03in] (E) -- (E2);
				\draw[line width=.03in] (E) -- (E3);
				\draw[line width=.03in] (E) -- (E4);
				\draw[line width=.03in] (E) -- (E5);
			\draw[gray!60, line width=.03in] (A) -- (F);
				\draw[gray!60, line width=.03in] (F) -- (F1);
			\draw[line width=.03in] (A) -- (A3);
				\draw[line width=.03in] (A3) -- (A3');
			\draw[line width=.03in] (A) -- (A4);
				\draw[gray!60, line width=.03in] (A4) -- (A4');	
			\draw[line width=.03in] (A) -- (G);
				\draw[line width=.03in] (G) -- (G1);
			\draw[gray!60, line width=.03in] (A) -- (H);
				\draw[gray!60, line width=.03in] (H) -- (H1);
		
				\node (M') at (-2,-5) {...};
			\node (H') at (-3,-3) {};
			\node (G') at (-1,-3) {};
			\node (F') at (1,-3) {...};
			\node (A') at (9,-10) {$e_0'a_0$};
				\node (N') at (6,-5) {...};
			\node (B') at (7,-3) {};
			\node (C') at (9,-3) {...};
			\node (D') at (11,-3) {};
			\node (E') at (17,-3) {...};
			\draw[line width=.03in] (A') -- (H');
			\draw[gray!60, line width=.03in] (A') -- (G');
			\draw[gray!60, line width=.03in] (A') -- (F');
			\draw[line width=.03in] (A') -- (B');
			\draw[line width=.03in] (A') -- (C');
			\draw[gray!60, line width=.03in] (A') -- (D');
			\draw[gray!60, line width=.03in] (A') -- (E');
			
			
		\end{tikzpicture} \\
		\\
		\begin{tikzpicture}[scale=0.3]
			\node (A) at (0,0) {$b_0$};
			\node (B) at (0,4) {\textcolor{gray!60}{$b_1$}};
			
			\draw[gray!60, line width=.03in] (A) -- (B);
		\end{tikzpicture}
		\arrow["eval"', from=1-1, to=3-1]
	\end{tikzcd}
	\caption{The map $eval : B^A \times A \rightarrow B : (\textbf{1}_\bot)^{\textbf{1}_\bot}\times \textbf{1}_\bot \rightarrow \textbf{1}_\bot$.\newline The pre-image of $b_1$ is colored in gray and that of $b_0$ in black. \newline
	In this case the nodes of the product are labeled with $e_ia_j$ to specify the left and right projections to $e_i$ and $a_j$ respectively.}
\end{figure}

\newpage
Remember that, as in the previous case for \emph{bushes}, we would like for the exponential object to encode the maps $\{ g_j \}_{j=0}^{17}$. The first two levels of $(\textbf{1}_\bot)^{\textbf{1}_\bot}$ can thus be labeled in a similar fashion as before. 

\begin{figure}[h]
	\centering
	\begin{tikzpicture}[scale=0.45, every node/.style={scale=0.8}]
		\node (A) at (3,0) {$g_0$};
			\node (A0) at (3,-5) {\textcolor{black}{$\bullet$}};
			\node (B0) at (1,-3) {\textcolor{black}{$\bullet$}};
				\node (B0') at (1,-1) {\textcolor{black}{$\bullet$}};
			\node (C0) at (3,-3) {\textcolor{black}{$\bullet$}};
			\node (D0) at (5,-3) {\textcolor{black}{$\bullet$}};
				\node (D0') at (5,-1) {\textcolor{black}{$\bullet$}};
			\draw[line width=.02in] (A0) -- (B0);
				\draw[line width=.02in] (B0) -- (B0');
			\draw[line width=.02in] (A0) -- (C0);
			\draw[line width=.02in] (A0) -- (D0);
				\draw[line width=.02in] (D0) -- (D0');
		\node (B) at (-5,5) {$g_3$};
			\node (B') at (-6,4) {.....};
			\node (A3) at (-5,6) {\textcolor{black}{$\bullet$}};
				\node (B3) at (-7,8) {\textcolor{gray!60}{$\bullet$}};
					\node (B3') at (-7,10) {\textcolor{gray!60}{$\bullet$}};
				\node (C3) at (-5,8) {\textcolor{black}{$\bullet$}};
				\node (D3) at (-3,8) {\textcolor{black}{$\bullet$}};
						\node (D3') at (-3,10) {\textcolor{black}{$\bullet$}};
			\draw[gray!60, line width=.02in] (A3) -- (B3);
				\draw[gray!60, line width=.02in] (B3) -- (B3');
			\draw[line width=.02in] (A3) -- (C3);
			\draw[line width=.02in] (A3) -- (D3);
				\draw[line width=.02in] (D3) -- (D3');
		\node (C) at (0,5) {$g_2$};
			\node (A2) at (0,6) {\textcolor{black}{$\bullet$}};
				\node (B2) at (-2,8) {\textcolor{black}{$\bullet$}};
					\node (B2') at (-2,10) {\textcolor{black}{$\bullet$}};
				\node (C2) at (0,8) {\textcolor{gray!60}{$\bullet$}};
				\node (D2) at (2,8) {\textcolor{black}{$\bullet$}};
					\node (D2') at (2,10) {\textcolor{black}{$\bullet$}};
				\draw[line width=.02in] (A2) -- (B2);
					\draw[line width=.02in] (B2) -- (B2');
				\draw[gray!60, line width=.02in] (A2) -- (C2);
				\draw[line width=.02in] (A2) -- (D2);
					\draw[line width=.02in] (D2) -- (D2');
		\node (D) at (5,5) {$g_1$};
			\node (A1) at (5,6) {\textcolor{black}{$\bullet$}};
				\node (B1) at (3,8) {\textcolor{gray!60}{$\bullet$}};
					\node (B1') at (3,10) {\textcolor{gray!60}{$\bullet$}};
				\node (C1) at (5,8) {\textcolor{gray!60}{$\bullet$}};
				\node (D1) at (7,8) {\textcolor{black}{$\bullet$}};
					\node (D1') at (7,10) {\textcolor{black}{$\bullet$}};
				\draw[gray!60, line width=.02in] (A1) -- (B1);
					\draw[gray!60, line width=.02in] (B1) -- (B1');
				\draw[gray!60, line width=.02in] (A1) -- (C1);
				\draw[line width=.02in] (A1) -- (D1);
					\draw[line width=.02in] (D1) -- (D1');
				\draw[line width=.01in] (A) -- (B);
				\draw[line width=.01in] (A) -- (C);
				\draw[line width=.01in] (A) -- (D);
		\node (E) at (10,5) {$g_7$};
		\node (e) at (11,4) {.....};
			\node (A7) at (10,6) {\textcolor{black}{$\bullet$}};
				\node (B7) at (8,8) {\textcolor{black}{$\bullet$}};
					\node (B7') at (8,10) {\textcolor{black}{$\bullet$}};
				\node (C7) at (10,8) {\textcolor{black}{$\bullet$}};
				\node (D7) at (12,8) {\textcolor{black}{$\bullet$}};
					\node (D7') at (12,10) {\textcolor{gray!60}{$\bullet$}};
				\draw[line width=.02in] (A7) -- (B7);
					\draw[line width=.02in] (B7) -- (B7');
				\draw[line width=.02in] (A7) -- (C7);
				\draw[line width=.02in] (A7) -- (D7);
					\draw[gray!60, line width=.02in] (D7) -- (D7');
				\draw[line width=.01in] (A) -- (B);
				\draw[line width=.01in] (A) -- (C);
				\draw[line width=.01in] (A) -- (D);
				\draw[line width=.01in] (A) -- (E);
		\node (F) at (15,5) {$g_4$};
			\node (A4) at (15,6) {\textcolor{black}{$\bullet$}};
				\node (B4) at (13,8) {\textcolor{black}{$\bullet$}};
					\node (B4') at (13,10) {\textcolor{gray!60}{$\bullet$}};
				\node (C4) at (15,8) {\textcolor{black}{$\bullet$}};
				\node (D4) at (17,8) {\textcolor{black}{$\bullet$}};
					\node (D4') at (17,10) {\textcolor{gray!60}{$\bullet$}};
				\draw[line width=.02in] (A4) -- (B4);
					\draw[gray!60, line width=.02in] (B4) -- (B4');
				\draw[line width=.02in] (A4) -- (C4);
				\draw[line width=.02in] (A4) -- (D4);
					\draw[gray!60, line width=.02in] (D4) -- (D4');
		\draw[line width=.01in] (A) -- (B);
		\draw[line width=.01in] (A) -- (C);
		\draw[line width=.01in] (A) -- (D);
		\draw[line width=.01in] (A) -- (E);	
		\draw[line width=.01in] (A) -- (F);			
			
		\node (A') at (15,-13) {$g_{12}$};
			\node (A0') at (15,-17) {\textcolor{black}{$\bullet$}};
				\node (B0') at (13,-15) {\textcolor{black}{$\bullet$}};
					\node (B0'') at (13,-13) {\textcolor{black}{$\bullet$}};
				\node (C0') at (15,-15) {\textcolor{gray!60}{$\bullet$}};
				\node (D0') at (17,-15) {\textcolor{gray!60}{$\bullet$}};	
					\node (D0'') at (17,-13) {\textcolor{gray!60}{$\bullet$}};
			\draw[line width=.02in] (A0') -- (B0');
			\draw[line width=.02in] (B0') -- (B0'');
			\draw[gray!60, line width=.02in] (A0') -- (C0');
			\draw[gray!60, line width=.02in] (A0') -- (D0');
			\draw[gray!60, line width=.02in] (D0') -- (D0'');
		\node (B'') at (10,-9) {...};
		\node (B') at (10,-8) {$g_{16}$};
			\node (A1') at (10,-7) {\textcolor{black}{$\bullet$}};
				\node (B1') at (8,-5) {\textcolor{black}{$\bullet$}};
					\node (B1'') at (8,-3) {\textcolor{black}{$\bullet$}};
				\node (C1') at (10,-5) {\textcolor{black}{$\bullet$}};				
				\node (D1') at (12,-5) {\textcolor{gray!60}{$\bullet$}};
					\node (D1'') at (12,-3) {\textcolor{gray!60}{$\bullet$}};
			\draw[ line width=.02in] (A1') -- (B1');
				\draw[ line width=.02in] (B1') -- (B1'');
			\draw[ line width=.02in] (A1') -- (C1');
			\draw[gray!60, line width=.02in] (A1') -- (D1');
				\draw[gray!60, line width=.02in] (D1') -- (D1'');		
		\node (C') at (15,-8) {$g_{15}$};
			\node (A2') at (15,-7) {\textcolor{black}{$\bullet$}};
				\node (B2') at (13,-5) {\textcolor{gray!60}{$\bullet$}};
					\node (B2'') at (13,-3) {\textcolor{gray!60}{$\bullet$}};
				\node (C2') at (15,-5) {\textcolor{black}{$\bullet$}};
				\node (D2') at (17,-5) {\textcolor{gray!60}{$\bullet$}};
					\node (D2'') at (17,-3) {\textcolor{gray!60}{$\bullet$}};
				\draw[gray!60, line width=.02in] (A2') -- (B2');
					\draw[gray!60, line width=.02in] (B2') -- (B2'');
				\draw[line width=.02in] (A2') -- (C2');
				\draw[gray!60, line width=.02in] (A2') -- (D2');
					\draw[gray!60, line width=.02in] (D2') -- (D2'');		
		\node (D') at (20,-8) {$g_{14}$};
		\node (D'') at (20,-9) {...};
			\node (A3') at (20,-7) {\textcolor{black}{$\bullet$}};
				\node (B3') at (18,-5) {\textcolor{gray!60}{$\bullet$}};
						\node (B3'') at (18,-3) {\textcolor{gray!60}{$\bullet$}};
				\node (C3') at (20,-5) {\textcolor{gray!60}{$\bullet$}};
				\node (D3') at (22,-5) {\textcolor{gray!60}{$\bullet$}};
					\node (D3'') at (22,-3) {\textcolor{gray!60}{$\bullet$}};
				\draw[gray!60, line width=.02in] (A3') -- (B3');
					\draw[gray!60, line width=.02in] (B3') -- (B3'');
				\draw[gray!60, line width=.02in] (A3') -- (C3');
				\draw[gray!60, line width=.02in] (A3') -- (D3');
					\draw[gray!60, line width=.02in] (D3') -- (D3'');
		\draw[line width=.01in] (A') -- (B');
		\draw[line width=.01in] (A') -- (C');
		\draw[line width=.01in] (A') -- (D');
	\end{tikzpicture}
	\caption{The encoded maps  $\{ g_j \}_{j=0}^{17}$ in $(\textbf{1}_\bot) ^ {\textbf{1}_\bot}$. \newline
	Not all of them have been displayed.}
\end{figure}
\newpage
Now we take a new $C = (\textbf{1}_\bot)_\bot = \{c_0 < c_1 < c_2\}$ (which is just the old $C$ with a new \emph{child} $c_2$ of $c_1$) and consider maps from the new product $C \times A = (\textbf{1}_\bot)_\bot \times \textbf{1}_\bot$ to $B = \textbf{1}_\bot$.\newline
The following holds from Theorem~\ref{thm:prodffn}:
\begin{lem}
	For any $F,G \in \mathbb{FF}$ $F_\bot \times G$ is a super-forest of $F \times G$. 
\end{lem}
\begin{remark}
	We have seen that $\textbf{1}_\bot \times \textbf{1}_\bot \in \mathbb{FF}_3$ and so
	 $(\textbf{1}_\bot)_\bot \times_{\mathbb{FF}_3} \textbf{1}_\bot$ is a super-forest of $\textbf{1}_\bot \times \textbf{1}_\bot$.
\end{remark}
With this in mind, we examine the new UMP diagram in the $\mathbb{FF}_3$ environment
and introduce Greek letters to mark the nodes of $C \times A$ and left and right colored bullets
on the product to display the left and right projections. As before we use dotted lines to highlight the new branches of the super-forest.  
 \begin{figure}[h]
	\centering
	\begin{tikzcd}
		{\begin{tikzpicture}[scale=0.4]
				\node (I) at (-18,4) {$e_{11}$};
				\node (J) at (-16,4) {$e_{10}$};
				\node (K) at (-14,4) {$e_9$};
				\node (L) at (-12,4) {$e_8$};
				\node (A) at (-8,0) {$e_0$};
				\node (B) at (-10,4) {$e_3$};
				\node (C) at (-8,4) {$e_2$};
				\node (C1) at (-8,6) {.....};
				\node (D) at (-6,4) {$e_1$};
				\node (E) at (-4,4) {$e_7$};
				\node (F) at (-2,4) {$e_6$};
				\node (G) at (0,4) {$e_5$};
				\node (H) at (2,4) {$e_4$};
				\draw[line width=.03in] (A) -- (B);
				\draw[line width=.03in] (A) -- (C);
				\draw[line width=.03in] (A) -- (D);
				\draw[line width=.03in] (A) -- (E);
				\draw[line width=.03in] (A) -- (F);
				\draw[line width=.03in] (A) -- (G);
				\draw[line width=.03in] (A) -- (H);
				\draw[line width=.03in] (A) -- (I);
				\draw[line width=.03in] (A) -- (J);
				\draw[line width=.03in] (A) -- (K);
				\draw[line width=.03in] (A) -- (L);
				
				\node (A') at (4,-3) {$e_0'$};	
				\node (B') at (0,1) {$e_5'$};
				\node (C') at (2,1) {$e_3'$};
				\node (D') at (4,1) {$e_2'$};
				\node (D1') at (4,3) {...};
				\node (E') at (6,1) {$e_1'$};
				\node (F') at (8,1) {$e_4'$};
				\draw[line width=.03in] (A') -- (B');
				\draw[line width=.03in] (A') -- (C');
				\draw[line width=.03in] (A') -- (D');
				\draw[line width=.03in] (A') -- (E');
				\draw[line width=.03in] (A') -- (F');	
			\end{tikzpicture} \times \begin{tikzpicture}[scale=0.4]
				\node (A) at (0,0) {\textcolor{blue}{$a_0$}};
				\node (B) at (0,3) {\textcolor{SkyBlue}{$a_1$}};
				
				\draw[SkyBlue, line width=.03in] (A) -- (B);
				
		\end{tikzpicture}} && 
		\begin{tikzpicture}[scale=0.4]
			\node (A) at (0,0) {$b_0$};
			\node (B) at (0,3) {\textcolor{gray!60}{$b_1$}};
			
			\draw[gray!60, line width=.03in] (A) -- (B);
			
		\end{tikzpicture} \\
		\\
			{\begin{tikzpicture}[thick,scale=0.45, every node/.style={scale=0.9}]
			\node (f')  at (2.5,-3.5) {\textcolor{Bittersweet}{\bullet}};
			\node (F)  at (3,-3) {\alpha};	
			\node (f)  at (3.5,-3.5) {\textcolor{blue}{\bullet}};
			\node (a')  at (-0.7,-0.5) {\textcolor{YellowOrange}{\bullet}};
			\node (A)  at (0,0) {\beta\;};
			\node (a)  at (0.3,-0.5) {\textcolor{blue}{\bullet}};
			\node (b')  at (-0.5,2.5) {\textcolor{YellowOrange}{\bullet}};
			\node (B)  at (0,3) {\eta};
			\node (f)  at (0.5,2.5) {\textcolor{SkyBlue}{\bullet}};
			\node (b1')  at (-3.5,2.5) {\textcolor{Lavender}{\bullet}};	
			\node (B1)  at (-3,3) {\zeta};
			\node (b1)  at (-2.5,2.5) {\textcolor{SkyBlue}{\bullet}};
			\node (b2')  at (-6.5,2.5) {\textcolor{Lavender}{\bullet}};
			\node (B2)  at (-6,3) {\epsilon};
			\node (b2)  at (-5.5,2.5) {\textcolor{blue}{\bullet}};
			\node (c')  at (2.5,-0.5) {\textcolor{YellowOrange}{\bullet}};		
			\node (C)  at (3,0) {\gamma};
			\node (c)  at (3.5,-0.5) {\textcolor{SkyBlue}{\bullet}};
			\node (c1')  at (2.5,2.5) {\textcolor{Lavender}{\bullet}};
			\node (C1)  at (3,3) {\theta};
			\node (c1)  at (3.5,2.5) {\textcolor{SkyBlue}{\bullet}};
			\node (d')  at (5.8,-0.5) {\textcolor{Bittersweet}{\bullet}};
			\node (D)  at (6,0) {\;\;\delta};
			\node (d)  at (6.8,-0.5) {\textcolor{SkyBlue}{\bullet}};
			\node (e')  at (5.5,2.5) {\textcolor{YellowOrange}{\bullet}};
			\node (E)  at (6,3) {\iota};
			\node (e)  at (6.5,2.5) {\textcolor{SkyBlue}{\bullet}};

			\draw [ line width=.03in] (F) -- (A);
			\draw [line width=.03in] (A) -- (B);
			\draw [dotted, line width=.03in] (A) -- (B1);
			\draw [dotted, line width=.03in] (A) -- (B2);
			\draw [ line width=.03in] (F) -- (C);
			\draw [dotted, line width=.03in] (C) -- (C1);
			\draw [ line width=.03in] (F) -- (D);
			\draw [line width=.03in] (D) -- (E);

	\end{tikzpicture}}
		\arrow["g"', from=3-1, to=1-3]
		\arrow["{\hat{g}\times id_A}"', from=3-1, to=1-1]
		\arrow["eval"', from=1-1, to=1-3]
	\end{tikzcd}
	\caption{UMP diagram for $A=\{ \textcolor{blue}{a_0} < \textcolor{SkyBlue}{a_1}\}$, $B=\{ \textcolor{black}{b_0} < \textcolor{gray!60}{b_1}\}$ \newline
		and $C=\{\textcolor{Bittersweet}{c_0} < \textcolor{YellowOrange}{c_1} < \textcolor{Lavender}{c_2}\}$.  }
\end{figure}

\begin{remark} 
	$(\textbf{1}_\bot)_\bot \times \textbf{1}_\bot = ( (\textbf{1}_\bot \times \textbf{1}_\bot) + (\textbf{1}_\bot \times \textbf{1}) + ( (\textbf{1}_\bot)_\bot \times \textbf{1} ))_\bot = ( (\textbf{1}_\bot \times \textbf{1}_\bot) + \textbf{1}_\bot +  (\textbf{1}_\bot)_\bot)_\bot$. Recall also that $(\textbf{1}_\bot)_\bot \times_{\mathbb{FF}_3} \textbf{1}_\bot$ is the sub-forest nodes of height at most 3.
\end{remark}
\newpage
We focus our attention now on maps from $C=\{c_0 < c_1 < c_2\} \times_{\mathbb{FF}_3} A=\{a_0 < a_1\}$ to $B= \{b_0 < b_1\}$ i.e., $C \times_{\mathbb{FF}_3} A = (\textbf{1}_\bot)_\bot \times_{\mathbb{FF}_3} \textbf{1}_\bot$ to $B = \textbf{1}_\bot$  that send every node of $\{c_0 < c_1 \} \times_{\mathbb{FF}_3} \{a_0 < a_1\}$ i.e., $\textbf{1}_\bot \times \textbf{1}_\bot$ to the root $b_0$.
We call these \emph{h-maps}.
\begin{definition}[\emph{h-map}]
	An h-map is a map $h:  (\textbf{1}_\bot)_\bot \times \textbf{1}_\bot \rightarrow \textbf{1}_\bot$ such that the sub-forest $\textbf{1}_\bot \times \textbf{1}_\bot  \text{ is mapped to the root of } \textbf{1}_\bot$.
\end{definition}

How many of these maps are there?

\begin{lem}
	There are in total 8 h-maps in $\mathbb{FF}_3$. 
\end{lem}

This is due to the fact that, once all the nodes of $\textbf{1}_\bot \times \textbf{1}_\bot$ are sent to the root, the remaining 3 nodes (labeled from left to right as $\epsilon,\zeta,\theta$) have no constraints and are incomparable, and are thus free to go to any of the 2 nodes of $\textbf{1}_\bot$ so $2^3=8$.
\newline

For example, one of these h-maps $h_1$ is:

\begin{figure}[h]
	\centering
	\begin{tikzpicture}[scale=0.25]
		
		\node (A) at (0,0) {$\textcolor{black}{\bullet}$};
		\node (B) at (-4,5) {$\textcolor{black}{\bullet}$};
		\node (B') at (-4,10) {$\textcolor{black}{\bullet}$};
		\node (B'') at (-8,10) {$\textcolor{gray!60}{\bullet}$};
		\node (B''') at (-12,10) {$\textcolor{gray!60}{\bullet}$};
		\node (C) at (0,5) {$\textcolor{black}{\bullet}$};
		\node (C') at (0,10) {$\textcolor{gray!60}{\bullet}$};
		\node (D) at (4,5) {$\textcolor{black}{\bullet}$};
		\node (D') at (4,10) {$\textcolor{black}{\bullet}$};
		\node (d) at (5,2) {};
		
		\draw[black, line width=.03in] (A) -- (B);
		\draw[black, line width=.03in] (B) -- (B');
		\draw[gray!60, line width=.03in] (B) -- (B'');
		\draw[gray!60, line width=.03in] (B) -- (B''');
		\draw[black, line width=.03in] (A) -- (C);
		\draw[line width=.03in] (A) -- (D);
		\draw[line width=.03in] (D) -- (D');
		\draw[gray!60, line width=.03in] (C) -- (C');
		
		\node (e) at (17,2) {};
		\node (E) at (18,0) {$\textcolor{black}{\bullet}$};
		\node (F) at (18,5) {$\textcolor{gray!60}{\bullet}$};
		\draw[gray!60, line width=.03in] (E) -- (F);
		
		\draw[->, dotted, line width=.01in] (d) -- node[anchor=north] {$h_1$} (e);
	\end{tikzpicture}
	
\end{figure}

How might these h-maps $\{h_i\}_{i=0}^7 : C \times A \rightarrow B$ be \emph{encoded} in the exponential $(\textbf{1}_\bot)^\textbf{1}_\bot$?
\newline\newline
Lets take a look at $\{\hat{h_i}\}_{i=0}^7 : C \rightarrow B^A$. 
\newline
Since $(\{c_0 < c_1\} = \textbf{1}_\bot) \times (\textbf{1}_\bot = \{a_0 < a_1\})$ is mapped by any $h_i$ to $b_0$ and this map (recall $g_0$) was already encoded, it follows from our previous work that
any $\hat{h_i}$ must send the nodes $c_0,c_1$ to the root $e_0$ of $B^A$. \newline 
What distinguishes these maps is the image of the node $c_2$  which can be any $e_i$, $i \in \{1,..,11\}$.
For example we can determine that  $\hat{h_0}: c_2 \mapsto e_0$ if we take $h_0$ as the h-map which sends every node of $C$ to the root $b_0$. 
\newline
A simple deduction is made from the fact that there are 8 h-maps that need to be encoded and at the same time there are 11 branches from the root $e_0$:
\begin{remark}
There are at least two distinct level-one  $i,j \in \{1,..,11\}$ $e_i,e_j$ nodes $i \neq j $ of $B^A$ that encode the same h-map $h_i$. 
We observe that the nodes $e_4$ and $e_7$ provide an example of this. 
\end{remark}
In other words: There are \emph{too many} branches for the h-maps. There are at least two distinct branches that encode the same h-map.
\newline
$C$ is again represented as $\{\textcolor{Bittersweet}{c_0} < \textcolor{YellowOrange}{c_1} < \textcolor{Lavender}{c_2}\}$ and the images of the nodes are displayed as bullets of the same color on $B^A$.
\begin{figure}[h]
	\centering
	\begin{tikzpicture}[scale=0.45, every node/.style={scale=0.8}]
		\node (A) at (0,-2) {$h_0$};
			\node (A') at (1,-2.5) {\textcolor{Bittersweet}{$\bullet$}};
			\node (A'') at (1,-2) {\textcolor{YellowOrange}{$\bullet$}};
		\node (A0) at (0,-7) {\textcolor{black}{$\bullet$}};
			\node (B0) at (-2,-5) {\textcolor{black}{$\bullet$}};
				\node (B0') at (-2,-3) {\textcolor{black}{$\bullet$}};
				\node (B0'') at (-4,-3) {\textcolor{black}{$\bullet$}};
				\node (B0''') at (-6,-3) {\textcolor{black}{$\bullet$}};
			\node (C0) at (0,-5) {\textcolor{black}{$\bullet$}};
				\node (C0') at (0,-3) {\textcolor{black}{$\bullet$}};
			\node (D0) at (2,-5) {\textcolor{black}{$\bullet$}};
				\node (D0') at (2,-3) {\textcolor{black}{$\bullet$}};
			\draw[line width=.02in] (A0) -- (B0);
				\draw[line width=.02in] (B0) -- (B0');
				\draw[dotted, line width=.02in] (B0) -- (B0'');
				\draw[dotted, line width=.02in] (B0) -- (B0''');
			\draw[line width=.02in] (A0) -- (C0);
				\draw[dotted, line width=.02in] (C0) -- (C0');
			\draw[line width=.02in] (A0) -- (D0);
				\draw[line width=.02in] (D0) -- (D0');
		
		\node (L) at (-16,5) {...};
		
		\node (K) at (-13,5) {...};
		
		\node (J) at (-10,5) {...};
		
		\node (I) at (-7,5) {...};
		
		\node (B) at (-4,5) {$h_3$};
			\node (A3) at (-4,6) {\textcolor{black}{$\bullet$}};
				\node (B3) at (-6,8) {\textcolor{black}{$\bullet$}};
					\node (B3') at (-6,10) {\textcolor{black}{$\bullet$}};
					\node (B3'') at (-8,10) {\textcolor{black}{$\bullet$}};
					\node (B3''') at (-10,10) {\textcolor{gray!60}{$\bullet$}};
				\node (C3) at (-4,8) {\textcolor{black}{$\bullet$}};
					\node (C3') at (-4,10) {\textcolor{black}{$\bullet$}};
				\node (D3) at (-2,8) {\textcolor{black}{$\bullet$}};
					\node (D3') at (-2,10) {\textcolor{black}{$\bullet$}};
			\draw[line width=.02in] (A3) -- (B3);
					\draw[line width=.02in] (B3) -- (B3');
					\draw[line width=.02in] (B3) -- (B3'');
					\draw[dotted, gray!60, line width=.02in] (B3) -- (B3''');
				\draw[line width=.02in] (A3) -- (C3);
					\draw[line width=.02in] (C3) -- (C3');
				\draw[line width=.02in] (A3) -- (D3);
					\draw[line width=.02in] (D3) -- (D3');
		
		\node (C) at (-1,5) {...};
		
		\node (D) at (2,5) {...};
			
		\node (E) at (5,5) {$h_4$};
			\node (A7) at (5,6) {\textcolor{black}{$\bullet$}};
				\node (B7) at (3,8) {\textcolor{black}{$\bullet$}};
					\node (B7') at (3,10) {\textcolor{black}{$\bullet$}};
					\node (B7'') at (1,10) {\textcolor{black}{$\bullet$}};
					\node (B7''') at (-1,10) {\textcolor{black}{$\bullet$}};
				\node (C7) at (5,8) {\textcolor{black}{$\bullet$}};
						\node (C7') at (5,10) {\textcolor{gray!60}{$\bullet$}};
				\node (D7) at (7,8) {\textcolor{black}{$\bullet$}};
					\node (D7') at (7,10) {\textcolor{black}{$\bullet$}};
				\draw[line width=.02in] (A7) -- (B7);
					\draw[line width=.02in] (B7) -- (B7');
					\draw[dotted, line width=.02in] (B7) -- (B7'');
					\draw[dotted, line width=.02in] (B7) -- (B7''');
				\draw[line width=.02in] (A7) -- (C7);
					\draw[gray!60, dotted, line width=.02in] (C7) -- (C7');
				\draw[line width=.02in] (A7) -- (D7);
				\draw[line width=.02in] (D7) -- (D7');
				
		\node (H) at (9,5) {...};
		
		\node (G) at (12,5) {...};
		
		\node (F) at (15,5) {$h_4$};
			\node (F') at (16,5) {\textcolor{Lavender}{$\bullet$}};
			\node (A4) at (15,6) {\textcolor{black}{$\bullet$}};
				\node (B4) at (13,8) {\textcolor{black}{$\bullet$}};
					\node (B4') at (13,10) {\textcolor{black}{$\bullet$}};
					\node (B4'') at (11,10) {\textcolor{black}{$\bullet$}};
					\node (B4''') at (9,10) {\textcolor{black}{$\bullet$}};
				\node (C4) at (15,8) {\textcolor{black}{$\bullet$}};
					\node (C4') at (15,10) {\textcolor{gray!60}{$\bullet$}};
				\node (D4) at (17,8) {\textcolor{black}{$\bullet$}};
					\node (D4') at (17,10) {\textcolor{black}{$\bullet$}};
			\draw[line width=.02in] (A4) -- (B4);
				\draw[line width=.02in] (B4) -- (B4');
				\draw[dotted, line width=.02in] (B4) -- (B4'');
				\draw[dotted, line width=.02in] (B4) -- (B4''');
			\draw[line width=.02in] (A4) -- (C4);
				\draw[gray!60, dotted, line width=.02in] (C4) -- (C4');
			\draw[line width=.02in] (A4) -- (D4);
				\draw[line width=.02in] (D4) -- (D4');
			
			\draw[line width=.01in] (A) -- (B);
			\draw[line width=.01in] (A) -- (C);
			\draw[line width=.01in] (A) -- (D);
			\draw[line width=.01in] (A) -- (E);	
			\draw[line width=.01in] (A) -- (F);			
			\draw[line width=.01in] (A) -- (G);
			\draw[line width=.01in] (A) -- (H);
			\draw[line width=.01in] (A) -- (I);
			\draw[line width=.01in] (A) -- (J);
			\draw[line width=.01in] (A) -- (K);
			\draw[line width=.01in] (A) -- (L);
	\end{tikzpicture}
	\caption{Some of the encoded h-maps $\{\hat{h_i}\}_{i=0}^7 $ in $(\textbf{1}_\bot) ^ {\textbf{1}_\bot}$.}
\end{figure}

%\begin{figure}[h]
%	\centering
%	\begin{tikzcd}
%			\begin{tikzpicture}[scale=0.3, every node/.style={scale=0.8}]
%			\node (A) at (0,-3) {\textcolor{Bittersweet}{\bullet}};
%			\node (B) at (0,0) {\textcolor{YellowOrange}{\bullet}};
%			\node (C) at (0,3) {\textcolor{Lavender}{\bullet}};
%			\draw[line width=.01in] (A) -- (B);
%			\draw[line width=.01in] (B) -- (C);
%		\end{tikzpicture} && 	
%		\begin{tikzpicture}[scale=0.3, every node/.style={scale=0.8}]
%			\node (A) at (0,-2) {e_0};
%			\node (A') at (1,-2.5) {\textcolor{Bittersweet}{\bullet}};
%			\node (A'') at (1,-2) {\textcolor{YellowOrange}{\bullet}};
%			\node (L) at (-16,5) {e_{11}};
%			\node (K) at (-13,5) {e_{10}};
%			\node (J) at (-10,5) {e_9};
%			\node (I) at (-7,5) {e_8};
%			\node (B) at (-4,5) {e_3};
%			\node (C) at (-1,5) {e_2};
%			\node (c) at (-1,6.5) {.......};
%			\node (D) at (2,5) {e_1};
%			\node (E) at (5,5) {e_7};
%				\node (E') at (6,5) {\textcolor{Lavender}{\bullet}};
%			\node (H) at (9,5) {e_6};
%			\node (G) at (12,5) {e_5};
%			\node (F) at (15,5) {e_4};
%			\draw[line width=.01in] (A) -- (B);
%			\draw[line width=.01in] (A) -- (C);
%			\draw[line width=.01in] (A) -- (D);
%			\draw[line width=.01in] (A) -- (E);	
%			\draw[line width=.01in] (A) -- (F);			
%			\draw[line width=.01in] (A) -- (G);
%			\draw[line width=.01in] (A) -- (H);
%			\draw[line width=.01in] (A) -- (I);
%			\draw[line width=.01in] (A) -- (J);
%			\draw[line width=.01in] (A) -- (K);
%			\draw[line width=.01in] (A) -- (L);
%		\end{tikzpicture}
%		\arrow["{\hat{h}_j}",dotted, from=1-1, to=1-3]
%	\end{tikzcd}
%	\caption{}
%\end{figure}

What this entails is the following:
\begin{thm}\label{thm:ff3nottopos}
	$\mathbb{FF}_3$ is \emph{not} a topos.
\end{thm}
\begin{proof}
	Let us assume that $\mathbb{FF}_3$ is in fact a topos and admits exponential objects $F^G$ for all \emph{finite forests} $F,G$. \newline
	The example we just presented is evidence of the lack of uniqueness of adjoint h-maps $\hat{h} : (\textbf{1}_\bot)_\bot \rightarrow (\textbf{1}_\bot) ^ {\textbf{1}_\bot}$ that \emph{encode} $h :  (\textbf{1}_\bot)_\bot \times_{\mathbb{FF}_3} \textbf{1}_\bot \rightarrow \textbf{1}_\bot$.
	\newline
	Thus $(\textbf{1}_\bot) ^ {\textbf{1}_\bot}$ does not satisfy the UMP for the h-maps $h_i$.
	\newline
	Thus $\mathbb{FF}_3$ does not admit the exponential object $(\textbf{1}_\bot) ^ {\textbf{1}_\bot}$ and we conclude that it cannot be a topos.
\end{proof}
So there is a \emph{uniqueness} problem for exponentiation in $\mathbb{FF}_3$.
\newline
However $(\textbf{1}_\bot)_\bot \times_{\mathbb{FF}_3} \textbf{1}_\bot$ is a proper sub-forest of $(\textbf{1}_\bot)_\bot \times_{\mathbb{FF}} \textbf{1}_\bot$. 
\newline
How does the situation change if we consider $\mathbb{FF}_k$ with $k>3$ and $\mathbb{FF}$ in general?



\newpage
\subsection{\hl{Counter-example for $\mathbb{FF}_{k \geq 3}$}}

If we move on to $\mathbb{FF}_4$ or any $\mathbb{FF}_{k>3}$ we get the full product, not just a sub-forest of, $C \times A= (\textbf{1}_\bot)_\bot \times \textbf{1}_\bot$ given by  $( (\textbf{1}_\bot \times \textbf{1}_\bot) + \textbf{1}_\bot +  (\textbf{1}_\bot)_\bot)_\bot$.
\newline
As before we would like for the following UMP diagram to commute, use Greek letters to mark the nodes of $C \times A$ and colored bullets on the product for the projections.

\begin{figure}[h]
	\centering
	\begin{tikzcd}
		{\begin{tikzpicture}[scale=0.4]
				\node (I) at (-18,4) {$e_{11}$};
				\node (J) at (-16,4) {$e_{10}$};
				\node (K) at (-14,4) {$e_9$};
				\node (L) at (-12,4) {$e_8$};
				\node (A) at (-8,0) {$e_0$};
				\node (B) at (-10,4) {$e_3$};
				\node (C) at (-8,4) {$e_2$};
				\node (C1) at (-8,6) {.....};
				\node (D) at (-6,4) {$e_1$};
				\node (E) at (-4,4) {$e_7$};
				\node (F) at (-2,4) {$e_6$};
				\node (G) at (0,4) {$e_5$};
				\node (H) at (2,4) {$e_4$};
				\draw[line width=.03in] (A) -- (B);
				\draw[line width=.03in] (A) -- (C);
				\draw[line width=.03in] (A) -- (D);
				\draw[line width=.03in] (A) -- (E);
				\draw[line width=.03in] (A) -- (F);
				\draw[line width=.03in] (A) -- (G);
				\draw[line width=.03in] (A) -- (H);
				\draw[line width=.03in] (A) -- (I);
				\draw[line width=.03in] (A) -- (J);
				\draw[line width=.03in] (A) -- (K);
				\draw[line width=.03in] (A) -- (L);
				
				\node (A') at (4,-3) {$e_0'$};	
				\node (B') at (0,1) {$e_5'$};
				\node (C') at (2,1) {$e_3'$};
				\node (D') at (4,1) {$e_2'$};
				\node (D1') at (4,3) {...};
				\node (E') at (6,1) {$e_1'$};
				\node (F') at (8,1) {$e_4'$};
				\draw[line width=.03in] (A') -- (B');
				\draw[line width=.03in] (A') -- (C');
				\draw[line width=.03in] (A') -- (D');
				\draw[line width=.03in] (A') -- (E');
				\draw[line width=.03in] (A') -- (F');	
			\end{tikzpicture} \times \begin{tikzpicture}[scale=0.4]
				\node (A) at (0,0) {\textcolor{blue}{$a_0$}};
				\node (B) at (0,3) {\textcolor{SkyBlue}{$a_1$}};
				
				\draw[SkyBlue, line width=.03in] (A) -- (B);
				
		\end{tikzpicture}} && 
		\begin{tikzpicture}[scale=0.4]
			\node (A) at (0,0) {$b_0$};
			\node (B) at (0,3) {\textcolor{gray!60}{$b_1$}};
			
			\draw[gray!60, line width=.03in] (A) -- (B);
			
		\end{tikzpicture} \\
		\\
		{\begin{tikzpicture}[thick,scale=0.5, every node/.style={scale=1.1}]
				\node (f')  at (2.5,-3.5) {\textcolor{Bittersweet}{\bullet}};
				\node (F)  at (3,-3) {\alpha};	
				\node (f)  at (3.5,-3.5) {\textcolor{blue}{\bullet}};
					\node (a')  at (-0.7,-0.5) {\textcolor{YellowOrange}{\bullet}};
					\node (A)  at (0,0) {\beta\;};
					\node (a)  at (0.3,-0.5) {\textcolor{blue}{\bullet}};
						\node (b')  at (-0.5,2.5) {\textcolor{YellowOrange}{\bullet}};
						\node (B)  at (0,3) {\eta};
						\node (f)  at (0.5,2.5) {\textcolor{SkyBlue}{\bullet}};
							\node (b'')  at (-0.5,5.5) {\textcolor{Lavender}{\bullet}};
							\node (B')  at (0,6) {\lambda};
							\node (b')  at (0.5,5.5) {\textcolor{SkyBlue}{\bullet}};
						\node (b1')  at (-3.5,2.5) {\textcolor{Lavender}{\bullet}};	
						\node (B1)  at (-3,3) {\zeta};
						\node (b1)  at (-2.5,2.5) {\textcolor{SkyBlue}{\bullet}};
						\node (b2')  at (-6.5,2.5) {\textcolor{Lavender}{\bullet}};
						\node (B2)  at (-6,3) {\epsilon};
						\node (b2)  at (-5.5,2.5) {\textcolor{blue}{\bullet}};
							\node (b2''')  at (-6.5,5.5) {\textcolor{Lavender}{\bullet}};
							\node (B2')  at (-6,6) {\kappa};
							\node (b2'')  at (-5.5,5.5) {\textcolor{SkyBlue}{\bullet}};
					\node (c')  at (2.5,-0.5) {\textcolor{YellowOrange}{\bullet}};		
					\node (C)  at (3,0) {\gamma};
					\node (c)  at (3.5,-0.5) {\textcolor{SkyBlue}{\bullet}};
						\node (c1')  at (2.5,2.5) {\textcolor{Lavender}{\bullet}};
						\node (C1)  at (3,3) {\theta};
						\node (c1)  at (3.5,2.5) {\textcolor{SkyBlue}{\bullet}};
					\node (d')  at (5.8,-0.5) {\textcolor{Bittersweet}{\bullet}};
					\node (D)  at (6,0) {\;\;\delta};
					\node (d)  at (6.8,-0.5) {\textcolor{SkyBlue}{\bullet}};
						\node (e')  at (5.5,2.5) {\textcolor{YellowOrange}{\bullet}};
						\node (E)  at (6,3) {\iota};
						\node (e)  at (6.5,2.5) {\textcolor{SkyBlue}{\bullet}};
							\node (e''')  at (5.5,5.5) {\textcolor{Lavender}{\bullet}};
							\node (E')  at (6,6) {\mu};
							\node (e'')  at (6.5,5.5) {\textcolor{SkyBlue}{\bullet}};
				\draw [ line width=.03in] (F) -- (A);
					\draw [line width=.03in] (A) -- (B);
						\draw [dotted, line width=.03in] (B) -- (B');
					\draw [dotted, line width=.03in] (A) -- (B1);
					\draw [dotted, line width=.03in] (A) -- (B2);
						\draw [dotted, line width=.03in] (B2) -- (B2');
				\draw [ line width=.03in] (F) -- (C);
					\draw [dotted, line width=.03in] (C) -- (C1);
				\draw [ line width=.03in] (F) -- (D);
					\draw [line width=.03in] (D) -- (E);
					\draw [dotted, line width=.03in] (E) -- (E');
	\end{tikzpicture}}
		\arrow["g"', from=3-1, to=1-3]
		\arrow["{\hat{g}\times id_A}"', from=3-1, to=1-1]
		\arrow["eval"', from=1-1, to=1-3]
	\end{tikzcd}
	\caption{UMP diagram for $A=\{ \textcolor{blue}{a_0} < \textcolor{SkyBlue}{a_1}\}$, $B=\{ \textcolor{black}{b_0} < \textcolor{gray!60}{b_1}\}$ and $C=\{\textcolor{Bittersweet}{c_0} < \textcolor{YellowOrange}{c_1} < \textcolor{Lavender}{c_2}\}$.   }
\end{figure}

As before we focus on \emph{h-maps} from $C \times A = (\textbf{1}_\bot)_\bot \times\textbf{1}_\bot$ to $B = \textbf{1}_\bot$.
	For example, one of these h-maps $h_1$ is:
\begin{figure}[h]
	\centering
	\begin{tikzpicture}[scale=0.25]
		
		\node (A) at (0,0) {$\textcolor{black}{\bullet}$};
		\node (B) at (-4,5) {$\textcolor{black}{\bullet}$};
		\node (B') at (-4,10) {$\textcolor{black}{\bullet}$};
		\node (B'''') at (-4,15) {$\textcolor{gray!60}{\bullet}$};
		\node (B'') at (-8,10) {$\textcolor{gray!60}{\bullet}$};
		\node (B''') at (-12,10) {$\textcolor{gray!60}{\bullet}$};
		\node (B''''') at (-12,15) {$\textcolor{gray!60}{\bullet}$};
		\node (C) at (0,5) {$\textcolor{black}{\bullet}$};
		\node (C') at (0,10) {$\textcolor{gray!60}{\bullet}$};
		\node (D) at (4,5) {$\textcolor{black}{\bullet}$};
		\node (D') at (4,10) {$\textcolor{black}{\bullet}$};
		\node (D'') at (4,15) {$\textcolor{gray!60}{\bullet}$};
		\node (d) at (5,2) {};
		
		\draw[black, line width=.03in] (A) -- (B);
		\draw[black, line width=.03in] (B) -- (B');
		\draw[gray!60, line width=.03in] (B') -- (B'''');
		\draw[gray!60, line width=.03in] (B) -- (B'');
		\draw[gray!60, line width=.03in] (B) -- (B''');
		\draw[gray!60, line width=.03in] (B''') -- (B''''');
		\draw[black, line width=.03in] (A) -- (C);
		\draw[line width=.03in] (A) -- (D);
		\draw[line width=.03in] (D) -- (D');
		\draw[gray!60, line width=.03in] (D') -- (D'');
		\draw[gray!60, line width=.03in] (C) -- (C');
		
		\node (e) at (17,2) {};
		\node (E) at (18,0) {$\textcolor{black}{\bullet}$};
		\node (F) at (18,5) {$\textcolor{gray!60}{\bullet}$};
		\draw[gray!60, line width=.03in] (E) -- (F);
		
		\draw[->, dotted, line width=.01in] (d) -- node[anchor=north] {$h_1$} (e);
	\end{tikzpicture}
	
\end{figure}

We ask again: How many of these maps are there?
\begin{lem}
There are in total 48 \emph{h-maps} in $\mathbb{FF}_{k > 3}$. 
\end{lem}

This can be computed by observing that compared to last time in $\mathbb{FF}_3$ we now have 3 new nodes to add to $C \times A$ so that $48=3 \cdot 2^4$.
\newline
The first one (reading the forest from left to right) $\kappa$ is the child of $\epsilon$, the second one $\lambda$ is the child of $\eta$ and the third one $\mu$ is the child of $\iota$.
If we observe the dotted branches we now have (still from left to right) one branch of the form $(\textbf{1}_\bot)_\bot$ and four of the form $(\textbf{1}_\bot)$. 
\newline
Recalling that $| Hom((\textbf{1}_\bot)_\bot, \textbf{1}_\bot) | =3$ and $| Hom(\textbf{1}_\bot, \textbf{1}_\bot) | =2$
a straightforward combinatorial argument tells us that the total number of \emph{h-maps} is $3 \cdot 2^4$.
\newline
 We ask again:
 How might these h-maps $\{h_i\}_{i=0}^{47} : C \times A \rightarrow B$ be \emph{encoded} in the exponential $(\textbf{1}_\bot)^{\textbf{1}_\bot}$?  \newline\newline
Lets take a look then at $\{\hat{h_i}\}_{i=0}^{47} : C \rightarrow B^A$. 
 \newline
The same argument we used last time tells us that any $\hat{h_i}$ must send the nodes $c_0,c_1$ to the root $e_0$ of $B^A$ and these maps should be distinguished by the image of the node $c_2$ which can be any $e_i$, $i \in \{1,..,11\}$.
 \newline
  Using the same example, we can determine that  $\hat{h_0}: c_2 \mapsto e_0$ if we take $h_0$ as the h-map which sends every node of $C$ to the root $b_0$.
  \newline
  In this case however there are 48 maps that need to be encoded with the same 11 branches from the root $e_0$.
   \newline
  This means that there are too \emph{few} branches to encode all the possible h-maps.
  
 \begin{lem}
 	There is at least one h-map that cannot be encoded in any level-one node $e_i$, $i \in \{0,..,11\}$ of $(\textbf{1}_\bot)^{\textbf{1}_\bot}$.
 \end{lem}
 The h-map $?h$ displayed in figure~\ref{fig:counterex} is an example of this.
  \newline
Let $\{\hat{h_i}\}_{i=1}^{11}$ be the maps from $C$ to $B^A$ such that $\hat{h_i}: \{c_0,c_1\}\mapsto e_0$ and $c_2 \mapsto e_i$. 
\newline
Take for example the h-map $h_7$ in display.
\newline
It should be in $1:1$ correspondence with $\hat{h_7}$.\newline 
Note that if we now focus our attention on $\{\beta < \eta < \lambda, \beta < \zeta,  \beta < \epsilon < \kappa\}$ i.e., the up-set $\uparrow \beta$ corresponding to $\{c_1 < c_2\} \times \{a_0 < a_1\}  $, the restriction of $\hat{h_7}$ already encodes a map which we called $g_7 :  \{c_1 < c_2\} \times \{a_0 < a_1\} \rightarrow \{b_0 < b_1\}$. \newline 
The yellow boxes in figure~\ref{fig:counterex} are meant to highlight this map. \newline	
If we construct a new h-map $?h$ which behaves exactly as $g_7$ on $\{c_1 < c_2\} \times \{a_0 < a_1\}$ but behaves differently on the two \emph{leaf} nodes $\theta, \mu$, then $?h \neq h_7$ and thus cannot be encoded by $\hat{h_7}$. \newline
Furthermore, $?h$ cannot be encoded by $\hat{h_i}$ with $i \neq 7$ since any of these $h_i$ when restricted to $\{c_1 < c_2\} \times \{a_0 < a_1\}$ behaves as $g_i$ with $g_i \neq g_7$.
\newline\newline
 We are thus in a bind and cannot find a corresponding $\hat{?h}$ for $?h$...
 \begin{figure}[h]
 	\centering
 	\begin{tikzpicture}[scale=0.45, every node/.style={scale=0.8}]
 		\node (A) at (0,-2) {$h_0$};
 		\node (A') at (1,-2.5) {\textcolor{Bittersweet}{$\bullet$}};
 		\node (A'') at (1,-2) {\textcolor{YellowOrange}{$\bullet$}};
	 	\node (A0) at (0,-9) {\textcolor{black}{$\bullet$}};
	 		\node (B0) at (-2,-7) {\textcolor{black}{$\bullet$}};
	 			\node (B0') at (-2,-5) {\textcolor{black}{$\bullet$}};
	 				\node (B0''''') at (-2,-3) {\textcolor{black}{$\bullet$}};
	 			\node (B0'') at (-4,-5) {\textcolor{black}{$\bullet$}};
	 			\node (B0''') at (-6,-5) {\textcolor{black}{$\bullet$}};
	 				\node (B0'''') at (-6,-3) {\textcolor{black}{$\bullet$}};
	 		\node (C0) at (0,-7) {\textcolor{black}{$\bullet$}};
	 			\node (C0') at (0,-5) {\textcolor{black}{$\bullet$}};
	 		\node (D0) at (2,-7) {\textcolor{black}{$\bullet$}};
	 			\node (D0') at (2,-5) {\textcolor{black}{$\bullet$}};
	 				\node (D0'') at (2,-3) {\textcolor{black}{$\bullet$}};
	 		\draw[line width=.02in] (A0) -- (B0);
		 		\draw[line width=.02in] (B0) -- (B0');
		 			\draw[dotted, line width=.02in] (B0') -- (B0''''');
		 		\draw[dotted, line width=.02in] (B0) -- (B0'');
		 		\draw[dotted, line width=.02in] (B0) -- (B0''');
		 			\draw[dotted, line width=.02in] (B0''') -- (B0'''');
		 	\draw[line width=.02in] (A0) -- (C0);
	 			\draw[dotted, line width=.02in] (C0) -- (C0');
	 		\draw[line width=.02in] (A0) -- (D0);
	 			\draw[line width=.02in] (D0) -- (D0');
					\draw[dotted, line width=.02in] (D0') -- (D0'');
		 		
 		\node (L) at (-16,5) {...};
 		
 		\node (K) at (-13,5) {...};
 		
 		\node (J) at (-10,5) {...};
 		
 		\node (I) at (-7,5) {...};
 		
 		\node (B) at (-4,5) {...};
 			\node (A3) at (-5,6) {\textcolor{black}{$\bullet$}};
	 		\node (A3') at (-4,6) {?h};
		 		\node (B3) at (-7,8) {\textcolor{black}{$\bullet$}};
		 			\node (B3') at (-7,10) {\textcolor{black}{$\bullet$}};
		 				\node (B3'''') at (-7,12) {\textcolor{gray!60}{$\bullet$}};
		 			\node (B3'') at (-9,10) {\textcolor{black}{$\bullet$}};
		 			\node (B3''') at (-11,10) {\textcolor{black}{$\bullet$}};
		 				\node (B3''''') at (-11,12) {\textcolor{black}{$\bullet$}};
		 		\node (C3) at (-5,8) {\textcolor{black}{$\bullet$}};
		 			\node (C3') at (-5,10) {\textcolor{black}{$\bullet$}};
		 		\node (D3) at (-3,8) {\textcolor{black}{$\bullet$}};
		 			\node (D3') at (-3,10) {\textcolor{black}{$\bullet$}};
		 				\node (D3'') at (-3,12) {\textcolor{gray!60}{$\bullet$}};
	 		\draw[line width=.02in] (A3) -- (B3);
	 			\draw[line width=.02in] (B3) -- (B3');
	 				\draw[dotted, gray!60, line width=.02in] (B3') -- (B3'''');
	 			\draw[dotted, line width=.02in] (B3) -- (B3'');
	 			\draw[dotted, line width=.02in] (B3) -- (B3''');
	 					\draw[dotted, line width=.02in] (B3''') -- (B3''''');
	 		\draw[line width=.02in] (A3) -- (C3);
	 			\draw[dotted, line width=.02in] (C3) -- (C3');
	 		\draw[line width=.02in] (A3) -- (D3);
	 			\draw[line width=.02in] (D3) -- (D3');
 					\draw[dotted, gray!60, line width=.02in] (D3') -- (D3'');
 			\draw[yellow, line width=.01in] (-6,7) -- (-6,13);
 			\draw[yellow, line width=.01in] (-6,13) -- (-12,13);
 			\draw[yellow, line width=.01in] (-12,13) -- (-12,7);
 			\draw[yellow, line width=.01in] (-12,7) -- (-6,7);
 			
 		\node (C) at (-1,5) {...};
 		
 		\node (D) at (2,5) {...};
 		
 		\node (E) at (5,5) {$h_7$};
 		 \node (E') at (6,5) {\textcolor{Lavender}{$\bullet$}};
 		\node (A7) at (5,6) {\textcolor{black}{$\bullet$}};
	 		\node (B7) at (3,8) {\textcolor{black}{$\bullet$}};
	 			\node (B7') at (3,10) {\textcolor{black}{$\bullet$}};
	 				\node (B7'''') at (3,12) {\textcolor{gray!60}{$\bullet$}};
	 			\node (B7'') at (1,10) {\textcolor{black}{$\bullet$}};
	 			\node (B7''') at (-1,10) {\textcolor{black}{$\bullet$}};
	 				\node (B7''''') at (-1,12) {\textcolor{black}{$\bullet$}};
	 		\node (C7) at (5,8) {\textcolor{black}{$\bullet$}};
	 			\node (C7') at (5,10) {\textcolor{gray!60}{$\bullet$}};
	 		\node (D7) at (7,8) {\textcolor{black}{$\bullet$}};
	 			\node (D7') at (7,10) {\textcolor{black}{$\bullet$}};
	 				\node (D7'') at (7,12) {\textcolor{gray!60}{$\bullet$}};
	 		\draw[line width=.02in] (A7) -- (B7);
	 			\draw[line width=.02in] (B7) -- (B7');
	 				\draw[dotted, gray!60, line width=.02in] (B7') -- (B7'''');
	 			\draw[dotted, line width=.02in] (B7) -- (B7'');
	 			\draw[dotted, line width=.02in] (B7) -- (B7''');
	 				\draw[dotted, line width=.02in] (B7''') -- (B7''''');
	 		\draw[line width=.02in] (A7) -- (C7);
	 			\draw[gray!60, dotted, line width=.02in] (C7) -- (C7');
	 		\draw[line width=.02in] (A7) -- (D7);
	 			\draw[line width=.02in] (D7) -- (D7');
	 				\draw[dotted, gray!60, line width=.02in] (D7') -- (D7'');
	 		\draw[yellow, line width=.01in] (4,7) -- (4,13);
	 		\draw[yellow, line width=.01in] (4,13) -- (-2,13);
	 		\draw[yellow, line width=.01in] (-2,13) -- (-2,7);
	 		\draw[yellow, line width=.01in] (-2,7) -- (4,7);
	 		
 		\node (H) at (9,5) {...};
 		
 		\node (G) at (12,5) {...};
 		
 		\node (F) at (15,5) {$h_4$};
	 		\node (A4) at (15,6) {\textcolor{black}{$\bullet$}};
		 		\node (B4) at (13,8) {\textcolor{black}{$\bullet$}};
		 			\node (B4') at (13,10) {\textcolor{black}{$\bullet$}};
		 					\node (B4'''') at (13,12) {\textcolor{gray!60}{$\bullet$}};
		 		\node (B4'') at (11,10) {\textcolor{black}{$\bullet$}};
		 		\node (B4''') at (9,10) {\textcolor{black}{$\bullet$}};
		 			\node (B4''''') at (9,12) {\textcolor{gray!60}{$\bullet$}};
		 		\node (C4) at (15,8) {\textcolor{black}{$\bullet$}};
		 			\node (C4') at (15,10) {\textcolor{gray!60}{$\bullet$}};
		 		\node (D4) at (17,8) {\textcolor{black}{$\bullet$}};
		 			\node (D4') at (17,10) {\textcolor{black}{$\bullet$}};
		 				\node (D4'') at (17,12) {\textcolor{gray!60}{$\bullet$}};
	 		\draw[line width=.02in] (A4) -- (B4);
	 			\draw[line width=.02in] (B4) -- (B4');
	 				\draw[gray!60, dotted, line width=.02in] (B4') -- (B4'''');
	 			\draw[dotted, line width=.02in] (B4) -- (B4'');
	 			\draw[dotted, line width=.02in] (B4) -- (B4''');
	 				\draw[gray!60, dotted, line width=.02in] (B4''') -- (B4''''');
	 		\draw[line width=.02in] (A4) -- (C4);
	 			\draw[gray!60, dotted, line width=.02in] (C4) -- (C4');
	 		\draw[line width=.02in] (A4) -- (D4);
	 			\draw[line width=.02in] (D4) -- (D4');
 					\draw[gray!60, dotted, line width=.02in] (D4') -- (D4'');
 		\draw[line width=.01in] (A) -- (B);
 		\draw[line width=.01in] (A) -- (C);
 		\draw[line width=.01in] (A) -- (D);
 		\draw[line width=.01in] (A) -- (E);	
 		\draw[line width=.01in] (A) -- (F);			
 		\draw[line width=.01in] (A) -- (G);
 		\draw[line width=.01in] (A) -- (H);
 		\draw[line width=.01in] (A) -- (I);
 		\draw[line width=.01in] (A) -- (J);
 		\draw[line width=.01in] (A) -- (K);
 		\draw[line width=.01in] (A) -- (L);
 		
 	\end{tikzpicture}
 	\caption{Some of the encoded h-maps $\{\hat{h_i}\}_{i=0}^{47} $ in $(\textbf{1}_\bot) ^ {\textbf{1}_\bot}$.}
 \label{fig:counterex}
 \end{figure}
 \newpage
 What this entails is the following:
 \begin{thm}\label{thm:ffknotatopos}
 	$\mathbb{FF}_{k > 3}$ for $k>3$ is \emph{not} a topos.
 \end{thm}
 \begin{proof}
 	Let us assume that $\mathbb{FF}_k$ is a topos for some $k > 3$ and admits exponential objects $F^G$ for all \emph{finite forests} $F,G$. 
\newline
 	The example we presented is evidence of the lack of existence of an adjoint h-map $\hat{h} : (\textbf{1}_\bot)_\bot \rightarrow (\textbf{1}_\bot) ^ {\textbf{1}_\bot}$ that \emph{encodes} $h :  (\textbf{1}_\bot)_\bot \times_{\mathbb{FF}_3} \textbf{1}_\bot \rightarrow \textbf{1}_\bot$. 
 	\newline
 	Thus $(\textbf{1}_\bot) ^ {\textbf{1}_\bot}$ does not satisfy the UMP for the h-maps $h_i$.
  \newline
 	Thus $\mathbb{FF}_k$ does not admit the exponential object $(\textbf{1}_\bot) ^ {\textbf{1}_\bot}$ and we conclude that it cannot be a topos.
 \end{proof}
 
 Summing up we have \emph{uniqueness} (Theorem~\ref{thm:ff3nottopos}) and \emph{existence} (Theorem~\ref{thm:ffknotatopos}) problems for exponentiation in $\mathbb{FF}_3$ and $\mathbb{FF}_{k> 3}$ respectively. The following results are now immediate:
 
 \begin{cor}
 	$\mathbb{FF}_{k\geq3}$ is not a topos.
 \end{cor}
 
 \begin{cor}
 	The only topoi in $\mathbb{FF}_*$ are $\mathbb{FF}_1 $ and $\mathbb{FF}_2$.
 \end{cor}



	\newpage
${}$ \newpage














 
 \chapter{Topos Semantics I}
 


In this section we will explore the topoi-semantics for $\mathbb{FF}_2$/\emph{bushes} by following the general approach outlined in \cite{goldblatt}. 
\newline
We give, whenever necessary, a general outline from \cite{goldblatt} for topoi-semantics and make a few considerations with regards to $\mathbb{FF}_2$/\emph{bushes}.
 \newline\newline
Recall the sub-object classifier for  $\mathbb{FF}_2$/\emph{bushes}. 
\newline 
This is given by $\Omega= \{\textcolor{red}{\textbf{f}}, \textcolor{OliveGreen}{\textbf{t}} < \textcolor{cyan}{*}\}$ and $\mathbf{1} \xrightarrow{true} \Omega$. 
\newline
For each \emph{sub-object} $f : F \rightarrowtail G $ (which determines a sub-forest $f[F] \subseteq G$) there is a unique \emph{characteristic} arrow $\chi_f : F \rightarrow \Omega$ making the following commutative diagram a \emph{pullback} of $\chi_f$ and $true$.

\begin{figure}[h]
	\centering 
	\begin{tikzcd}
		F && G \\
		& {{}} \\
		1 && \Omega
		\arrow["s"', tail, from=1-1, to=1-3]
		\arrow["{\chi_f}"', from=1-3, to=3-3]
		\arrow["{!_F}", dashed, from=1-1, to=3-1]
		\arrow["true", from=3-1, to=3-3]
		\arrow[draw=none, from=1-1, to=2-2]
		\arrow[ from=1-1, to=2-2, phantom, "\scalebox{1.5}{$\lrcorner$}",  very near start, color=black]
	\end{tikzcd}\
	\caption{the characteristic diagram of $f$.}
\end{figure}   

Recall also that $\mathbb{Set}_{fin} \simeq \mathbb{FF}_1$. \newline
In $\mathbb{Set}_{fin}$ and more generally in $\mathbb{Set}$ the sub-object classifier is $\textbf{2} = \{0,1\} $ together with $\textbf{1}=\{0\} \xrightarrow{true} \textbf{2},\; 0 \mapsto 1$.\newline This can be translated in $\mathbb{FF}_1$ as \textbf{2} = \{\textcolor{red}{\textbf{f}}, \textcolor{OliveGreen}{\textbf{t}}\} with $\textbf{1}=\{\bullet\} \xrightarrow{true} \textbf{2}, \; \bullet \mapsto \textcolor{OliveGreen}{\textbf{t}}$.


\section{The Propositional Layer}
\label{chapter3}

Moving away from the usual environment of sets we wish to recover propositional Logic in the context of topoi.


\subsection{\hl{The Language of Topoi}}
 
\emph{	(Unless otherwise specified, the usual coloring notation to display maps is used. In the domain the fiber/pre-image of a node will have the same color of the node in the co-domain).} \newline

We begin the translation from the language of $\mathbb{Set}$ to that of a topos $\mathcal{E}$ from the truth constants $\top$,$\bot$ and the logical connectives $\neg,\land,\lor,\Rightarrow$. \newline
In all these cases they will be interpreted as arrows into $\Omega$ of the form $\Omega^n \rightarrow \Omega$, a.k.a. \emph{Truth-arrows} in which $n$ corresponds to the arity of the connective. \newline

Let's start by translating the constants true/top $\top$ and false/bottom $\bot$ in the context of our topos of \emph{bushes}.
\newline

Note that $\top$ has already been introduced as an arrow from the zero-th power of $\Omega$ to $\Omega$ i.e., $\textbf{1}= \Omega^0  \rightarrow \Omega$.
\newline In any topos $\mathcal{E}$, $\top$ is defined as $\textbf{1} \xrightarrow{true} \Omega$. In our case:
\begin{definition}[$\top$]
	$\top$ is $\textbf{1} \xrightarrow{true} \Omega$ the map $\bullet \mapsto \textcolor{OliveGreen}{\textbf{t}}$ .
\end{definition}

\begin{figure}[h]
	\centering
	\begin{tikzcd}
		{\textcolor{OliveGreen}{\bigcdot}} && \begin{tikzpicture}[scale=0.4]
			\node (A) at (0,0) {\textcolor{red}{\textbf{f}}};
			\node (B) at (3,0) {\textcolor{OliveGreen}{\textbf{t}}};
			\node (C) at (3,3) {\textcolor{cyan}{$*$}};
			\draw[cyan, line width=.03in] (B) -- (C);
		\end{tikzpicture}
		\arrow["true", from=1-1, to=1-3]
	\end{tikzcd}
	\caption{$\top : \mathbf{1} \xrightarrow{true} \Omega$. (\emph{bushes})}
\end{figure}

This can also be seen as the \emph{characteristic arrow} for $\textbf{1} \xrightarrow{id} \textbf{1}$ i.e., for the \emph{maximal sub-object} of \textbf{1}. 
\newline
This generalizes from $\mathbb{Set}$ in which $\top : \textbf{1}=\{0\} \xrightarrow{true} \textbf{2}=\{0,1\}, \; 0 \mapsto 1$ is also the characteristic arrow for $\textbf{1} \xrightarrow{id} \textbf{1}$.

\newpage

In a similar vein, $\bot : \textbf{1}=\{0\} \xrightarrow{false} \textbf{2}=\{0,1\}, \; 0 \mapsto 0 $ in $\mathbb{Set}$ is the characteristic of $\emptyset \xrightarrow{\emptyset} \textbf{1}$ i.e., of the \emph{minimal sub-object} of \textbf{1}. 
\newline

It is straightforward to generalize this to an arbitrary topos $\mathcal{E}$ as the characteristic arrow $\bot$ of $0 \xrightarrow{!_0} 1$. In $\mathbb{FF}_2$ this becomes:

\begin{definition}[$\bot$]
	$\bot$ is $\textbf{1} \xrightarrow{false} \Omega$ the map $\bullet \mapsto \textcolor{red}{\textbf{f}}$.
\end{definition}

\begin{figure}[h]
	\centering
	\begin{tikzcd}
		{\textcolor{red}{\bigcdot}} && \begin{tikzpicture}[scale=0.4]
			\node (A) at (0,0) {\textcolor{red}{\textbf{f}}};
			\node (B) at (3,0) {\textcolor{OliveGreen}{\textbf{t}}};
			\node (C) at (3,3) {\textcolor{cyan}{$*$}};
			\draw[cyan, line width=.03in] (B) -- (C);
		\end{tikzpicture}
		\arrow["false", from=1-1, to=1-3]
	\end{tikzcd}
	\caption{$\bot : \mathbf{1} \xrightarrow{false} \Omega$. (\emph{bushes})}
\end{figure}

This is the characteristic arrow of the empty sub-forest $\textbf{0} \xrightarrow{0_1} \textbf{1}$.
\newline
Moving on to logical connectives.. 
\newpage

Starting with \emph{negation} $\neg$: This is a \emph{unary} operator and we expect a truth-arrow from $\Omega=\Omega^1 \rightarrow \Omega$. 
\newline
In $\mathbb{Set}$ $\textbf{2} \xrightarrow{\neg} \textbf{2}$ is the \emph{switch} function $\neg : 0 \mapsto 1, 1 \mapsto 0$. This coincides with the characteristic function of $\bot : \textbf{1}=\{0\} \xrightarrow{false} \textbf{2} = \{0,1\}$ which determines the sub-set $\{\bot(0)\}\subseteq \textbf{2}$. \newline
In the language of topoi: $\neg$ is the characteristic arrow of $\bot$ as defined previously. So:

\begin{definition}[$\neg$]
	$\neg : \Omega \rightarrow \Omega$ is the characteristic arrow of $\bot : \textbf{1} \xrightarrow{false} \Omega$.
\end{definition}

This means that the following commutative diagram is a pullback in $\mathbb{FF}_2$.

\begin{figure}[h]
	\centering 
	\begin{tikzcd}
		\begin{tikzpicture}[scale=0.4]
		\node (A) at (0,0) {\textcolor{black}{$\bigcdot$}};
	\end{tikzpicture} &&  \begin{tikzpicture}[scale=0.4]
			\node (A) at (0,0) {\textcolor{OliveGreen}{\textbf{f}}};
			\node (A') at (0,-0.7) {$\bullet$};
			\node (B) at (3,0) {\textcolor{red}{\textbf{t}}};
			\node (C) at (3,3) {\textcolor{red}{$*$}};
			\draw[red, line width=.03in] (B) -- (C);
		\end{tikzpicture} \\
		& {{}} \\
		\begin{tikzpicture}[scale=0.4]
			\node (A) at (0,0) {\textcolor{OliveGreen}{$\bigcdot$}};
			\node (A') at (0,-0.7) {\textcolor{black}{$\bullet$}};	
	\end{tikzpicture} && \begin{tikzpicture}[scale=0.4]
		\node (A) at (0,0) {\textcolor{red}{\textbf{f}}};
		\node (B) at (3,0) {\textcolor{OliveGreen}{\textbf{t}}};
		\node (C) at (3,3) {\textcolor{cyan}{$*$}};
		\draw[cyan, line width=.03in] (B) -- (C);
	\end{tikzpicture}
		\arrow["\bot"', tail, from=1-1, to=1-3]
		\arrow["{\neg}"', from=1-3, to=3-3]
		\arrow["{!_1}", dashed, from=1-1, to=3-1]
		\arrow["true", from=3-1, to=3-3]
		\arrow[draw=none, from=1-1, to=2-2]
		\arrow[ from=1-1, to=2-2, phantom, "\scalebox{1.5}{$\lrcorner$}",  very near start, color=black]
	\end{tikzcd}\
	\caption{the characteristic diagram of $\bot$. The images of \textbf{1} by $\bot$ and $!_1$ are shown with a black bullet underneath. }
\end{figure}   
\newpage
In the case of \emph{conjunction} $\land$, as a binary operator we expect a truth arrow $\Omega \times \Omega = \Omega^2 \rightarrow \Omega$. 
\newline
For $\mathbb{Set}$ $\land$ is the characteristic function of $\{ (1,1) \} \subset \textbf{2}\times\textbf{2}$ since by the correspondence $0 \leftrightarrow $ \textcolor{red}{\textbf{f}} and $1 \leftrightarrow $ \textcolor{OliveGreen}{\textbf{t}}, $\land$ classically outputs \textcolor{OliveGreen}{\textbf{t}} only on the pair (\textcolor{OliveGreen}{\textbf{t}},\textcolor{OliveGreen}{\textbf{t}}). As an arrow in $\mathbb{Set}$ this sub-set is determined by the \emph{product-function} $\top \times \top : \textbf{1} \times \textbf{1} \rightarrow \textbf{2} \times \textbf{2}$. \newline
We can define $\land$ in any topos as the character of the product arrow $\top \times \top$.\newline
In \emph{bushes}:

\begin{definition}[$\land$]
	$\land$ is the characteristic arrow of $\top \times \top : \textbf{1} \times \textbf{1} \rightarrow \Omega \times \Omega$.
\end{definition}

Recalling that  $ \Omega \times \Omega = (\textbf{1} + \textbf{1}_\bot) \times (\textbf{1} + \textbf{1}_\bot) = \textbf{1} + \textbf{1}_\bot + \textbf{1}_\bot + (\textbf{1}_\bot \times \textbf{1}_\bot)   $ and $\textbf{1} \times \textbf{1} = \textbf{1} $
the characteristic diagram is the following:

\begin{figure}[h]
	\centering 
	\begin{tikzcd}
		\begin{tikzpicture}[scale=0.4]
			\node (A) at (0,0) {\textcolor{black}{$\bigcdot$}};
		\end{tikzpicture} && \begin{tikzpicture}[scale=0.4]
		\node (A) at (0,0) {\textcolor{red}{$\bigcdot$}};
		\node (B) at (2,0) {\textcolor{red}{$\bigcdot$}};
			\node (C) at (2,3) {\textcolor{red}{$\bigcdot$}};
		\node (D) at (4,0) {\textcolor{red}{$\bigcdot$}};
			\node (E) at (4,3) {\textcolor{red}{$\bigcdot$}};
		\node (F) at (8,0) {\textcolor{OliveGreen}{$\bigcdot$}};
		\node (F') at (8,-0.7) {\textcolor{black}{$\bullet$}};	
			\node (G) at (6,3) {\textcolor{cyan}{$\bigcdot$}};
			\node (H) at (8,3) {\textcolor{cyan}{$\bigcdot$}};
			\node (I) at (10,3) {\textcolor{cyan}{$\bigcdot$}};
		\draw[red, line width=.03in] (B) -- (C);
		\draw[red, line width=.03in] (D) -- (E);
		\draw[cyan, line width=.03in] (F) -- (G);
		\draw[cyan, line width=.03in] (F) -- (H);
		\draw[cyan, line width=.03in] (F) -- (I);	
\end{tikzpicture} \\
		& {{}} \\
		\begin{tikzpicture}[scale=0.45]
			\node (A) at (0,0) {\textcolor{OliveGreen}{$\bigcdot$}};
			\node (A') at (0,-0.7) {\textcolor{black}{$\bullet$}};	
		\end{tikzpicture} && \begin{tikzpicture}[scale=0.4]
			\node (A) at (0,0) {\textcolor{red}{\textbf{f}}};
			\node (B) at (3,0) {\textcolor{OliveGreen}{\textbf{t}}};
			\node (C) at (3,3) {\textcolor{cyan}{$*$}};
			\draw[cyan, line width=.03in] (B) -- (C);
		\end{tikzpicture}
		\arrow["\top \times \top"', tail, from=1-1, to=1-3]
		\arrow["{\land}"', from=1-3, to=3-3]
		\arrow["{!_{1 \times 1}}", dashed, from=1-1, to=3-1]
		\arrow["true", from=3-1, to=3-3]
		\arrow[draw=none, from=1-1, to=2-2]
		\arrow[ from=1-1, to=2-2, phantom, "\scalebox{1.5}{$\lrcorner$}",  very near start, color=black]
	\end{tikzcd}\
	\caption{the characteristic diagram of $\top \times \top$.  }
\end{figure}   

\newpage
The case of $\lor$ is analogous to $\land$: \newline
Starting as usual from $\mathbb{Set}$, $\lor$ is the characteristic function of the sub-set $D = \{(1,1),(0,1)\} \cup \{(1,1),(1,0)\}=\{(0,1),(1,0),(1,1)\} \subset \textbf{2}\times\textbf{2}$. The two sub-sets $\{(1,1),(0,1)\}$ and $\{(1,1),(1,0)\}$ are determined by the arrows $id_2 \times \top, \top \times id_2 : \textbf{2} \rightarrow \textbf{2} \times \textbf{2}$ so their union $D$ is the image of the \emph{sum-function} $(id_2 \times \top) + (\top \times id_2) : \textbf{2} + \textbf{2} \rightarrow \textbf{2} \times \textbf{2} $. \newline
As such we can define $\lor$ in any topos as the character of the image of the \emph{coproduct-arrow} $f = (id_\Omega \times \top) + (\top \times id_\Omega)$ i.e., the characteristic arrow of $Im(f) \xrightarrow{m} \Omega \times \Omega$ \footnote{obtained through epi-mono factorization $f=m \circ e$.}.
In \emph{bushes}:

\begin{definition}[$\lor$] 
$\lor$ is the character of the image of  $(id_\Omega \times \top) + (\top \times id_\Omega) : \Omega + \Omega \rightarrow \Omega \times \Omega $.
\end{definition}


\begin{figure}[h]
	\centering
	\begin{tikzcd}
		\begin{tikzpicture}[scale=0.4]
			\node (A) at (0,0) {\textcolor{purple}{$\bigcdot$}};
			\node (B) at (2,0) {\textcolor{YellowGreen}{$\bigcdot$}};
			\node (C) at (2,3) {\textcolor{blue}{$\bigcdot$}};
			\node (D) at (4,0) {\textcolor{orange}{$\bigcdot$}};
			\node (E) at (6,0) {\textcolor{green}{$\bigcdot$}};
			\node (F) at (6,3) {\textcolor{CadetBlue}{$\bigcdot$}};
			\draw[blue, line width=.01in] (B) -- (C);	
			\draw[CadetBlue, line width=.01in] (E) -- (F);	
		\end{tikzpicture} \\
		\\
		\begin{tikzpicture}[scale=0.4]
			
			\node (B) at (2,0) {\textcolor{black}{$\bigcdot$}};
			\node (b) at (2,-0.7) {\textcolor{purple}{$\bullet$}};
			\node (D) at (4,0) {\textcolor{black}{$\bigcdot$}};
			\node (d) at (4,-0.7) {\textcolor{orange}{$\bullet$}};
			\node (F) at (8,0) {\textcolor{black}{$\bigcdot$}};
			\node (f) at (7.8,-0.7) {\textcolor{YellowGreen}{$\bullet$}};	
			\node (f') at (8.3,-0.7) {\textcolor{green}{$\bullet$}};
			\node (G) at (6,3) {\textcolor{black}{$\bigcdot$}};
			\node (g) at (6,2.3) {\textcolor{blue}{$\bullet$}};
			\node (I) at (10,3) {\textcolor{black}{$\bigcdot$}};
			\node (i) at (10,2.3) {\textcolor{CadetBlue}{$\bullet$}};

			\draw[line width=.01in] (F) -- (G);
			\draw[line width=.01in] (F) -- (I);	
		\end{tikzpicture} && \begin{tikzpicture}[scale=0.4]
			\node (A) at (0,0) {\textcolor{red}{$\bigcdot$}};
			\node (B) at (2,0) {\textcolor{OliveGreen}{$\bigcdot$}};
			\node (b) at (2,-0.7) {\textcolor{purple}{$\bullet$}};
			\node (C) at (2,3) {\textcolor{cyan}{$\bigcdot$}};
			\node (D) at (4,0) {\textcolor{OliveGreen}{$\bigcdot$}};
			\node (d) at (4,-0.7) {\textcolor{orange}{$\bullet$}};
			\node (E) at (4,3) {\textcolor{cyan}{$\bigcdot$}};
			\node (F) at (8,0) {\textcolor{OliveGreen}{$\bigcdot$}};
			\node (f) at (7.8,-0.7) {\textcolor{YellowGreen}{$\bullet$}};	
			\node (f') at (8.3,-0.7) {\textcolor{green}{$\bullet$}};
			\node (G) at (6,3) {\textcolor{OliveGreen}{$\bigcdot$}};
			\node (g) at (6,2.3) {\textcolor{blue}{$\bullet$}};
			\node (H) at (8,3) {\textcolor{cyan}{$\bigcdot$}};
			\node (I) at (10,3) {\textcolor{OliveGreen}{$\bigcdot$}};
			\node (i) at (10,2.3) {\textcolor{CadetBlue}{$\bullet$}};
			\draw[cyan, line width=.03in] (B) -- (C);
			\draw[cyan, line width=.03in] (D) -- (E);
			\draw[OliveGreen, line width=.03in] (F) -- (G);
			\draw[cyan, line width=.03in] (F) -- (H);
			\draw[OliveGreen, line width=.03in] (F) -- (I);	
		\end{tikzpicture} \\
		\\
			\begin{tikzpicture}[scale=0.45]
			\node (A) at (0,0) {\textcolor{OliveGreen}{$\bigcdot$}};
		
		\end{tikzpicture} && \begin{tikzpicture}[scale=0.4]
		\node (A) at (0,0) {\textcolor{red}{\textbf{f}}};
		\node (B) at (3,0) {\textcolor{OliveGreen}{\textbf{t}}};
		\node (C) at (3,3) {\textcolor{cyan}{$*$}};
		\draw[cyan, line width=.03in] (B) -- (C);
	\end{tikzpicture}
		\arrow["true", from=5-1, to=5-3]
		\arrow["{!_{Im (f)}}", from=3-1, to=5-1]
		\arrow[tail, "m", from=3-1, to=3-3]
		\arrow["{\lor}", from=3-3, to=5-3]
		\arrow["{(id_\Omega \times \top) + (\top \times id_\Omega)}", from=1-1, to=3-3]
		\arrow[two heads,"e"', from=1-1, to=3-1]
		\arrow[draw=none, from=3-1, to=5-3]
		\arrow[ from=3-1, to=5-3, phantom, "\scalebox{1.5}{$\lrcorner$}",  very near start, color=black]
	\end{tikzcd}

\caption{the characteristic diagram of $Im((id_\Omega \times \top) + (\top \times id_\Omega))$ with the epi-mono factorization on display. \newline
	The images of the nodes $\Omega + \Omega$ are shown with bullets of matching color.  }
\end{figure}


\newpage
Finally we arrive at implication $\Rightarrow$. \newline
Starting from $\mathbb{Set}$ we observe that the sub-set that we want to \emph{characterize} for implication is precisely $\leq \;= \{(0,0), (0,1), (1,1)\} \subset \textbf{2} \times \textbf{2}$ i.e., the partial order relation on the \emph{lattice} \textbf{2} given by $\leq \;= \{(x,y) \in \textbf{2} \times \textbf{2} : x \leq y\}$. $\leq$ can also be described as $\leq \;= \{(x,y) \in \textbf{2} \times \textbf{2} : x \land y = x\}$ i.e., the \emph{equalizer}  of $\land$ and $\pi_1$ the first projection of $\textbf{2} \times \textbf{2}$ denoted by $Eq(\land,\pi_1) = \;\leq \;\overset{e}{\rightarrowtail} \textbf{2} \times \textbf{2}\; \substack{\xrightarrow[\pi_1]{}\\[-2.2em] \xrightarrow[]{\land\;}} \; \textbf{2}$. \newline
This allows us to state that in any topos: \newline $\Rightarrow$ is the character of the equalizer $e$ of $\land$ as previously defined and $\pi_1$ the first projection of $\Omega \times \Omega$. As such in \emph{bushes}:

\begin{definition}[$\Rightarrow$] 
$\Rightarrow$ is the character of $e$ with $Eq(\land,\pi_1) = E \overset{e}{\rightarrowtail} \Omega \times \Omega\; \substack{\xrightarrow[\pi_1]{}\\[-2.2em] \xrightarrow[]{\land\;}} \; \Omega$.
\end{definition}
The characteristic diagram is thus given by:

\begin{figure}[h]
	\centering 
	\begin{tikzcd}
		\begin{tikzpicture}[scale=0.4]
			\node (A) at (0,0) {\textcolor{orange}{$\bigcdot$}};
			\node (B) at (2,0) {\textcolor{GreenYellow}{$\bigcdot$}};
			\node (C) at (2,3) {\textcolor{blue}{$\bigcdot$}};
			\node (F) at (8,0) {\textcolor{green}{$\bigcdot$}};
			\node (G) at (6,3) {\textcolor{CadetBlue}{$\bigcdot$}};
			\node (H) at (8,3) {\textcolor{RoyalBlue}{$\bigcdot$}};
			\draw[blue, line width=.01in] (B) -- (C);
			\draw[CadetBlue, line width=.01in] (F) -- (G);
			\draw[RoyalBlue, line width=.01in] (F) -- (H);
		\end{tikzpicture} && \begin{tikzpicture}[scale=0.4]
			\node (A) at (0,0) {\textcolor{OliveGreen}{$\bigcdot$}};
			\node (a) at (0,-0.7) {\textcolor{orange}{$\bullet$}};
			\node (B) at (2,0) {\textcolor{OliveGreen}{$\bigcdot$}};
			\node (b) at (2,-0.7) {\textcolor{GreenYellow}{$\bullet$}};
			\node (C) at (2,3) {\textcolor{OliveGreen}{$\bigcdot$}};
			\node (c) at (2,2.3) {\textcolor{blue}{$\bullet$}};
			\node (D) at (4,0) {\textcolor{red}{$\bigcdot$}};
			\node (E) at (4,3) {\textcolor{red}{$\bigcdot$}};
			\node (F) at (8,0) {\textcolor{OliveGreen}{$\bigcdot$}};
			\node (f) at (8,-0.7) {\textcolor{green}{$\bullet$}};	
			\node (G) at (6,3) {\textcolor{OliveGreen}{$\bigcdot$}};
			\node (g) at (6,2.3) {\textcolor{CadetBlue}{$\bullet$}};
			\node (H) at (8,3) {\textcolor{OliveGreen}{$\bigcdot$}};
			\node (h) at (8,2.3) {\textcolor{RoyalBlue}{$\bullet$}};
			\node (I) at (10,3) {\textcolor{cyan}{$\bigcdot$}};
			\draw[OliveGreen, line width=.03in] (B) -- (C);
			\draw[red, line width=.03in] (D) -- (E);
			\draw[OliveGreen, line width=.03in] (F) -- (G);
			\draw[OliveGreen, line width=.03in] (F) -- (H);
			\draw[cyan, line width=.03in] (F) -- (I);	
		\end{tikzpicture} \\
		& {{}} \\
		\begin{tikzpicture}[scale=0.45]
			\node (A) at (0,0) {\textcolor{OliveGreen}{$\bigcdot$}};
		\end{tikzpicture} && \begin{tikzpicture}[scale=0.4]
			\node (A) at (0,0) {\textcolor{red}{\textbf{f}}};
			\node (B) at (3,0) {\textcolor{OliveGreen}{\textbf{t}}};
			\node (C) at (3,3) {\textcolor{cyan}{$*$}};
			\draw[cyan, line width=.03in] (B) -- (C);
		\end{tikzpicture}
		\arrow["e", tail, from=1-1, to=1-3]
		\arrow["{\Rightarrow}"', from=1-3, to=3-3]
		\arrow["{!_E}", dashed, from=1-1, to=3-1]
		\arrow["true", from=3-1, to=3-3]
		\arrow[draw=none, from=1-1, to=2-2]
		\arrow[ from=1-1, to=2-2, phantom, "\scalebox{1.5}{$\lrcorner$}",  very near start, color=black]
	\end{tikzcd}\
	\caption{the characteristic diagram of $e$. \newline
		The images of the nodes of $E$ are shown with bullets of matching color.  }
\end{figure}  

\newpage
We give some general notions from \cite{goldblatt}: \newline
\newline
Let's consider a generic topos $\mathcal{E}$. \newline
We defined the \emph{truth-arrows} we need for the semantics of propositional formulae.  \newline
The \emph{truth-values} are given by the \emph{hom-set} $\mathcal{E}(\textbf{1},\Omega)$. \newline Note that these truth-values are in fact \emph{generalized elements} of $\Omega$ which generalize the set-elements of $\mathbf{2}$ in $\mathbb{Set}$. \newline
An \emph{$\mathcal{E}-valuation$} of a propositional variable is just an assignment of a truth-value. This can be extended inductively to formulae using the connectives we just defined:

\begin{definition}[$\mathcal{E}$ -valuation]
	An \emph{$\mathcal{E}$-valuation} is a function $V : \mathbf{Prop} \rightarrow \mathcal{E}(\mathbf{1},\Omega)$ which is extended to $\mathbf{Form}$: \newline
	  $\forall \phi,\psi \in \mathbf{Form}$ :
	\begin{itemize}
		\item $V(\neg \phi) = \neg \circ V(\phi)$.
		\item $V(\phi \land \psi) =\; \land \circ  [V(\phi)\times V(\chi)] $.
		\item $V(\phi \lor \psi) =\; \lor \circ  [V(\phi)\times V(\chi)] $.
		\item $V(\phi \Rightarrow \psi) =\; \Rightarrow \circ  [V(\phi)\times V(\chi)] $.
	\end{itemize}
\end{definition}

We can talk about \emph{topos-validity} of a formula $\phi$ if every valuation gives the truth-arrow $\top$:

\begin{definition}[$\mathcal{E}$-validity (propositional)]
	A formula $\phi$ is $\mathcal{E}$-valid, denoted by $\mathcal{E} \models \phi$ or $\models_\mathcal{E}  \phi$, when for every $\mathcal{E}$-valuation $V$, $V(\phi)=\top$.  
\end{definition} 




\newpage
\subsection{The Algebra of Sub-Objects}

Let $\mathcal{E}$ be a topos and $\textbf{d}\in \mathcal{E}$ one of its objects. \newline
The truth arrows defined previously can be used to give an algebraic structure to $Sub(\textbf{d})$ the collection of sub-objects of $\textbf{d}$. \newline
In the case of $\mathbb{Set}$, consider sub-objects or simply sub-sets $A$ and $B$ of a set $D$ with their characteristic functions $\chi_A, chi_B : D \rightarrow \textbf{2} $ then we have the following:
\begin{itemize}
	\item $\chi_{A^c} = \neg \circ \chi_A$.
	\item $\chi_{A \cap B} = \land \circ (\chi_A \times \chi_B)$.
	\item $\chi_{A \cup B} = \lor \circ (\chi_A \times \chi_B)$.	
\end{itemize}  

Generalizing to $\mathcal{E}$:
\newline

We define the operations in $Sub(\textbf{d})$ by specifying the \emph{characteristic arrow} $\chi : \textbf{d} \rightarrow \Omega$ of the new sub-object which will then be obtained via the pullback of $\top$ and $\chi$. \newline
Let $f,g \in Sub(d)$ we define the complement of $f$ as:

\begin{definition}[$-f$]
	The complement of $f$ is the sub-object $-f$ whose characteristic arrow is $\neg \circ \chi_f$.  
\end{definition}

The intersection of $f$ and $g$ as:

\begin{definition}[$f \cap g$]
	The intersection of $f$ and $g$ is the sub-object $f \cap g$ whose characteristic arrow is $\land \circ (\chi_f \times \chi_g)$.	
\end{definition}

The union of $f$ and $g$ as:

\begin{definition}[$f \cup g$]
	The union of $f$ and $g$ is the sub-object $f \cup g$ whose characteristic arrow is $\lor \circ (\chi_f \times \chi_g)$.	
\end{definition}

The implication of $g$ by $f$ as:

\begin{definition}[$f \Rightarrow g$]
	The implication of $g$ by $f$ is the sub-object $f \Rightarrow g$ whose characteristic arrow is $\Rightarrow \circ (\chi_f \times \chi_g)$.	
\end{definition}

Recall that just like ($\mathcal{P}(D), \leq$) is a partial ordering with set-inclusion, $(Sub(d), \sqsubseteq)$
becomes a partial ordering by \emph{sub-object inclusion} i.e., $f \sqsubseteq g$ if there is an arrow $h: Dom(f) \rightarrow Dom(g)$ such that $f = g \circ h$.\newline
We now have the following result:

\begin{prop}
	$(Sub(d),\sqsubseteq)$ is a \emph{bounded lattice} in which:
	\begin{itemize}
		\item $f \cap g$ is the \emph{greatest lower bound} of $f$ and $g$ i.e., the \emph{meet} of $f$ and $g$.
		\item $f \cup g$ is the \emph{least upper bound} of $f$ and $g$ i.e., the \emph{join} of $f$ and $g$.
		\item the arrow from the initial object $0_d : \textbf{0} \rightarrow \textbf{d}$ is the \emph{bottom} element.
		 \item the identity arrow $1_d : \textbf{d} \rightarrow \textbf{d}$ is the \emph{top} element.
	\end{itemize}
\end{prop}

Also the following holds for all $f,g,h \in Sub(\textbf{d})$ making $\Rightarrow$ a \emph{pseudo-complement}:

\begin{lem}
	$h \sqsubseteq (f \Rightarrow g) $ iff $ (f \cap h) \sqsubseteq g $.
\end{lem}

We are thus in a position to state that:

\begin{thm}
	$(Sub(d),\sqsubseteq)$ is a \emph{Heyting Algebra} with top element $1_d$, bottom element $0_d$ and $f\cap g$,$f \cup g$ and $f \Rightarrow g$ are respectively the \emph{meet}, \emph{join} and \emph{pseudo-complement} operations.
\end{thm}

We also have:

\begin{thm}
	$(\mathcal{E}(\textbf{d},\Omega), \sqsubseteq)$ is a \emph{Heyting Algebra} with top element $true_d$ \footnote{the character of $1_d$ i.e., the identity arrow on $d$. }, bottom element $false_d$ \footnote{the character of $0_d: 0 \rightarrow d$ i.e., the initial arrow on $d$.} and the truth-arrows as operations.
	This can be seen by the following definitions:
	\begin{gather*}
		\chi_f \sqsubseteq \chi_g \text{ iff }(\chi_f \times \chi_g)\text{ factors through }e: \leq \hookrightarrow \Omega \times \Omega. \\
		\chi_f \cap \chi_g :=\;\; \land \circ (\chi_f \times \chi_g). \\
		\chi_f \cup \chi_g :=\;\; \lor \circ (\chi_f \times \chi_g). \\
		\neg\chi_f :=\;\; \neg \circ \chi_f. \\
		\chi_f \Rightarrow \chi_g :=\;\; \Rightarrow \circ (\chi_f \times \chi_g). \\
	\end{gather*}
\end{thm} 
\newpage
Recall that the $\Omega$-Axiom from \ref{chapter02} gave us a bijection $Sub(d) \cong \mathcal{E}(\textbf{d},\Omega)$ in which $f \xleftrightarrow{1:1} \chi_f$.\newline
We can show that this bijection is in fact a \emph{Heyting Algebra isomorphism} $\Delta: Sub(d) \cong \mathcal{E}(\textbf{d},\Omega)$ where: 
\begin{gather*}
	\Delta(-f) :=\;\; \neg \circ \Delta(f). \\
	\Delta(f \cap g) :=\;\; \land \circ (\Delta(f) \times \Delta(g)). \\
	\Delta(f \cup g) :=\;\; \lor \circ (\Delta(f) \times \Delta(g)). \\
	\Delta(f \Rightarrow g) =\;\; \Rightarrow \circ (\Delta(f) \times \Delta(g)). 
\end{gather*}


\newpage
\subsection{Soundness and Completeness for CPL and IPL}

We proceed by giving a few more general results from \cite{goldblatt}:\newline

We have seen that in any topos $\mathcal{E}$ the truth arrow $\top = \chi_{id_\textbf{1}}$ and $\bot = \chi_{0_\textbf{1}}$. \newline
If we now focus on the sub-objects of the terminal \textbf{1} i.e., $Sub(\textbf{1})$  we can derive the following: 
\begin{itemize}
	\item $\top \land \top = \chi_{id_\textbf{1}} \land \chi_{id_\textbf{1}} = \chi_{id_\textbf{1} \cap id_\textbf{1}}$ = $\chi_{id_\textbf{1}} = \top$. 
	\item $\top \land \bot = \chi_{id_\textbf{1}} \land \chi_{0_\textbf{1}} = \chi_{id_\textbf{1} \cap 0_\textbf{1}}$ = $\chi_{0_\textbf{1}} = \bot$.
	\item $\bot \land \top = \chi_{0_\textbf{1}} \land \chi_{id_\textbf{1}} = \chi_{0_\textbf{1} \cap id_\textbf{1}}$ = $\chi_{0_\textbf{1}} = \bot$.
	 \item $\bot \land \bot = \chi_{0_\textbf{1}} \land \chi_{0_\textbf{1}} = \chi_{0_\textbf{1} \cap 0_\textbf{1}}$ = $\chi_{0_\textbf{1}}  = \bot$.
	\item ... 	 
\end{itemize}
and so forth for $\lor$ and $\Rightarrow$.
What we found is that the truth arrows $\top$ and $\bot$ behave \emph{classically} with respect to the $\mathcal{E}$-connectives we defined. \footnote{the same result could have been obtained without reference to sub-objects and simply unfolding the definitions, putting one pullback square atop another.} 

With this in mind and remembering how we defined $\mathcal{E}$-validity, we can give some first results about Soundness and Completeness for \emph{Classical} and \emph{Intuitionistic} Propositional Logic \textbf{CPL} and \textbf{IPL}. \newline

Namely, \emph{Completeness} for \textbf{CPL}: 

\begin{thm}
	For any topos $\mathcal{E}$, $\alpha \in \mathbf{Form}$:
	\begin{equation*}
		\text{If }\mathcal{E} \models \alpha\text{ then }\vdash_{CPL} \alpha.
	\end{equation*} 
\end{thm}

\textbf{CPL} is not always \emph{sound} with respect $\mathcal{E}$-validity.. \newline
However, if we restrict ourselves to \emph{bivalent} topoi i.e., topoi with just two truth values i.e., $|\mathcal{E}(\textbf{1}, \Omega)|=2$ we have soundness and completeness for \textbf{CPL}:
\begin{prop}\label{bivalence}
	If $\mathcal{E}$ is bivalent, then:
	\begin{gather*}
		\forall \alpha \in \textbf{Form}:\\ 
		\mathcal{E} \models \alpha \text{ iff } \vdash_{CPL} \alpha.
	\end{gather*}
	
	For example:\newline
	$\mathbb{Set}$ is bivalent as $\Omega = \mathbb{2} = \{0,1\}$ and thus:
	\begin{equation*}
		\vdash_{CPL} \alpha \; \text{ iff } \; \mathbb{Set} \models \alpha.
	\end{equation*}
\end{prop}

What about \emph{intuitionistic} Logic? \newline

Recall that Heyting Algebras provide a sound and complete semantics for \textbf{IPL} i.e.,:

\begin{remark}
	For any Heyting Algebra \emph{HA}, $\alpha \in \mathbf{Form}$:
	\begin{equation*}
		\emph{HA} \models \alpha\text{ iff }\vdash_{IPL} \alpha.
	\end{equation*}
	 
\end{remark}

As we have seen, for any $\mathcal{E}$-object $\mathbf{d}$, there is an isomorphism $Sub(d) \cong \mathcal{E}(\textbf{d},\Omega)$ which transfers the H.A. \footnote{from now on H.A. will be more commonly used instead of \emph{Heyting Algebra}. } structure of the sub-objects of $Sub(d)$ to the truth arrows of $\mathcal{E}(\textbf{d},\Omega)$. \newline
 This gives us the following equivalence which links the semantics of topoi to that of Heyting Algebras:\newline
($\models_{\mathcal{E}}$ denotes topos validity whilst $\models_{H.A.}$ Heyting algebra validity).
\begin{prop}
	For any topos $\mathcal{E}$, $\alpha \in \mathbf{Form}$:
	\begin{equation*}
		\models_\mathcal{E} \alpha\text{ iff }\;\;\mathcal{E}(\textbf{1},\Omega) \models_{H.A.} \alpha\text{ iff }\;\;Sub(\textbf{1}) \models_{H.A.} \alpha.
	\end{equation*}
\end{prop}
To see why this is the case, notice that an $\mathcal{E}$-valuation is an H.A.-valuation for $\mathcal{E}(\textbf{1},\Omega)$ which is isomorphic to $Sub(\textbf{1})$ and that the \emph{unit} $\top$ of the H.A. $\mathcal{E}(\textbf{1},\Omega)$ is precisely the truth arrow $\top : \textbf{1}\rightarrow \Omega$ so that $\mathcal{E}$-validity and $\mathcal{E}(\textbf{1},\Omega)$-validity amount to the same thing. 


This allows us to say that if $\vdash_{IPL} \alpha$, then by soundness for Heyting Algebras $Sub(\textbf{1})\models \alpha$ and $\mathcal{E}(\textbf{1}, \Omega) \models \alpha$ which means $\models_\mathcal{E}  \alpha$. \newline
What we have shown is, contrary to the case of \textbf{CPL}, that topoi provide a sound semantics for \textbf{IPL}.

\begin{thm}[Soundness for topos-validity]
		For any topos $\mathcal{E}$, $\alpha \in \mathbf{Form}$:
		\begin{equation*}
			\text{If }\vdash_{IPL} \alpha\text{ then }\models_\mathcal{E}  \alpha.
		\end{equation*}		
\end{thm}


\newpage
\subsection{\hl{External and Internal Logics}}
\label{externalandint}

Let's take a step back and revisit our definitions for the logical connectives in the topos of \emph{bushes}/$\mathbb{FF_2}$. 
\newline
Negation $\neg$ for example yielded:
\begin{figure}[h]
	\centering
	\begin{tikzcd}
		\begin{tikzpicture}[scale=0.4]
			\node (A) at (0,0) {\textcolor{OliveGreen}{\textbf{f}}};
			\node (B) at (3,0) {\textcolor{red}{\textbf{t}}};
			\node (C) at (3,3) {\textcolor{red}{$*$}};
			\draw[red, line width=.03in] (B) -- (C);
		\end{tikzpicture} && \begin{tikzpicture}[scale=0.4]
		\node (A) at (0,0) {\textcolor{red}{\textbf{f}}};
		\node (B) at (3,0) {\textcolor{OliveGreen}{\textbf{t}}};
		\node (C) at (3,3) {\textcolor{cyan}{$*$}};
		\draw[cyan, line width=.03in] (B) -- (C);
	\end{tikzpicture}
		\arrow["\neg", from=1-1, to=1-3]
	\end{tikzcd}
\caption{$\neg : \Omega \rightarrow \Omega$ in $\mathbb{FF_2}$.}
\end{figure}

Recalling the truth table for $\neg$ in $\mathcal{G}_3$:

	
	\begin{figure}[h]
		\centering
		\begin{tabular}{||c || c ||}  
			\hline
			& $\neg $ \\  
			\hline\hline
			\textcolor{OliveGreen}{\textbf{t}} & \textcolor{red}{\textbf{f}}  \\ 
			\hline
			\textcolor{cyan}{$*$} & \textcolor{red}{\textbf{f}} \\
			\hline
			\textcolor{red}{\textbf{f}} & \textcolor{OliveGreen}{\textbf{t}}  \\
			\hline
		\end{tabular}
		\caption{$\neg$ in $\mathcal{G}_3$.}
	\end{figure}

Looking \emph{from outside} at the $\neg$ arrow for \emph{bushes}, one finds that the nodes \textcolor{OliveGreen}{t} and \textcolor{cyan}{$*$} are mapped to \textcolor{red}{f} and that \textcolor{red}{f} is mapped to \textcolor{OliveGreen}{t}.
\newline
This is precisely the truth function for $\neg$ in $\mathcal{G}_3$. 
\newline
Note that $\neg$ is a unary operator and all one had to do was to look at the nodes in order to \emph{visualize} externally the corresponding truth-table. For binary connectives one needs to look at the product object $\Omega \times \Omega$ and its projections $\pi_1, \pi_2$ and label the nodes appropriately: 
\begin{figure}[h]
	
	\centering
		\begin{tikzcd}
			& \begin{tikzpicture}[scale=0.4]
				\node (A) at (0,0) {\textcolor{red}{\textbf{f}} \textcolor{red}{\textbf{f}}};
				
				\node (B) at (4,0) {\textcolor{red}{\textbf{f}} \textcolor{OliveGreen}{\textbf{t}}};
				\node (C) at (4,3) {\textcolor{red}{\textbf{f}} \textcolor{cyan}{\textbf{$*$}}};
				
				\node (D) at (8,0) {\textcolor{OliveGreen}{\textbf{t}} \textcolor{red}{\textbf{f}}};
				\node (E) at (8,3) {\textcolor{cyan}{\textbf{$*$}} \textcolor{red}{\textbf{f}}};
				
				\node (F) at (16,0) {\textcolor{OliveGreen}{\textbf{t}} \textcolor{OliveGreen}{\textbf{t}}};
					\node (G) at (12,3) {\textcolor{cyan}{$*$} \textcolor{OliveGreen}{\textbf{t}}};
					\node (H) at (16,3) {\textcolor{cyan}{$*$} \textcolor{cyan}{$*$}};
					\node (I) at (20,3) {\textcolor{OliveGreen}{\textbf{t}} \textcolor{cyan}{$*$}};
				\draw[line width=.02in] (B) -- (C);
				\draw[line width=.02in] (D) -- (E);
				\draw[line width=.02in] (F) -- (G);
				\draw[line width=.02in] (F) -- (H);
				\draw[line width=.02in] (F) -- (I);	
			\end{tikzpicture}
			 \\
			\begin{tikzpicture}[scale=0.4]
				\node (A) at (0,0) {\textcolor{red}{\textbf{f}}};
				\node (B) at (3,0) {\textcolor{OliveGreen}{\textbf{t}}};
				\node (C) at (3,3) {\textcolor{cyan}{$*$}};
				\draw[cyan, line width=.03in] (B) -- (C);
			\end{tikzpicture} && \begin{tikzpicture}[scale=0.4]
			\node (A) at (0,0) {\textcolor{red}{\textbf{f}}};
			\node (B) at (3,0) {\textcolor{OliveGreen}{\textbf{t}}};
			\node (C) at (3,3) {\textcolor{cyan}{$*$}};
			\draw[cyan, line width=.03in] (B) -- (C);
		\end{tikzpicture}
			\arrow["{\pi_1}",dotted, two heads, from=1-2, to=2-1]
			\arrow["{\pi_2}"',dotted, two heads, from=1-2, to=2-3]
		\end{tikzcd}\
		\caption{$\Omega \times \Omega$ and the projections $\pi_1, \pi_2$ to $\Omega$. \newline The nodes of the product are labeled
		in the format $l r$ in which $l$ and $r$ specify the first and second projections respectively:}
\end{figure}

\newpage

Keeping this representation in mind, the same phenomenon we observed for the truth-arrow $\neg$ occurs now for the other connectives.\newline

In other words, we re-discover the truth functions for $\mathcal{G}_3$:
	
	
	\begin{figure}[h]
	\centering
	\begin{subfigure}[h]{0.4\textwidth}
		\begin{tikzcd}
			\begin{tikzpicture}[scale=0.4]
				\node (A) at (0,0) {\textcolor{red}{$\bigcdot$}};
				\node (B) at (2,0) {\textcolor{red}{$\bigcdot$}};
				\node (C) at (2,3) {\textcolor{red}{$\bigcdot$}};
				\node (D) at (4,0) {\textcolor{red}{$\bigcdot$}};
				\node (E) at (4,3) {\textcolor{red}{$\bigcdot$}};
				\node (F) at (8,0) {\textcolor{OliveGreen}{$\bigcdot$}};
				\node (G) at (6,3) {\textcolor{cyan}{$\bigcdot$}};
				\node (H) at (8,3) {\textcolor{cyan}{$\bigcdot$}};
				\node (I) at (10,3) {\textcolor{cyan}{$\bigcdot$}};
				\draw[red, line width=.03in] (B) -- (C);
				\draw[red, line width=.03in] (D) -- (E);
				\draw[cyan, line width=.03in] (F) -- (G);
				\draw[cyan, line width=.03in] (F) -- (H);
				\draw[cyan, line width=.03in] (F) -- (I);	
			\end{tikzpicture} && \begin{tikzpicture}[scale=0.4]
				\node (A) at (0,0) {\textcolor{red}{\textbf{f}}};
				\node (B) at (3,0) {\textcolor{OliveGreen}{\textbf{t}}};
				\node (C) at (3,3) {\textcolor{cyan}{$*$}};
				\draw[cyan, line width=.03in] (B) -- (C);
			\end{tikzpicture}
			\arrow["\land", from=1-1, to=1-3]
		\end{tikzcd}
		\caption{$\land : \Omega \times \Omega \rightarrow \Omega$ in $\mathbb{FF_2}$.}
	\end{subfigure}
	\hfill
	\centering
	\begin{subfigure}[h]{0.2\textwidth}
	\begin{tabular}{||c || c | c | c ||}  
		\hline
		$ \land $ & \textcolor{OliveGreen}{\textbf{t}} & \textcolor{cyan}{$*$} & \textcolor{red}{\textbf{f}} \\  
		\hline\hline
		\textcolor{OliveGreen}{\textbf{t}} & \textcolor{OliveGreen}{\textbf{t}} & \textcolor{cyan}{$*$} & \textcolor{red}{\textbf{f}}  \\ 
		\hline
		\textcolor{cyan}{$*$} & \textcolor{cyan}{$*$} & \textcolor{cyan}{$*$} & \textcolor{red}{\textbf{f}} \\
		\hline
		\textcolor{red}{\textbf{f}} & \textcolor{red}{\textbf{f}} & \textcolor{red}{\textbf{f}} & \textcolor{red}{\textbf{f}}  \\
		\hline
	\end{tabular}
	\caption{$\land$ in $\mathcal{G}_3$.}
	\end{subfigure}
	
\end{figure}

	\begin{figure}[h]
	\centering
	\begin{subfigure}[h]{0.4\textwidth}
		\begin{tikzcd}
			\begin{tikzpicture}[scale=0.4]
			\node (A) at (0,0) {\textcolor{red}{$\bigcdot$}};
			\node (B) at (2,0) {\textcolor{OliveGreen}{$\bigcdot$}};
			\node (C) at (2,3) {\textcolor{cyan}{$\bigcdot$}};
			\node (D) at (4,0) {\textcolor{OliveGreen}{$\bigcdot$}};
			\node (E) at (4,3) {\textcolor{cyan}{$\bigcdot$}};
			\node (F) at (8,0) {\textcolor{OliveGreen}{$\bigcdot$}};
			\node (G) at (6,3) {\textcolor{OliveGreen}{$\bigcdot$}};
			\node (H) at (8,3) {\textcolor{cyan}{$\bigcdot$}};
			\node (I) at (10,3) {\textcolor{OliveGreen}{$\bigcdot$}};
			\draw[cyan, line width=.03in] (B) -- (C);
			\draw[cyan, line width=.03in] (D) -- (E);
			\draw[OliveGreen, line width=.03in] (F) -- (G);
			\draw[cyan, line width=.03in] (F) -- (H);
			\draw[OliveGreen, line width=.03in] (F) -- (I);	
		\end{tikzpicture} &&  \begin{tikzpicture}[scale=0.4]
		\node (A) at (0,0) {\textcolor{red}{\textbf{f}}};
		\node (B) at (3,0) {\textcolor{OliveGreen}{\textbf{t}}};
		\node (C) at (3,3) {\textcolor{cyan}{$*$}};
		\draw[cyan, line width=.03in] (B) -- (C);
		\end{tikzpicture}
		\arrow["\lor", from=1-1, to=1-3]
		\end{tikzcd}
		\caption{$\lor : \Omega \times \Omega \rightarrow \Omega$ in $\mathbb{FF_2}$.}
	\end{subfigure}
	\hfill
	\centering
	\begin{subfigure}[h]{0.2\textwidth}
		\begin{tabular}{||c || c | c | c ||}  
			\hline
			$ \lor $ & \textcolor{OliveGreen}{\textbf{t}} & \textcolor{cyan}{$*$} & \textcolor{red}{\textbf{f}} \\  
			\hline\hline
			\textcolor{OliveGreen}{\textbf{t}} & \textcolor{OliveGreen}{\textbf{t}} & \textcolor{OliveGreen}{\textbf{t}} & \textcolor{OliveGreen}{\textbf{t}}  \\ 
			\hline
			\textcolor{cyan}{$*$} & \textcolor{OliveGreen}{\textbf{t}} & \textcolor{cyan}{$*$} & \textcolor{cyan}{$*$} \\
			\hline
			\textcolor{red}{\textbf{f}} & \textcolor{OliveGreen}{\textbf{t}} & \textcolor{cyan}{$*$} & \textcolor{red}{\textbf{f}}  \\
			\hline
		\end{tabular}
		\caption{$\lor$ in $\mathcal{G}_3$.}
	\end{subfigure}
	
\end{figure}

	\begin{figure}[h]
	\centering
	\begin{subfigure}[h]{0.4\textwidth}
		\begin{tikzcd}
		\begin{tikzpicture}[scale=0.4]
			\node (A) at (0,0) {\textcolor{OliveGreen}{$\bigcdot$}};
			\node (B) at (2,0) {\textcolor{OliveGreen}{$\bigcdot$}};
			\node (C) at (2,3) {\textcolor{OliveGreen}{$\bigcdot$}};
			\node (D) at (4,0) {\textcolor{red}{$\bigcdot$}};
			\node (E) at (4,3) {\textcolor{red}{$\bigcdot$}};
			\node (F) at (8,0) {\textcolor{OliveGreen}{$\bigcdot$}};
			\node (G) at (6,3) {\textcolor{OliveGreen}{$\bigcdot$}};
			\node (H) at (8,3) {\textcolor{OliveGreen}{$\bigcdot$}};
			\node (I) at (10,3) {\textcolor{cyan}{$\bigcdot$}};
			\draw[OliveGreen, line width=.03in] (B) -- (C);
			\draw[red, line width=.03in] (D) -- (E);
			\draw[OliveGreen, line width=.03in] (F) -- (G);
			\draw[OliveGreen, line width=.03in] (F) -- (H);
			\draw[cyan, line width=.03in] (F) -- (I);	
		\end{tikzpicture} &&  \begin{tikzpicture}[scale=0.4]
				\node (A) at (0,0) {\textcolor{red}{\textbf{f}}};
				\node (B) at (3,0) {\textcolor{OliveGreen}{\textbf{t}}};
				\node (C) at (3,3) {\textcolor{cyan}{$*$}};
				\draw[cyan, line width=.03in] (B) -- (C);
			\end{tikzpicture}
			\arrow["\Rightarrow", from=1-1, to=1-3]
		\end{tikzcd}
		\caption{$\Rightarrow : \Omega \times \Omega \rightarrow \Omega$ in $\mathbb{FF_2}$.}
	\end{subfigure}
	\hfill
	\centering
	\begin{subfigure}[h]{0.2\textwidth}
			\begin{tabular}{||c || c | c | c ||}  
			\hline
			$ \Rightarrow $ & \textcolor{OliveGreen}{\textbf{t}} & \textcolor{SkyBlue}{$*$} & \textcolor{red}{\textbf{f}} \\  
			\hline\hline
			\textcolor{OliveGreen}{\textbf{t}} & \textcolor{OliveGreen}{\textbf{t}} & \textcolor{SkyBlue}{$*$} & \textcolor{red}{\textbf{f}}  \\ 
			\hline
			\textcolor{SkyBlue}{$*$} & \textcolor{OliveGreen}{\textbf{t}} & \textcolor{OliveGreen}{\textbf{t}} & \textcolor{red}{\textbf{f}} \\
			\hline
			\textcolor{red}{\textbf{f}} & \textcolor{OliveGreen}{\textbf{t}} & \textcolor{OliveGreen}{\textbf{t}} & \textcolor{OliveGreen}{\textbf{t}}  \\
			\hline
		\end{tabular}
		\caption{$\Rightarrow$ in $\mathcal{G}_3$.}
	\end{subfigure}
	
\end{figure}

\newpage

\emph{From the inside} however we just defined truth values for  $\mathbb{FF_2}$ as elements of the hom-set $\mathbb{FF_2}(\textbf{1},\Omega) = \{ \top, \bot\}$ i.e., the set containing the arrow $\top$ for which $\bullet \mapsto \textcolor{OliveGreen}{\textbf{t}}$ and $\bot$ for which $\bullet \mapsto \textcolor{red}{\textbf{f}}$.\newline 
Having just two truth-values, the topos $\mathbb{FF_2}$ of \emph{bushes} is called \emph{bivalent}. \newline 
The third node which we labeled \textcolor{cyan}{$*$} and which externally seems to correspond to the same-name truth-value \textcolor{cyan}{$*$} \emph{not-false} in $\mathcal{G}_3$ cannot be a truth-value of the form $\textbf{1} \rightarrow \Omega$ since \textbf{1} as an open map can only be mapped to a root so either to \textcolor{red}{\textbf{f}} or \textcolor{OliveGreen}{\textbf{t}}.\newline
Informally we could say:
\begin{remark}
	 $\mathbb{FF_2}$/\emph{bushes} is \emph{internally} bivalent while \emph{from the outside} $\Omega$ has three elements.
\end{remark}
In other words:
\begin{remark}
	On the propositional level, the \emph{internal} logic of $\mathbb{FF_2}$/\emph{bushes} is classical whilst the \emph{external} logic is that of $\mathcal{G}_3$.
\end{remark}


What does it mean if the topos behaves \emph{classically} on the \emph{inside} and \emph{non-classically} on the \emph{outside}? 
\newpage

To clarify this situation,
let's take a look at what happens with \emph{double negation}. Recall that in classical logic $\alpha$ \emph{is equivalent to} $\neg \neg \alpha$ so we would expect in the language of topoi for the following to hold: $\neg \circ \neg = id_\Omega$.


\begin{figure}[h]
	\centering
	\begin{tikzcd}
		\begin{tikzpicture}[scale=0.4]
			\node (A) at (0,0) {\textcolor{red}{$\bullet_{\textbf{f}}$}};
		\node (B) at (3,0) {\textcolor{OliveGreen}{$\bullet_{\textbf{t}}$}};
		\node (C) at (3,3) {\textcolor{cyan}{$\bullet_{*}$}};
		\draw[line width=.03in] (B) -- (C);
		\end{tikzpicture} & 	\begin{tikzpicture}[scale=0.4]
		\node (A) at (0,0) {\textcolor{OliveGreen}{\textbf{f}}};
		\node (a) at (-0.5,-0.5) {\textcolor{OliveGreen}{$\bullet_{\textbf{t}}$}};
		\node (a') at (0.5,-0.5) {\textcolor{cyan}{$\bullet_{*}$}};
		\node (b) at (3.5,-0.5) {\textcolor{red}{$\bullet_{\textbf{f}}$}};
		\node (B) at (3,0) {\textcolor{red}{\textbf{t}}};
		\node (C) at (3,3) {\textcolor{red}{$*$}};
		\draw[red, line width=.03in] (B) -- (C);
		\end{tikzpicture} \\
		& \begin{tikzpicture}[scale=0.4]
			\node (A) at (0,0) {\textcolor{red}{\textbf{f}}};
			\node (a) at (3.5,-0.5) {\textcolor{OliveGreen}{$\bullet_{\textbf{t}}$}};
			\node (a') at (2.5,-0.5) {\textcolor{cyan}{$\bullet_{*}$}};
			\node (b) at (0.5,-0.5) {\textcolor{red}{$\bullet_{\textbf{f}}$}};
			\node (B) at (3,0) {\textcolor{OliveGreen}{\textbf{t}}};
			\node (C) at (3,3) {\textcolor{cyan}{$*$}};
			\draw[cyan, line width=.03in] (B) -- (C);
		\end{tikzpicture}
		\arrow["\neg", from=1-2, to=2-2]
		\arrow["\neg", from=1-1, to=1-2]
		\arrow["{id_\Omega}"', from=1-1, to=2-2]
	\end{tikzcd}
	\caption{$\neg \circ \neg \neq id_\Omega$ in $\mathbb{FF_2}$. \newline Here we displayed the images of the nodes of $\Omega$ in the top left corner as labeled bullets. }
\end{figure}
	
	This is clearly not the case for our topos.
	\newline
	Note that while the \emph{classical} nodes \textcolor{OliveGreen}{$\bullet_{\textbf{t}}$} $\mapsto \textcolor{OliveGreen}{\textbf{t}}$, \textcolor{red}{$\bullet_{\textbf{f}}$} $\mapsto \textcolor{red}{\textbf{f}}$ are fixed by $\neg \circ \neg$, the \emph{non-classical} node \textcolor{cyan}{$\bullet_{*}$}$\mapsto \textcolor{OliveGreen}{\textbf{t}}$ is not.\footnote{this corresponds to the fact in $\mathcal{G}_3$ that the negation of $*$ i.e., \emph{not-false} is \emph{false}. }\newline
	
	Another approach is to consider the \emph{Algebra of sub-objects}. \newline
	When in a topos the algebra of sub-objects for an arbitrary object, which we have seen is a Heyting Algebra, is also a Boolean Algebra:
	
	\begin{definition}[Boolean topos]
		A topos $\mathcal{E}$ is \emph{Boolean} if for every object $\textbf{d}$, $(Sub(\textbf{d}), \sqsubseteq)$ is a Boolean Algebra.
	\end{definition} 
	
	\begin{ex}[Sets]
		The prototypical Boolean topos is unsurprisingly $\mathbb{Set}$ where for every set $D$ it is the case that  $(Sub(\textbf{D}), \sqsubseteq) \cong (\mathcal{P}(D), \subseteq)$ which is the \emph{power-set} Boolean Algebra.
	\end{ex}
	
	We can use the following results from \cite{goldblatt} and \cite{lambekscott}:
	
	\begin{lem}
		For any topos $\mathcal{E}$, $\mathcal{E}$ is Boolean iff $(Sub(\Omega),\sqsubseteq)$ is a Boolean Algebra. 
	\end{lem}

	\begin{prop}
		A topos $\mathcal{E}$ is Boolean iff $\textbf{1} \xrightarrow{\top} \Omega \xleftarrow{\bot} \textbf{1}$ is a Co-product diagram.
	\end{prop}
	This would entail $\Omega \cong \textbf{1} + \textbf{1}$ and $\neg\circ\neg = id_\Omega$ which are both false for \emph{bushes}.\newline 

We also know that $\Omega = Spec(\mathcal{F}_1)$ i.e., the prime spectrum of the free Gödel Algebra on one generator $\mathcal{F}_1$. $Sub(Spec(\mathcal{F}_1))$ are in this case the sub-forests of the prime spectrum of $\mathcal{F}_1$ and by duality we have shown that  $Sub(Spec(\mathcal{F}_1)) \cong \mathcal{F}_1$. 
\newline
	$\mathcal{F}_1$ is not a Boolean Algebra ($(x \lor \neg x) \neq 1$) and so, summing up: 
	
	\begin{thm}[propositional logic of $\mathbb{FF_2}$]
		${}$ \newline
		The topos of \emph{bushes} $\mathbb{FF_2}$ is bivalent and non-Boolean.
	\end{thm}
	
	By a direct application of (\ref{bivalence}) we obtain:
	\begin{cor}
		\begin{equation*}
			\forall \alpha \in \textbf{Form} : \;\vdash_{CPL} \alpha \;\text{ iff }\; \mathbb{FF_2} \models \alpha.
		\end{equation*}
	\end{cor}

\newpage	
\subsection{A few Topoi Examples}
\label{examples}
	
	To better understand why for topoi \emph{bivalent} and \emph{Boolean} are, in a sense, independent attributes, consider the following examples:\newline
	(We omit in both cases the construction of exponential objects and focus on their sub-object classifier and truth-arrows).
	
	\begin{ex}[Pair of Sets]
		The topos $ \mathbb{Set}^2 $ of \emph{Pairs of Sets} has as objects all set pairs $\langle A,B \rangle$ and arrows pairs of set functions $\langle f, g \rangle :  \langle A,B \rangle \rightarrow \langle C,D \rangle$ with $f: A \rightarrow C$ and $g: B \rightarrow D$.\newline
		
		One can verify that the sub-object classifier $\Omega$ for $ \mathbb{Set}^2 $ is none-other than 
		$\langle \textbf{2}, \textbf{2} \rangle $ with the \emph{true} arrow given by $\langle \top,\top \rangle : \langle \{0\},\{0\} \rangle \rightarrow \langle \textbf{2}, \textbf{2} \rangle $.\newline
		As such, the \emph{truth values} are precisely the four elements $\{ \langle \bot,\bot \rangle, \langle \bot,\top \rangle, \langle \top,\bot \rangle, \langle \top,\top \rangle \}$. 
	\end{ex} 
	So:
	\begin{remark}
		$ \mathbb{Set}^2 $ fails to be bivalent. It is however a Boolean topos as the sub-objects of $\Omega$ form a power-set Boolean Algebra.\footnote{see \cite{goldblatt} for details.}  
	\end{remark}
	
	\begin{ex}[Functions between sets]
		The \emph{functor category} $\mathbb{Set}^{0 \rightarrow 1}$ of \emph{functions between sets}, with objects the functors $F :\mathbb{2} \rightarrow \mathbb{Set}$ from the poset category $\mathbf{2}:=\{0 \xrightarrow{\leq_0}$ 1\}\footnote{$\mathbf{2}$ has only objects 0 and 1 and the only non-identity arrow 01 from 0 to 1.} to $\mathbb{Set}$ and arrows the natural transformations $\tau : F \Rightarrow G$ between these functors, is a topos.
		 \newline
		We use the notation $F_i, G_j$ for $F(i),G(j)$. Also $f$ replaces $F(\leq_0)$ and $g$ replaces $G(\leq_0)$.\newline
		
		An arrow $\tau$ from the objects $F$ and $G$ is realized in $\mathbb{Set}$ as a commutative diagram:
		
		\begin{figure}[h]
			\centering
			\begin{tikzcd}
				0 && {F_0} & {G_0} \\
				1 && {F_1} & {G_1}
				\arrow[from=1-1, to=2-1]
				\arrow["{\tau_0}"', from=1-3, to=1-4]
				\arrow["f", from=1-3, to=2-3]
				\arrow["{\tau_1}"', from=2-3, to=2-4]
				\arrow["g", from=1-4, to=2-4]
			\end{tikzcd}\
			\caption{The poset category $\mathbb{2}$ (left) and the commutative diagram in $\mathbb{Set}$ (right).}
		\end{figure}  
		
		The terminal object \textbf{1}, one can verify, is the identity function $\{0\} \xrightarrow{id} \{0\}$ on the singleton set $\{0\}$. \newline
		The sub-object $\mu : F \Rightarrow G$ is realized again as a commutative diagram and we will assume without loss of generality  that the components $\mu_0, \mu_1$ (which are injective functions in $\mathbb{Set}$) be set-inclusions $F_0 \subseteq G_0, F_1 \subseteq G_1$  so that $f$ will in fact be the restriction of $g$ to $F_0$. 
		\begin{figure}[h]
			\centering
			\begin{tikzcd}
				{F_0} & {G_0} \\
				{F_1} & {G_1}
				\arrow["{\mu_0}", tail, from=1-1, to=1-2]
				\arrow["{f = g \restriction F_0}"', from=1-1, to=2-1]
				\arrow["{\mu_1}"', tail, from=2-1, to=2-2]
				\arrow["g", from=1-2, to=2-2]
			\end{tikzcd}
			\caption{The sub-object $\mu : F \Rightarrow G$.}
		\end{figure}
		\newline
		Note that an element $x \in G_0$ can be \emph{classified} in three ways:
		\begin{enumerate}[label=(\roman*)]
			\item $x \in F_0$.
			\item $x \notin F_0$ but $g(x) \in F_1$.
			\item $x \notin F_0$ and $g(x) \notin F_1$.
		\end{enumerate}
		 For this purpose, we introduce the set $\{0,\frac{1}{2},1\}$ and define $\psi : F_0 \rightarrow \{0,\frac{1}{2},1\}$ by:
		 \begin{equation*}
		 	\psi(x) =
		 	\begin{cases}
		 		1 & \text{if (i) holds}\\
		 		\frac{1}{2} & \text{if (ii) holds}\\
		 		0 & \text{if (iii) holds}
		 	\end{cases}       
		 \end{equation*}  
		  
		  \begin{figure}[h]
		  	\centering
		  	
		  	\begin{tikzpicture}
		  		\node (a) at (-0.4,0.4) {$F_0$};
		  		\node (b) at (3.5,0.4) {$G_0$};
		  		\node (A0) at (0,0) {$\bullet_{x_0}$};
		  		\node (B0) at (1.5,0) {$\bullet_{x_0}$};
		  		\node (C0) at (2.5,0) {$\bullet_{x_0}$};
		  		\draw (0.5,0) ellipse (2.75 and 1.5);
		  		\draw (0,0) ellipse (1 and 0.75);
		  		
		  		\draw (0.5,-4) ellipse (2.5 and 1.5);
		  		\draw (0,-4) ellipse (1 and 0.75);
		  		\node (A1) at (-0.5,-4) {$\bullet_{x_1}$};
		  		\node (B1) at (0.7,-4) {$\bullet_{x_1}$};
		  		\node (C1) at (2,-4) {$\bullet_{x_1}$};
		  		\node (c) at (-0.2,-4.5) {$F_1$};
		  		\node (d) at (3.3,-3.6) {$G_1$};
		  		
		  		\draw[->, dotted, line width=.01in] (A0) -- node[anchor=west] {$1$} (A1);
		  		\draw[->, dotted, line width=.01in] (B0) -- node[anchor=west] {$\frac{1}{2}$} (B1);
		  		\draw[->, dotted, line width=.01in] (C0) -- node[anchor=west] {$0$} (C1);
		  		\draw[->, line width=.01in] (b) -- node[anchor=west] {$g$} (d);
		  	\end{tikzpicture}
		  	\caption{The sub-object $F \xRightarrow{\mu} G$ and the function $\psi$.}
		  \end{figure}
		  
		  This suggests that: \newline
		   $\Omega(0) := \{0,\frac{1}{2},1\}$ and $\Omega(1) := \{0,1\}$ with $\Omega(\leq_0):= t: 0 \mapsto 0, \frac{1}{2} \mapsto 1, 1 \mapsto 1$. 
\newpage
		  The sub-object classifier is thus given by $\top : \textbf{1} \Rightarrow \Omega$:
		  
		  \begin{figure}[h]
		  	\centering
			\begin{tikzcd}
				{\{0\}} && {\{0,\frac{1}{2},1\}} \\
				& {{}} \\
				{\{0\}} && {\{0,1\}}
				\arrow[draw=none, from=1-1, to=2-2]
				\arrow["{t'}", from=1-1, to=1-3]
				\arrow["id"', from=1-1, to=3-1]
				\arrow["true"', from=3-1, to=3-3]
				\arrow["t", from=1-3, to=3-3]
			\end{tikzcd}
		  	\caption{$true: 0 \mapsto 1$, $t': 0 \mapsto 1$, $t:  0 \mapsto 0, \frac{1}{2} \mapsto 1, 1 \mapsto 1$.}
		  \end{figure}
		  
		  We have in addition to $\top$ other two truth-arrows $*,  \bot : \textbf{1} \Rightarrow \Omega$:
		  	\begin{figure}[h]
		  		\centering
		  			\begin{tikzcd}
		  			{\{0\}} && {\{0,\frac{1}{2},1\}} \\
		  			& {{}} \\
		  			{\{0\}} && {\{0,1\}}
		  			\arrow[draw=none, from=1-1, to=2-2]
		  			\arrow["{*'}", from=1-1, to=1-3]
		  			\arrow["id"', from=1-1, to=3-1]
		  			\arrow["true"', from=3-1, to=3-3]
		  			\arrow["t", from=1-3, to=3-3]
		  		\end{tikzcd}
		  		\caption{$* : \textbf{1} \Rightarrow \Omega$ with
		  			$true: 0 \mapsto 1$, $*': 0 \mapsto \frac{1}{2}$.}
		  	\end{figure}
		  
		  	\begin{figure}[h]
		  		\centering
		  		\begin{tikzcd}
		  			{\{0\}} && {\{0,\frac{1}{2},1\}} \\
		  			& {{}} \\
		  			{\{0\}} && {\{0,1\}}
		  			\arrow[draw=none, from=1-1, to=2-2]
		  			\arrow["{f'}", from=1-1, to=1-3]
		  			\arrow["id"', from=1-1, to=3-1]
		  			\arrow["false"', from=3-1, to=3-3]
		  			\arrow["t", from=1-3, to=3-3]
		  		\end{tikzcd}
		  		\caption{$\bot : \textbf{1} \Rightarrow \Omega$ with
		  			$false: 0 \mapsto 0$, $f': 0 \mapsto 0$.}
		  	\end{figure}
	
		  \newpage
		  The characteristic diagram is now a cube instead of the usual square:
		  
		  \begin{figure}[h]
		  	\centering
		  	\begin{tikzcd}
		  	{F_0} && {G_0} \\
		  	& {F_1} && {G_1} \\
		  	{\{0\}} && {\{0,\frac{1}{2},1\}} \\
		  	& {\{0\}} && {\{0,1\}}
		  	\arrow["true"', from=4-2, to=4-4]
		  	\arrow["id"', from=3-1, to=4-2]
		  	\arrow["{t'}"'{pos=0.6}, from=3-1, to=3-3]
		  	\arrow["t", from=3-3, to=4-4]
		  	\arrow["{\tau_0}", tail, from=1-1, to=1-3]
		  	\arrow[""{name=0, anchor=center, inner sep=0}, "f"', from=1-1, to=2-2]
		  	\arrow["g"', from=1-3, to=2-4]
		  	\arrow["{\tau_1}", tail, from=2-2, to=2-4]
		  	\arrow["{\chi_{F_1}}", from=2-4, to=4-4]
		  	\arrow[dashed, from=2-2, to=4-2]
		  	\arrow[dashed, from=1-1, to=3-1]
		  	\arrow["\psi"'{pos=0.6}, dotted, from=1-3, to=3-3]
		  		\arrow[draw=none, from=2-2, to=3-3]
		  			\arrow[ from=2-2, to=3-3, phantom, "\scalebox{1.5}{$\lrcorner$}",  very near start, color=black]
		  		\arrow[draw=none, from=1-1, to=0]
		  			\arrow[ from=1-1, to=0, phantom, "\scalebox{1.5}{$\lrcorner$}",  very near start, color=black]
		  	\end{tikzcd}
		  	\caption{The front and back faces of the cube are each pull-backs in $\mathbb{Set}$.}
		  \end{figure}
		  	  	  
	\end{ex}
	
The following holds: \footnote{to prove that  $\mathbb{Set}^{0 \rightarrow 1}$ is not a Boolean topos requires some further considerations that will be made later on.}

	\begin{remark}
		 $\mathbb{Set}^{0 \rightarrow 1}$ is neither bivalent (it has three truth-arrows) nor Boolean.
	\end{remark}

\newpage
A generalized version of $\mathbb{Set}^{0 \rightarrow 1}$, which will prove rather important later on, is that of the functor category of \emph{sets through time}:

\begin{ex}[Sets through time]
	Let $\mathbf{\omega}:=(\omega, \leq) $ be the poset category of natural numbers with their standard ordering $0 \xrightarrow{\leq_0} 1 \xrightarrow{\leq_1} 2 \xrightarrow{\leq_2}...$ which will be our so-called \emph{moments in time}. \newline 
	$\mathbb{Set}^\mathbf{\omega}$, a.k.a. the category of \emph{sets through time} has objects \emph{sequences} of sets and arrows \emph{commutative diagrams} between them:
	
	\begin{figure}[h]
		\centering
	\begin{tikzcd}
		{F:} & {F_0} & {F_1} & {F_2} & {...} & {F_m} & {F_{m+1}} & {...} \\
		{G:} & {G_0} & {G_1} & {G_2} & {...} & {G_m} & {G_{m+1}} & {...}
		\arrow["{F_{12}}"', from=1-3, to=1-4]
		\arrow[from=1-4, to=1-5]
		\arrow[from=1-5, to=1-6]
		\arrow["{F_{m\;m+1}}", from=1-6, to=1-7]
		\arrow["{F_{01}}"', from=1-2, to=1-3]
		\arrow[from=1-7, to=1-8]
		\arrow[from=2-2, to=2-3]
		\arrow["{G_{12}}"', from=2-3, to=2-4]
		\arrow[from=2-4, to=2-5]
		\arrow[from=2-5, to=2-6]
		\arrow["{G_{m\;m+1}}"', from=2-6, to=2-7]
		\arrow[from=2-7, to=2-8]
		\arrow["{\tau_{m+1}}", dashed, from=1-7, to=2-7]
		\arrow["{\tau_m}", dashed, from=1-6, to=2-6]
		\arrow["{\tau_2}", dashed, from=1-4, to=2-4]
		\arrow["{\tau_1}", dashed, from=1-3, to=2-3]
		\arrow["{\tau_0}", dashed, from=1-2, to=2-2]
		\arrow["\tau", squiggly, from=1-1, to=2-1]
	\end{tikzcd}
		\caption{$\tau : F \Rightarrow G$.}
	\end{figure}  
	
	 It is worth pausing to give an intuition due to \emph{John C. Baez} for the notion of \emph{sets through time}:
	
	\begin{remark}
		Imagine a set of theorems proven by an infallible mathematician at various times $0,1,2.. \in \omega$.\newline
		As time passes, in steps from $0$ to $1$ and from $1$ to $2$ and so on, This set can get new elements as the mathematician produces new theorems.\newline
		Also, two distinct theorems can \emph{merge} into one if an equivalence is found between them. \newline
		However, we can never remove an element from the set as once a theorem has been proven it remains so forever and cannot be dis-proven.  
	\end{remark}
	
	Notice that: 
	Any non-empty up-set $S\subseteq \omega$ has a minimum $m_S$ and so we have:
	\begin{equation*}
		S = [m_S) = \{m_S, m_S +1, m_S +2...\}.
	\end{equation*}
	 i.e., $S$ coincides with the principal up-set generated by its minimum. \newline
	
	Now let us add a symbol for \emph{infinity} to replace the empty-set so that $S=\emptyset$ becomes $S=\{\infty\}$. \newline
	So we work with $\mathbf{\omega}^+ := \mathbf{\omega} \cup \{\infty\}$.\newline
	
	The terminal object, a.k.a. \emph{the infinite telephone pole}  is the constant functor $\textbf{1}$ :
	\begin{gather*}
		\forall m\in \omega : \textbf{1}(m) := \{0\}. \\
		\textbf{1}(\leq_m) := id_{\{0\}}. 
	\end{gather*}
	 
	
	For the sub-object classifier, $\Omega$ and $\top : \textbf{1} \Rightarrow \Omega$ are defined in the following way for each $m \in \mathbf{\omega}$ : \footnote{we make use of the same notation we introduced for the previous example.}
	\begin{gather*}
		m \;\mapsto\; \Omega_m := [m) = \{m,m+1,..,\infty\}. \\
		m \xrightarrow{\leq} n \;\mapsto\;\; [m)  \xrightarrow{\Omega_{m\;n}} [n) \text{ where } 
		\Omega_{m\;n}(p) :=
		\begin{cases}
			n & \text{if } m \leq p \leq n \\
			p & \text{if } n \leq p \\
			\infty & \text{if } p = \infty
		\end{cases}. \\
		 \top_m: \{0\} \rightarrow  [m) \text{ where } \top_m(0) := m.
	\end{gather*} 
	\newpage
	We can display $\Omega$ as:
	
	\begin{figure}[h]
		\centering
	\begin{tikzcd}
		{\Omega_m =} & {\{} & {m,} & {m+1,} & {m+2,} & {...} & \infty & {\}} \\
		{\Omega_{m+1} =} & {\{} & {m+1,} & {m+2,} & {...} & {...} & \infty & {\}} \\
		{\Omega_{m+2} =} & {\{} & {m+2,} & {...} & {...} & {...} & \infty & {\}} \\
		&& {} & {} & {...} & {...} \\
		{\Omega_n =} & {\{} & {n,} & {n+1,} & {...} & {...} & \infty & {\}}
		\arrow["{\Omega_{m \; m+1}}"', from=1-1, to=2-1]
		\arrow[maps to, from=1-3, to=2-3]
		\arrow[maps to, from=1-4, to=2-3]
		\arrow[maps to, from=1-5, to=2-4]
		\arrow["{\Omega_{m+1 \; m+2}}"', from=2-1, to=3-1]
		\arrow[maps to, from=2-3, to=3-3]
		\arrow[maps to, from=2-4, to=3-3]
		\arrow["{\Omega_{m+2 \; n}}"', dotted, from=3-1, to=5-1]
		\arrow[dashed, maps to, from=3-3, to=4-3]
		\arrow[dashed, maps to, from=4-3, to=5-3]
		\arrow[dashed, maps to, from=4-4, to=5-3]
		\arrow[dashed, maps to, from=1-6, to=2-5]
		\arrow[dashed, maps to, from=2-5, to=3-4]
		\arrow[dashed, maps to, from=3-4, to=4-3]
		\arrow[maps to, from=1-7, to=2-7]
		\arrow[maps to, from=2-7, to=3-7]
		\arrow[dashed, maps to, from=3-7, to=5-7]
	\end{tikzcd}
		\caption{$\Omega$ displayed in a few of its components.}
	\end{figure}
	
	\newpage
	The characteristic arrow $\chi_\tau$ of a sub-object $\tau : F \Rightarrow G$ (as before w.l.o.g. we assume $F_m \subseteq G_m$) is given by the so-called \emph{time till truth}:
	\begin{equation*}
		(\chi_\tau)_m(x) := \begin{cases}
			min\{n : n\geq m|\; G_{m\;n}(x) \in F_n \} & \text{ if such $n$ exists. } \\
			\infty & \text{ if } G_{m\;n}(x) \notin F_n \text{ for every } n \geq m. 
		\end{cases}
	\end{equation*} 
	
	This is rendered visually as:
	
	\begin{figure}[h]
		\centering
		
		\begin{tikzpicture}
			\node (a) at (0,0.3) {$F_m$};
			\node (b) at (3.5,0) {$G_m$};
			\node (C0) at (2.3,0) {$\bullet_{x_m}$};
			\draw (0.5,0) ellipse (2.45 and 1.2);
			\draw (0,0) ellipse (1.2 and 0.9);
			
			\draw (0.5,-3) ellipse (2.3 and 1.2);
			\draw (0,-3) ellipse (1 and 0.7);
			\node (C1) at (2.2,-3) {$\bullet_{x_{m+1}}$};
			\node (c) at (0,-3) {$F_{m+1}$};
			\node (d) at (3.5,-3) {$G_{m+1}$};
			
			\draw (0.5,-6) ellipse (2.15 and 1.3);
			\draw (0,-6) ellipse (1.2 and 0.9);
			\node (C2) at (2,-6) {$\bullet_{x_{m+2}}$};
			\node (e) at (0,-6) {$F_{m+2}$};
			\node (f) at (3.3,-6) {$G_{m+2}$};
			
			\draw (0.5,-9) ellipse (2.2 and 1.3);
			\draw (0,-9) ellipse (0.9 and 0.7);
			\node (C3) at (0.5,-9) {$\bullet_{x_{n}}$};
			\node (e) at (0,-9) {$F_{n}$};
			\node (f) at (3.2,-9) {$G_{n}$};
			
			\draw[|->, line width=.01in] (C0) -- node[anchor=east] {$G_{m \; m+1}$} (C1);
			\draw[|->, line width=.01in] (C1) -- node[anchor=east] {$G_{m+1 \;m+2}$} (C2);
			\draw[|->, dotted, line width=.01in] (C2) -- node[anchor=east] {$G_{m+2 \;n}$} (C3);
		\end{tikzpicture}
		\caption{Here $(\chi_\tau)_m(x)=n$. \newline
			The sub-object $F \xRightarrow{\tau} G$ is displayed in some of its components $F_m \subseteq G_m$,  $F_{m+1} \subseteq G_{m+1},$ and $F_n \subseteq G_n$  and the transition function $G_{m\;n}$ is shown in its constituents $G_{m\;n}= G_{m\;m+1} G_{m+1\;m+2}G_{m+2 \;n}$.}
	\end{figure}
	
	\newpage
	Notice that we now have a countable infinity of \emph{truth-values} between 
	$\bot : \textbf{1} \Rightarrow \Omega $:
	\begin{gather*}
		\forall m \in \omega :
		(\bot)_m(0) := \{\infty\}. 
	\end{gather*}
	and $\top : \textbf{1} \Rightarrow \Omega$ given by $\{ *_q \}_{q\in \omega}$ where each
	$*_q : \textbf{1} \Rightarrow \Omega$ is given by:
	\begin{gather*}
		\forall m \in \omega :
		(*_q)_m(0) := \begin{cases}
			q & \text{ if }m \leq q. \\
			m & \text{ if }m > q. 
		\end{cases}.
	\end{gather*}
	
	
	
\end{ex}





	\newpage
${}$ \newpage






 
 \chapter{Topos Semantics II}
 \section{First Order}
\label{chapter4}

We move on from the \emph{Propositional} level to the \emph{Predicate} or \emph{First Order} level of logic following \cite{goldblatt}'s \emph{first-principles} approach before changing course to \cite{lambekscott}'s \emph{BKJ Semantics} in order to better realize quantifiers in our topos of study.\newline
For our purposes we have chosen at this stage not to add multiple constants or predicates and to omit function symbols\footnote{any function symbol could be substituted with a relation or predicate symbol which specifies the function's \emph{graph}.}. Also we start off with a single \emph{sort} or \emph{type} of variables.\newline
\newline
We have also decided, for the purpose of this work, not to mention Soundness \& Completeness Theorems which generalize to first order level soundness \& completeness of topoi (\ref{soundcompl}) with respect to intuitionistic logic.\newline We leave as references chapter XI of \cite{goldblatt} and part II of \cite{lambekscott}. 


\newpage
\subsection{First Order Logic and Topoi}

In order to interpret an elementary language $\mathcal{L}$ in a topos $\mathcal{E}$ it is necessary first to reformulate Tarski Semantics for $\mathcal{L}$-terms and $\mathcal{L}$-formulae (which from now on will be called just 'terms' and 'formulae') in a given \emph{context}.
\begin{definition}[\emph{context}]
	We specify a \emph{context} for a formula $\phi$ by fixing an integer $m \geq 1$ which will be called \emph{appropriate} to $\phi$ if all the variables that occur in $\phi$ free or bound are all elements of the list $\{x_1,..,x_m\}$. We will refer to this $\phi$ as a \emph{formula-in-context}. \newline
	Similarly a \emph{term-in-context}  is a term \textbf{t} in which all occurrences of variables in \textbf{t} belong to $\{x_1,..,x_m\}$. 
\end{definition}

We re-define satisfaction for $\phi$ by $m$-length sequences $\{ \textbf{a}_1,..,\textbf{a}_m \}$ by requiring that $\mathcal{M} \vDash \phi[\textbf{a}_1,..,\textbf{a}_n]$ iff $\mathcal{M} \vDash \phi[\textbf{y}]$ for some assignment \textbf{y} for which $\textbf{y}_i = \textbf{a}_i$ whenever $x_i$ is free in $\phi$.  \newline
Note that given an $\mathcal{L}$-model $\mathcal{M}= \langle \textbf{A}, \textfrak{P}, \textfrak{c} \rangle$ and a \emph{context} $m \geq 1$, each formula-in-context $\phi$ determines a sub-set $\phi^m \subseteq \textbf{A}^m$ namely the set of all m-tuples satisfying $\phi$:
\begin{equation*}
	\phi^m = \{(a_1,..,a_m) : \mathcal{M} \vDash \phi[a_1,..,a_m]\}.
\end{equation*}
Note also that by this definition:
\begin{align*} 
	(\neg \phi)^m &=  \textbf{A} \setminus \phi^m \\ 
	(\phi \land \psi)^m &=  \phi^m \cap \psi^m. \\
	(\phi \lor \psi)^m &=  \phi^m \cup \psi^m. \\
	(\phi \Rightarrow \psi)^m &=  (\textbf{A} \setminus \phi^m) \cup \psi^m. \\ etc..
\end{align*}
This is a translation of $\mathcal{L}$-formulae into sub-sets or sub-objects of the product domain $\textbf{A}^m$ which in turn can be replaced by their characteristic functions $\llbracket \phi^m \rrbracket : \textbf{A}^m \rightarrow \textbf{2}$. \newline
With this in mind, we can finally generalize from $\mathbb{Set}$ to a generic topos.
\newline
\newline
Let $\mathcal{E}$ be a topos and a fixed $\mathcal{E}$-object  $\textfrak{a}$ \footnote{the fixed object corresponds to a fixed type or sort.}.\newline

We first give some preliminary definitions:
\begin{definition}[$\Delta_\textfrak{a}$ and $\delta_\textfrak{a}$ ]
	$\Delta_\textfrak{a} : \textfrak{a} \rightarrowtail \textfrak{a} \times \textfrak{a}$ is the product arrow $id_\textfrak{a} \times id_\textfrak{a}$.\newline
	$\delta_\textfrak{a} : \textfrak{a} \times \textfrak{a} \rightarrow \Omega$ is the characteristic arrow of $\Delta_\textfrak{a}$. 
\end{definition}

\begin{definition}[$true_\textfrak{o}$]
	For any $\mathcal{E}$-object \textfrak{o}, \newline
	$true_\textfrak{o}$ is the composite arrow $true \;\circ\; !_\textfrak{o}$ \newline
	(where $!_\textfrak{o}$ as usual is the unique arrow from \textfrak{o} to the terminal \textbf{1}).
	
\end{definition}


We are finally ready to define a topos-model or $\mathcal{E}$-model for First Order Logic:
(From now on we fix an appropriate $m \geq 1$ context).

\begin{definition}[$\mathcal{E}$-model] \footnote{note that we can generalize this definition by taking multiple objects in I (multiple sorts), multiple predicates in II and generalized elements in III.}
	An $\mathcal{E}$-model for $\mathcal{L}$ is a structure \textfrak{M}=$\langle \textfrak{a}, \textfrak{p}, \textfrak{f}_c \rangle$ where
	\begin{enumerate}[label=\Roman*]
		\item 	\;\;\textfrak{a} is an $\mathcal{E}$-object for the \emph{domain} that is \emph{not empty} i.e., $\mathcal{E}(\textbf{1},\textfrak{a})\neq \emptyset$.
		\item 	 \;\;\textfrak{p} 	:	$\textfrak{a}^n \rightarrow \Omega$ is an $\mathcal{E}$-arrow for the \emph{predicate/relation}. \footnote{we will assume $0 \leq n \leq m$.}
		\item 	\;\;$\textfrak{f}_c$	:	$\textbf{1} \rightarrow \textfrak{a}$ is an $\mathcal{E}$-\emph{element} of \textfrak{a} for the \emph{particular individual}.
	\end{enumerate}
\end{definition}

\begin{remark}
	Notice that if the arity of the predicate-arrow is $n=0$ we recover \emph{propositions}, if the arity is $n=1$ we obtain unary \emph{predicates} and if $n>1$ n-ary \emph{relations}. 
\end{remark}
We \emph{realize} i.e., interpret the terms \textbf{t} as arrows $\textfrak{a}^m \rightarrow \textfrak{a}$ :

\begin{definition}[$\llbracket \textbf{t} \rrbracket^m$]
	\begin{equation*}
		\llbracket \textbf{t} \rrbracket^m =
		\begin{cases}
			pr_i^m : \textfrak{a}^m \rightarrow \textfrak{a} & \text{if \textbf{t} is the variable $x_i$}\\
			\textfrak{f}_c \;\circ\; !_\textfrak{a} : \textfrak{a}^m \rightarrow \textfrak{a}  & \text{if \textbf{t} is the constant \textbf{c}}
		\end{cases}   
	\end{equation*}  
	For $m>1$ the $m$ variables in context $\{x_i\}_{i=1}^m$ are realized by the $m$ projections $pr_i^m$ from $\textfrak{a}^m$ to \textfrak{a}. \newline
	If $m=1$ the only variable in context $x = x_1$ is realized by the identity arrow $id_\textfrak{a}$.
\end{definition}


We now define for each $\mathcal{L}$-formula $\phi$ a realization/interpretation $\llbracket  \phi \rrbracket^m$ as an arrow $\textfrak{a}^m \rightarrow \Omega$:
\newline
(From now on $\llbracket  \phi \rrbracket, \llbracket  \textbf{t} \rrbracket $  will be used instead of $\llbracket  \phi \rrbracket^m, \llbracket  \textbf{t} \rrbracket^m$ if the context is already specified and there is no ambiguity).

\begin{definition}[$\llbracket  \phi \rrbracket$]
	The atomic formulae admit the following \emph{realizations}:
	\begin{enumerate}
		\item $\llbracket \textbf{t} \approx \textbf{u}  \rrbracket = \delta_\textfrak{a} \circ (\llbracket \textbf{t} \rrbracket \times \llbracket \textbf{u} \rrbracket$)  .
		\begin{figure}[h]
			\centering
			\begin{tikzcd}
				{\textfrak{a}^m} && {\textfrak{a}^2} \\
				\\
				&& \Omega
				\arrow["{\llbracket \textbf{t} \rrbracket \times \llbracket \textbf{u} \rrbracket}"', from=1-1, to=1-3]
				\arrow["{\delta_\textfrak{a}}"', from=1-3, to=3-3]
				\arrow["{\llbracket \textbf{t} \approx \textbf{u} \rrbracket}"', from=1-1, to=3-3]
			\end{tikzcd}\
		\end{figure}
		
		\item $\llbracket \textbf{P}(\textbf{t}_1,..,\textbf{t}_n) \rrbracket = \textfrak{p} \circ (\llbracket \textbf{t}_1 \rrbracket \times ...\times\llbracket \textbf{t}_n \rrbracket)$.
		\begin{figure}[h]
			\centering
			\begin{tikzcd}
				{a^m} && {a^n} \\
				\\
				&& \Omega
				\arrow["{\llbracket \textbf{t}_1 \rrbracket \times ...\times\llbracket \textbf{t}_n \rrbracket}", from=1-1, to=1-3]
				\arrow["p"', from=1-3, to=3-3]
				\arrow["{\llbracket \textbf{P}(\textbf{t}_1,..,\textbf{t}_n) \rrbracket}"', from=1-1, to=3-3]
			\end{tikzcd}
		\end{figure}
		
	\end{enumerate}
	The rest follow by induction:
	\begin{enumerate}
		\setcounter{enumi}{2}
		\item $\llbracket \phi \land \psi \rrbracket = \llbracket \phi \rrbracket \land \llbracket \psi \rrbracket = \land \circ (\llbracket \phi \rrbracket \times \llbracket \psi \rrbracket) $.
		\begin{figure}[h]
			\centering
			\begin{tikzcd}
				{a^m} && {\Omega \times \Omega} \\
				\\
				&& \Omega
				\arrow["{\llbracket \phi \rrbracket \times \llbracket \psi \rrbracket}", from=1-1, to=1-3]
				\arrow["\land"', from=1-3, to=3-3]
				\arrow["{\llbracket \phi \land \psi \rrbracket}"', from=1-1, to=3-3]
			\end{tikzcd}\
		\end{figure}
		\item $\llbracket \phi \lor \psi \rrbracket = \llbracket \phi \rrbracket \lor \llbracket \psi \rrbracket = \lor \circ (\llbracket \phi \rrbracket \times \llbracket \psi \rrbracket) $.
		
		\item $\llbracket \neg \phi \rrbracket = \neg \circ \llbracket \phi \rrbracket$.
		
		\item $\llbracket \phi \Rightarrow \psi \rrbracket = \llbracket \phi \rrbracket \Rightarrow \llbracket \psi \rrbracket =\; \Rightarrow \circ (\llbracket \phi \rrbracket \times \llbracket \psi \rrbracket) $.
	\end{enumerate}		

\end{definition}

We can now define $\mathcal{E}$-validity for a formula $\phi$ starting by what it means for \textfrak{M} to \emph{model} $\phi$ denoted by $\textfrak{M} \vDash_\mathcal{E} \phi$. \newline
Let $\phi = \phi(x_{i_1},..x_{i_n})$ be a formula-in-context and take any arrow $g: \textfrak{a}^n \rightarrow \textfrak{a}$ 	,we construct a product arrow $f: p_1 \times .. \times p_m$ where: ($pr_k^n$ as usual denotes the k-th projection from $\textfrak{a}^n$)
\begin{equation*}
	p_i =
	\begin{cases}
		pr_k^n : \textfrak{a}^n \rightarrow \textfrak{a} & \text{if } j=i_k \text{ for some } \  1 \leq k \leq n.\\
		g  & \text{otherwise}.
	\end{cases}   
\end{equation*}  
\textfrak{M} is thus an "$\mathcal{E}$-model of $\phi = \phi(x_{i_1},..x_{i_n})$" i.e., $\textfrak{M} \vDash_\mathcal{E} \phi$ if:

\begin{definition}[\textfrak{M} models $\phi(x_{i_1},..x_{i_n})$]
	\begin{equation*}
		\textfrak{M} \vDash_\mathcal{E} \phi\text{ iff }\llbracket \phi \rrbracket_\textfrak{M} = true_{\textfrak{a}^n}.
	\end{equation*}
		Where the arrow $\llbracket \phi \rrbracket_\textfrak{M} = true_{\textfrak{a}^n} : \textfrak{a}^n \rightarrow \Omega$ is defined as $ true_{\textfrak{a}^m} \circ f$:
	\begin{figure}[h]
		\centering
		\begin{tikzcd}
			{\textfrak{a}^n} && {\textfrak{a}^m} \\
			\\
			&& \Omega
			\arrow["f", from=1-1, to=1-3]
			\arrow["{\llbracket \phi \rrbracket}", from=1-3, to=3-3]
			\arrow["{\llbracket \phi \rrbracket_\textfrak{M}}"', from=1-1, to=3-3]
		\end{tikzcd}\
	\end{figure}
\end{definition}

By the categorical properties of the arrows $true_\textfrak{o}$ \footnote{by using the fact that any arrow "that factors through \emph{true} is \emph{true}".} we find that:

\begin{remark}
	$\llbracket \phi \rrbracket_\textfrak{M} = true_{\textfrak{a}^n}$ iff $\llbracket \phi \rrbracket = true_{\textfrak{a}^m}$ .
\end{remark}
Thus:
\begin{definition}[$\mathcal{E}$-validity] Let $\phi = \phi(x_{i_1},..x_{i_n})$ be a formula-in-context,	then: 
	\begin{equation*}
		\textfrak{M} \vDash_\mathcal{E} \phi \;\text{	iff	}\; \llbracket \phi \rrbracket = true_{\textfrak{a}^m} .
	\end{equation*}
	The formula $\phi$ is $\mathcal{E}$-valid i.e., $ \vDash_\mathcal{E} \phi $ if for every $\mathcal{E}$-model \textfrak{M} one has $ \textfrak{M}\vDash_\mathcal{E} \phi $.
\end{definition}



\subsection{\hl{Another look at $\mathcal{G}_3$ through Predicates}}
\label{anotherlook}

We continue making some considerations on our topos of \emph{bushes}/$\mathbb{FF}_2$ like the following:
\newline
Recall our discourse about external and internal topos-logic in the propositional case, we left off with the result about \emph{bushes}/$\mathbb{FF}_2$ being bivalent and non-Boolean.
\newline
The culprit was the definition of propositional semantics for topoi.
\newline
The propositional truth-values are given by the two arrows $\top$ and $\bot$ of $\mathbb{FF_2}(\textbf{1},\Omega)$ that pick out the two roots 't' and 'f' of $\Omega= \textbf{1} + \textbf{1}_\bot$: 

\begin{figure}[h]
	\centering
	\begin{subfigure}[h]{0.2\textwidth}
		\begin{tikzcd}
			{\textcolor{OliveGreen}{\bigcdot}} && \begin{tikzpicture}[scale=0.4]
				\node (A) at (0,0) {\textcolor{red}{\textbf{f}}};
				\node (B) at (3,0) {\textcolor{OliveGreen}{\textbf{t}}};
				\node (C) at (3,3) {\textcolor{cyan}{$*$}};
				\draw[cyan, line width=.03in] (B) -- (C);
			\end{tikzpicture}
			\arrow["true", from=1-1, to=1-3]
		\end{tikzcd}
		\caption{$\top : \mathbf{1} \rightarrow \Omega$.}
	\end{subfigure}
	\hfil
	\centering
	\begin{subfigure}[h]{0.2\textwidth}
		\begin{tikzcd}
			{\textcolor{red}{\bigcdot}} && \begin{tikzpicture}[scale=0.4]
				\node (A) at (0,0) {\textcolor{red}{\textbf{f}}};
				\node (B) at (3,0) {\textcolor{OliveGreen}{\textbf{t}}};
				\node (C) at (3,3) {\textcolor{cyan}{$*$}};
				\draw[cyan, line width=.03in] (B) -- (C);
			\end{tikzpicture}
			\arrow["false", from=1-1, to=1-3]
		\end{tikzcd}
		\caption{$\bot : \mathbf{1} \rightarrow \Omega$.}
	\end{subfigure}
\end{figure} 


However, as in the case of $\mathbb{Set}$, truth values are meant to be \emph{generalized elements} of the sub-object classifier $\Omega$. The arrows from \textbf{1} to $\Omega$ are clearly insufficient to obtain the top element $*$ of $\textbf{1}_\bot \subset \Omega$. This is remedied if we take arrows from $\textbf{1}_\bot$. 
The object $\textbf{1}_\bot$ in fact is shown to be a \emph{representing object} for the category of \emph{bushes}.
\newline \newline
The hom-set $\mathbb{FF_2}(\textbf{1}_\bot,\Omega)$ has exactly three arrows which we denote by $\textfrak{p}_t, \textfrak{p}_f, \textfrak{p}_*$ each determined by the image of the top element of $\textbf{1}_\bot$.     


\begin{figure}[h]
	\centering
	\begin{tikzcd}
		\begin{tikzpicture}[scale=0.4]
			\node (A) at (0,0) {\textcolor{OliveGreen}{$\bullet$}};
			\node (B) at (0,3) {\textcolor{OliveGreen}{$\bullet$}};
			\draw[OliveGreen, line width=.03in] (A) -- (B);
		\end{tikzpicture} && \begin{tikzpicture}[scale=0.4]
			\node (A) at (0,0) {\textcolor{red}{\textbf{f}}};
			\node (B) at (3,0) {\textcolor{OliveGreen}{\textbf{t}}};
			\node (C) at (3,3) {\textcolor{cyan}{$*$}};
			\draw[cyan, line width=.03in] (B) -- (C);
		\end{tikzpicture}
		\arrow["\textfrak{p}_t", from=1-1, to=1-3]
	\end{tikzcd}
	\caption{$\textfrak{p}_t : \mathbf{1}_\bot \rightarrow \Omega$. }
\end{figure}	

\begin{figure}[h]
	\centering
	\begin{tikzcd}
		\begin{tikzpicture}[scale=0.4]
			\node (A) at (0,0) {\textcolor{red}{$\bullet$}};
			\node (B) at (0,3) {\textcolor{red}{$\bullet$}};
			\draw[red, line width=.03in] (A) -- (B);
		\end{tikzpicture} && \begin{tikzpicture}[scale=0.4]
			\node (A) at (0,0) {\textcolor{red}{\textbf{f}}};
			\node (B) at (3,0) {\textcolor{OliveGreen}{\textbf{t}}};
			\node (C) at (3,3) {\textcolor{cyan}{$*$}};
			\draw[cyan, line width=.03in] (B) -- (C);
		\end{tikzpicture}
		\arrow["\textfrak{p}_f", from=1-1, to=1-3]
	\end{tikzcd}
	\caption{$\textfrak{p}_f : \mathbf{1}_\bot \rightarrow \Omega$. }
\end{figure}	


\begin{figure}[h]
	\centering
	\begin{tikzcd}
		\begin{tikzpicture}[scale=0.4]
			\node (A) at (0,0) {\textcolor{OliveGreen}{$\bullet$}};
			\node (B) at (0,3) {\textcolor{cyan}{$\bullet$}};
			\draw[cyan, line width=.03in] (A) -- (B);
		\end{tikzpicture} && \begin{tikzpicture}[scale=0.4]
			\node (A) at (0,0) {\textcolor{red}{\textbf{f}}};
			\node (B) at (3,0) {\textcolor{OliveGreen}{\textbf{t}}};
			\node (C) at (3,3) {\textcolor{cyan}{$*$}};
			\draw[cyan, line width=.03in] (B) -- (C);
		\end{tikzpicture}
		\arrow["\textfrak{p}_*", from=1-1, to=1-3]
	\end{tikzcd}
	\caption{$\textfrak{p}_* : \mathbf{1}_\bot \rightarrow \Omega$. }
\end{figure}	

Note that these arrows can be viewed as realizations of unary \emph{predicates} which we name: $\textbf{P}_t,\textbf{P}_f,\textbf{P}_*$. 
\newline
Consider the following $\mathcal{E}$-model \textfrak{X} with specified context $m = 1$:
\begin{ex} (m=1)
	\begin{gather*}
		\textfrak{X}= \langle \textbf{1}_\bot, \{\textfrak{p}_t, \textfrak{p}_f, \textfrak{p}_*\}, \textfrak{f}_c \rangle. \\ \\
		\llbracket x \rrbracket = id_{\textbf{1}_\bot}: \textbf{1}_\bot \rightarrow \textbf{1}_\bot. \\ \llbracket \textbf{c} \rrbracket = \textfrak{f}_c \circ !_{\textbf{1}_\bot} : \textbf{1}_\bot \rightarrow \textbf{1}_\bot \; \text{ i.e., the constant map on the root of $\textbf{1}_\bot$}.  \\
		\llbracket \textbf{P}_t (x) \rrbracket = \textfrak{P}_t \circ \llbracket x \rrbracket = \textfrak{P}_t : \textbf{1}_\bot \rightarrow \Omega.\\
		\llbracket \textbf{P}_f (x) \rrbracket = \textfrak{P}_f \circ \llbracket x \rrbracket = \textfrak{P}_f : \textbf{1}_\bot \rightarrow \Omega.\\
		\llbracket \textbf{P}_* (x) \rrbracket = \textfrak{P}_* \circ \llbracket x \rrbracket = \textfrak{P}_* : \textbf{1}_\bot \rightarrow \Omega. 
	\end{gather*}
\end{ex}
Note that by construction $\textfrak{P}_t = true_{\textbf{1}_\bot}$ and so we have $\textfrak{X} \vDash_\mathcal{E} \textbf{P}_t(x) $. 
\newline
If we now compose these predicates with truth-arrows, we obtain:
\begin{gather*}
	\llbracket \neg \textbf{P}_t (x) \rrbracket = \neg \circ \llbracket \textbf{P}_t (x) \rrbracket = \neg \circ  \textfrak{P}_t = \textfrak{P}_f = \llbracket \textbf{P}_f (x) \rrbracket.\\
	\llbracket \neg \textbf{P}_* (x) \rrbracket = \neg \circ \llbracket \textbf{P}_* (x) \rrbracket = \neg \circ  \textfrak{P}_* = \textfrak{P}_f = \llbracket \textbf{P}_f (x) \rrbracket.\\
	\llbracket \neg \textbf{P}_f (x) \rrbracket = \neg \circ \llbracket \textbf{P}_f (x) \rrbracket = \neg \circ  \textfrak{P}_f = \textfrak{P}_t = \llbracket \textbf{P}_t (x) \rrbracket.
\end{gather*}

\begin{figure}[h]
	\centering
	\begin{tikzcd}
		\begin{tikzpicture}[scale=0.4]
			\node (A) at (0,0) {\textcolor{red}{$\bullet$}};
			\node (B) at (0,3) {\textcolor{red}{$\bullet$}};
			\draw[red, line width=.03in] (A) -- (B);
		\end{tikzpicture} && \begin{tikzpicture}[scale=0.4]
			\node (A) at (0,0) {\textcolor{OliveGreen}{\textbf{f}}};
			\node (B) at (3,0) {\textcolor{red}{\textbf{t}}};
			\node (b) at (3.5,-0.5) {\textcolor{black}{$\bullet$}};
			\node (C) at (3,3) {\textcolor{red}{$*$}};
			\node (c) at (3.5,2.5) {\textcolor{black}{$\bullet$}};
			\draw[red, line width=.03in] (B) -- (C);
		\end{tikzpicture}\\
		\\
		&& \begin{tikzpicture}[scale=0.4]
			\node (A) at (0,0) {\textcolor{red}{\textbf{f}}};
			\node (B) at (3,0) {\textcolor{OliveGreen}{\textbf{t}}};
			\node (C) at (3,3) {\textcolor{cyan}{$*$}};
			\draw[cyan, line width=.03in] (B) -- (C);
		\end{tikzpicture}
		\arrow["{\llbracket \textbf{P}_*(x)\rrbracket}", from=1-1, to=1-3]
		\arrow["\neg", from=1-3, to=3-3]
		\arrow["{\llbracket \neg\textbf{P}_*(x)\rrbracket}"', from=1-1, to=3-3]
	\end{tikzcd}
	\caption{$\llbracket \neg\textbf{P}_*(x)\rrbracket : \textbf{1}_\bot \rightarrow \Omega$.  The usual coloring notation is applied for this arrow. The images of the nodes of $\textbf{1}_\bot$ by $\llbracket \textbf{P}_*(x)\rrbracket$ are also shown as black bullets.}
\end{figure}	

\begin{gather*}
	\llbracket \textbf{P}_* (x) \land \textbf{P}_t (x) \rrbracket = \land \circ (\llbracket \textbf{P}_* (x) \rrbracket \times \llbracket \textbf{P}_t (x) \rrbracket) = \textfrak{P}_*. \\
	etc.. \\ \\
	\llbracket \textbf{P}_* (x) \lor \textbf{P}_t (x) \rrbracket = \lor \circ (\llbracket \textbf{P}_t (x) \rrbracket \times \llbracket \textbf{P}_* (x) \rrbracket) = \textfrak{P}_t. \\
	etc.. \\ \\
	\llbracket \textbf{P}_t (x) \Rightarrow \textbf{P}_* (x) \rrbracket = \Rightarrow \circ (\llbracket \textbf{P}_t (x) \rrbracket \times \llbracket \textbf{P}_* (x) \rrbracket) = \textfrak{P}_*. \\
	etc..
\end{gather*}
\newpage
\begin{remark}
	Again we find the (propositional) truth functions of $\mathcal{G}_3$.
	Though this time \emph{internally} by composition of the \emph{new} truth arrows for First Order Logic that we defined. In this case, these arrows are of the form $\textbf{1}_\bot \rightarrow \Omega$ which is precisely what we need for generalized elements of $\mathbb{FF_2}$. 
\end{remark}
 
Notice that $\textbf{P}_t$,$\textbf{P}_*$ and $\textbf{P}_f$ can also be viewed as characteristic arrows for the sub-forests $\{\textbf{1}_\bot\}$, $\{\bot\}$ and $\emptyset$ respectively.
\newline
This leads to the following consideration:
\begin{remark}
	What we found with these predicates is an application of the duality between \emph{bushes} $\mathbb{FF_2}$ and $(\mathbb{G_3})_{fin}$. \newline
	 $Sub(\textbf{1}_\bot)$ is isomorphic (as Gödel Algebras) to $C_3$ the 3-element chain which semantically characterizes our logic $\mathcal{G}_3$ i.e., $C_3 \models \mathcal{G}_3$.
\end{remark}


\newpage

\section{\hl{What about Quantifiers?}}
\label{whataboutquant}
To treat First Order Logic and \emph{quantifiers} we leave aside the method in \cite{goldblatt} we introduced earlier and instead apply a new approach outlined in \cite{lambekscott}, inspired by \emph{type theory} and the \emph{Curry-Howard} correspondence between objects and types, and arrive at a few conclusions.

\subsection{A Type-Theoretic Approach}

We introduce some new jargon:\newline
(The notation $\langle a,b \rangle$ is equivalent to $a \times b$.)	

\begin{itemize}
	\item The \emph{special} type $\Omega$ is the object $\Omega$.
	\item Variables of type $A_i$  i.e., $x_i:A_i$ are realized as \emph{indeterminate} arrows $1 \xrightarrow{x_i} A_i.$ 
	\item From  ($a:A$) $1 \xrightarrow{a} A$  and ($b:B$) $1 \xrightarrow{b} B$  one obtains  
	($\langle a,b \rangle : A \times B$)  $1 \xrightarrow{\langle a,b \rangle} A \times B$.
	\item $a=a'$ denotes \emph{internal} equality\footnote{as a first order relation.} and is realized as\footnote{recall that $\delta_A$  characteristic arrow of $\langle id_{A},id_{A} \rangle$.} 
	$1\xrightarrow{\langle a,a' \rangle}A \times A \xrightarrow{\delta_A}\Omega$.
	\item $\cdot = \cdot$ instead denotes \emph{external} equality\footnote{as equality of arrows.} between arrows in the topos.
	\item $\textfrak{T} \models p$ means that the topos \textfrak{T} satisfies the proposition $p$: In which case $p \cdot = \cdot \top$ as arrows in \textfrak{T}.	 
\end{itemize}
	
\begin{definition}[realization of $\phi(x)$]
	A predicate formula $\phi(x)$ which takes as argument an $x:A$ is realized as $1 \xrightarrow{x} A \xrightarrow{f} \Omega$ i.e., $\ulcorner \phi(x) \urcorner \equiv fx$ where $f = \ulcorner \phi \urcorner$ is the realization of the predicate $\phi$.
\end{definition}

Generalizing to an arbitrary context:

\begin{definition}[realization of $\psi(x_1,..,x_n)$]
	$\psi(x_1,x_2,...,x_n)$ with  $h = \ulcorner \psi \urcorner$ the realization of the n-ary predicate $\psi$ and $x_i : A_i$  is realized as \newline
	$1 \xrightarrow{\langle x_1,..,x_n \rangle} A_1 \times ... \times A_n \xrightarrow{h} \Omega$ .
\end{definition}

\begin{definition}[realization of $\phi(a)$]
	We extend this notion to $\ulcorner \phi(a) \urcorner \equiv f a$ realized as $C \xrightarrow{a} A \xrightarrow{f} \Omega$ (by a slight abuse of notation) where $a$ is one of the \emph{generalized elements} of $A$ at stage $C$.	
\end{definition}

Generalizing to an arbitrary context:

\begin{definition}[realization of $\psi(a_1,..,a_n)$]
	 $\psi(a_1,a_2,...,a_n)$ with generalized elements $C \xrightarrow{a_i} A_i$ at stage $C$ is realized as
	 $C \xrightarrow{\langle a_1,a_2..,a_n \rangle} A_1 \times ... \times A_n \xrightarrow{h} \Omega$.
\end{definition}


\begin{definition}[truth at a stage]
	$\phi(a)$ \emph{holds at stage C} or \emph{C forces $\phi(a)$} denoted by $C \VDashA \phi(a) $ if the following diagram commutes:
	\begin{figure}[h]
		\centering
		\begin{tikzcd}
			C & A & \Omega \\
			& 1
			\arrow["a", from=1-1, to=1-2]
			\arrow["f", from=1-2, to=1-3]
			\arrow["{!_C}"',dotted, from=1-1, to=2-2]
			\arrow["\top"', from=2-2, to=1-3]
		\end{tikzcd}\
		\caption{$f \; a \cdot = \cdot \top \; !_C $}	
	\end{figure}
	\newline
In general:
$\psi(a_1,..,a_n)$ \emph{holds at stage C} or \emph{C forces $\psi(a_1,..,a_n)$} denoted by $C \VDashA \psi(a_1,..,a_n) $ if the following diagram commutes:
\begin{figure}[h]
	\centering
	\begin{tikzcd}
		C && {A_1 \times .. \times A_n} && \Omega \\
		&& 1
		\arrow["h", from=1-3, to=1-5]
		\arrow["{\langle a_1,a_2,..,a_n\rangle}", dashed, from=1-1, to=1-3]
		\arrow["{!_C}"', dotted, from=1-1, to=2-3]
		\arrow["\top"', from=2-3, to=1-5]
	\end{tikzcd}
	\caption{$h \; \langle a_1,a_2,..,a_n\rangle \cdot = \cdot \top \; !_C $}	
\end{figure}
 
\end{definition}


As a consequence of these definitions the following hold:
\begin{prop}
	\begin{enumerate}
		\item If $C \VDashA \phi(a)$ and $D \xrightarrow{h} C$ then $D \VDashA \phi(ah)$. \footnote{by $ah$ we mean $a \circ h$ generalized element of $A$ at stage $D$.}
		\item If $h: D \twoheadrightarrow C $ is an epi and $D \VDashA \phi(ah)$, then $C \VDashA \phi(a)$. 
	\end{enumerate}
\end{prop}

We wish to characterize \emph{truth} in a topos as \emph{truth} at all stages and for all generalized elements.\newline

We can do better by restricting the stages:

\begin{definition}[generating set]
	A set $\mathcal{C}$ of objects of \textfrak{T} is a \emph{generating set} if for any two arrows $f,g :A \rightarrow B$ we have $f \cdot = \cdot g$ iff for all $C\in \mathcal{C}$ and all $C \xrightarrow{h} A$
	$fh \cdot = \cdot gh$.
\end{definition} 

\begin{definition}[truth in \textfrak{T}]
	A formula $\phi(x)$ is \emph{true} in a topos \textfrak{T} denoted by $\models_{\textfrak{T}} \phi(x)$ iff for all objects $C \in \mathcal{C}$ and all generalized elements $C \xrightarrow{a} A$ of $A$ at stage $C$, $C \VDashA \phi(a)$. 
\end{definition}

We also give a preliminary definition that will come in useful:

\begin{definition}[indecomposable]
	The object $C$ is \emph{indecomposable} if for all arrows $D \xrightarrow{k} C$ and $E \xrightarrow{l} C$ such that $[k,l]: D+E \twoheadrightarrow C$ \footnote{the notation $[k,l]$ is equivalent to $k+l$ the unique arrow from the co-product.} is an epi, either $k$ or $l$ is an epi.
\end{definition} 

The so-called \emph{Beth-Kripke-Joyal} Semantics, which we will use, are given by:

\begin{definition}[BKJ Semantics]
	Given $C \xrightarrow{a} A$ generalized element of the topos \textfrak{T}:
	\begin{enumerate}[label=(\roman*)]
		\item $C \VDashA a$ (in case $A=\Omega$) iff $a \cdot = \cdot \top !_C$. 
		\item $C \VDashA \top$ always holds. \footnote{this can be also thought as \emph{there is always an arrow-witness from C}.}
		\item $C \VDashA \bot$ iff $C \cong 0$ i.e., is an initial object in \textfrak{T}. \footnote{in our case of \emph{bushes} this would be the empty forest which can only give the \emph{trivial} generalized element.}
		\item $C \VDashA \phi(a) \land \psi(a)$ iff $C \VDashA \phi(a) $ and $C \VDashA \psi(a)$.
		\item $C \VDashA \phi(a) \lor \psi(a)$ iff there is an epi $[k,l]: D+E \twoheadrightarrow C$ such that $D \VDashA \phi(ak)$ and $E \VDashA \psi(al)$.
		\item $C \VDashA \phi(a) \Rightarrow \psi(a)$ iff for all $D \xrightarrow{h} C$ if $D \VDashA \phi(ah)$ then $D \VDashA \psi(ah)$.
		\item $C \VDashA \neg\phi(a)$ iff for all $D \xrightarrow{h} C$ if $D \VDashA \phi(ah)$ then $D \cong 0$.
	\end{enumerate}
	However, if $C$ is indecomposable, then we can replace (v) with the much simpler:
	\begin{enumerate}[label=(\roman*)']
		\setcounter{enumi}{4}
		\item $C \VDashA \phi(a) \lor \psi(a)$ iff either $C \VDashA \phi(a)$ or $C \VDashA \psi(a)$.
	\end{enumerate} 

\newpage	 
What interests us above all are the semantics for quantifiers where variables range over some sort.
We give the definitions for the cases of unary predicates and binary relations and leave implicit the successive generalizations for arbitrary arities.
	\newline
	 (In this case we quantify over $x:A$) :
\begin{enumerate}[label=(\roman*)]
	\setcounter{enumi}{6}
	\item $C \VDashA \forall_{x : A} \phi(x)$ iff, for all generalized elements $C \xrightarrow{a} A$, then $C \VDashA \phi(a)$.
	\item $C \VDashA \exists_{x : A} \phi(x)$ iff there is a generalized element $C \xrightarrow{a} A$ such that $C \VDashA \phi(a)$.
\end{enumerate}
	(If we introduce an additional variable or \emph{parameter} $y : B$  one has:)
	\begin{gather*}
		\ulcorner \phi(y,x) \urcorner \equiv g \langle y,x \rangle\\
		\ulcorner \phi(y,a) \urcorner \equiv g \; \langle y !_C, a \rangle \text{ with }\ulcorner \phi \urcorner \equiv g: B \times A \rightarrow \Omega
	\end{gather*}
	\begin{enumerate}[label=(\roman*)']
	 \setcounter{enumi}{6}
	\item $C \VDashA \forall_{y : B} \psi(y,a)$ iff, for all $D \xrightarrow{h} C$ and $D \xrightarrow{b} B$, then $D \VDashA \psi(b,ah)$.
	\item $C \VDashA \exists_{y : B} \psi(y,a)$ iff there is an epi $h : D\twoheadrightarrow C$ and a $D \xrightarrow{b} B$ such that $D \VDashA \psi(b,ah)$.
	\end{enumerate}
\end{definition}

\newpage
\subsection{\hl{Quantifying Predicates}}

Let's re-interpret the predicate symbols $p_t, p_*, p_f$ in this environment.
\newline
These are still arrows in $\mathbb{FF_2}$ of the form $\textfrak{p}_t, \textfrak{p}_*,\textfrak{p}_f  :\textbf{1}_\bot \rightarrow \Omega$.
\begin{remark}
	The predicates in question are $p_t(x), p_*(x), p_f(x)$ with $x : \textbf{1}_\bot$ variable of type $\textbf{1}_\bot$.
\end{remark}
Remembering from (\ref{representing}) that in our topos $\mathbb{FF_2}$ we have that $\textbf{1}_\bot$ is also the representing object i.e., $\mathbb{FF_2}(\textbf{1}_\bot, F) \cong |F|$, we can restrict ourselves to the only stage given by $\textbf{1}_\bot$ and let the \emph{generating set} be $\mathcal{C}:=\{\textbf{1}_\bot\}$.
\begin{ex}
What does it mean for say $p_*(x)$ to be \emph{true} at stage $\textbf{1}_\bot$ for some generalized element $\textbf{1}_\bot \xrightarrow{a_0} \textbf{1}_\bot$ i.e., $\textbf{1}_\bot \VDashA p_*(a_0)$? We require the following diagram to commute.

%\begin{tikzcd}
%		\begin{tikzpicture}[scale=0.4]
%		\node (a) at (0.5,-1) {\textcolor{Plum}{$\bullet$}};
%		\node (b) at (0.5,2) {\textcolor{Plum}{$\bullet$}};
%		\draw[Plum, line width=.03in] (a) -- (b);
%	\end{tikzpicture} && 	\begin{tikzpicture}[scale=0.4]
%	\node (a) at (0.5,-1) {\textcolor{Plum}{$\bullet$}};
%	\node (b) at (0.5,2) {\textcolor{Lavender}{$\bullet$}};
%	\draw[Lavender, line width=.03in] (a) -- (b);
%	\end{tikzpicture}
%	\arrow["{a_0}", from=1-1, to=1-3]
%\end{tikzcd}
%
%\begin{tikzcd}
%	\begin{tikzpicture}[scale=0.4]
%		\node (a) at (0.5,-1) {\textcolor{Plum}{$\bullet$}};
%		\node (b) at (0.5,2) {\textcolor{Lavender}{$\bullet$}};
%		\draw[Lavender, line width=.03in] (a) -- (b);
%	\end{tikzpicture} && 	\begin{tikzpicture}[scale=0.4]
%		\node (a) at (0.5,-1) {\textcolor{Plum}{$\bullet$}};
%		\node (b) at (0.5,2) {\textcolor{Lavender}{$\bullet$}};
%		\draw[Lavender, line width=.03in] (a) -- (b);
%	\end{tikzpicture}
%	\arrow["{a_1}", from=1-1, to=1-3]
%\end{tikzcd}


\begin{figure}[h]
	\centering
	\begin{tikzcd}
			\begin{tikzpicture}[scale=0.4]
			\node (A) at (0,0) {\textcolor{OliveGreen}{$\bigcdot$}};
			\node (a) at (0.5,0) {\textcolor{Plum}{$\bullet$}};
			\node (B) at (0,3) {\textcolor{OliveGreen}{$\bigcdot$}};
			\node (b) at (0.5,3) {\textcolor{Lavender}{$\bullet$}};
			\draw[OliveGreen, line width=.03in] (A) -- (B);
		\end{tikzpicture} & 	\begin{tikzpicture}[scale=0.4]
		\node (A) at (0,0) {\textcolor{OliveGreen}{$\bigcdot$}};
		\node (a) at (0.5,0) {\textcolor{Plum}{$\bullet$}};
		\node (b) at (0.5,0.5) {\textcolor{Lavender}{$\bullet$}};
		\node (B) at (0,3) {\textcolor{cyan}{$\bigcdot$}};
		\draw[cyan, line width=.03in] (A) -- (B);
		\end{tikzpicture} & \begin{tikzpicture}[scale=0.4]
		\node (A) at (0,0) {\textcolor{red}{\textbf{f}}};
		\node (B) at (3,0) {\textcolor{OliveGreen}{\textbf{t}}};
		\node (C) at (3,3) {\textcolor{cyan}{$*$}};
		\draw[cyan, line width=.03in] (B) -- (C);
		\end{tikzpicture} \\
		& 	\begin{tikzpicture}[scale=0.4]
			\node (A) at (0,0) {\textcolor{OliveGreen}{$\bigcdot$}};
		\end{tikzpicture}
		\arrow["{a_0}", from=1-1, to=1-2]
		\arrow["\textfrak{p}_*", from=1-2, to=1-3]
		\arrow["{!_{\textbf{1}_\bot}}"', from=1-1, to=2-2]
		\arrow["\top"', from=2-2, to=1-3]
	\end{tikzcd}
	\caption{ $\textfrak{p}_* a_0 \cdot = \cdot \top !_{\textbf{1}_\bot} $ }
\end{figure}
Of course, taking the other generalized element  $\textbf{1}_\bot \xrightarrow{a_1} \textbf{1}_\bot$ the diagram fails to commute so $\textbf{1}_\bot \nVDash p_*(a_1)$.
\begin{figure}[h]
	\centering
	\begin{tikzcd}
		\begin{tikzpicture}[scale=0.4]
			\node (A) at (0,0) {\textcolor{OliveGreen}{$\bigcdot$}};
			\node (a) at (0.5,0) {\textcolor{Plum}{$\bullet$}};
			\node (B) at (0,3) {\textcolor{OliveGreen}{$\bigcdot$}};
			\node (b) at (0.5,3) {\textcolor{Lavender}{$\bullet$}};
			\draw[OliveGreen, line width=.03in] (A) -- (B);
		\end{tikzpicture} & 	\begin{tikzpicture}[scale=0.4]
			\node (A) at (0,0) {\textcolor{OliveGreen}{$\bigcdot$}};
			\node (a) at (0.5,0) {\textcolor{Plum}{$\bullet$}};
			\node (b) at (0.5,3) {\textcolor{Lavender}{$\bullet$}};
			\node (B) at (0,3) {\textcolor{cyan}{$\bigcdot$}};
			\draw[cyan, line width=.03in] (A) -- (B);
		\end{tikzpicture} & \begin{tikzpicture}[scale=0.4]
			\node (A) at (0,0) {\textcolor{red}{\textbf{f}}};
			\node (B) at (3,0) {\textcolor{OliveGreen}{\textbf{t}}};
			\node (C) at (3,3) {\textcolor{cyan}{$*$}};
			\draw[cyan, line width=.03in] (B) -- (C);
		\end{tikzpicture} \\
		& 	\begin{tikzpicture}[scale=0.4]
			\node (A) at (0,0) {\textcolor{OliveGreen}{$\bigcdot$}};
		\end{tikzpicture}
		\arrow["{a_1}", from=1-1, to=1-2]
		\arrow["\textfrak{p}_*", from=1-2, to=1-3]
		\arrow["{!_{\textbf{1}_\bot}}"', from=1-1, to=2-2]
		\arrow["\top"', from=2-2, to=1-3]
	\end{tikzcd}
	\caption{ $\textfrak{p}_* a_1 \cdot \neq \cdot \top !_{\textbf{1}_\bot} $ }
\end{figure}

By the previous considerations we conclude that:
\begin{gather*}
	\not\models_{\mathbb{FF_2}} p_*(x). \;\;\;\;
	\not\models_{\mathbb{FF_2}} p_f(x). \;\;\;\;
	\models_{\mathbb{FF_2}} p_t(x).
\end{gather*}

\end{ex}

What happens now if we quantify over $(x:\textbf{1}_\bot)$?

\begin{ex}
	For $\textbf{1}_\bot \VDashA \forall_{x:1_\bot} p_*(x)$ we need to check whether $\textbf{1}_\bot \VDashA p_*(a)$ for all $\textbf{1}_\bot \xrightarrow{a} \textbf{1}_\bot$.
	\begin{figure}[h]
		\centering
		\begin{tikzcd}
			\begin{tikzpicture}[scale=0.4]
				\node (A) at (0,0) {\textcolor{black}{$\bigcdot$}};
				\node (B) at (0,3) {\textcolor{black}{$\bigcdot$}};
				\draw[black, line width=.03in] (A) -- (B);
			\end{tikzpicture} & 	\begin{tikzpicture}[scale=0.4]
			\node (A) at (0,0) {\textcolor{OliveGreen}{$\bigcdot$}};
			\node (B) at (0,3) {\textcolor{cyan}{$\bigcdot$}};
			\draw[cyan, line width=.03in] (A) -- (B);
			\end{tikzpicture} &  \begin{tikzpicture}[scale=0.4]
			\node (A) at (0,0) {\textcolor{red}{\textbf{f}}};
			\node (B) at (3,0) {\textcolor{OliveGreen}{\textbf{t}}};
			\node (C) at (3,3) {\textcolor{cyan}{$*$}};
			\draw[cyan, line width=.03in] (B) -- (C);
			\end{tikzpicture} \\
			& \begin{tikzpicture}[scale=0.4]
				\node (A) at (0,0) {\textcolor{OliveGreen}{$\bigcdot$}};
			\end{tikzpicture}
			\arrow["{a_0}", curve={height=6pt}, from=1-1, to=1-2]
			\arrow["\textfrak{p}_*", from=1-2, to=1-3]
			\arrow["{!_{\textbf{1}_\bot}}"', from=1-1, to=2-2]
			\arrow["\top"', from=2-2, to=1-3]
			\arrow["{a_1}", curve={height=-6pt}, from=1-1, to=1-2]
		\end{tikzcd}
	\end{figure}
	\newline
	We find, without surprises, that:
	\begin{gather*}
		\not\models_{\mathbb{FF_2}} \forall_{x:1_\bot} p_f(x). \;\;\;\;
		\not\models_{\mathbb{FF_2}} \forall_{x:1_\bot} p_*(x). \;\;\;\;
		\models_{\mathbb{FF_2}} \forall_{x:1_\bot} p_t(x).
	\end{gather*}

Moving on to existentials:
\newline	
	For $\textbf{1}_\bot \VDashA \exists_{x:1_\bot}\textfrak{p}_*(x)$ we need to check whether $\textbf{1}_\bot \VDashA p_*(a)$ for some $\textbf{1}_\bot \xrightarrow{a} \textbf{1}_\bot$.
	We find:
\begin{gather*}
		\not\models_{\mathbb{FF_2}} \exists_{x:1_\bot} p_f(x). \;\;\;\;
		\models_{\mathbb{FF_2}} \exists_{x:1_\bot} p_*(x). \;\;\;\;
		\models_{\mathbb{FF_2}} \exists_{x:1_\bot} p_t(x).
\end{gather*}
\end{ex}


\newpage
\subsection{\hl{Quantifying Relations}}
	 		 
In a new example, we introduce a \emph{relation} symbol $r$.
 \newline
This is realized as an arrow \textfrak{r} from $\textbf{1}_\bot \times \textbf{1}_\bot$ to $\Omega$ ($x$ and $y$ have both the same type $1_\bot$) of the following form:

\begin{figure}[h]
	\centering
	\begin{tikzcd}
		\begin{tikzpicture}[scale=0.35]
			\node (A) at (0,0) {\textcolor{OliveGreen}{$\bigcdot$}};
			\node (B) at (-3,3) {\textcolor{OliveGreen}{$\bigcdot$}};
			\node (C) at (0,3) {\textcolor{OliveGreen}{$\bigcdot$}};
			\node (D) at (3,3) {\textcolor{cyan}{$\bigcdot$}};
			\draw[OliveGreen, line width=.03in] (A) -- (B);
			\draw[OliveGreen, line width=.03in] (A) -- (C);
			\draw[cyan, line width=.03in] (A) -- (D);
		\end{tikzpicture} && \begin{tikzpicture}[scale=0.35]
			\node (A) at (0,0) {\textcolor{red}{\textbf{f}}};
			\node (B) at (3,0) {\textcolor{OliveGreen}{\textbf{t}}};
			\node (C) at (3,3) {\textcolor{cyan}{$*$}};
			\draw[cyan, line width=.03in] (B) -- (C);
		\end{tikzpicture}
		\arrow["\textfrak{r}", from=1-1, to=1-3]
	\end{tikzcd}
	\caption{$ \textbf{1}_\bot \times \textbf{1}_\bot \xrightarrow{\textfrak{r}} \Omega$:}
\end{figure}		 
		 		 
\begin{ex}
	Fixing a generalized element $\textbf{1}_\bot \xrightarrow{a'} \textbf{1}_\bot$ we aim to establish if $\textbf{1}_\bot \VDashA \forall_{x:1_\bot} r(x,a')$.
\newline
	Unfolding the previous definitions we require that the following commutative diagram for all arrows $D \xrightarrow{h} \textbf{1}_\bot$ and $D \xrightarrow{a} \textbf{1}_\bot$ results in $D \VDashA r(a, \; a'h)$ i.e.,  $\textfrak{r}\langle a, \;a'h \rangle \cdot = \cdot \top!_D$. 
	\newline
	Notice however that since $\textbf{1}_\bot$ is a representing object for \emph{bushes} (\ref{representing}) any arrows $D \xrightarrow{h} \textbf{1}_\bot$ and $D \xrightarrow{a} \textbf{1}_\bot$ admit a family of \emph{liftings} $\{\textbf{1}_\bot \xrightarrow{e_j} D\}_{e_j \in D}$ for every node of D \footnote{$e_j$ picks out the node of $D$ by the image of the top node of $\textbf{1}_\bot$.}  and corresponding families of arrows $\{\textbf{1}_\bot \xrightarrow{h_j} \textbf{1}_\bot\}_{e_j \in D}$ and $\{\textbf{1}_\bot \xrightarrow{a_j} \textbf{1}_\bot\}_{e_j \in D}$ such that $\forall e_j \in D :$ $h e_j = h_j$ and $a e_j = a_j$. 
	\newline
	All this to say that, without loss of generality, we may assume $D=\textbf{1}_\bot$. This is motivated by the fact that \emph{truth} of the formula is obtained when every node of $D$ is sent to the node $t$ of $\Omega$ by $\textfrak{r}\langle a,a'h \rangle$ and this is the same as requiring that every image of $\textbf{1}_\bot$ into $D$ is sent to $t$.
	\begin{figure}[h]
		\centering
		\begin{tikzcd}
			& {\textbf{1}_\bot} & {\textbf{1}_\bot} \\
			{D=\textbf{1}_\bot} && {\textbf{1}_\bot \times \textbf{1}_\bot} & \Omega \\
			& {\textbf{1}_\bot}
			\arrow[curve={height=6pt}, from=2-1, to=1-2]
			\arrow[from=2-1, to=1-2]
			\arrow["{\forall h}", curve={height=-6pt}, from=2-1, to=1-2]
			\arrow[curve={height=6pt}, from=2-1, to=3-2]
			\arrow["{\forall a}"', curve={height=12pt}, from=2-1, to=3-2]
			\arrow[from=2-1, to=3-2]
			\arrow[squiggly, two heads, from=2-3, to=3-2]
			\arrow["{\langle a,a'h \rangle}"', dashed, from=2-1, to=2-3]
			\arrow[squiggly, two heads, from=2-3, to=1-3]
			\arrow["{a'}", from=1-2, to=1-3]
			\arrow["{\textfrak{r}}", from=2-3, to=2-4]
		\end{tikzcd}
	\end{figure}
	\newpage
	Let's start by examining $\forall_{x:1_\bot} r(x,a_0)$:
	\begin{figure}[h]
		\centering
		\begin{tikzcd}
			& 	\begin{tikzpicture}[scale=0.4]
				\node (A) at (0,0) {\textcolor{black}{$\bigcdot$}};
				\node (a) at (0.5,0) {\textcolor{orange}{$\bullet$}};
				\node (b') at (0.5,0.5) {\textcolor{yellow}{$\bullet$}};
				\node (B) at (0,3) {\textcolor{black}{$\bigcdot$}};
				\node (b) at (0.5,3) {\textcolor{yellow}{$\bullet$}};
				\draw[black, line width=.03in] (A) -- (B);
			\end{tikzpicture} & 	\begin{tikzpicture}[scale=0.4]
			\node (A) at (0,0) {\textcolor{black}{$\bigcdot$}};
			\node (a) at (0.5,0) {\textcolor{orange}{$\bullet$}};
			\node (B) at (0,3) {\textcolor{black}{$\bigcdot$}};
			\node (b) at (0.5,0.5) {\textcolor{yellow}{$\bullet$}};
			\draw[black, line width=.03in] (A) -- (B);
			\end{tikzpicture} \\
				\begin{tikzpicture}[scale=0.4]
					\node (A) at (0,0) {\textcolor{OliveGreen}{$\bigcdot$}};
					\node (a) at (0.5,0) {\textcolor{orange}{$\bullet$}};
					\node (B) at (0,3) {\textcolor{OliveGreen}{$\bigcdot$}};
					\node (b) at (0.5,3) {\textcolor{yellow}{$\bullet$}};
					\draw[OliveGreen, line width=.03in] (A) -- (B);
				\end{tikzpicture} && 	\begin{tikzpicture}[scale=0.35]
				\node (A) at (0,0) {\textcolor{OliveGreen}{$\bigcdot$}};
				\node (a) at (0.5,0) {\textcolor{orange}{$\bullet$}};
				\node (b') at (0.5,0.5) {\textcolor{yellow}{$\bullet$}};
				\node (B) at (-3,3) {\textcolor{OliveGreen}{$\bigcdot$}};
				\node (b) at (-2.5,3) {\textcolor{yellow}{$\bullet$}};
				\node (C) at (0,3) {\textcolor{OliveGreen}{$\bigcdot$}};
				\node (D) at (3,3) {\textcolor{cyan}{$\bigcdot$}};
				\draw[OliveGreen, line width=.03in] (A) -- (B);
				\draw[OliveGreen, line width=.03in] (A) -- (C);
				\draw[cyan, line width=.03in] (A) -- (D);
			\end{tikzpicture} & \begin{tikzpicture}[scale=0.35]
			\node (A) at (0,0) {\textcolor{red}{\textbf{f}}};
			\node (B) at (3,0) {\textcolor{OliveGreen}{\textbf{t}}};
			\node (C) at (3,3) {\textcolor{cyan}{$*$}};
			\draw[cyan, line width=.03in] (B) -- (C);
			\end{tikzpicture} \\
			& 	\begin{tikzpicture}[scale=0.4]
				\node (A) at (0,0) {\textcolor{black}{$\bigcdot$}};
				\node (a) at (0.5,0) {\textcolor{orange}{$\bullet$}};
				\node (b') at (0.5,0.5) {\textcolor{yellow}{$\bullet$}};
				\node (B) at (0,3) {\textcolor{black}{$\bigcdot$}};
				\node (b) at (0.5,3) {\textcolor{yellow}{$\bullet$}};
				\draw[black, line width=.03in] (A) -- (B);
			\end{tikzpicture}
			\arrow["{a_j}", curve={height=-6pt}, from=2-1, to=1-2]
			\arrow[curve={height=-4pt}, from=2-1, to=3-2]
			\arrow[squiggly, two heads, from=2-3, to=3-2]
			\arrow["{\langle a_i,\;a_0a_j \rangle}"', dashed, from=2-1, to=2-3]
			\arrow[squiggly, two heads, from=2-3, to=1-3]
			\arrow["{a_0}", from=1-2, to=1-3]
			\arrow["{\textfrak{r}}", from=2-3, to=2-4]
			\arrow[curve={height=6pt}, from=2-1, to=1-2]
			\arrow["{a_i}"', curve={height=6pt}, from=2-1, to=3-2]
		\end{tikzcd}
		\caption{The images of $a_i,a_j$ with $i,j \in \{0,1\}$ are displayed with bullets of matching color.}
	\end{figure}
	We discover that:
	\begin{gather*}
		\models_{\mathbb{FF_2}} \forall_{x:1_\bot} r(x,a_0). \\
		\not\models_{\mathbb{FF_2}} \forall_{x:1_\bot} r(x,a_1).
	\end{gather*}
\end{ex}		 

\newpage
Moving on to existential quantification, we aim to realize in analogy to $\forall$, the formula $\exists_{x:1_\bot} r(x,a_1)$:

\begin{ex}
	If we unfold the definitions, in order for $\exists_{x:1_\bot} r(x,a_1)$ to be true at stage $\textbf{1}_\bot$ there must exist an epi $e: D \twoheadrightarrow \textbf{1}_\bot$ and an arrow $D \xrightarrow{a} \textbf{1}_\bot$ such that $D \VDashA r(a,  \;a_1e)$.
	\newline
	Let $D=\textbf{1}_\bot$,$e=a_1$ and $a=a_1$.
	\begin{figure}[h]
		\centering
		\begin{tikzcd}
			& {\textbf{1}_\bot} & {\textbf{1}_\bot} \\
			{D=\textbf{1}_\bot} && {\textbf{1}_\bot \times \textbf{1}_\bot} & \Omega \\
			& {\textbf{1}_\bot}
			\arrow["{e=a_1}", two heads, from=2-1, to=1-2]
			\arrow[squiggly, two heads, from=2-3, to=3-2]
			\arrow["{\langle a_1, a_1 a_1 \rangle}"', dashed, from=2-1, to=2-3]
			\arrow[squiggly, two heads, from=2-3, to=1-3]
			\arrow["{a_1}", from=1-2, to=1-3]
			\arrow["{\textfrak{r}}", from=2-3, to=2-4]
			\arrow["{a=a_1}"', from=2-1, to=3-2]
		\end{tikzcd}
	\end{figure}
		\begin{figure}[h]
		\centering
		\begin{tikzcd}
			& \begin{tikzpicture}[scale=0.4]
				\node (A) at (0,0) {\textcolor{black}{$\bigcdot$}};
				\node (a) at (0.5,0) {\textcolor{orange}{$\bullet$}};
				\node (B) at (0,3) {\textcolor{black}{$\bigcdot$}};
				\node (b) at (0.5,3) {\textcolor{yellow}{$\bullet$}};
				\draw[black, line width=.03in] (A) -- (B);
			\end{tikzpicture} & \begin{tikzpicture}[scale=0.4]
			\node (A) at (0,0) {\textcolor{black}{$\bigcdot$}};
			\node (a) at (0.5,0) {\textcolor{orange}{$\bullet$}};
			\node (B) at (0,3) {\textcolor{black}{$\bigcdot$}};
			\node (b) at (0.5,3) {\textcolor{yellow}{$\bullet$}};
			\draw[black, line width=.03in] (A) -- (B);
			\end{tikzpicture} \\
		\begin{tikzpicture}[scale=0.4]
			\node (A) at (0,0) {\textcolor{OliveGreen}{$\bigcdot$}};
			\node (a) at (0.5,0) {\textcolor{orange}{$\bullet$}};
			\node (B) at (0,3) {\textcolor{OliveGreen}{$\bigcdot$}};
			\node (b) at (0.5,3) {\textcolor{yellow}{$\bullet$}};
			\draw[OliveGreen, line width=.03in] (A) -- (B);
		\end{tikzpicture} &&	\begin{tikzpicture}[scale=0.35]
		\node (A) at (0,0) {\textcolor{OliveGreen}{$\bigcdot$}};
		\node (a) at (0.5,0) {\textcolor{orange}{$\bullet$}};
		\node (B) at (-3,3) {\textcolor{OliveGreen}{$\bigcdot$}};
		\node (b) at (0.5,3) {\textcolor{yellow}{$\bullet$}};
		\node (C) at (0,3) {\textcolor{OliveGreen}{$\bigcdot$}};
		\node (D) at (3,3) {\textcolor{cyan}{$\bigcdot$}};
		\draw[OliveGreen, line width=.03in] (A) -- (B);
		\draw[OliveGreen, line width=.03in] (A) -- (C);
		\draw[cyan, line width=.03in] (A) -- (D);
		\end{tikzpicture} & \begin{tikzpicture}[scale=0.35]
		\node (A) at (0,0) {\textcolor{red}{\textbf{f}}};
		\node (B) at (3,0) {\textcolor{OliveGreen}{\textbf{t}}};
		\node (C) at (3,3) {\textcolor{cyan}{$*$}};
		\draw[cyan, line width=.03in] (B) -- (C);
		\end{tikzpicture} \\
			& \begin{tikzpicture}[scale=0.4]
				\node (A) at (0,0) {\textcolor{black}{$\bigcdot$}};
				\node (a) at (0.5,0) {\textcolor{orange}{$\bullet$}};
				\node (B) at (0,3) {\textcolor{black}{$\bigcdot$}};
				\node (b) at (0.5,3) {\textcolor{yellow}{$\bullet$}};
				\draw[black, line width=.03in] (A) -- (B);
			\end{tikzpicture}
			\arrow["{e=a_1}", two heads, from=2-1, to=1-2]
			\arrow[squiggly, two heads, from=2-3, to=3-2]
			\arrow["{\langle a_1, a_1 a_1 \rangle}"', dashed, from=2-1, to=2-3]
			\arrow[squiggly, two heads, from=2-3, to=1-3]
			\arrow["{a_1}", from=1-2, to=1-3]
			\arrow["{\textfrak{r}}", from=2-3, to=2-4]
			\arrow["{a=a_1}"', from=2-1, to=3-2]
		\end{tikzcd}
	\caption{The usual coloring notation is applied.}	
	\end{figure}
	\newline
	This results in:
	\begin{gather*}
		\models_{\mathbb{FF_2}} \exists_{x:1_\bot} r(x,a_1). \\
		\models_{\mathbb{FF_2}} \exists_{x:1_\bot} r(x,a_0). \\
	\end{gather*}
\end{ex} 

We present a final example in which we introduce a new type $A' := \textbf{1}_\bot + \textbf{1}$ and a new relation $r'$ which \emph{extends} $r$ and is realized by $\textbf{1}_\bot \times \textbf{A}' \xrightarrow{\textfrak{r}'} \Omega$.
\begin{ex}
	We want to check the validity of $\exists_{x: 1_\bot} r(x,a'_2)$:
	\begin{figure}[h]
		\centering
		\begin{tikzcd}
			& \begin{tikzpicture}[scale=0.4]
				\node (A) at (0,0) {\textcolor{black}{$\bigcdot$}};
				\node (a) at (0.5,0) {\textcolor{orange}{$\bullet$}};
				\node (B) at (0,3) {\textcolor{black}{$\bigcdot$}};
				\node (b) at (0.5,3) {\textcolor{yellow}{$\bullet$}};
				\draw[black, line width=.03in] (A) -- (B);
			\end{tikzpicture} & \begin{tikzpicture}[scale=0.4]
				\node (A) at (0,0) {\textcolor{black}{$\bigcdot$}};
				\node (B) at (0,3) {\textcolor{black}{$\bigcdot$}};
				\node (C) at (2,0) {\textcolor{black}{$\bigcdot$}};
				\node (a) at (2.5,0) {\textcolor{orange}{$\bullet$}};
				\node (b) at (2.5,0.5) {\textcolor{yellow}{$\bullet$}};
				\draw[black, line width=.03in] (A) -- (B);
			\end{tikzpicture} \\
			\begin{tikzpicture}[scale=0.4]
				\node (A) at (0,0) {\textcolor{red}{$\bigcdot$}};
				\node (a) at (0.5,0) {\textcolor{orange}{$\bullet$}};
				\node (B) at (0,3) {\textcolor{red}{$\bigcdot$}};
				\node (b) at (0.5,3) {\textcolor{yellow}{$\bullet$}};
				\draw[red, line width=.03in] (A) -- (B);
			\end{tikzpicture} &&	\begin{tikzpicture}[scale=0.35]
				\node (A) at (0,0) {\textcolor{OliveGreen}{$\bigcdot$}};
				\node (B) at (-3,3) {\textcolor{OliveGreen}{$\bigcdot$}};
				\node (C) at (0,3) {\textcolor{OliveGreen}{$\bigcdot$}};
				\node (D) at (3,3) {\textcolor{cyan}{$\bigcdot$}};
				\node (E) at (5,0) {\textcolor{red}{$\bigcdot$}};
				\node (F) at (5,3) {\textcolor{red}{$\bigcdot$}};
					\node (a) at (5.5,0) {\textcolor{orange}{$\bullet$}};
					\node (b) at (5.5,0.5) {\textcolor{yellow}{$\bullet$}};
					\node (b') at (5.5,3) {\textcolor{yellow}{$\bullet$}};
				\draw[OliveGreen, line width=.03in] (A) -- (B);
				\draw[OliveGreen, line width=.03in] (A) -- (C);
				\draw[cyan, line width=.03in] (A) -- (D);
				\draw[red, line width=.03in] (E) -- (F);
			\end{tikzpicture} & \begin{tikzpicture}[scale=0.35]
				\node (A) at (0,0) {\textcolor{red}{\textbf{f}}};
				\node (B) at (3,0) {\textcolor{OliveGreen}{\textbf{t}}};
				\node (C) at (3,3) {\textcolor{cyan}{$*$}};
				\draw[cyan, line width=.03in] (B) -- (C);
			\end{tikzpicture} \\
			& \begin{tikzpicture}[scale=0.4]
				\node (A) at (0,0) {\textcolor{black}{$\bigcdot$}};
				\node (a) at (0.5,0) {\textcolor{orange}{$\bullet$}};
				\node (B) at (0,3) {\textcolor{black}{$\bigcdot$}};
				\node (b) at (0.5,3) {\textcolor{yellow}{$\bullet$}};
				\draw[black, line width=.03in] (A) -- (B);
			\end{tikzpicture}
			\arrow["{e=a_1}", two heads, from=2-1, to=1-2]
			\arrow[squiggly, two heads, from=2-3, to=3-2]
			\arrow["{\langle a_1, a_1 a_1 \rangle}"', dashed, from=2-1, to=2-3]
			\arrow[squiggly, two heads, from=2-3, to=1-3]
			\arrow["{a'_2}", from=1-2, to=1-3]
			\arrow["{\textfrak{r}}", from=2-3, to=2-4]
			\arrow["{?a}"', from=2-1, to=3-2]
		\end{tikzcd}
		\caption{In this case $?a$ can be either $a_0$ or $a_1$.}	
	\end{figure}
	\newline
	This results in:
	\begin{equation*}
		\not\models_{\mathbb{FF_2}} \exists_{x:1_\bot} r'(x,a'_2).
	\end{equation*}
\end{ex}


\newpage
\subsection{\hl{Recovering First-Order $\mathcal{G}_3$}}		

Let's take a look back at what we found using BKJ Semantics.	
		Notice that:
		\begin{remark}
			In order to establish the validity of the universal quantified predicate $\forall_{x:1_\bot} p(x)$ we have to \emph{check} a \emph{finite} number of generalized elements $a_i$.
			In fact, this was equivalent to checking if $\textbf{1}_\bot \VDashA p(a_0)$ and $\textbf{1}_\bot \VDashA p(a_1)$  i.e., if $\textbf{1}_\bot \VDashA p(a_0) \land p(a_1) $ which is equivalent to establishing if $\models_{\mathbb{FF_2}} p(a_0) \land p(a_1) $.
		\end{remark}
		Also:
		\begin{remark}
			 The same phenomenon occurs in order to establish the validity of a universal quantified relation $\forall_{x:1_\bot} r(x,a_0)$. We also have to \emph{check} a \emph{finite} number of generalized elements $a_i,a_j$.
			In fact, this was equivalent to checking if $\textbf{1}_\bot \VDashA r(a_1,a_0)$ and $\textbf{1}_\bot \VDashA r(a_0,a_0)$  i.e., if 
			$\textbf{1}_\bot \VDashA r(a_0,a_0) \land r(a_1,a_0)$ which in turn is equivalent to establishing if $\models_{\mathbb{FF_2}} r(a_0,a_0) \land r(a_1,a_0) $.
		\end{remark} 
		 
		 This suggests that:
		 \begin{lem}
		 	The semantics of a universally quantified formula in $\mathbb{FF_2}$ correspond to  finitary\footnote{\emph{bushes}, recall, are \emph{finite} forests.} conjunction of instanced formulae.
		 	\footnote{the instances to consider are those of $\phi(a_i,..)$ where $x_i : A_i$ is the variable being quantified over and $a_i$ is a generalized element of $A_i$.} 
		 \end{lem}
		 
In a similar manner:

\begin{remark}
	In order to establish the validity of the existential quantified predicate $\exists_{x:1_\bot} p(x)$ we have to \emph{check} a \emph{finite} number of generalized elements $a_i$. 
	In fact, this was equivalent to checking if either $\textbf{1}_\bot \VDashA p(a_0)$ or $\textbf{1}_\bot \VDashA p(a_1)$  i.e., if $\textbf{1}_\bot \VDashA p(a_0) \lor p(a_1) $ which is equivalent to establishing if $\models_{\mathbb{FF_2}} p(a_0) \lor p(a_1) $.
\end{remark}

Now, a categorical consideration:

\begin{lem}
	$\textbf{1}_\bot$ is \emph{indecomposable}.
\end{lem} 
	This becomes quite apparent as the epis in $\mathbb{FF}_*$ are the surjective arrows. Assuming $[k,l] : D + E \twoheadrightarrow \textbf{1}_\bot$ is an epi, if neither $D \xrightarrow{k} \textbf{1}_\bot$ nor $E \xrightarrow{l} \textbf{1}_\bot$ are epis then they must both be constant maps into the root of $\textbf{1}_\bot$ and so $[k,l]$ must be a constant map into the root of $\textbf{1}_\bot$ which brings us to a contradiction.

\begin{remark}
	In order to establish the validity of the existential quantified formula $\exists_{x:1_\bot} r(x,a_0)$ we have to \emph{check} a \emph{finite} number of generalized elements $a_i,a_j$. \newline
	In fact, this was equivalent to checking if either $\textbf{1}_\bot \VDashA r(a_1,a_0)$ or $\textbf{1}_\bot \VDashA r(a_0,a_0)$  i.e., if $\textbf{1}_\bot \VDashA r(a_0,a_0) \lor r(a_1,a_0) $ which in turn is equivalent to establishing if $\models_{\mathbb{FF_2}} r(a_0,a_0) \lor r(a_1,a_0) $.
\end{remark} 

This suggests, analogously to the previous case:

\begin{lem}
	The semantics of an existentially quantified formula in $\mathbb{FF_2}$ corresponds to a generalized finitary disjunction of instanced formulae.
\end{lem}

Finally, let us observe that:

\begin{remark}
	$\mathbb{FF_2}$-validity of instanced atomic formulae like $\phi(a_1,..,a_n)$ is always reduced to $\mathbb{FF_2}$-validity at the stage $\textbf{1}_\bot$ which in turn corresponds to checking if the fiber of $t$ by $\top !_{1_\bot}$ coincides with the maximal sub-forest of $\textbf{1}_\bot$ i.e., $(\top !_{1_\bot})^{-1}[t] = \{\textbf{1}_\bot\}$.
\end{remark}
\newpage
Recall now that $Sub(\textbf{1}_\bot) \cong C_3$ and semantically characterizes $\mathcal{G}_3$.
 \begin{figure}[h]
	\centering
	\begin{tikzpicture}[thick,scale=0.6, every node/.style={scale=0.8}]
		\node (A) at (0,0) {\textcolor{red}{$\emptyset$}};
		\node (B) at (0,2) {\textcolor{cyan}{$\{ \bot \}$}};
		\node (C) at (0,4) {\textcolor{OliveGreen}{$\{ \textbf{1}_\bot \}$}};
		\draw[line width=.01in] (A) -- (B);
		\draw[line width=.01in] (B) -- (C);
		\node (D) at (4,0) {\textcolor{black}{$\bot$}};
		\node (E) at (4,2) {\textcolor{black}{$\bigcdot$}};
		\draw[line width=.01in] (D) -- (E);
	\end{tikzpicture}
	\caption{ $Sub(\textbf{1}_\bot)$ (left) and $\textbf{1}_\bot$ (right).}
\end{figure}

\begin{remark}
We can define an assignment $\mathcal{A}$ from instanced atomic formulae $\phi$ to the Gödel set $\textfrak{T}= \{0, \frac{1}{2}, 1\}$ whereby if we consider the sub-forest of $\textbf{1}_\bot$ determined by  $(\top !_{1_\bot})^{-1}[t]$:
			\begin{equation*}
				\mathcal{A}(\phi) := \begin{cases}
					1 &	\text{if } (\top !_{1_\bot})^{-1}[t] = \{\textbf{1}_\bot\} \\
					\frac{1}{2} & \text{if } (\top !_{1_\bot})^{-1}[t] = \{\bot\}  \\
					0 & \text{if } (\top !_{1_\bot})^{-1}[t] = \emptyset 
				\end{cases}     .  
			\end{equation*}
\end{remark}

Recall also that the semantics of first order $\mathcal{G}_3$ in \ref{fosemantics} whereby $\forall$ and $\exists$ where essentially interpreted as generalized $\land$ and $\lor$, i.e., the interpretation of $\forall x. A(x)$ and $\exists x.A(x)$ was respectively the $min$ and $max$ of the interpretations of the instances $A(u)$ where $u$ ranged over some domain or universe $\textfrak{U}$.
\newline\newline
This suggests, summing up all these results, the following:

\begin{thm}[first-order logic of $\mathbb{FF_2}$]
${}$ \newline
The first-order logic of the topos of $\mathbb{FF_2}$/\emph{bushes} corresponds to first-order three-valued Gödel-Dummett Logic on \emph{finite} domains.  
\end{thm}		 

In other words:	 
\begin{remark}
	The topos of \emph{bushes} provides first order finite models for three-valued Gödel-Dummett Logic.
\end{remark}		 
		 
		 
	\newpage
${}$ \newpage		 
 
 \chapter{Topos Semantics III}
 

\section{The Logic of Variable Sets}
We now give an account of the logic of variable sets as outlined in \cite{goldblatt} and by so doing present another approach to the topos semantics of $\mathcal{G}_3$.   \newline\newline


Informally speaking, in the \emph{classical} world the truth of a statement $\phi(x)$ regarding some \emph{thing} $x$ determines once and for all a set $\{x: \phi(x)\}$ of all things of which the statement is true. \newline
However, in the \emph{non-classical} world the truth-value of a statement is not \emph{absolute} but rather, \emph{context-dependent}. \newline As we saw in the Introduction varies according to the states of knowledge at some particular time. Similarly as we did before, we say that $\phi$ determines for each state $p$ a set $\phi_p$ of all the things of which the statement is known to be true at $p$ called the \emph{extension} of $\phi$ at $p$. 

\begin{definition}[$\phi_p$]
	The extension of $\phi$ at $p$ is denoted by: \begin{equation*}
		\phi_p := \{x: \phi(x)\textit{ is known at p to be true}\}.
	\end{equation*}
\end{definition}

We also require, as we saw for Kripke Semantics, that what is known now to be true remain true in the future, i.e., that truth \emph{persist} in time. 

Formalizing this construction:

\begin{definition}[$F: \textbf{P} \rightarrow \mathbb{Set}$]
	Given a frame \textbf{P} seen as a pre-order category, the assignments $p \mapsto \phi_p$ and $p \rightarrow q $ $\mapsto$ $\phi_p \subseteq \phi_q$: 
	\begin{gather*}
		\phi_p := \{x: \phi(x)\textit{ is known at p to be true}\}. \\
		\text{If } p \sqsubseteq q \text{ then } \phi_p \subseteq \phi_q
	\end{gather*}
	yield a functor $F : \textbf{P} \rightarrow \mathbb{Set}$.
\end{definition}

The category $\mathbb{Set}^\textbf{P}$ has as objects these functors which can be seen as the \emph{variable sets} $\{ \phi_p \}_{p \in \textbf{P}}$, i.e., the \emph{extensions} of $\phi$ at each stage.
\newline
The remarkable result about this category is:
\begin{prop}
	$\mathbb{Set}^\textbf{P}$  is a topos. 
\end{prop}
	This comes from a more general fact:
	\begin{thm}
		For any \emph{small} category $\mathcal{C}$ the (functor) \emph{category of diagrams} $\mathbb{Set}^\mathcal{C}$ is a topos. 
	\end{thm}
	

	In chapter 3 (\ref{examples}) we saw a few instances of this category first with $\textbf{P}= \mathbf{2} = 0 \xrightarrow{\leq_0} 1$, i.e., \emph{functions between sets} and secondly with $\textbf{P}= \mathbf{\omega} = 0 \xrightarrow{\leq_0} 1 \xrightarrow{\leq_1} 2 \xrightarrow{\leq_2}...$, i.e., \emph{sets through time}.



\newpage
\subsection{Back to Kripke Frames}

When we introduced a Kripke Frame $\textbf{P}$ as a finite poset $(P, \sqsubseteq)$ of \emph{possible worlds}, we constructed $\textbf{P}^+$ the collection of \emph{up-sets}, a.k.a. \emph{hereditary sub-sets} of the Frame and found out it could be made a Heyting algebra. \newline
Having fixed a Frame, we focus on \emph{principal} up-sets :

\begin{definition}[principal up-set]
	The principal up-set generated by an element $p \in \textbf{P}$ is:
	\begin{equation*}
		[p) := \{q : p \sqsubseteq q\}.
	\end{equation*}
	, i.e., the elements of the Frame \emph{above} $p$ in the ordering $\sqsubseteq$.
\end{definition}

The operations introduced to make  $\textbf{P}^+$ a Heyting algebra can now be characterized as:

\begin{lem} For any $S,T \in \textbf{P}^+$:
	\begin{gather*}
		S \Rightarrow T = \{p : S \cap [p) \subseteq T\}. \\
		\neg S = \{p: [p) \cap S = \emptyset\}.
	\end{gather*}
\end{lem}

If we restrict $\sqsubseteq$ to $[p)$, we can talk about the \emph{principal set generated by an element}
$q \in [p)$ denoted by $[q)_p$:

\begin{definition}
	$[q)_p := [p) \cap [q)$.
\end{definition}

More generally if $S \subseteq P$ we can \emph{relativize} $S$ to $[p)$:

\begin{definition}
	$S_p := S \cap [p)$.
\end{definition}


What we can obtain of particular interest is:

\begin{prop}
	The poset $([p)^+, \subseteq)$ of up-sets of $[p)$ ordered by inclusion forms a \emph{sub-directly irreducible} Heyting algebra with the operations defined for any up-sets $S,T \subseteq [p)$:
	\begin{gather*}
		S \cap_p T := S \cap T. \\
		S \cup_p T := S \cup T. \\
		S \Rightarrow_p T := \{q: q \in [p) \text{ and } S \cap [q)_p \subseteq T\}. \\
		\neg_p S := \{q: q \in [p) \text{ and } [q)_p \cap S = \emptyset\}.
	\end{gather*}
\end{prop} 

The notable result is that:

	If we start from an up-set $S \subseteq P$ we may choose to first relativize $S$ to $[p)$ and then apply the operations of the H.A.\footnote{H.A. stands for Heyting algebra.}  $([p)^+, \subseteq)$ or first apply the corresponding operations of the H.A. $\textbf{P}^+$ and then relativize to obtain the same result. 
	
Formally:	
\begin{lem}
	for any $S,T \in \textbf{P}^+$,
	\begin{gather*}
		(S_p) \cap_p (T_p) = (S \cap T)_p. \\
		(S_p) \cup_p (T_p) = (S \cup T)_p.\\
		\neg_p(S_p) = (\neg S)_p. \\
		(S_p) \Rightarrow_p (T_p) = (S \Rightarrow T)_p.
	\end{gather*}
\end{lem}	
	  
In fact one can prove that:

\begin{prop}
	The assignment $S \mapsto S_p$ is a surjective H.A. \emph{homomorphism} from $\textbf{P}^+$ to $[p)^+$. 
\end{prop}

\newpage
\subsection{Topos Structure}

We fix some notation:
\begin{remark}
For a functor $F: \textbf{P} \rightarrow \mathbb{Set}$ we denote $F_p$ for $F(p)$ and $F_{pq}$ for the transition map between $F_p$ and $F_q$ when $p \sqsubseteq q$.
\end{remark}

Let's use what we just proved to provide a sub-object classifier for $\mathbb{Set}^\textbf{P}$. \newline

The terminal object is  given by (a generalization of what we saw in the chapter 3):

\begin{lem}[terminal object for $\mathbb{Set}^\textbf{P}$]
	The terminal object is given by the \emph{constant} functor $1: \textbf{P} \rightarrow \mathbb{Set}$
	where every component $1_p := \{0\}$ and the transition maps are the identity $1_pq = id_{0}$.
\end{lem}

The functor we have in mind $\Omega: \textbf{P} \rightarrow \mathbb{Set}$ is the following: \footnote{This is a particular case of a more general construction for $\mathbb{Set}^\mathcal{C}$ for $\mathcal{C}$ small. }

\begin{lem}[sub-object classifier for $\mathbb{Set}^\textbf{P}$]
	The sub-object classifier $\Omega$ is defined by the following assignments:
	\begin{gather*}
		p \mapsto [p)^+. \\
		p \sqsubseteq q \mapsto [p)^+ \xrightarrow{\Omega_{pq}} [q)^+ \text{ where }
		 \Omega_{pq} : S \mapsto S_q = S \cap [q)^+.
	\end{gather*}
	The truth arrow \emph{true}, i.e.,  $\top : 1 \Rightarrow \Omega$ in its components $\{\top_p\}_{p\in \textbf{P}}$ is given by the assignment of the maximal or \emph{unit} element of each $[p)^+$, i.e.,:
	\begin{equation*}
		\top_p(0) := [p).
	\end{equation*}
\end{lem}

A sub-object $\tau: F \Rightarrow G$ in its components, again w.l.o.g., can be assumed to be a set of inclusions $\{\tau_p : F_p \hookrightarrow G_p\}_{p \in P}$. \newline
The characteristic arrow of $\tau$ is given by:

\begin{lem}[characteristic arrow in $\mathbb{Set}^\textbf{P}$]
	$\chi_\tau : G \Rightarrow \Omega$ for each $x \in G_p$:
	\begin{equation*}
		(\chi_\tau)_p(x) := \{q : p \sqsubseteq q \text{ and } G_{p\;q}(x) \in F_q\}. \footnote{One can check first that $\chi_\tau$ is a natural transformation and secondly that $(\chi_\tau)_p(x)$ is an up-set in \textbf{P}.}
	\end{equation*}
	By this definition we have:
	\begin{equation*}
		F_p = \{x : (\chi_\tau)_p(x)= [p)\}.
	\end{equation*}
\end{lem}


We already defined $true$, i.e., $\top : 1 \Rightarrow \Omega$ which picks out the \emph{unit} element $[p)$ from each H.A. $[p)^+$ and can now define the rest of the truth arrows.
\newline
Note that the initial object is given by:

\begin{lem}[initial in $\mathbb{Set}^\textbf{P}$]
	The initial object $0: \textbf{P} \rightarrow \mathbb{Set}$ is the constant functor:
	\begin{gather*}
		\forall p \in P : 0_p := \emptyset. \\
		\forall p \sqsubseteq q \in P : 0_{p\;q} := id_{\emptyset}.
	\end{gather*}
\end{lem}

The unique arrow into the terminal, i.e., $!_0 : 0 \Rightarrow 1 $ is made up of inclusions $\emptyset \hookrightarrow \{0\}$ for each $p \in P$.\newline
 The character of this arrow is defined as the truth arrow \emph{false}:

\begin{definition}[\emph{false} in $\mathbb{Set}^\textbf{P}$]
	\emph{false}, i.e., $\bot : 1 \Rightarrow \Omega$ for each $p\in P$ is:
	\begin{gather*}
		\bot_p (0) = \{ q : p \sqsubseteq q \text{ and } 1_{p \;q}(0) \in 0_q\} = \\
		= \{ q : p \sqsubseteq q \text{ and } 1_{p \;q}(0) \in \emptyset\} = \emptyset.
	\end{gather*}
	, i.e., $\bot$ picks out the \emph{zero} element from each  H.A. $[p)^+$.
\end{definition}

The negation arrow $\neg: \Omega \Rightarrow \Omega$ is the character of $\bot$ where $\bot_p: \{\emptyset\} \subseteq \Omega_p$ \footnote{We identify $\bot_p$ with $\{\emptyset\}$.}:

\begin{definition}[negation in $\mathbb{Set}^\textbf{P}$]
	$\neg: \Omega \Rightarrow \Omega$ in its components $p\in P$ is defined as $\neg_p : \Omega_p \rightarrow \Omega_p$ on $S \subseteq \Omega_p$ as:
	\begin{gather*}
		\neg_p (S) = \{ q : p \sqsubseteq q \text{ and } \Omega_{p \; q} \in \{\emptyset\}\} = \\
		= \{ q : p \sqsubseteq q \text{ and } S \cap [q) = \emptyset\}= \\
		= [p) \cap \neg S = \\
		= (\neg S)_p.
	\end{gather*}	
\end{definition}

\begin{remark}
	There appears to be conflicting notation in the form of $\neg_p$ for the component in $p$ of the natural transformation $\neg$ and the pseudo-complement in the H.A. $[p)^+$. \newline
	Notice however from the result $\neg_p (S) = (\neg S)_p$ that these operations are \emph{compatible} and so the notation is actually consistent.\newline
	This phenomenon appears for all the other truth arrows.
\end{remark}

Notice for instance that in $\mathbb{Set}^\textbf{P}$ products are defined \emph{component-wise}, i.e., $(F \times G)_p := F_p \times G_p$ and $(F \times G) : p \sqsubseteq q \mapsto F_{pq} \times G_{pq}$. \newline 

The conjunction arrow $\land : \Omega \times \Omega \Rightarrow \Omega$ is thus defined as the character of
$\top \times \top : 1 \Rightarrow \Omega \times \Omega$ where $(\top \times \top)_p(0) = ([p),[p))$.
\newline

We make similar considerations for implication and disjunction obtaining:

\begin{gather*}
	\land_p (S,T) = (S \land T)_p. \\
	\Rightarrow_p (S,T) = (S \Rightarrow T)_p.\\
	\lor_p (S,T) = (S \lor T)_p. 
\end{gather*}

Let's take stock of what we just learned:

\begin{remark}
	The components of the \emph{truth-arrows} in $\mathbb{Set}^\textbf{P}$ are essentially the same as the corresponding connectives on the Heyting algebras we defined from \textbf{P}. \newline
	This suggests, as we foretold, that the logic of Variable Sets is \emph{intuitionistic}.  
\end{remark}

\newpage
\subsection{Validity and Applications}

We want now to clarify the link we envisioned between topos validity in $\mathbb{Set}^\textbf{P}$ and H.A. validity on $[p)^+$. \newline

The main result about this is the following, which links topos, Kripke and H.A. validity:
\begin{thm}[Validity Theorem]
	The notation $ \models_{\mathcal{E}}, \models_{K.}, \models_{H.A.}$ stands for respectively topoi, Kripke and Heyting algebra validity. \newline
	For any (propositional) formula $\phi$:
	
	\begin{equation*}
		\mathbb{Set}^\textbf{P}  \models_{\mathcal{E}} \phi\; \text{ iff } \;\textbf{P} \models_{K.} \phi\; \text{ iff } \; P^+ \models_{H.A.} \phi.  
	\end{equation*}

\end{thm}

Furthermore, \newline from what we know about topos validity:

\begin{prop}
	
	\begin{equation*}
		\mathbb{Set}^\textbf{P}  \models_{\mathcal{E}} \phi\; \text{ iff } \;\mathbb{Set}^\textbf{P}(1,\Omega)  \models_{H.A.} \phi; \text{ iff } \; Sub(1) \models_{H.A.} \phi.  
	\end{equation*}

\end{prop}

A sketch of the proof of the \emph{Validity Theorem} is given:

\begin{remark}
	Let $\mathcal{M}=(\textbf{P},V)$ be a Kripke Model based on $\textbf{P}$ with a valuation $V: \textbf{Prop} \rightarrow P^+$. \newline
	A $\mathbb{Set}^\textbf{P}$-valuation $V': \textbf{Prop} \rightarrow \mathbb{Set}^\textbf{P}(1,\Omega)$ can be constructed by defining each component of $V'(\textbf{r}): 1 \Rightarrow \Omega$:
	\begin{equation*}
		V'(\textbf{r})_p(0) := V(\textbf{r}) \cap [p) = V(\textbf{r})_p.
	\end{equation*}
	, i.e., $V'(\textbf{r})_p$ picks out the states in $[p)$ at which \textbf{r} is true in $\mathcal{M}$.\newline
	In fact, from this one can show something stronger:
	\begin{equation*}
		V'(\phi)_p(0) = \mathcal{M}(\phi)_p.
	\end{equation*}
	, i.e., $V'(\phi)_p$ picks out the states in $[p)$ at which $\phi$ is true in $\mathcal{M}$.\newline
	If $\mathbb{Set}^\textbf{P} \models \phi$ then $V'(\phi)=\top$ and so for each state p: $V'(\phi)_p(0) = \mathcal{M}(\phi)_p = [p) $ meaning $\mathcal{\phi}=P$. In other words, $\textbf{P} \models \phi$.   
\end{remark}

\begin{remark}
	For the converse, let $V' : \textbf{P} \rightarrow \mathbb{Set}^\textbf{P}$ be a  $\mathbb{Set}^\textbf{P}$-valuation. \newline
	Each arrow $V'(\textbf{r}) : 1 \Rightarrow \Omega$ determines a collection of up-sets of $[q)$ $V'(\textbf{r})_q(0)$ for each stage $q \in P$.\newline
	We define a \textbf{P}-valuation $V: \textbf{Prop} \rightarrow \textbf{P}^+$ by taking their union at all stages:
	\begin{equation*}
		V(\textbf{r}) := \bigcup_{q \in P} V'(\textbf{r})_q(0).
	\end{equation*} 
	, i.e., $p \in V(\textbf{r})$ iff for some $q$ we have $p \in V'(\textbf{r})_q(0)$.\newline
	If we now define a $\mathbb{Set}^\textbf{P}$-valuation $V''$ from the $\textbf{P}$-valuation $V$ in the same manner as before:
	\begin{equation*}
		V''(\textbf{r})_p(0) = V(\textbf{r}) \cap [p).
	\end{equation*} 
	This just gives us back the original $V'$, i.e.,
	\begin{equation*}
		V(\textbf{r}) \cap [p) = V'(\textbf{r})_p(0)
	\end{equation*}
	In a similar manner if we start from a \textbf{P}-valuation $V$, define as before $V'$ a $\mathbb{Set}^\textbf{P}$-valuation with $V'(\textbf{r})_p(0)=V(\textbf{r})_p$ and try to construct a \textbf{P}-valuation:
	\begin{equation*}
		\bigcup_{p \in P} V'(\textbf{r})_p(0) = \bigcup_{p \in P} V(\textbf{r})_p = V(\textbf{r}). 
	\end{equation*}  
	We return back to the original $V$.
\end{remark}

We may conclude:

\begin{remark}
	There exists a bijection between $\mathbb{Set}^\textbf{P}$-valuations and $\textbf{P}$-valuations.
\end{remark}
 
One of the first consequences of the Validity Theorem is the \emph{characterisation} of topos-valid sentences:

\begin{prop}
	Take the canonical Kripke Frame $\textbf{P}_{IPL}$. We now know that:
	\begin{equation*}
		\vdash_{IPL} \phi \;\text{ iff }\; \textbf{P}_{IPL} \models_{K.} \phi  \;\text{ iff }\; \mathbb{Set}^{\textbf{P}_{IPL}} \models_{\mathcal{E}} \phi.
	\end{equation*}
\end{prop}

From this we get \emph{Completeness} for topos-validity:

\begin{thm}[Completeness Theorem for topos-Validity]
	If $\phi$ is valid on every topos $\mathcal{E}$, i.e., $\mathcal{E} \models \phi$, then
	$\phi$ is an intuitionistic tautology, i.e., $\vdash_{IPL} \phi$.
\end{thm}

Together with the result about \emph{Soundness} for topos-validity, we conclude:

\begin{thm}[Soundness and Completeness Theorem for topos-Validity]\label{soundcompl}
	For all formulae $\phi$ and topoi $\mathcal{E}$:
	\begin{equation*}
		\vdash_{IPL} \phi \; \text{ iff } \; \models_\mathcal{E} \phi.
	\end{equation*}
\end{thm}
In other words: \emph{sentences valid on all topoi are precisely the IPL theorems} or \emph{topoi provide a sound and complete semantics for IPL}.\newline

Furthermore, the Validity Theorem turns gives us a very interesting application for Gödel-Dummett Logic.\newline
Recall from the introduction that $\mathcal{G} = \bigcap_{k\geq2} \mathcal{G}_k$ and that $\mathcal{G}_n \vdash \phi \;\text{ iff }\; C_n \models \phi$. \newline

We can give a topos-semantics characterization for the family $\{\mathcal{G}_n\}_{n \geq 2}$:

\begin{prop}
	Let \textbf{N} be the N-chain Kripke frame,
	$\forall N \geq 1$:
	\begin{equation*}
		C_{N+1} \models_{H.A.} \mathcal{G}_{N+1} \; \text{ iff } \; \textbf{N} \models_{K.} \mathcal{G}_{N+1}
		\; \text{ iff } \; \mathbb{Set}^\textbf{N} \models_{\mathcal{E}} \mathcal{G}_{N+1}.
	\end{equation*}
\end{prop}

 This is because $N^+ \cong C_{N+1}$.
\newline
For $\mathcal{G}_3$ this translates into:
\begin{cor}
	\begin{equation*}
		\mathcal{G}_3 \vdash \phi\; \text{ iff } \; C_{3} \models_{H.A.} \phi \; \text{ iff } \; \textbf{2} \models_{K.} \phi
		\; \text{ iff } \; \mathbb{Set}^\textbf{2} \models_{\mathcal{E}} \phi.
	\end{equation*}
\end{cor}


Recalling now the examples made in \ref{examples}: \newline

This also allows us to motivate the assertion we made about the category of \emph{functions between sets}  $\mathbb{Set}^{\textbf{2}}$ not being Boolean. \newline
Note that if we identify the poset category $\textbf{2}=0 \xrightarrow{\leq_0} 1$ with the 2-chain Kripke frame $\textbf{2}=0 \leq 1 $ we have the following evidence for being non-Boolean:

\begin{prop}
	$ \textbf{2} \not\models_{K.} \alpha \lor \neg \alpha $ holds for the two-element Kripke frame and thus:
	\begin{equation*}
		\mathbb{Set}^\textbf{2} \not\models_{\mathcal{E}} \alpha \lor \neg \alpha.
	\end{equation*} 
	, i.e., the law of excluded middle is not valid in the topos $ \mathbb{Set}^\textbf{2}$.
\end{prop}

\newpage
We conclude with a noteworthy result from \emph{Dummett \& Segerberg} reported in \cite{goldblatt}:\newline
	Applying what we learned for variable sets we discover:

\begin{prop}
	Let $\omega$ be the linear Kripke frame on natural numbers, i.e., $\omega := \{0 \leq 1 \leq 2..\}$:
	\begin{gather*}
		\omega \models_{K.} \alpha \; \text{ iff } \; \mathcal{G} \vdash \alpha \\
		\mathcal{G} \vdash \alpha \; \text{ iff } \;   \mathbb{Set}^\omega \models_{\mathcal{E}} \alpha 
	\end{gather*}
\end{prop}

What this tells us is:

\begin{prop}
	$\mathcal{G}$ is the logic of \emph{sets through time}.
\end{prop}

Furthermore, the structure of $\omega$ which corresponds to \emph{discrete time} can be altered to correspond to \emph{continuous time}:

\begin{prop}
	\begin{equation*}
		\omega \models \alpha \; \text{ iff } \; \mathbb{Q} \models \alpha \; \text{ iff } \; \mathbb{R} \models \alpha  
	\end{equation*}	 
\end{prop}
 
 In fact, if \textbf{C} is \emph{any} infinite chain: 
 
 \begin{prop}
 	\begin{equation*}
 		\mathbb{Set}^{\textbf{C}} \models_{\mathcal{E}} \alpha \; \text{ iff } \; \mathcal{G} \vdash \alpha. 
 	\end{equation*}
 \end{prop}
 


\newpage
\section{Sheaf Semantics} 
To conclude, we give yet another approach to topoi-semantics of $\mathcal{G}_n$ using \emph{sheaves on locales} as seen in \cite{borceaux} and building on the work of  \cite{lisboa}.\newline
 
We have been working with \emph{covariant} functor categories of the form $\mathbb{Set}^\mathbb{C}$ with $\mathbb{C}$ a small category like the poset category $\textbf{P}$. \newline
We could have considered \emph{contra-variant} functor categories like \emph{presheafs}:

\begin{definition}[presheaf]
	A presheaf $F: \mathbb{C}^{op} \rightarrow \mathbb{Set}$ is a contra-variant set-valued functor.
\end{definition} 

Note that since $(-)^{op}$ is an involution \footnote{formally this is an endo-functor in $\mathbb{Cat}$ that is an involution, i.e., $((-)^{op})^{op}$ is the identity functor $id$.}:

\begin{lem}
	Any functor category $\mathbb{Set}^\mathbb{C}$ is a presheaf $\mathbb{Set}^{({\mathbb{C}^{op})}^{op}}$. 
\end{lem}

 Traditionally \emph{pre-sheaves} were used in Topology as functors from $\mathcal{O}(X)$ the lattice of open subsets of a space $X$ to $\mathbb{Set}$. \newline
  
 One can generalize from $\mathcal{O}(X)$ to more general lattices called \emph{locales}:

 
 \begin{definition}[Locale]
 		We say a lattice is \emph{complete} if every $S \subseteq \mathcal{L}$ has a join $(\bigvee_{a \in S} a) \in \mathcal{L}$ \footnote{i.e., a \emph{least upper bound}.} and meet $(\bigwedge_{a \in S} a) \in \mathcal{L}$ \footnote{i.e., a \emph{greatest lower bound}.}.   \newline
 	A \emph{locale} $\mathcal{L}$ is a \emph{complete} lattice in which arbitrary joins distribute over finite meets, i.e., for an arbitrary indexing set $I$ and elements $a_i,b \in \mathcal{L}$:
 	\begin{equation*}
 		a \land (\bigvee_{i \in I} b_i ) = \bigvee_{i \in I} (a \land b_i)
 	\end{equation*}
 \end{definition}
 
 \begin{ex}[the locale $O(X)$]
 	Given a topological space $(X,\mathcal{O}(X))$, the locale of \emph{open subsets} $(\mathcal{O}(X),\subseteq)$ is a \emph{sub-lattice} of the locale of subsets $(\mathcal{P}(X), \subseteq)$ that is closed under arbitrary joins \footnote{an arbitrary union of opens is open.} and finite meets \footnote{any finite intersection of opens is open.}. 
 \end{ex}

 Note that by this definition a locale can be seen as a \emph{co-complete} and small category in which for each $b\in \mathcal{L}$ every functor $(- \land b)$ preserves co-limits, i.e., the distributive condition for arbitrary joins.
 This is equivalent \footnote{by the Adjoint Functor Theorem, see \cite{awodey}.} to each functor $(- \land b)$ having a right adjoint $(b \Rightarrow -)$, so that:
 \begin{prop}
 	\begin{gather*}
 		 b \Rightarrow c = \bigvee \{a \in \mathcal{L} \;|\; a \land b \leq c\} \\
 		\mathcal{L}\text{ is a locale } \;\text{ iff } \; \mathcal{L} \text{ is a complete Heyting algebra}. 
 	\end{gather*}
 	A H.A. is \emph{complete} if it is so as a lattice.
 \end{prop} 
 
 \begin{ex}[the locale $C_n$]
 	Every n-chain $C_n$, as a complete Heyting algebra, is a locale.
 \end{ex}
 
 If we consider the corresponding poset category on a locale, the pre-sheaves on a locale $\mathcal{L}$ are thus:
 
 \begin{definition}[presheaf]
 	A presheaf on a locale $\mathcal{L}$ is a contra-variant functor $F: \mathcal{L} \rightarrow \mathbb{Set}$.\newline
 	If $v \xrightarrow{\leq_{uv}} u$ in $\mathcal{L}$, the action of the transition map $F(u) \xrightarrow{F(\leq_{uv})} F(v)$ on elements is denoted by $x \mapsto x \restriction_v$. \footnote{the reason for this notation will become soon apparent.} \newline
 	Notice that the \emph{functoriality} of $F$ is given by:
 	\begin{enumerate}
 		\item $\forall u \in \mathcal{L}$, $\forall x \in F(u)$ $x \restriction_u = x$.
 		\item $\forall w \leq v \leq u \in \mathcal{L}$ $\forall x \in F(u)$ $x \restriction_w = (x \restriction_v)\restriction_w$. 
 	\end{enumerate}
 \end{definition}
 
 Letting $F$ be a presheaf on a locale $\mathcal{L}$, we introduce the following notion:
 
 \begin{definition}[compatible family]
	Taking an arbitrary family $(u_i)_{i\in I}$ in $\mathcal{L}$, a family of elements $ \{x_i \in F(u_i) \}_{i \in I} $ is \emph{compatible} when:
	\begin{equation*}
		\forall i,j \in I : x_i \restriction_{u_i \land u_j} = x_j \restriction_{u_i \land u_j}.
	\end{equation*}  	
 \end{definition}
 
 
 On a locale $\mathcal{L}$, when does a presheaf become a \emph{sheaf} ?
 
 \begin{definition}[sheaf]
 	A presheaf is a \emph{sheaf} when, given a so-called \emph{covering} of $u \in \mathcal{L}$, i.e., $u= (\bigvee_{i \in I} u_i)$ and  a compatible family $(x_i \in F(u_i))_{i \in I}$, there exists a unique element, a.k.a. \emph{gluing} $x\in F(u)$ such that $x \restriction_{u_i} = x_i$ for each $i\in I$. \newline 
 		In short, given some \emph{covering} there exists a unique \emph{gluing} for every \emph{compatible family}.
 \end{definition}
 
 \begin{ex}[continuous functions]
 	Let $(X,\tau)$ and $(Y,\sigma)$ be topological spaces and fix $U \in \tau$. \newline
 	If we consider the set $\mathcal{C}(U,Y)$ of \emph{continuous} functions $f: U \rightarrow Y$ and choose the usual restriction mappings $\restriction_V$ of a function to a subset $V \subseteq U$, this yields a presheaf $\mathcal{C}(-,Y)$ on the locale $(\mathcal{O}(X),\subseteq)$.\newline
 	Furthermore, this is a \emph{sheaf} since a compatible family $\{f_i : U_i \rightarrow Y\}$ on an open covering $U= \bigcup_{i \in I} U_i$ entails that any $f_i$ and $f_j$ with $i,j \in I$ coincide on $U_i \cap U_j$, i.e., $f_i \restriction_{U_i \land U_j} \equiv f_j \restriction_{U_i \land U_j}$. \newline
 	The unique \emph{gluing} is given as the \emph{collation} of all the functions in the family:
 	\begin{equation*}
 		f: \bigcup_{i \in I} U_i \rightarrow Y, \;\; x \mapsto f_i(x) \;\;\text{ if }x \in U_i.
 	\end{equation*}   
 \end{ex}
 
 \begin{remark}
 	An obvious counter-example to show that \emph{not every presheaf is a sheaf} is given by taking $Y= \mathbb{R}$ with the standard topology.\newline The presheaf $\mathcal{B}(-,\mathbb{R})$ of \emph{bounded} functions fails to be a sheaf since the collation of functions that are bounded may yield an unbounded function. 
 \end{remark}
 
 
 Let $\mathcal{H}$ be a complete H.A., or equivalently a locale, and $\mathbb{C}_{\mathcal{H}}$ be the corresponding poset category.\newline
 We now take pre-sheaves on $\mathbb{C}_{\mathcal{H}}$ and define the following category:
 
 \begin{remark}
 	We use the following notation: $\mathbf{H}$ instead of $\mathbb{C}_{\mathcal{H}}$.
 \end{remark} 
 
 \begin{definition}[presheaf \& sheaf categories]
 	The category $\mathbf{PreSh}(\mathbf{H})$ has objects the pre-sheaves on $\mathbf{H}$ and arrows given by natural transformations between them. \newline
 	The category $\mathbf{Sh}(\mathbf{H})$ is the category with objects sheaves on $\mathbf{H}$ and arrows given by natural transformations between them.
 \end{definition} 
 
 We use the following results from \cite{borceaux}:
 
 \begin{prop}
 	If $H$ is a complete Heyting algebra, then $\mathbf{Sh}(\mathbf{H})$ is a topos.
 \end{prop}

  \begin{prop}
  	If $H$ is a complete H.A., then:
  	\begin{equation*}
  		 H \cong Sub_{ \mathbf{Sh}(\mathbf{H})} (1). 
  	\end{equation*}
  \end{prop}
  \newpage
  
  We are now in a position to \emph{semantically characterize} using sheaves the family of intermediate logics $\{\mathcal{G}_N\}_{N \geq 2}$:\newline
  (The following result is the main conclusion of \cite{lisboa})
  \begin{thm}[Sheaf semantics for $\mathcal{G}_N$]

  	For every $N\geq 2$ and (propositional) formula $\phi$:
  	\begin{gather*}
  		\mathcal{G}_N \vdash \phi \;\text{ iff }\; C_N \models_{H.A.} \phi
  		\\  \;\text{ iff }\; \; Sub_{ \mathbf{Sh}(\mathbf{C_N})} (1) \models_{H.A.} \phi \;\text{ iff }\; 
  		\; \mathbf{Sh}(\mathbf{C_N}) \models_{\mathcal{E}} \phi. \\
  		\\
  		\mathcal{G}_N \vdash \phi \;\text{ iff }\;  \mathbf{Sh}(\mathbf{C_N}) \models_{\mathcal{E}} \phi. \\
  	\end{gather*}
  \end{thm}
  
  As an immediate application of the above:
  \begin{cor}
  	  \begin{equation*}
  		\mathcal{G}_{3} \vdash \phi \;\text{ iff }\;\mathbf{Sh}(\mathbf{C_{3}}) \models_{\mathcal{E}} \phi.
  	\end{equation*}
  \end{cor}
  
  \newpage
  \subsection{\hl{Sheaves and Variable Sets}}
  We make a few concluding remarks about the relationship between topoi-semantics of sheaves and variable sets. \newline
  
  Note that in a sheaf $F$ on a locale $\mathcal{L}$ the bottom element $0 \in \mathcal{L}$ is the \emph{zero}-th join and:
  
  \begin{remark} 
  	\begin{equation*}
  		0 = \bigvee_{i \in \emptyset} u_i.
  	\end{equation*}
  	, i.e., the \emph{empty covering} of $0$.\newline The empty family $(x_i \in F(u_i))_{i \in \emptyset}$ is trivially compatible and admits a unique gluing $* \in F(0)$.\newline
  \end{remark}
  
  What this implies is:
  
  \begin{lem}
  	$F(0)$ is a singleton set $F(0)=\{*\}$. \footnote{this is a common result found in \cite{borceaux} among others.}
  \end{lem}
  This allows us to link the semantic characterizations of $\mathcal{G}_n$ on variable sets $\mathbb{Set}^\mathbf{N}$ (or equivalently pre-sheaves on $\mathbf{N}^{op}$) and sheaves on $\mathbf{C_n}$:
  
  \begin{prop}
  	  For every $N \geq 2$:
  	  	\begin{gather*}
  		\mathcal{G}_{N} \vdash \phi   \;\text{ iff }\;\mathbf{Sh}(\mathbf{N}) \models_{\mathcal{E}} \phi 
  		\\ \;\text{ iff }\; C_{N} \models_{H.A.} \phi\;\text{ iff }\;\mathbf{N-1} \models_{K.} \phi \;\text{ iff }\ \mathbb{Set}^\mathbf{N-1} \models_{\mathcal{E}} \phi ;   
  	\end{gather*}
  \end{prop}
  Recall that $\mathcal{G}_{2}= CPL$, i.e., classical propositional logic, $C_{2}$ is the Boolean algebra of binary truth values, $\mathbf{1}$ is the one-world Kripke frame and that $\mathbb{Set}^\mathbf{1} \cong \mathbb{Set}$ is a bivalent and boolean topos.
  \newline
  For $N=2$ we recover classical logic:
  \begin{prop}
  	\begin{gather*}
  		\mathcal{G}_{2} \vdash \phi \;\text{ iff }\;\mathbf{Sh}(\mathbf{C_{2}}) 
  		\models_{\mathcal{E}} \phi \\ \;\text{ iff }\; C_{2} \models_{H.A.} \phi  \;\text{ iff }\; \mathbf{1} \models_{K.} \phi  \;\text{ iff }\; \mathbb{Set}^\mathbf{1} \models_{\mathcal{E}} \phi.     
  	\end{gather*}
  	\end{prop}
  
  \newpage
  Since the poset category $\mathbf{N} \cong \mathbf{C_N}$ for any $N \in \mathbb{N}$: 
  
  \begin{lem}
  	  For every $N \geq 2 :
  		 \mathbf{Sh}(\mathbf{{N}}) \cong  \mathbb{Set}^\mathbf{N-1}.$   
  \end{lem}
  
  To see why this isomorphism holds, recall that $\mathbb{Set}^\mathbf{N}$, i.e., variable sets on $\mathbf{{N}}$ are the same as pre-sheaves on $\mathbf{{N}^{op}}$ and  consider the case of $N=3$ :
  \newline
Let's consider the sheaves on the 3-chain $C_3 = \{u_0 \leq u_1 \leq u_2\}$:
  
  \begin{prop}
  	 \begin{gather*}
  		\mathcal{G}_{3} \vdash \phi \;\text{ iff }\;\mathbf{Sh}(\mathbf{C_{3}}) \models_{\mathcal{E}} \phi \\
  		\;\text{ iff }\; C_{3} \models_{H.A.} \phi  \;\text{ iff }\; \mathbf{2} \models_{K.} \phi  \;\text{ iff }\; \mathbb{Set}^\mathbf{2} \models_{\mathcal{E}} \phi.     
  	\end{gather*}
  	\end{prop}
  	\begin{remark}
  		The image of a sheaf $F$ of this form is given by sets and restriction maps between them $F_2 \xrightarrow{\restriction_1} F_1 \xrightarrow{\restriction_0} F_0=\{*\}$ where $ \restriction_0 = !_{F_1} $ is the unique map on the singleton set, i.e., the terminal object in $\mathbb{Set}$.
  		\newline
  		Arrows between these sheaves are, as in the case of variable sets, natural transformations $\tau: F \Rightarrow G$ which form commutative diagrams. 
  		\newline
  			Any covering $u = \bigvee_{i \in I} u_i$ with $I \subseteq \{0,1,2\}$ in the chain $C_3$ must have an element $u_{i'}$ such that $u_{i'} = u$ and of course all elements $u_i \leq u$. 
  		\newline
  			What this means for compatible families $\{x_i \in F_i\}_{i \in I}$ is that given $i,j \in I$ if $i \geq j$ then $\restriction_{u_j}: x_i \mapsto x_j$.
  		\newline
  			Considering the \emph{forest structure} of the variable sets in question,  $\{x_i \in F_i\}_{i \in I}$ is nothing more than a bunch of nodes on a \emph{branch}.
  			The highest node in this case $x_{u_{i'}} = x_u$ corresponds to the gluing.
  	\end{remark}

    The following figure gives an example of this situation:
    
  \begin{figure}[h]
  	\centering
  	\begin{tikzcd}
  		{F_2=} & {\{} & {a_2,} & {b_2} & {c_2} & {d_2} & {e_2} & {\}} \\
  		\\
  		{F_1=} & {\{} & {a_1,} & {b_1,} & {c_1} & {...} & {...} & {\}} \\
  		\\
  		{F_0=} & {\{} &&& {*} &&& {\}}
  		\arrow[maps to, from=1-3, to=3-4]
  		\arrow[maps to, from=1-4, to=3-4]
  		\arrow[maps to, from=1-5, to=3-4]
  		\arrow[maps to, from=1-6, to=3-4]
  		\arrow[maps to, from=1-7, to=3-5]
  		\arrow[dashed, maps to, from=3-3, to=5-5]
  		\arrow[dashed, maps to, from=3-4, to=5-5]
  		\arrow[dashed, maps to, from=3-5, to=5-5]
  		\arrow[dashed, maps to, from=3-6, to=5-5]
  		\arrow[dashed, maps to, from=3-7, to=5-5]
  	\end{tikzcd}
  	\caption{The image of a sheaf $F$ on $C_3$. 
  	Branches like $\{b_2,b_1,*\}$ and $\{e_2,c_1\}$ are compatible families with gluings respectively $b_2$ and $e_2$.}
  \end{figure}
 \newpage
 \begin{remark}
	If we remove the bottom level of $F_0$, which is present in every sheaf of this form, what we are left with is a presheaf on $C_2$.
 	\newline
 	Vice-versa, if we start from a presheaf $X$ on $C_2$, add a new set $X_0 := \{ * \} $ and transition function $X_1 \xrightarrow{c_*} X_0$, i.e., the constant function on the singleton set $x \mapsto *$ we return to sheaves on $C_3$. \newline
 	What this implies is that there is a one to one correspondence between variable sets on $C_2$ and sheaves on $C_3$ which is functorial and establishes an isomorphism.
 \end{remark}
 

 \begin{remark} 
The 3-chain $C_3$ in a topological context can be thought of as the locale of open subsets of the \emph{Sierpinski Space} $\mathcal{S}:=\{0,1\}$, i.e., $\mathcal{O(S)}:= \emptyset \subseteq \{1\} \subseteq \{0,1\}$.
 \begin{figure}[h]
 	\centering
 		\begin{tikzpicture}
 		\node (a) at (-1,0) {$1$};
 		\node (b) at (-3,0) {$0$};
 		\draw[dotted] (-2,0) ellipse (2.5 and 1.2);
 		\draw[dotted] (-1,0) ellipse (0.6 and 0.6);
 	\end{tikzpicture}
 	\caption{The \emph{Sierpinski} topology on the space $\{0,1\}$ where the only non-trivial open sub-set is $\{1\}$. }
 \end{figure}
\end{remark}
\newpage
 This gives us the following characterization inspired by \cite{elephant}:
 \newline
\emph{Sheaves over the Sierpinski space $\textbf{Sh}(\mathcal{S})$, a.k.a. the \emph{Sierpinski topos} is equivalent to the category of presheaves over \textbf{2} or $\textbf{PSh}(\textbf{2})$ which in turn is equivalent to \emph{functions between sets.}}
\begin{prop}	
	\begin{gather*}
		\textbf{Sh}(\mathcal{S}) \simeq \textbf{PSh}(\textbf{2}) \simeq \mathbb{Set}^{\textbf{2}}. \\ \\
		\mathcal{G}_{3} \vdash \phi \;\text{ iff }\;\mathbf{Sh}(\mathcal{S}) \models_{\mathcal{E}} \phi.
	\end{gather*}
\end{prop}
  In other words, 
  \begin{prop}
  	$\mathcal{G}_3$  is the logic of the Sierpinski topos.
  \end{prop}
  
  
 
 
 \newpage
 \subsection{\hl{Forests and Variable Sets}}
\label{forestsvar}
 Here we give some new insight about the relationship between \emph{forests} and \emph{variable sets}.  
\newline
 
 Remember from \ref{examples} that $\mathbb{Set}^{0 \rightarrow 1}$/\emph{functions between sets} is a tri-valent and non-Boolean topos. 
 \newline
 
 Restricting ourselves to the sub-category $\mathbb{Set}_{fin}^{0 \rightarrow 1}$, i.e., \emph{functions between finite sets}, let's revisit the truth-arrows $\top,*,\bot: \textbf{1} \Rightarrow \Omega$:
 \newpage
 
 \begin{figure}[h]
 	\centering
 	\begin{tikzcd}
 		{\{0\}} &&& {\{0,} & {\frac{1}{2},} & {1\}} \\
 		\\
 		{\{0\}} &&& {\{0,} & {1\}}
 		\arrow["t", maps to, from=1-6, to=3-5]
 		\arrow["t", maps to, from=1-5, to=3-5]
 		\arrow[""{name=0, anchor=center, inner sep=0}, "t", maps to, from=1-4, to=3-4]
 		\arrow[""{name=1, anchor=center, inner sep=0}, "id", maps to, from=1-1, to=3-1]
 		\arrow["true", curve={height=12pt}, squiggly, maps to, from=3-1, to=3-5]
 		\arrow["{t'}", curve={height=-18pt}, squiggly, maps to, from=1-1, to=1-6]
 		\arrow["\top", shorten <=19pt, shorten >=19pt, Rightarrow, from=1, to=0]
 	\end{tikzcd}
 	\caption{$\top: \textbf{1} \Rightarrow \Omega$ with $true: 0 \mapsto 1$, $t': 0 \mapsto 1$.}
 \end{figure}
 
 \begin{figure}[h]
 	\centering
 	\begin{tikzcd}
 		{\{0\}} &&& {\{0,} & {\frac{1}{2},} & {1\}} \\
 		\\
 		{\{0\}} &&& {\{0,} & {1\}}
 		\arrow["t", maps to, from=1-6, to=3-5]
 		\arrow["t", maps to, from=1-5, to=3-5]
 		\arrow[""{name=0, anchor=center, inner sep=0}, "t", maps to, from=1-4, to=3-4]
 		\arrow[""{name=1, anchor=center, inner sep=0}, "id", maps to, from=1-1, to=3-1]
 		\arrow["{*'}", curve={height=-12pt}, squiggly, maps to, from=1-1, to=1-5]
 		\arrow["true", curve={height=12pt}, squiggly, maps to, from=3-1, to=3-5]
 		\arrow["{*}", shorten <=19pt, shorten >=19pt, Rightarrow, from=1, to=0]
 	\end{tikzcd}
 	\caption{$* : \textbf{1} \Rightarrow \Omega$ with
 		$true: 0 \mapsto 1$, $*': 0 \mapsto \frac{1}{2}$.}
 \end{figure}
 
 
 \begin{figure}[h]
 	\centering
 	\begin{tikzcd}
 		{\{0\}} &&& {\{0,} & {\frac{1}{2},} & {1\}} \\
 		\\
 		{\{0\}} &&& {\{0,} & {1\}}
 		\arrow["t", maps to, from=1-6, to=3-5]
 		\arrow["t", maps to, from=1-5, to=3-5]
 		\arrow[""{name=0, anchor=center, inner sep=0}, "t", maps to, from=1-4, to=3-4]
 		\arrow["{f'}", curve={height=-9pt}, squiggly, maps to, from=1-1, to=1-4]
 		\arrow["false", curve={height=9pt}, squiggly, maps to, from=3-1, to=3-4]
 		\arrow[""{name=1, anchor=center, inner sep=0}, "id", maps to, from=1-1, to=3-1]
 		\arrow["\bot", shorten <=19pt, shorten >=19pt, Rightarrow, from=1, to=0]
 	\end{tikzcd}
 	\caption{$\bot : \textbf{1} \Rightarrow \Omega$ with
 		$false: 0 \mapsto 0$, $f': 0 \mapsto 0$.}
 \end{figure}
 
 This \emph{structure} is very similar to what we saw in \emph{bushes} or \emph{finite forests}. 
 
 \newpage
 In fact, the \emph{translation} from $\mathbb{Set}_{fin}^{0 \rightarrow 1}$ to $\mathbb{FF_2}$ is readily given in this case:\newline
 (The usual coloring notation is applied.)
 \begin{figure}[h]
 	\centering
 	\begin{tikzpicture}[thick,scale=0.9, every node/.style={scale=0.9}]
 		\node (A) at (-2,0) {\textcolor{PineGreen}{0}};
 		\node (B) at (-2,3) {\textcolor{SpringGreen}{0'}};
 		\draw[line width=.03in, SpringGreen] (A) -- (B);
 		
 		\node (C) at (3,0) {\textcolor{red}{0}};
 		\node (D) at (3,3) {\textcolor{YellowOrange}{0'}};
 		\draw[line width=.03in, YellowOrange] (C) -- (D);
 		
 		\node (E) at (5,0) {\textcolor{PineGreen}{1}};
 		\node (F) at (5,3) {\textcolor{cyan}{$\frac{1}{2}$'}};
 		\node (G) at (7,3) {\textcolor{SpringGreen}{1'}};
 		\draw[line width=.03in, cyan] (E) -- (F);
 		\draw[line width=.03in, SpringGreen] (E) -- (G);
 		
 		\node (a) at (-1,1) {};
 		\node (b) at (2,1) {};
 		\draw[->, dotted, line width=.01in] (a) -- node[anchor=north] {$f_{\text{t}}$} (b); 			 	
 	\end{tikzpicture}
 	\caption{$f_t: \textbf{1}_\bot \rightarrow \textbf{1}_\bot + (2\cdot \textbf{1})_\bot$ as $\mathbb{FF_2}$-arrow where $0 \mapsto 1, 0' \mapsto 1'$.}
 \end{figure}
  

 \begin{figure}[h]
 	\centering
 		\begin{tikzpicture}[thick,scale=0.9, every node/.style={scale=0.9}]
 		\node (A) at (-2,0) {\textcolor{PineGreen}{0}};
 		\node (B) at (-2,3) {\textcolor{cyan}{0'}};
 		\draw[line width=.03in, cyan] (A) -- (B);
 		
 		\node (C) at (3,0) {\textcolor{red}{0}};
 		\node (D) at (3,3) {\textcolor{YellowOrange}{0'}};
 		\draw[line width=.03in, YellowOrange] (C) -- (D);
 		
 		\node (E) at (5,0) {\textcolor{PineGreen}{1}};
 		\node (F) at (5,3) {\textcolor{cyan}{$\frac{1}{2}$'}};
 		\node (G) at (7,3) {\textcolor{SpringGreen}{1'}};
 		\draw[line width=.03in, cyan] (E) -- (F);
 		\draw[line width=.03in, SpringGreen] (E) -- (G);
 		
 		\node (a) at (-1,1) {};
 		\node (b) at (2,1) {};
 		\draw[->, dotted, line width=.01in] (a) -- node[anchor=north] {$f_{*}$} (b); 			 	
 	\end{tikzpicture}
 	\caption{$f_*: \textbf{1}_\bot \rightarrow \textbf{1}_\bot + (2\cdot \textbf{1})_\bot$ as $\mathbb{FF_2}$-arrow where $0 \mapsto 1, 0' \mapsto \frac{1}{2}'$.}
 \end{figure}
 
 
 \begin{figure}[h]
 	\centering
 		\begin{tikzpicture}[thick,scale=0.9, every node/.style={scale=0.9}]
 		\node (A) at (-2,0) {\textcolor{red}{0}};
 		\node (B) at (-2,3) {\textcolor{YellowOrange}{0'}};
 		\draw[line width=.03in, YellowOrange] (A) -- (B);
 		
 		\node (C) at (3,0) {\textcolor{red}{0}};
 		\node (D) at (3,3) {\textcolor{YellowOrange}{0'}};
 		\draw[line width=.03in, YellowOrange] (C) -- (D);
 		
 		\node (E) at (5,0) {\textcolor{PineGreen}{1}};
 		\node (F) at (5,3) {\textcolor{cyan}{$\frac{1}{2}$'}};
 		\node (G) at (7,3) {\textcolor{SpringGreen}{1'}};
 		\draw[line width=.03in, cyan] (E) -- (F);
 		\draw[line width=.03in, SpringGreen] (E) -- (G);
 		
 		\node (a) at (-1,1) {};
 		\node (b) at (2,1) {};
 		\draw[->, dotted, line width=.01in] (a) -- node[anchor=north] {$f_{\text{f}}$} (b); 			 	
 	\end{tikzpicture}
 	\caption{$f_\text{f}: \textbf{1}_\bot \rightarrow \textbf{1}_\bot + (2\cdot \textbf{1})_\bot$ as $\mathbb{FF_2}$-arrow where $0 \mapsto 0, 0' \mapsto 0$.}
 \end{figure}
 
 
 The objects are translated as follows:
 \begin{itemize}
 	\item At levels $0$ and $1$ the elements of the sets $F_0,F_1$ and $G_0,G_1$ become distinct nodes.
 	\item The transition functions $f: F_0 \rightarrow F_1$ and $g: G_0 \rightarrow G_1$ specify the partial ordering by requiring $\forall_{a\in F_0} \;f(a) \leq a$ and $\forall_{b\in G_0} \;g(b) \leq b$.\newline
 	What we are left with is two finite forests $F$ and $G$. 
 \end{itemize}
 As for the arrows:
 \begin{itemize}
 	\item The natural transformation $\tau: F \Rightarrow G$ in its components $\tau_0,\tau_1$ determines the image of an arrow $f_\tau$ for each node.\newline
 	Notice that the naturality of $\tau$, i.e., $g \tau_0 = \tau_1 f$ makes $f_\tau$ an order-preserving and open map, i.e., an arrow in $\mathbb{FF}$.
 \end{itemize}
 
This method can be generalized from the poset category \textbf{2}, i.e., $0 \xrightarrow{\leq_0} 1$ to a generic functor category of \emph{variable sets}, a.k.a. \emph{finite sets through finite time} $\mathbb{Set}_{fin}^\textbf{N}$ where $\textbf{N}$ is the analogous poset category \textbf{N}, i.e., $0 \xrightarrow{\leq_0} 1 \xrightarrow{\leq_1} 2.. \xrightarrow{\leq_{N-1}} N$ and provides a translation from $\mathbb{Set}_{fin}^\textbf{N}$ to $\mathbb{FF_N}$. \newline
 So:
 \begin{remark}
 	There is a \emph{translation} available from the categories $\mathbb{Set}_{fin}^\textbf{N}$ to the category of finite forests $\mathbb{FF}$. 
 \end{remark} 
 
 At first glance there seems to be a \emph{reverse translation} available from $\mathbb{FF_N}$ to $\mathbb{Set}_{fin}^\textbf{N}$. 
 \newline
 Consider the following example:
 \begin{ex}
 	Let $f: \textbf{1} + \textbf{1}_\bot \times \textbf{1}_\bot \rightarrow \textbf{1}_\bot + (\textbf{1}_\bot)_\bot  $ be an arrow in $\mathbb{FF_3}$:
 	\begin{figure}[h]
 		\centering
 			\begin{tikzpicture}[thick,scale=0.5, every node/.style={scale=0.9}]
 				\node (D) at (-2,0) {\textcolor{red}{$a_2$}};
 				
 				\node (F) at (2,0) {\textcolor{OliveGreen}{$b_2$}};
 				\node (G) at (2,3) {\textcolor{Aquamarine}{$b_1'$}};
 				\node (H) at (-1,3) {\textcolor{Aquamarine}{$b_1''$}};
 				\node (I) at (5,3) {\textcolor{Aquamarine}{$b_1$}};
 				\node (J) at (-1,6) {\textcolor{cyan}{$b_0''$}};
 				\node (K) at (5,6) {\textcolor{cyan}{$b_0$}};
 				
 				\node (a) at (6,1) {};
 				\node (b) at (11,1) {};
 				\draw[->, dotted, line width=.01in] (a) -- node[anchor=north] {$f$} (b);
 				
 				\node (L) at (14,0) {\textcolor{BrickRed}{$\alpha_2$}};
 				\node (L') at (14,3) {\textcolor{red}{$\alpha_1$}};
 				
 				\node (M) at (17,0) {\textcolor{OliveGreen}{$\beta_2$}};
 				\node (N) at (17,3) {\textcolor{Aquamarine}{$\beta_1$}};
 				\node (0) at (17,6) {\textcolor{cyan}{$\beta_0$}};
 				
 				\draw[line width=.03in, red] (L) -- (L');
 				\draw[line width=.03in, Aquamarine] (F) -- (G);
 				\draw[line width=.03in, Aquamarine] (F) -- (H);
 				\draw[line width=.03in, Aquamarine] (F) -- (I);
 				\draw[line width=.03in, cyan] (H) -- (J);
 				\draw[line width=.03in, cyan] (I) -- (K);
 				
 				\draw[line width=.03in, Aquamarine] (M) -- (N);
 				\draw[line width=.03in, cyan] (N) -- (0);
 		\end{tikzpicture}
 		\caption{The usual coloring notation is used.\newline So $f: a_2 \mapsto \alpha_2$, $b_1,b_1',b_1'' \mapsto \beta_1$ and $b_0, b_0'' \mapsto \beta_0$.}
 	\end{figure}
 	\newpage
 	If we follow the translation steps in \emph{reverse}, we obtain:
 	
 	\begin{figure}[h]
 		\centering
 		\begin{tikzcd}[scale=0.9]
 			{\{b_0'',} && {b_0\}} &&&& {\{\beta_0} & {\}} \\
 			{\{b_1'',} & {b_1',} & {b_1\}} &&&& {\{\beta_1,} & {\gamma_1\}} \\
 			{\{a_2,} & {b_2} & {\}} &&&& {\{\beta_2,} & {\gamma_2\}}
 			\arrow[maps to, from=1-1, to=2-1]
 			\arrow["{f_0}", maps to, from=1-3, to=2-3]
 			\arrow[maps to, from=2-1, to=3-2]
 			\arrow[maps to, from=2-2, to=3-2]
 			\arrow["{f_1}", maps to, from=2-3, to=3-2]
 			\arrow["{g_0}"', maps to, from=1-7, to=2-7]
 			\arrow["{g_1}"', maps to, from=2-7, to=3-7]
 			\arrow[maps to, from=2-8, to=3-8]
 			\arrow["{\tau_0}"', squiggly, from=1-3, to=1-7]
 			\arrow["{\tau_1}"', squiggly, from=2-3, to=2-7]
 			\arrow["{\tau_2}"', squiggly, from=3-3, to=3-7]
 		\end{tikzcd}
 		\caption{The transition maps are displayed as $F_0 \xrightarrow{f_0} F_1 \xrightarrow{f_1} F_2$ and
 			$G_0 \xrightarrow{g_0} G_1 \xrightarrow{g_1} G_2$}
 	\end{figure}
 	The nodes at each level correspond to sets at a particular time \footnote{in this case either 0,1 or 2.} and the edges between them indicate the transition maps.
 	\newline
 	Notice that, differently from \ref{chapter02}, finite forests seem to \emph{grow} from top to bottom \footnote{the notation reflects this as the top-most nodes have a 0 for subscript} where the transition maps $f_0,f_1$ need not be injective \footnote{this is seen for example with $b_1,b_1',b_1'' \mapsto \beta_1$.} or surjective \footnote{this corresponds for example to the emergence of $b_1' \in F_1$ which is not in the image of $f_0$.}.
 \end{ex}
 

 
 This is reflected in the fact that every transition function from say $F_m$ to $F_n$ with $m \leq n$ is an arbitrary set-function and, as we just said, need not be injective or surjective.\newline
 This simple fact gives the \emph{forest structure} we observed for variable sets. 
 
 \newpage
 The reverse translation from $\mathbb{FF_N}$ to $\mathbb{Set}^\textbf{N}$, however, breaks down when we consider arrows from finite forests of different height like in this simple case:
 
 \begin{ex}
 	Let $!_{\textbf{1}_\bot}$ be the only arrow from $\textbf{1}_\bot$ to the terminal $\textbf{1}$:
 	\begin{figure}[h]
 		\centering
 			\begin{tikzpicture}[thick,scale=0.7, every node/.style={scale=0.9}]
 				\node (A) at (0,0) {\textcolor{orange}{$a_1$}};
 				\node (a) at (0.5,0) {};
 				\node (B) at (0,3) {\textcolor{orange}{$a_0$}};
 				\node (b) at (0.5,3) {};
 				
 				\node (c) at (5.5,0) {};
 				\node (d) at (5.5,0.5) {};
 				\node (C) at (6,0) {\textcolor{orange}{$\alpha_1$}};
 				\draw[line width=.03in, orange] (A) -- (B);
 					\draw[|->, dotted, line width=.01in] (a) -- (c);
 					\draw[|->, dotted, line width=.01in] (b) -- (d);
 			\end{tikzpicture}
 		\caption{$\textbf{1}_\bot \xrightarrow{!_{\textbf{1}_\bot}} \textbf{1}$.}
 	\end{figure}
 	
 	We would need the following to be a commutative diagram:
 	\begin{figure}[h]
 		\centering
 		\begin{tikzcd}
 			& {\{ a_0 \}} && \emptyset \\
 			{} & {\{ a_1 \}} & {} & {\{\alpha_1\}}
 			\arrow["{f_0}", maps to, from=1-2, to=2-2]
 			\arrow["{\tau_1}"', squiggly, from=2-2, to=2-4]
 			\arrow["{g_0= \emptyset_{G_1}}", dashed, maps to, from=1-4, to=2-4]
 			\arrow["{?\tau_0}"', squiggly, from=1-2, to=1-4]
 		\end{tikzcd}
 		\caption{$f_0 : a_0 \mapsto a_1$ and $\tau_1 : a_1 \mapsto \alpha_1$ with $g_0= \emptyset_{G_1}$ the (unique) empty map from $\emptyset$ to $G_1 = \{\beta_1\}$. }
 	\end{figure}
 	\newline
 	But:
 	\begin{remark}
 		There exists no map from a non-empty set like $\{a_0\}$ to the empty-set $\emptyset$ which would be the component $\tau_0$ of $\tau: F \Rightarrow G$.
 	\end{remark}
 \end{ex}
 
 We conclude with the following considerations:
 
 \begin{remark}
 	The \emph{translation} $\tilde{F}$ from $\mathbb{Set}^\textbf{N}$ to $\mathbb{FF_N}$ defines an assignment for objects and morphisms and is functorial.\footnote{the functor $\tilde{F}$ is such that $\tilde{F}(id_A)=id_{\tilde{F}(A)}$ and $F(fg)=F(f)F(g)$.} 
 	\newline
 	The \emph{reverse translation} from $\mathbb{FF_N}$ to $\mathbb{Set}^\textbf{N}$ fails to be an assignment for all $\mathbb{FF_N}$-arrows.
 \end{remark}
 
 
 
 \begin{remark}
 	If, using our imperfect translation, we \emph{compare} the arrows in the two categories one notices for instance that the terminal object in $\mathbb{Set}^\textbf{N}$ is $\{0\} \xrightarrow{id} \{0\} \xrightarrow{id}... \xrightarrow{id}\{0\}$ a.k.a the \emph{finite telephone pole} whilst the terminal object in $\mathbb{FF}_*$ is the singleton forest \textbf{1}. \newline
 	For $N>1$ the \emph{telephone pole} in $\mathbb{FF}_N$, i.e., $((\textbf{1}_{\bot_1})_{\bot_2}..)_{\bot_{N-1}}$ is of course \emph{not} terminal as there can be many distinct arrows into it. \newline
 	Furthermore if we compare the sub-object classifiers (both seen as forests) between $\mathbb{FF_2}$/\emph{bushes} and $\mathbb{Set}^\textbf{2}$ we find a rather different structure:
 \begin{figure}[h]
 	\centering
 	\begin{subfigure}[h]{0.5\textwidth}
 		\centering
 		\begin{tikzcd}
 		{\textcolor{OliveGreen}{\bigcdot}} && \begin{tikzpicture}[scale=0.6]
 			\node (A) at (0,0) {\textcolor{red}{\textbf{f}}};
 			\node (B) at (3,0) {\textcolor{OliveGreen}{\textbf{t}}};
 			\node (C) at (3,3) {\textcolor{cyan}{$*$}};
 			\draw[cyan, line width=.03in] (B) -- (C);
 		\end{tikzpicture}
 		\arrow["true", from=1-1, to=1-3]
 	\end{tikzcd}
 	\caption{$\top : \mathbf{1} \xrightarrow{true} (\textbf{1} + \textbf{1}_\bot)$.}
 	\end{subfigure}
 	\hfil
 	\centering
 	\begin{subfigure}[h]{0.5\textwidth}
 		\centering
		\begin{tikzpicture}[thick,scale=0.6, every node/.style={scale=0.9}]
			\node (A) at (-2,0) {\textcolor{PineGreen}{0}};
			\node (B) at (-2,3) {\textcolor{SpringGreen}{0'}};
			\draw[line width=.03in, SpringGreen] (A) -- (B);
			
			\node (C) at (3,0) {\textcolor{red}{0}};
			\node (D) at (3,3) {\textcolor{YellowOrange}{0'}};
			\draw[line width=.03in, YellowOrange] (C) -- (D);
			
			\node (E) at (5,0) {\textcolor{PineGreen}{1}};
			\node (F) at (5,3) {\textcolor{cyan}{$\frac{1}{2}$'}};
			\node (G) at (7,3) {\textcolor{SpringGreen}{1'}};
			\draw[line width=.03in, cyan] (E) -- (F);
			\draw[line width=.03in, SpringGreen] (E) -- (G);
			
			\node (a) at (-1,1) {};
			\node (b) at (2,1) {};
			\draw[->, dotted, line width=.01in] (a) -- node[anchor=north] {$f_{\text{t}}$} (b); 			 	
		\end{tikzpicture}
		\caption{$f_t: \textbf{1}_\bot \rightarrow \textbf{1}_\bot + (2\cdot \textbf{1})_\bot$.}
 	\end{subfigure}
 \end{figure}
 \end{remark}

 	We conclude with the following observation:
%check 	
 	\begin{remark}
 		Having fixed an $N>0$, the arrows in $\mathbb{FF}_N$ that can be translated into natural transformations in $\mathbb{Set}^\textbf{N}$ must have the following requirements:
 		\begin{itemize}
 			\item the finite forests in the domain and co-domain must have the same height equal to $N$.
 			\item each node must be sent to a node of equal height. 
 		\end{itemize}
 	\end{remark}
 	
 	In a nutshell, for $N > 1$: 
 	
 	\begin{remark}
 		There are \emph{more} arrows in $\mathbb{FF}_N$ than in $\mathbb{Set}^\textbf{N}$.	
 	\end{remark}
 	
 	     
\begin{remark}
	Though remarkably similar, for any $N>0$ the categories of \emph{Variable Finite Sets} $\mathbb{Set}^\textbf{N}$ and \emph{finite forests of height at most N} $\mathbb{FF}_N$ in fact have very different structures.
	\newline
	The most relevant distinction for our concerns is that: $\mathbb{Set}^\textbf{N}$ is a topos for all $N$, while $\mathbb{FF}_N$, as we have proven through the lengthy counterexample in \ref{counterex},  is a topos only for $N \leq 2$.
\end{remark}
 
 	\newpage
 ${}$ \newpage
 
 \chapter{Conclusive Remarks}
 

We take stock of what has been achieved in this work:
\newline

The primary aim was to explore the topos semantics of $\mathcal{G}_3$. This has been done extensively thanks to the dual-algebraic semantics of $\mathcal{G}_3$ given by the sub-category of \emph{finite forests} known as $\mathbb{FF_2}/$\emph{bushes}.\newline
%check
In fact, citing \cite{towards}:
\begin{remark}
	we have seen that the category of \emph{bushes} represents a sort of \emph{best of both worlds} semantics as it already completely characterizes $\mathcal{G}_3$ at the propositional level through its duality with finite three-valued \emph{Gödel algebras} and at the same time is a \emph{topos} which provides a path to develop first order semantics for $\mathcal{G}_3$ based on \emph{bush}-concepts instead of \emph{sets}.
\end{remark}
 
 We used the tools of Categorical Logic available in \cite{goldblatt} \& \cite{lambekscott} to arrive at the following results:
 \newline
 At the propositional layer we found in \ref{externalandint}:
 \begin{prop*}
 		The topos of \emph{bushes}/$\mathbb{FF_2}$ is bivalent and non-Boolean.
 	\end{prop*}
In other words: \emph{internally} the propositional logic of \emph{bushes} is classical whilst \emph{externally} it is not.\newline
However, moving on to the first-order predicate level, we remedied this fact and recovered $\mathcal{G}_3$ \emph{internally} in \ref{anotherlook}.\newline
 
 Recall from \ref{intermediate} that:
 	First-order semantics for $\mathcal{G}_3$ was defined in the usual way using set-concepts and interpreted the quantifiers $\forall$ and $\exists$ as generalized $\land$ and $\lor$, i.e., in this case $min$ and $max$ of truth-values.	\newline
What we found in \ref{whataboutquant}, from the perspective of the first-order semantics of \emph{bushes}, is that:
\emph{universal and existential quantification are equivalent to finite conjunction and disjunction over generalized elements}, i.e.,:

\begin{prop*}
		The  first-order logic of the topos of $\mathbb{FF_2}$/\emph{bushes} corresponds to first-order three-valued Gödel-Dummett Logic on \emph{finite} domains.
\end{prop*}

\begin{remark}
	This justifies our approach building up from the propositional to the first order layer of topos-semantics of \emph{bushes} as it \emph{recovers} the first-order set-based semantics for $\mathcal{G}_3$ defined in \ref{intermediate}.
\end{remark}
	 
In chapter 5,  we compared our findings with alternative (propositional) topos-semantics for $\mathcal{G}_3$ given by \emph{variable sets} and \emph{sheaves on locales} found in \cite{goldblatt} and \cite{lisboa}. \newline
We reviewed the fact that \emph{$\mathcal{G}$ is the logic of sets through time} $\mathbb{Set}^\textbf{$\omega$}$ and observed the following corollary for \emph{bushes} linking  topos $ \models_{\mathcal{E}}$, Kripke $ \models_{K.}$ and Heyting algebra $\models_{H.A.}$ semantics:
\begin{cor*}
	\begin{equation*}
		\mathcal{G}_3 \vdash \phi\; \text{ iff } \; C_{3} \models_{H.A.} \phi \; \text{ iff } \; \textbf{2} \models_{K.} \phi
		\; \text{ iff } \; \mathbb{Set}^\textbf{2} \models_{\mathcal{E}} \phi.
	\end{equation*}
\end{cor*}
A topos semantics for $\mathcal{G}_3$ is thus given by $\mathbb{Set}^\textbf{2}$, i.e., the category of \emph{functions between sets}. \newline
With regards to \emph{sheaves on locales}, we proposed the following:\newline
The 3-chain $C_3$ in a topological context can be thought of as the locale of open subsets of the \emph{Sierpinski Space}.\newline
This gives us the following characterization inspired by  \cite{elephant}  which links Sheaves over the Sierpinski space $\textbf{Sh}(\mathcal{S})$, a.k.a. the \emph{Sierpinski topos}, to the category of presheaves over \textbf{2} or $\textbf{PSh}(\textbf{2})$, which in turns is equivalent to the category of \emph{functions between sets}.
\begin{gather*}
	\textbf{Sh}(\mathcal{S}) \simeq \textbf{PSh}(\textbf{2}) \simeq \mathbb{Set}^{\textbf{2}}. \\
	\mathcal{G}_{3} \vdash \phi \;\text{ iff }\;\mathbf{Sh}(\mathcal{S}) \models_{\mathcal{E}} \phi.
\end{gather*}
In other words:   
\begin{prop}
	$\mathcal{G}_3$  is the logic of the Sierpinski topos.
\end{prop}

What is the link between \emph{variable sets} and \emph{finite forests}?\newline\newline
It is the case, as we saw in \ref{forestsvar},  that for any $N>0$ objects and arrows in $\mathbb{Set}_{fin}^{\textbf{N}}$ can be easily translated into objects and arrows in $\mathbb{FF_N}$.\newline
However, the converse translation breaks down for arbitrary arrows in $\mathbb{FF}_*$.\newline
In fact back in \ref{chapter02} we introduced the categorical structure of \emph{finite forests} and, as it turns out, though strikingly similar, is remarkably different from that of \emph{variable sets}:
For $N>1$:
\begin{remark}
	The most relevant distinction for our concerns is that: $\mathbb{Set}^\textbf{N}$ is a topos for all $N$, while $\mathbb{FF}_N$, as we have proven through the lengthy counterexample in \ref{counterex},  is a topos only for $N \leq 2$.
\end{remark}

Recall also the observation made in \ref{forestsvar}: 

\begin{remark}
	In a nutshell, for $N > 1$, there are \emph{more} arrows in $\mathbb{FF}_N$ than in $\mathbb{Set}^\textbf{N}$.	
\end{remark}

This gives a new insight for the phrase \emph{"best of both worlds"} used to describe the category of \emph{bushes}:

\begin{remark}
	$\mathbb{FF_2}$/\emph{bushes} has a \emph{richer arrow structure} than its \emph{variable set}/\emph{presheaf} counterpart $\mathbb{Set}^{\textbf{2}}$ and, unlike all the other finite forests of greater height, has a topos structure.
\end{remark}

We conclude with a short selection of proposals for
further areas of inquiry:
\begin{itemize}
	\item A semantic approach using \emph{Lawvere Theories} for Gödel-Dummett Logic and other fuzzy logics like \emph{Nilpotent-Minimum Logic} $\mathcal{NM}$.
	\item A deeper categorical understanding of the \emph{non-toposness} of $\mathbb{FF_{k\geq3}}$.
	\item An improvement and expansion of the Python code used.
	\item Ind/Pro-finite completion of \emph{bushes} and associated topos semantics.
\end{itemize}

The link to the Python implementation of the tool used for computing operations between finite forests and the number of arrows between them in chapter 2 is given:\newline\newline \textbf{https://github.com/albertpaner/finiteforest.git}

	\newpage
${}$ \newpage

\begin{thebibliography}{30}
	\bibitem{towards}
	\emph{S. Aguzzoli and P. Codara}, \textbf{Towards an Algebraic Topos Semantics for Three-valued Gödel Logic} 2021 IEEE International Conference on Fuzzy Systems (FUZZ-IEEE), Luxembourg, 2021, pp. 1-6.
	
	\bibitem{fuzzy}
	\emph{Cintula P., Hájek P., Noguera C.} (2011).\textbf{ Handbook of Mathematical Fuzzy Logic} - volume 2. London : College Publications.
	
	\bibitem{metamath}
	\emph{Hájek, Petr} (1998). \textbf{Metamathematics of Fuzzy Logic}. Dordrecht, Boston and London: Kluwer Academic Publishers.
	
	\bibitem{firstorder}
	\emph{Baaz, Matthias et al.} \textbf{First-order Gödel logics.} Ann. Pure Appl. Log. 147 (2006): 23-47.
	
	\bibitem{manyval}
	\emph{S.Aguzzoli, P. Codara \& V.Marra.} \textbf{Gödel-Dummett logic, the category of forests and topoi} 
	ManyVal 2019 Department of C.S. University of Bucharest.
	
	\bibitem{recursive}
	\emph{S. Aguzzoli and P. Codara}, \textbf{Recursive formulas to compute coproducts of finite Gödel algebras and related structures} 2016 IEEE International Conference on Fuzzy Systems (FUZZ-IEEE), Vancouver, BC, Canada, 2016, pp. 201-208.
	
	\bibitem{computing}
	\emph{D’Antona, Ottavio M. \& Marra, Vincenzo} (2006). \textbf{Computing coproducts of finitely presented Gödel algebras}. Annals of Pure and Applied Logic 142 (1):202-211.
	
	\bibitem{stone}
	\emph{Stone, M.H.} \textbf{The theory of representation of Boolean algebras.} Trans. Amer. Math Soc. 40, 1936.
	
	\bibitem{goldblatt}
	\emph{Goldblatt, R. I.} (1982). \textbf{Topoi: The Categorial Analysis of Logic}. British Journal for the Philosophy of Science 33 (1):95-97.
	
	\bibitem{awodey}
	\emph{Awodey, Steve} (2006). \textbf{Category Theory}. Oxford, England: Oxford University Press.
	
	\bibitem{godel}
	\emph{Solomon Feferman, John W. Dawson, Stephen C. Kleene, Gregory H. Moore, Robert M. Solovay, and Jean van Heijenoort} (Eds.). 1986. \textbf{Kurt Godel: collected works. Vol. 1: Publications 1929-1936}. Oxford University Press, Inc., USA.
	
	
	\bibitem{lambekscott}
	\emph{Lambek, J. \& Scott, P. J.} (1989). \textbf{Introduction to Higher Order Categorical Logic}. Journal of Symbolic Logic.
	
	\bibitem{borceaux}
	\emph{Borceux, F.} (1994). \textbf{Handbook of Categorical Algebra 3: Categories of Sheaves} (Encyclopedia of Mathematics and its Applications). Cambridge: Cambridge University Press.
	
	\bibitem{lisboa}
	\emph{Pedro Filipe.} \textbf{A topoi characterization of Gödel intermediate logics.} Master’s
	thesis, Instituto Superior Técnico, 2017
	
	\bibitem{elephant}
	\emph{Johnstone, Peter T.} (2002). \textbf{Sketches of an Elephant: A Topos Theory Compendium}, Volume 1. Oxford, England: Clarendon Press.
	
	\bibitem{heyting}
	\emph{A. Heyting} (1930), \textbf{Die Formalen Regeln der intuitionistischen Logik} Sitzungsberichte der Preussischen Akademie von Wissenschaften. Physikalisch Mathematische Klasse 42–56. A.
	
	\bibitem{maclane}
	\emph{MacLane, S.} (1971). \textbf{Categories for the Working Mathematician}. New York: Springer-Verlag.
	
	\bibitem{mtl}
	\emph{Esteva, Francesc \& Godo, Lluis.} (2001) \textbf{Monoidal t-norm Based Logic: Towards a Logic for Left-continuous t-norms}. Fuzzy Sets and Systems. Published by Elsevier BV. 
	
	
\end{thebibliography}



 
% \appendix
% \chapter{Appendix}
% \begin{thebibliography}{30}
	\bibitem{towards}
	\emph{S. Aguzzoli and P. Codara}, \textbf{Towards an Algebraic Topos Semantics for Three-valued Gödel Logic} 2021 IEEE International Conference on Fuzzy Systems (FUZZ-IEEE), Luxembourg, 2021, pp. 1-6.
	
	\bibitem{fuzzy}
	\emph{Cintula P., Hájek P., Noguera C.} (2011).\textbf{ Handbook of Mathematical Fuzzy Logic} - volume 2. London : College Publications.
	
	\bibitem{metamath}
	\emph{Hájek, Petr} (1998). \textbf{Metamathematics of Fuzzy Logic}. Dordrecht, Boston and London: Kluwer Academic Publishers.
	
	\bibitem{firstorder}
	\emph{Baaz, Matthias et al.} \textbf{First-order Gödel logics.} Ann. Pure Appl. Log. 147 (2006): 23-47.
	
	\bibitem{manyval}
	\emph{S.Aguzzoli, P. Codara \& V.Marra.} \textbf{Gödel-Dummett logic, the category of forests and topoi} 
	ManyVal 2019 Department of C.S. University of Bucharest.
	
	\bibitem{recursive}
	\emph{S. Aguzzoli and P. Codara}, \textbf{Recursive formulas to compute coproducts of finite Gödel algebras and related structures} 2016 IEEE International Conference on Fuzzy Systems (FUZZ-IEEE), Vancouver, BC, Canada, 2016, pp. 201-208.
	
	\bibitem{computing}
	\emph{D’Antona, Ottavio M. \& Marra, Vincenzo} (2006). \textbf{Computing coproducts of finitely presented Gödel algebras}. Annals of Pure and Applied Logic 142 (1):202-211.
	
	\bibitem{stone}
	\emph{Stone, M.H.} \textbf{The theory of representation of Boolean algebras.} Trans. Amer. Math Soc. 40, 1936.
	
	\bibitem{goldblatt}
	\emph{Goldblatt, R. I.} (1982). \textbf{Topoi: The Categorial Analysis of Logic}. British Journal for the Philosophy of Science 33 (1):95-97.
	
	\bibitem{awodey}
	\emph{Awodey, Steve} (2006). \textbf{Category Theory}. Oxford, England: Oxford University Press.
	
	\bibitem{godel}
	\emph{Solomon Feferman, John W. Dawson, Stephen C. Kleene, Gregory H. Moore, Robert M. Solovay, and Jean van Heijenoort} (Eds.). 1986. \textbf{Kurt Godel: collected works. Vol. 1: Publications 1929-1936}. Oxford University Press, Inc., USA.
	
	
	\bibitem{lambekscott}
	\emph{Lambek, J. \& Scott, P. J.} (1989). \textbf{Introduction to Higher Order Categorical Logic}. Journal of Symbolic Logic.
	
	\bibitem{borceaux}
	\emph{Borceux, F.} (1994). \textbf{Handbook of Categorical Algebra 3: Categories of Sheaves} (Encyclopedia of Mathematics and its Applications). Cambridge: Cambridge University Press.
	
	\bibitem{lisboa}
	\emph{Pedro Filipe.} \textbf{A topoi characterization of Gödel intermediate logics.} Master’s
	thesis, Instituto Superior Técnico, 2017
	
	\bibitem{elephant}
	\emph{Johnstone, Peter T.} (2002). \textbf{Sketches of an Elephant: A Topos Theory Compendium}, Volume 1. Oxford, England: Clarendon Press.
	
	\bibitem{heyting}
	\emph{A. Heyting} (1930), \textbf{Die Formalen Regeln der intuitionistischen Logik} Sitzungsberichte der Preussischen Akademie von Wissenschaften. Physikalisch Mathematische Klasse 42–56. A.
	
	\bibitem{maclane}
	\emph{MacLane, S.} (1971). \textbf{Categories for the Working Mathematician}. New York: Springer-Verlag.
	
	\bibitem{mtl}
	\emph{Esteva, Francesc \& Godo, Lluis.} (2001) \textbf{Monoidal t-norm Based Logic: Towards a Logic for Left-continuous t-norms}. Fuzzy Sets and Systems. Published by Elsevier BV. 
	
	
\end{thebibliography}



 	
 \end{document}